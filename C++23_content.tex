\documentclass[C++.tex]{subfiles}
\begin{document}

\section{C++23 \og{}\textit{Pandemic Edition}\fg{}}
\subsection*{Présentation}
\begin{frame}
	\frametitle{Présentation}
	\begin{itemize}
		\item Travaux techniques terminés en février 2023
		\item Document final en cours de validation internationale
		\item Dernier \textit{Working Draft} : \href{https://github.com/cplusplus/draft/releases/download/n4917/n4917.pdf}{n4917}
	\end{itemize}
\end{frame}

\subsection*{Organisation}
\begin{frame}
	\frametitle{Changements d'organisation du comité}
	\begin{itemize}
		\item Création d'un \textit{ABI Review Group} : étude des impacts des évolutions sur l'ABI
		\item Création d'un \textit{Study Group} pour la liaison C/C++ (SG22)
		\item Création d'un \textit{Study Group} \textit{safety and security} (SG23)
	\end{itemize}
\end{frame}

\subsection*{Dépréciations et suppressions}
\begin{frame}[fragile]
	\frametitle{Dépréciations et suppressions}
	\begin{itemize}
		\item Suppression des fonctionnalités liées au support d'un GC

\note[item]{Suppression directe, sans dépréciation. Pas utilisées en pratique}

		\item Dépréciation de \lstinline|std::aligned_storage| et \lstinline|std::aligned_union|
		\item Dépréciation de \lstinline|std::std::numeric_limits::has_denorm|
	\end{itemize}
\end{frame}

\subsection*{Syntaxe}
\begin{frame}[fragile]
	\frametitle{Espaces en fin de ligne}
	\begin{itemize}
		\item Espaces ignorés après le \lstinline|\| de séparation de ligne
	\end{itemize}

	\begin{lstlisting}[language=C++]
// Toujours une chaine vide en C++23
auto str = "\<space>
";\end{lstlisting}

\note[item]{Avant soit une chaîne vide, soit \lstinline|"\ "| selon le compilateur}
\end{frame}

\begin{frame}[fragile]
	\frametitle{Label}
	\begin{itemize}
		\item Label autorisé en fin de bloc
		\item Reprise d'une évolution C2X
	\end{itemize}

	\begin{lstlisting}[language=C++]
void foo(void) {
  int x;
  x = 1;
last: }\end{lstlisting}
\end{frame}

\subsection*{Compilation conditionnelle et macro}
\begin{frame}[fragile]
	\frametitle{Compilation conditionnelle}
	\begin{itemize}
		\item Ajout de \lstinline|#elifdef| et \lstinline|#elifndef|
		\item Équivalent à \lstinline|#elif defined| et \lstinline|#elif not defined|
	\end{itemize}

	\begin{lstlisting}[language=C++]
#ifdef FOO
...
#elifdef BAR
...
#endif\end{lstlisting}

	\begin{itemize}
		\item Se combine avec \lstinline|#if| et \lstinline|#elif|
	\end{itemize}

	\vskip 10mm plus 10fill
	\hfill
	\href{https://godbolt.org/#g:!((g:!((g:!((h:codeEditor,i:(filename:'1',fontScale:14,fontUsePx:'0',j:1,lang:c%2B%2B,selection:(endColumn:3,endLineNumber:13,positionColumn:3,positionLineNumber:13,selectionStartColumn:3,selectionStartLineNumber:13,startColumn:3,startLineNumber:13),source:'%23include+%3Ciostream%3E%0A+%0A//%23define+FOO%0A//%23define+BAR%0A%0Aint+main()%0A%7B%0A%23ifdef+FOO%0A++std::cout+%3C%3C+%22FOO%5Cn%22%3B%0A%23elifdef+BAR%0A++std::cout+%3C%3C+%22BAR%5Cn%22%3B%0A%23else%0A++std::cout+%3C%3C+%22Autre%5Cn%22%3B%0A%23endif%0A%7D%0A'),l:'5',n:'0',o:'C%2B%2B+source+%231',t:'0')),k:50,l:'4',n:'0',o:'',s:0,t:'0'),(g:!((h:executor,i:(argsPanelShown:'1',compilationPanelShown:'0',compiler:g122,compilerName:'',compilerOutShown:'0',execArgs:'',execStdin:'',fontScale:14,fontUsePx:'0',j:1,lang:c%2B%2B,libs:!((name:boost,ver:'175')),options:'-std%3Dc%2B%2B23+-Wall+-Wextra+-pedantic',source:1,stdinPanelShown:'1',tree:'1',wrap:'0'),l:'5',n:'0',o:'Executor+x86-64+gcc+12.2+(C%2B%2B,+Editor+%231)',t:'0')),header:(),k:50,l:'4',n:'0',o:'',s:0,t:'0')),l:'2',n:'0',o:'',t:'0')),version:4}{\includegraphics[height=0.5cm]{input_src/play-button.png}}
\end{frame}

\begin{frame}[fragile]
	\frametitle{Avertissement}
	\begin{itemize}
		\item \lstinline|#warning| génère un avertissement à la compilation

\note[item]{Déjà présent dans la plupart des compilateurs}
	\end{itemize}

	\vskip 10mm plus 10fill
	\hfill
	\href{https://godbolt.org/#g:!((g:!((g:!((h:codeEditor,i:(filename:'1',fontScale:14,fontUsePx:'0',j:1,lang:c%2B%2B,selection:(endColumn:1,endLineNumber:11,positionColumn:1,positionLineNumber:11,selectionStartColumn:1,selectionStartLineNumber:1,startColumn:1,startLineNumber:1),source:'%23include+%3Ciostream%3E%0A+%0Aint+main()%0A%7B%0A%23warning+%22Programme+vide%22%0A%0A%23if+0%0A%23error+%22Programme+vide%22%0A%23endif%0A%7D%0A'),l:'5',n:'0',o:'C%2B%2B+source+%231',t:'0')),k:50,l:'4',n:'0',o:'',s:0,t:'0'),(g:!((h:executor,i:(argsPanelShown:'1',compilationPanelShown:'0',compiler:g122,compilerName:'',compilerOutShown:'0',execArgs:'',execStdin:'',fontScale:14,fontUsePx:'0',j:1,lang:c%2B%2B,libs:!((name:boost,ver:'175')),options:'-std%3Dc%2B%2B23+-Wall+-Wextra+-pedantic',source:1,stdinPanelShown:'1',tree:'1',wrap:'0'),l:'5',n:'0',o:'Executor+x86-64+gcc+12.2+(C%2B%2B,+Editor+%231)',t:'0')),header:(),k:50,l:'4',n:'0',o:'',s:0,t:'0')),l:'2',n:'0',o:'',t:'0')),version:4}{\includegraphics[height=0.5cm]{input_src/play-button.png}}
\end{frame}

\subsection*{Durée de vie}
\begin{frame}[fragile]
	\frametitle{Gestion explicite de la durée de vie}
	\begin{itemize}
		\item \lstinline|std::start_lifetime_as| et \lstinline|std::start_lifetime_as_array| indiquent qu'un objet est créé mais sans initialisation
	\end{itemize}

	\begin{lstlisting}[language=C++]
struct X { int a, b; };

X* p = start_lifetime_as<X>(malloc(sizeof(struct X));
p->a = 1;
p->b = 2;\end{lstlisting}
\end{frame}

\subsection*{Types}
\begin{frame}[fragile]
	\frametitle{Types flottants étendus}
	\begin{itemize}
		\item Ajout de \lstinline|std::float16_t|, \lstinline|std::float32_t|, \lstinline|std::float64_t|, \lstinline|std::float128_t|
		\begin{itemize}
			\item Types IEEE N-bit
			\item Support optionnel
		\end{itemize}
		\item Ajout de \lstinline|std::bfloat16_t|
		\begin{itemize}
			\item Type IEEE binary16
			\item Support optionnel
		\end{itemize}
		\item Ajout des suffixes littéraux correspondant (\lstinline|f16|, \lstinline|f32|, \lstinline|f64|, \lstinline|f128| et \lstinline|bf16|)
		\item Prise en compte par \lstinline|std::format|, \lstinline|std::ostream| et \lstinline|std::istream|
		\item Prise en compte par \lstinline|std::numeric_limits| et \lstinline|std::is_floating_point|
		\item Ajout de surcharges dans \lstinline|<cmath>|, \lstinline|<complex>| et \lstinline|<atomic>|
	\end{itemize}

	\begin{alertblock}{Types indépendants}
		Types indépendants (pas d'alias) de \lstinline|float|, \lstinline|double| ou \lstinline|long double|
	\end{alertblock}
\end{frame}

\begin{frame}[fragile]
	\frametitle{Évolutions de \lstinline|char8_t|}
	\begin{itemize}
		\item Initialisation d'un tableau de \lstinline|char| ou d'\lstinline|unsigned char| depuis une chaîne littérale UTF-8
	\end{itemize}
\end{frame}

\begin{frame}[fragile]
	\frametitle{Relâchement des contraintes de \lstinline|wchar_t|}
	\begin{itemize}
		\item Suppression de la contrainte
	\end{itemize}

	\begin{block}{}
		The values of type \lstinline|wchar_t| can represent distinct codes for all members of the largest extended character set specified among the supported locale
	\end{block}

	\begin{itemize}		
		\item Permet l'utilisation de \lstinline|wchar_t| pour représenter des caractères UTF-16 ou UCS-2 sur des systèmes supportant UTF-8

\note[item]{Tous les codes points Unicode (21 bits) ne sont pas représentables sur un \lstinline|wchar_t| 16 bits}
\note[item]{Déjà le cas sur Visual C++}
	\end{itemize}
\end{frame}

\begin{frame}[fragile]
	\frametitle{Conversions}
	\begin{itemize}
		\item Ajout d'une conversion implicite en booléen
		\begin{itemize}
			\item Dans les \lstinline|static_assert|
			\item Dans les \lstinline|if constexpr|
		\end{itemize}
	\end{itemize}

	\begin{lstlisting}[language=C++]
// Valide en C++23, pas en C++20
if constexpr(flags & 0x01) { ... }
else { ... }\end{lstlisting}

	\begin{lstlisting}[language=C++]
// Valide en C++23, pas en C++20
template <std::size_t N>
class Foo {	static_assert(N, "Message"); };\end{lstlisting}

\note[item]{En pratique, ces codes étaient déjà acceptées, au moins sous GCC et clang}

	\begin{itemize}
		\item \lstinline|auto(x)| et \lstinline|auto{x}| convertissent \lstinline|x| en \textit{prvalue}

\note[item]{C'est une \textit{decay-copy} (copie faible ?)}
\note[item]{\lstinline|auto a = x| construit une \textit{lvalue}, non une \textit{prvalue}}
	\end{itemize}
\end{frame}

\begin{frame}[fragile]
	\frametitle{Énumérations}
	\begin{itemize}
		\item \lstinline|std::to_underlying| convertit une énumération vers le type sous-jacent
	\end{itemize}

	\begin{lstlisting}[language=C++]
enum class FOO : uint32_t { A = 0xABCDEF };

auto bar = to_underlying(FOO::A); // uint32_t\end{lstlisting}

	\vskip 10mm plus 10fill
	\hfill
	\href{https://godbolt.org/#g:!((g:!((g:!((h:codeEditor,i:(filename:'1',fontScale:14,fontUsePx:'0',j:1,lang:c%2B%2B,selection:(endColumn:6,endLineNumber:17,positionColumn:6,positionLineNumber:17,selectionStartColumn:6,selectionStartLineNumber:17,startColumn:6,startLineNumber:17),source:'%23include+%3Ciostream%3E%0A%23include+%3Cutility%3E%0A%23include+%3Ccstdint%3E%0A%23include+%3Ctypeinfo%3E%0A%23include+%3Ccassert%3E%0A%0Aenum+class+FOO+:+uint32_t+%0A%7B+%0A++A+%3D+0xABCDEF,%0A%7D%3B%0A%0Aint+main()%0A%7B%0A++auto+bar+%3D+std::to_underlying(FOO::A)%3B%0A%0A++assert(typeid(bar)+%3D%3D+typeid(uint32_t))%3B%0A%23if+0%0A++assert(typeid(bar)+%3D%3D+typeid(int))%3B%0A%23endif%0A%7D%0A'),l:'5',n:'0',o:'C%2B%2B+source+%231',t:'0')),k:50,l:'4',n:'0',o:'',s:0,t:'0'),(g:!((h:executor,i:(argsPanelShown:'1',compilationPanelShown:'0',compiler:g122,compilerName:'',compilerOutShown:'0',execArgs:'',execStdin:'',fontScale:14,fontUsePx:'0',j:1,lang:c%2B%2B,libs:!((name:boost,ver:'175')),options:'-std%3Dc%2B%2B23+-Wall+-Wextra+-pedantic',source:1,stdinPanelShown:'1',tree:'1',wrap:'0'),l:'5',n:'0',o:'Executor+x86-64+gcc+12.2+(C%2B%2B,+Editor+%231)',t:'0')),header:(),k:50,l:'4',n:'0',o:'',s:0,t:'0')),l:'2',n:'0',o:'',t:'0')),version:4}{\includegraphics[height=0.5cm]{input_src/play-button.png}}
\end{frame}

\subsection*{constexpr}
\begin{frame}[fragile]
	\frametitle{\lstinline|constexpr|}
	\begin{itemize}
		\item Relâchement de contrainte sur les fonctions \lstinline|constexpr|
		\begin{itemize}
			\item Code non évalué au \textit{compile-time}

\note[item]{Soit la fonction n'est en pratique jamais appelé au \textit{compile-time} soit la branche contenant ce code n'est exécutée qu'au \textit{run-time}}

			\begin{itemize}
				\item Variables non littérales
				\item Utilisation de \lstinline|goto|
				\item Retour non littéral
				\item Paramètres non littéraux
				\item Appel de fonctions non \lstinline|constexpr|
			\end{itemize}
			\item Code non évalué au \textit{compile-time} ou utilisable dans un contexte constant
			\begin{itemize}
				\item Variables \lstinline|static| ou \lstinline|thread_local|

\note[item]{P.ex. une variable \lstinline|static| \lstinline|constexpr|}
			\end{itemize}
			\item Valeur non utilisée
			\begin{itemize}
				\item Utilisation de pointeurs ou références inconnus

\note[item]{P.ex. \lstinline|sizeof *ptr|}
			\end{itemize}
		\end{itemize}
		\item Conversion implicite de fonctions \lstinline|constexpr| en \lstinline|consteval| 

\note[item]{Lorsque la fonction ne peut qu'être invoquée au \textit{compile-time}}

		\item Davantage de \lstinline|constexpr| dans la bibliothèque standard
	\end{itemize}
\end{frame}

\subsection*{if consteval}
\begin{frame}[fragile]
	\frametitle{\lstinline|if consteval|}
	\begin{itemize}
		\item Branche prise en compte si le code est évalué au \textit{compile-time}
		\item Peut appeler des fonctions immédiate

\note[item]{P.ex. dans une fonction \lstinline|constexpr|}

		\item \lstinline|else| pour le code évalué au \textit{run-time}
	\end{itemize}

	\begin{lstlisting}[language=C++]
consteval int foo(int i) { return i; }

constexpr int bar(int i) {
  if consteval { return foo(i) + 1; } 
  else { return 42; } }\end{lstlisting}

	\vskip 10mm plus 10fill
	\hfill
	\href{https://godbolt.org/#g:!((g:!((g:!((h:codeEditor,i:(filename:'1',fontScale:14,fontUsePx:'0',j:1,lang:c%2B%2B,selection:(endColumn:1,endLineNumber:27,positionColumn:1,positionLineNumber:27,selectionStartColumn:1,selectionStartLineNumber:1,startColumn:1,startLineNumber:1),source:'%23include+%3Ciostream%3E%0A%0Aconsteval+int+foo(int+i)%0A%7B%0A++return+i%3B%0A%7D%0A%0Aconstexpr+int+bar(int+i)%0A%7B%0A++if+consteval%0A++%7B%0A++++return+foo(i)+%2B+1%3B%0A++%7D%0A++else%0A++%7B%0A++++return+42%3B%0A++%7D%0A%7D%0A%0Aint+main()%0A%7B%0A++std::cout+%3C%3C+bar(5)+%3C%3C+%22%5Cn%22%3B%0A%0A++constexpr+int+baz+%3D+bar(10)%3B%0A++std::cout+%3C%3C+baz+%3C%3C+%22%5Cn%22%3B%0A%7D%0A'),l:'5',n:'0',o:'C%2B%2B+source+%231',t:'0')),k:50,l:'4',n:'0',o:'',s:0,t:'0'),(g:!((h:executor,i:(argsPanelShown:'1',compilationPanelShown:'0',compiler:g122,compilerName:'',compilerOutShown:'0',execArgs:'',execStdin:'',fontScale:14,fontUsePx:'0',j:1,lang:c%2B%2B,libs:!((name:boost,ver:'175')),options:'-std%3Dc%2B%2B23+-Wall+-Wextra+-pedantic',source:1,stdinPanelShown:'1',tree:'1',wrap:'0'),l:'5',n:'0',o:'Executor+x86-64+gcc+12.2+(C%2B%2B,+Editor+%231)',t:'0')),header:(),k:50,l:'4',n:'0',o:'',s:0,t:'0')),l:'2',n:'0',o:'',t:'0')),version:4}{\includegraphics[height=0.5cm]{input_src/play-button.png}}
\end{frame}

\begin{frame}[fragile]
	\frametitle{\lstinline|if consteval|}
	\begin{itemize}
		\item Négation possible
	\end{itemize}

	\begin{lstlisting}[language=C++]
if not consteval { ... }
// ou
if ! consteval { ... }\end{lstlisting}

	\begin{alertblock}{Attention}
		\begin{itemize}
			\item Accolades obligatoires, même avec une unique instruction
		\end{itemize}
	\end{alertblock}
\end{frame}

\subsection*{Sémantique de déplacement}
\begin{frame}[fragile]
	\frametitle{Évolutions de la sémantique de déplacement}
	\begin{itemize}
		\item Simplification des règles de déplacement implicite
		\item \lstinline|std::move_only_function| équivalent \textit{move-only} de \lstinline|std::function|
	\end{itemize}
\end{frame}

\subsection*{Range-based for loop}
\begin{frame}[fragile]
	\frametitle{Durée de vie des temporaires}
	\begin{itemize}
		\item Extension de la durée de vie des objets temporaires créés dans l'initialisation d'un range-based for loop jusqu'à la fin de la boucle
	\end{itemize}

	\begin{lstlisting}[language=C++]
const vector<int>& foo(const vector<int>& t) { return t; }
vector<int> bar( return vector<int>{1, 2, 3}; );

// Valide, duree de vie du retour de bar est etendu
for (auto e : foo(bar())) { ... }\end{lstlisting}
\end{frame}

\subsection*{init-statement}
\begin{frame}[fragile]
	\frametitle{\textit{init-statement}}
	\begin{itemize}
		\item \lstinline|using| possible dans l'\textit{init-statement} de \lstinline|if|, \lstinline|switch| et \lstinline|for|
	\end{itemize}

	\begin{lstlisting}[language=C++]
for(using T = int; T e : v) {
  ... }\end{lstlisting}

\note[item]{En C++20, il est possible d'utiliser \lstinline|typedef| dans l'\textit{init-statement}}

	\vskip 10mm plus 10fill
	\hfill
	\href{https://godbolt.org/#g:!((g:!((g:!((h:codeEditor,i:(filename:'1',fontScale:14,fontUsePx:'0',j:1,lang:c%2B%2B,selection:(endColumn:1,endLineNumber:13,positionColumn:1,positionLineNumber:13,selectionStartColumn:1,selectionStartLineNumber:1,startColumn:1,startLineNumber:1),source:'%23include+%3Ciostream%3E%0A%23include+%3Cvector%3E%0A%0Aint+main()%0A%7B%0A++std::vector%3Cint%3E+v%7B1,+2,+5%7D%3B%0A+%0A++for(using+T+%3D+int%3B+T+e+:+v)%0A++%7B%0A++++std::cout+%3C%3C+e+%3C%3C+%22%5Cn%22%3B%0A++%7D%0A%7D%0A'),l:'5',n:'0',o:'C%2B%2B+source+%231',t:'0')),k:50,l:'4',n:'0',o:'',s:0,t:'0'),(g:!((h:executor,i:(argsPanelShown:'1',compilationPanelShown:'0',compiler:g122,compilerName:'',compilerOutShown:'0',execArgs:'',execStdin:'',fontScale:14,fontUsePx:'0',j:1,lang:c%2B%2B,libs:!((name:boost,ver:'175')),options:'-std%3Dc%2B%2B23+-Wall+-Wextra+-pedantic',source:1,stdinPanelShown:'1',tree:'1',wrap:'0'),l:'5',n:'0',o:'Executor+x86-64+gcc+12.2+(C%2B%2B,+Editor+%231)',t:'0')),header:(),k:50,l:'4',n:'0',o:'',s:0,t:'0')),l:'2',n:'0',o:'',t:'0')),version:4}{\includegraphics[height=0.5cm]{input_src/play-button.png}}
\end{frame}

\subsection*{Littéraux}
\begin{frame}[fragile]
	\frametitle{Encodage}
	\begin{itemize}
		\item Support des fichiers sources en UTF-8

\note[item]{Auparavant supporté par plusieurs compilateurs, pas imposé par la norme}

		\item Encodage identique entre le préprocesseur et le code C++

\note[item]{Auparavant les deux encodages pouvaient être différents}
\note[item]{Et ainsi \lstinline|\#if 'A' == '\\x41'| et \lstinline|if('A' == 0x41)| donner des résultats différents}
	\end{itemize}
\end{frame}

\begin{frame}[fragile]
	\frametitle{Suffixes littéraux}
	\begin{itemize}
		\item Suffixe \lstinline|uz| pour \lstinline|size_t|
		\item Suffixe \lstinline|z| pour le type entier signé correspondant à \lstinline|size_t|

\note[item]{Typiquement le retour de \lstinline|std::ssize()|}
\note[item]{Correspond à \lstinline|ptrdiff_t| ou \lstinline|ssize_t|}

		\item \lstinline|z| utilisable pour les littéraux binaires, octaux ou hexadécimaux de \lstinline|size_t|
	\end{itemize}

	\vskip 10mm plus 10fill
	\hfill
	\href{https://godbolt.org/#g:!((g:!((g:!((h:codeEditor,i:(filename:'1',fontScale:14,fontUsePx:'0',j:1,lang:c%2B%2B,selection:(endColumn:1,endLineNumber:15,positionColumn:1,positionLineNumber:15,selectionStartColumn:1,selectionStartLineNumber:1,startColumn:1,startLineNumber:1),source:'%23include+%3Ciostream%3E%0A%23include+%3Ctypeinfo%3E%0A%0Aint+main()%0A%7B%0A++auto+i+%3D+5uz%3B%0A++auto+j+%3D+5z%3B%0A%0A++std::cout+%3C%3C+std::boolalpha%3B%0A++std::cout+%3C%3C+(typeid(i)+%3D%3D+typeid(size_t))+%3C%3C+%22%5Cn%22%3B%0A++std::cout+%3C%3C+(typeid(i)+%3D%3D+typeid(int))+%3C%3C+%22%5Cn%22%3B%0A++std::cout+%3C%3C+(typeid(j)+%3D%3D+typeid(size_t))+%3C%3C+%22%5Cn%22%3B%0A++std::cout+%3C%3C+(typeid(j)+%3D%3D+typeid(ptrdiff_t))+%3C%3C+%22%5Cn%22%3B%0A%7D%0A'),l:'5',n:'0',o:'C%2B%2B+source+%231',t:'0')),k:50,l:'4',n:'0',o:'',s:0,t:'0'),(g:!((h:executor,i:(argsPanelShown:'1',compilationPanelShown:'0',compiler:g122,compilerName:'',compilerOutShown:'0',execArgs:'',execStdin:'',fontScale:14,fontUsePx:'0',j:1,lang:c%2B%2B,libs:!((name:boost,ver:'175')),options:'-std%3Dc%2B%2B23+-Wall+-Wextra+-pedantic',source:1,stdinPanelShown:'1',tree:'1',wrap:'0'),l:'5',n:'0',o:'Executor+x86-64+gcc+12.2+(C%2B%2B,+Editor+%231)',t:'0')),header:(),k:50,l:'4',n:'0',o:'',s:0,t:'0')),l:'2',n:'0',o:'',t:'0')),version:4}{\includegraphics[height=0.5cm]{input_src/play-button.png}}
\end{frame}

\begin{frame}[fragile]
	\frametitle{Chaînes littérales}
	\begin{itemize}
		\item Plus de concaténation de chaînes littérales adjacentes d'encodage différent
	\end{itemize}

	\begin{lstlisting}[language=C++]
L"" u"";	// Invalide
L"" u8"";	// Invalide
L"" U"";	// Invalide
u8"" L"";	// Invalide
u8"" u"";	// Invalide
u8"" U"";	// Invalide
u"" L"";	// Invalide
u"" u8"";	// Invalide
u"" U"";	// Invalide
U"" L"";	// Invalide
U"" u"";	// Invalide
U"" u8"";	// Invalide\end{lstlisting}

	\begin{block}{Et si  ?}
		Si une des chaînes n'a pas d'encodage, on utilise celui de la seconde
	\end{block}
\end{frame}

\begin{frame}[fragile]
	\frametitle{Caractères littéraux}
	\begin{itemize}
		\item Caractères Unicode conservés durant la phase du préprocesseur
	\end{itemize}

	\begin{lstlisting}[language=C++]
#define S(x) # x
// C++23 : "Köppe"
const char * s1 = S(Köppe);
const char * s2 = S(K\u00f6ppe);\end{lstlisting}

\note[item]{En pratique, avant C++23, le traitement dépend du compilateur}

	\begin{itemize}
		\item Suppression des caractères littéraux larges non codables ou multi-caractères
		\item Ajout de séquences d'échappement délimitées

\note[item]{Délimitées et pas de taille fixe}

		\begin{itemize}
			\item \lstinline|\u{}| prenant un nombre arbitraire de chiffres hexadécimaux
			\item \lstinline|\x{}| prenant un nombre arbitraire de chiffres hexadécimaux
			\item \lstinline|\o{}| prenant un nombre arbitraire de chiffres octaux
		\end{itemize}
		\item Ajout de séquences d'échappement nommés \lstinline|\N{...}|
	\end{itemize}

	\begin{lstlisting}[language=C++]
cout << "\N{GREEK SMALL LETTER ETA WITH PSILI}";\end{lstlisting}
\end{frame}

\subsection*{Opérateurs}
\begin{frame}[fragile]
	\frametitle{Évolutions des opérateurs d'égalité}
	\begin{itemize}
		\item Modification des règles de résolution de \lstinline|operator==| et \lstinline|operator!=|
		\item Corrige des ambiguïtés introduites par la réécriture de \lstinline|==| et \lstinline|!=| en C++20

\note[item]{En pratique, les compilateurs, notamment GCC, sont plus laxistes que ce qu'impose la norme et accepte silencieusement des cas ambigus}

		\item \lstinline|operator==| est utilisé pour réécrire \lstinline|operator!=| et la forme inverse de \lstinline|operator==| uniquement si \lstinline|operator!=| n'existe pas 
	\end{itemize}

\begin{lstlisting}[language=C++]
struct Foo {
  bool operator==(const Foo&) { return true; }
  bool operator!=(const Foo&) { return false; } };

// Ambigu en C++20
bool b = Foo{} != Foo{};\end{lstlisting}

\note[item]{Ambiguïté provoquée par l'absence de \lstinline|const|, l'opérateur défini ne matche pas directement et deux réécritures (via \lstinline|operator==| et via \lstinline|operator!=|) sont éligibles}
\end{frame}

\begin{frame}[fragile]
	\frametitle{\lstinline|operator[]| multidimensionnel}
	\begin{itemize}
		\item Définition de \lstinline|operator[]| avec aucun ou plusieurs arguments
		\item Y compris des arguments \textit{variadic}
	\end{itemize}

	\begin{lstlisting}[language=C++]
T& operator[]();
T& operator[](size_t x, size_t y, size_t z);

foo[3, 2, 1] = 42\end{lstlisting}

\note[item]{Auparavant : utilisation de \lstinline|operator()|, de chaîne d'appel de \lstinline|[]| ou d'une surcharge de \lstinline|operator[]| prenant un agrégat \textit{tuple-like}}

	\begin{block}{Au-delà de C++23}
		\begin{itemize}
			\item Réécritures
			\begin{itemize}
				\item De \lstinline|a[x][y][z]| en \lstinline|a[x, y, z]|
				\item De \lstinline|a(x, y, z)| en \lstinline|a[x][y][z]| (et  \lstinline|a(x)| en \lstinline|a[x]|)
				\item De \lstinline|a[x, y, z]| en \lstinline|a[x][y][z]|
			\end{itemize}
			\item Extension aux tableaux C et aux \lstinline|operator[]| non-membres
		\end{itemize}
	\end{block}

	\vskip 5mm plus 10fill
	\hfill
	\href{https://godbolt.org/#g:!((g:!((g:!((h:codeEditor,i:(filename:'1',fontScale:14,fontUsePx:'0',j:1,lang:c%2B%2B,selection:(endColumn:3,endLineNumber:13,positionColumn:3,positionLineNumber:13,selectionStartColumn:3,selectionStartLineNumber:13,startColumn:3,startLineNumber:13),source:'%23include+%3Ciostream%3E%0A%23include+%3Carray%3E%0A%0Atemplate+%3Ctypename+T,+std::size_t+Z,+std::size_t+Y,+std::size_t+X%3E%0Astruct+Array3d%0A%7B%0A++T%26+operator%5B%5D(std::size_t+z,+std::size_t+y,+std::size_t+x)%0A++%7B%0A++++return+foo%5Bz+*+Y+*+X+%2B+y+*+X+%2B+x%5D%3B%0A++%7D%0A%0A++std::array%3CT,+X+*+Y+*+Z%3E+foo%3B%0A%7D%3B%0A+%0Aint+main()%0A%7B%0A++Array3d%3Cint,+4,+3,+2%3E+bar%3B%0A++bar%5B3,+2,+1%5D+%3D+42%3B%0A++std::cout+%3C%3C+bar%5B3,+2,+1%5D+%3C%3C+!'%5Cn!'%3B%0A%7D%0A'),l:'5',n:'0',o:'C%2B%2B+source+%231',t:'0')),k:50,l:'4',n:'0',o:'',s:0,t:'0'),(g:!((h:executor,i:(argsPanelShown:'1',compilationPanelShown:'0',compiler:g122,compilerName:'',compilerOutShown:'0',execArgs:'',execStdin:'',fontScale:14,fontUsePx:'0',j:1,lang:c%2B%2B,libs:!((name:boost,ver:'175')),options:'-std%3Dc%2B%2B23+-Wall+-Wextra+-pedantic',source:1,stdinPanelShown:'1',tree:'1',wrap:'0'),l:'5',n:'0',o:'Executor+x86-64+gcc+12.2+(C%2B%2B,+Editor+%231)',t:'0')),header:(),k:50,l:'4',n:'0',o:'',s:0,t:'0')),l:'2',n:'0',o:'',t:'0')),version:4}{\includegraphics[height=0.5cm]{input_src/play-button.png}}
\end{frame}

\begin{frame}[fragile]
	\frametitle{Opérateurs \lstinline|static|}
	 \begin{itemize}
	 	\item Possibilité de déclarer \lstinline|static| des \lstinline|operator()|
	 \end{itemize}
	 	
 	\begin{lstlisting}[language=C++]
struct Foo {
  static constexpr bool operator()(int i, int j) {
    return i < j; } };

static_assert(Foo::operator()(1, 2));\end{lstlisting}

	 \begin{itemize}
	 	\item Possibilité de déclarer \lstinline|static| des \lstinline|operator[]|
	 \end{itemize}

 	\begin{lstlisting}[language=C++]
struct Foo {
  static int operator[](int i) {
    return v[i]; }
  static constexpr array<int, 4> v{5, 8, 9, 12}; };

cout << Foo::operator[](2) << "\n";\end{lstlisting}

	\hfill
	\href{https://godbolt.org/#g:!((g:!((g:!((h:codeEditor,i:(filename:'1',fontScale:14,fontUsePx:'0',j:1,lang:c%2B%2B,selection:(endColumn:1,endLineNumber:19,positionColumn:1,positionLineNumber:19,selectionStartColumn:1,selectionStartLineNumber:1,startColumn:1,startLineNumber:1),source:'%23include+%3Ciostream%3E%0A%0Astruct+Foo%0A%7B%0A%23if+1%0A++static+constexpr+bool+operator()(int+i,+int+j)%0A%23else%0A++constexpr+bool+operator()(int+i,+int+j)%0A%23endif%0A++%7B%0A++++return+i+%3C+j%3B%0A++%7D%0A%7D%3B%0A%0Aint+main()%0A%7B%0A++static_assert(Foo::operator()(1,+2))%3B%0A%7D%0A'),l:'5',n:'0',o:'C%2B%2B+source+%231',t:'0')),k:50,l:'4',n:'0',o:'',s:0,t:'0'),(g:!((h:executor,i:(argsPanelShown:'1',compilationPanelShown:'0',compiler:gsnapshot,compilerName:'',compilerOutShown:'0',execArgs:'',execStdin:'',fontScale:14,fontUsePx:'0',j:1,lang:c%2B%2B,libs:!((name:boost,ver:'175')),options:'-std%3Dc%2B%2B23+-Wall+-Wextra+-pedantic',source:1,stdinPanelShown:'1',tree:'1',wrap:'0'),l:'5',n:'0',o:'Executor+x86-64+gcc+(trunk)+(C%2B%2B,+Editor+%231)',t:'0')),header:(),k:50,l:'4',n:'0',o:'',s:0,t:'0')),l:'2',n:'0',o:'',t:'0')),version:4}{\includegraphics[height=0.5cm]{input_src/play-button.png}}
	\href{https://godbolt.org/#g:!((g:!((g:!((h:codeEditor,i:(filename:'1',fontScale:14,fontUsePx:'0',j:1,lang:c%2B%2B,selection:(endColumn:1,endLineNumber:22,positionColumn:1,positionLineNumber:22,selectionStartColumn:1,selectionStartLineNumber:1,startColumn:1,startLineNumber:1),source:'%23include+%3Ciostream%3E%0A%23include+%3Carray%3E%0A%0Astruct+Foo%0A%7B%0A%23if+1%0A++static+int+operator%5B%5D(int+i)%0A%23else%0A++int+operator%5B%5D(int+i)%0A%23endif%0A++%7B%0A++++return+v%5Bi%5D%3B%0A++%7D%0A%0A++static+constexpr+std::array%3Cint,+4%3E+v%7B5,+8,+9,+12%7D%3B%0A%7D%3B%0A%0Aint+main()%0A%7B%0A++std::cout+%3C%3C+Foo::operator%5B%5D(2)+%3C%3C+%22%5Cn%22%3B%0A%7D%0A'),l:'5',n:'0',o:'C%2B%2B+source+%231',t:'0')),k:50,l:'4',n:'0',o:'',s:0,t:'0'),(g:!((h:executor,i:(argsPanelShown:'1',compilationPanelShown:'0',compiler:gsnapshot,compilerName:'',compilerOutShown:'0',execArgs:'',execStdin:'',fontScale:14,fontUsePx:'0',j:1,lang:c%2B%2B,libs:!((name:boost,ver:'175')),options:'-std%3Dc%2B%2B23+-Wall+-Wextra+-pedantic',source:1,stdinPanelShown:'1',tree:'1',wrap:'0'),l:'5',n:'0',o:'Executor+x86-64+gcc+(trunk)+(C%2B%2B,+Editor+%231)',t:'0')),header:(),k:50,l:'4',n:'0',o:'',s:0,t:'0')),l:'2',n:'0',o:'',t:'0')),version:4}{\includegraphics[height=0.5cm]{input_src/play-button.png}}
\end{frame}

\subsection*{Programmation fonctionnelle}
\begin{frame}[fragile]
	\frametitle{Évolutions des lambdas}
	\begin{itemize}
		\item \lstinline|()| optionnelles en l'absence de paramètres dans les lambdas mutables

\note[item]{Optionnelles dans tous les autres cas en C++20 et précédents}

		\item Utilisation du \textit{name lookup} du corps de la lambda pour son retour
	\end{itemize}

	\begin{lstlisting}[language=C++]
// Ne compile pas en C++20 et precedents
auto foo = [j=0]() mutable -> decltype(j) { return j++; };\end{lstlisting}

\note[item]{Auparavant, la résolution de nom du retour ne prenait pas en compte les variables capturées}

	\begin{itemize}
		\item Ajout du support d'attributs pour les lambdas
	\end{itemize}

	\begin{lstlisting}[language=C++]
[] [[ attr ]] () ->int { return 42; };\end{lstlisting}

	\begin{itemize}
		\item Support des attributs \lstinline|[[ nodiscard ]]|, \lstinline|[[ deprecated ]]|, \lstinline|[[ noreturn ]]|
	\end{itemize}
\end{frame}

\begin{frame}[fragile]
	\frametitle{\lstinline|std::invoke_r()|}
	\begin{itemize}
		\item Similaire à \lstinline|std::invoke()|
		\item Retour convertit vers le premier paramètre template
		\item Ou ignoré si le premier paramètre template est \lstinline|void|
	\end{itemize}
\end{frame}

\subsection*{Attributs}
\begin{frame}[fragile]
	\frametitle{Évolutions des attributs}
	\begin{itemize}
		\item Duplication possible d'un attribut dans une liste d'attributs

		\begin{lstlisting}[language=C++]
// Valide en C++23, pas en C++20
[[ nodiscard, nodiscard ]]
int foo();\end{lstlisting}
	\end{itemize}
\end{frame}

\begin{frame}[fragile]
	\frametitle{Nouveaux attributs}
	\begin{itemize}
		\item \lstinline|[[ assume(expression) ]]| permet au compilateur d'optimiser en supposant la véracité de l'expression
	\end{itemize}

	\begin{alertblock}{Contrainte}
		\begin{itemize}
			\item Expression doit être vraie à l'emplacement de \lstinline|assume|

\note[item]{UB dans le cas contraire}
		\end{itemize}
	\end{alertblock}
\end{frame}

\subsection*{Classe}
\begin{frame}[fragile]
	\frametitle{\textit{Layout}}
	\begin{itemize}
		\item Suppression de la possibilité donnée aux compilateurs de réordonner les données d'accessibilité différente
	\end{itemize}
\end{frame}

\begin{frame}[fragile]
	\frametitle{Paramètre \lstinline|this| explicite / \textit{deducing \lstinline|this|}}
	\begin{itemize}
		\item Objectif : limiter les surcharges \lstinline|const| / non \lstinline|const| de fonctions membres
		\item Utilisation d'un premier paramètre, préfixé \lstinline|this|, notant l'instance de classe
	\end{itemize}

	\begin{lstlisting}[language=C++]
struct Foo {
  void bar(this Foo const&); }\end{lstlisting}

	\begin{alertblock}{Restrictions}
		\begin{itemize}
			\item Ne peuvent pas être \lstinline|virtual| ni \lstinline|static|
			\item Ne peuvent pas avoir de \textit{cv-qualifier} ni de \textit{ref-qualifier}
		\end{itemize}
	\end{alertblock}
\end{frame}

\begin{frame}[fragile]
	\frametitle{Paramètre \lstinline|this| explicite / \textit{deducing \lstinline|this|}}
	\begin{itemize}
		\item Utilisation des règles classiques de déduction de types 
	\end{itemize}

	\begin{lstlisting}[language=C++]
struct Foo {
  template <typename Self>
  void bar(this Self&&, int); };
	
void ex(Foo& foo, D& d) {
  foo.bar(1);       // Self=Foo&
  move(foo).bar(2); // Self=Foo}\end{lstlisting}
\end{frame}

\begin{frame}[fragile]
	\frametitle{Paramètre \lstinline|this| explicite / \textit{deducing \lstinline|this|}}
	\begin{itemize}
		\item Permet le passage de \lstinline|this| par valeur
	\end{itemize}

	\begin{lstlisting}[language=C++]
struct Foo {
  void bar()(this Foo, int i); };

Foo{}(4);\end{lstlisting}
\end{frame}

\subsection*{Template}
\begin{frame}[fragile]
	\frametitle{Déduction dans les constructeurs hérités}
	\begin{itemize}
		\item Déduction des paramètres templates d'un constructeur hérité
	\end{itemize}

	\begin{lstlisting}[language=C++]
template <typename T> struct A
{ A(T); };
template <typename T> struct B : public A<T>
{ using  A<T>::A; };

B b(42);	// OK B<int>\end{lstlisting}
\end{frame}

\subsection*{Exceptions}
\begin{frame}[fragile]
	\frametitle{\lstinline|noexcept|}
	\begin{itemize}
		\item Ajout de \lstinline|noexcept| à plusieurs fonctions de la bibliothèque standard
	\end{itemize}
\end{frame}

\subsection*{Traits}
\begin{frame}[fragile]
	\frametitle{Traits}
	\begin{itemize}
		\item \lstinline|std::is_scoped_enum| indique si un type est un \lstinline|enum class|
	\end{itemize}

	\begin{lstlisting}[language=C++]
class A {};
enum E {};
enum struct Es {};
enum class Ec : int {};

is_scoped_enum_v<A>;		// Faux
is_scoped_enum_v<E>;		// Faux
is_scoped_enum_v<Es>;		// Vrai
is_scoped_enum_v<Ec>;		// Vrai
is_scoped_enum_v<int>;	// Faux\end{lstlisting}

	\vskip 10mm plus 10fill
	\hfill
	\href{https://godbolt.org/#g:!((g:!((g:!((h:codeEditor,i:(filename:'1',fontScale:14,fontUsePx:'0',j:1,lang:c%2B%2B,selection:(endColumn:1,endLineNumber:29,positionColumn:1,positionLineNumber:29,selectionStartColumn:1,selectionStartLineNumber:1,startColumn:1,startLineNumber:1),source:'%23include+%3Ciostream%3E%0A%23include+%3Ctype_traits%3E%0A%0Aclass+A%0A%7B%0A%7D%3B%0A%0Aenum+E%0A%7B%0A%7D%3B%0A%0Aenum+struct+Es+%0A%7B%0A%7D%3B%0A%0Aenum+class+Ec+:+int+%0A%7B%0A%7D%3B%0A%0Aint+main()%0A%7B%0A++std::cout+%3C%3C+std::boolalpha%3B%0A++std::cout+%3C%3C+std::is_scoped_enum_v%3CA%3E+%3C%3C+%22%5Cn%22%3B%0A++std::cout+%3C%3C+std::is_scoped_enum_v%3CE%3E+%3C%3C+%22%5Cn%22%3B%0A++std::cout+%3C%3C+std::is_scoped_enum_v%3CEs%3E+%3C%3C+%22%5Cn%22%3B%0A++std::cout+%3C%3C+std::is_scoped_enum_v%3CEc%3E+%3C%3C+%22%5Cn%22%3B%0A++std::cout+%3C%3C+std::is_scoped_enum_v%3Cint%3E+%3C%3C+%22%5Cn%22%3B%0A%7D%0A'),l:'5',n:'0',o:'C%2B%2B+source+%231',t:'0')),k:50,l:'4',n:'0',o:'',s:0,t:'0'),(g:!((h:executor,i:(argsPanelShown:'1',compilationPanelShown:'0',compiler:g122,compilerName:'',compilerOutShown:'0',execArgs:'',execStdin:'',fontScale:14,fontUsePx:'0',j:1,lang:c%2B%2B,libs:!((name:boost,ver:'175')),options:'-std%3Dc%2B%2B23+-Wall+-Wextra+-pedantic',source:1,stdinPanelShown:'1',tree:'1',wrap:'0'),l:'5',n:'0',o:'Executor+x86-64+gcc+12.2+(C%2B%2B,+Editor+%231)',t:'0')),header:(),k:50,l:'4',n:'0',o:'',s:0,t:'0')),l:'2',n:'0',o:'',t:'0')),version:4}{\includegraphics[height=0.5cm]{input_src/play-button.png}}
\end{frame}

\begin{frame}[fragile]
	\frametitle{Traits}
	\begin{itemize}
		\item \lstinline|std::is_implicit_lifetime| indique si un objet à une durée de vie implicite
		\item \lstinline|std::reference_constructs_from_temporary| et \lstinline|std::reference_converts_from_temporary| indiquent si la référence est construite depuis un temporaire
	\end{itemize}
\end{frame}

\subsection*{Conteneurs}
\begin{frame}[fragile]
	\frametitle{Chaînes de caractères}
	\begin{itemize}
		\item \lstinline|contains()| teste la présence d'une sous-chaîne dans une chaîne ou une vue
	\end{itemize}

	\begin{lstlisting}[language=C++]
string foo = "Hello world";
foo.contains("Hello");   // true
foo.contains("monde");   // false

string_view bar = foo;
bar.contains("Hello");   // true
bar.contains("monde");   // false\end{lstlisting}

\note[item]{Comparaison \textit{case-sensitive} comme toutes les fonctions de recherche dans \lstinline|std::string|}

	\begin{itemize}
		\item Interdiction de la construction de \lstinline|std::string| depuis \lstinline|nullptr|

\note[item]{Erreur de compilation, avant c'était un comportement indéfini}

		\item Construction de \lstinline|std::string_view| depuis un range
		\item Ajout de la contrainte trivialement copiable à \lstinline|std::string_view|
	\end{itemize}

	\vskip 10mm plus 10fill
	\hfill
	\href{https://godbolt.org/#g:!((g:!((g:!((h:codeEditor,i:(filename:'1',fontScale:14,fontUsePx:'0',j:1,lang:c%2B%2B,selection:(endColumn:1,endLineNumber:16,positionColumn:1,positionLineNumber:16,selectionStartColumn:1,selectionStartLineNumber:1,startColumn:1,startLineNumber:1),source:'%23include+%3Ciostream%3E%0A%23include+%3Cstring%3E%0A%23include+%3Cstring_view%3E%0A%0Aint+main()%0A%7B%0A++const+std::string+foo+%3D+%22Hello+world%22%3B%0A++std::cout+%3C%3C+std::boolalpha%3B%0A++std::cout+%3C%3C+foo.contains(%22Hello%22)+%3C%3C+%22%5Cn%22%3B%0A++std::cout+%3C%3C+foo.contains(%22monde%22)+%3C%3C+%22%5Cn%22%3B%0A%0A++std::string_view+bar+%3D+foo%3B%0A++std::cout+%3C%3C+bar.contains(%22Hello%22)+%3C%3C+%22%5Cn%22%3B%0A++std::cout+%3C%3C+bar.contains(%22monde%22)+%3C%3C+%22%5Cn%22%3B%0A%7D%0A'),l:'5',n:'0',o:'C%2B%2B+source+%231',t:'0')),k:50,l:'4',n:'0',o:'',s:0,t:'0'),(g:!((h:executor,i:(argsPanelShown:'1',compilationPanelShown:'0',compiler:g122,compilerName:'',compilerOutShown:'0',execArgs:'',execStdin:'',fontScale:14,fontUsePx:'0',j:1,lang:c%2B%2B,libs:!((name:boost,ver:'175')),options:'-std%3Dc%2B%2B23+-Wall+-Wextra+-pedantic',source:1,stdinPanelShown:'1',tree:'1',wrap:'0'),l:'5',n:'0',o:'Executor+x86-64+gcc+12.2+(C%2B%2B,+Editor+%231)',t:'0')),header:(),k:50,l:'4',n:'0',o:'',s:0,t:'0')),l:'2',n:'0',o:'',t:'0')),version:4}{\includegraphics[height=0.5cm]{input_src/play-button.png}}
\end{frame}

\begin{frame}[fragile]
	\frametitle{Chaînes de caractères}
	\begin{itemize}
		\item \lstinline|resize_and_overwrite()| redimensionne et met à jour une chaîne
		\begin{itemize}
			\item Allocation d'un tableau de \lstinline|count + 1| caractères
			\item Copie du contenu de la chaîne dans ce tableau

\note[item]{En se limitant à la taille allouée bien sûr}

			\item Appel à la fonction pour valoriser les caractères et déterminer la taille finale
			\item Mise à jour du contenu de la chaîne avec celui du tableau
		\end{itemize}
	\end{itemize}

	\begin{lstlisting}[language=C++]
string foo = "Hello ", bar = "world!";

foo.resize_and_overwrite(20, 
  [sz = foo.size(), bar] (char* buf, size_t buf_size) {
    auto to_copy = min(buf_size - sz, bar.size());
    memcpy(buf + sz, bar.data(), to_copy);
    return sz + to_copy; });	// Hello world!\end{lstlisting}

	\begin{block}{Motivation}
		Éviter des initialisations, des tests et des copies inutiles
	\end{block}

	\hfill
	\href{https://godbolt.org/#g:!((g:!((g:!((h:codeEditor,i:(filename:'1',fontScale:14,fontUsePx:'0',j:1,lang:c%2B%2B,selection:(endColumn:1,endLineNumber:20,positionColumn:1,positionLineNumber:20,selectionStartColumn:1,selectionStartLineNumber:1,startColumn:1,startLineNumber:1),source:'%23include+%3Ciostream%3E%0A%23include+%3Cstring%3E%0A%23include+%3Calgorithm%3E%0A%23include+%3Ccstring%3E%0A%0Aint+main()%0A%7B%0A++std::string+foo+%3D+%22Hello+%22%3B%0A++std::string+bar+%3D+%22world!!%22%3B%0A%0A++foo.resize_and_overwrite(20,+%0A++++++++++++++++++++++++++%5Bsz+%3D+foo.size(),+bar%5D+(char*+buf,+size_t+buf_size)+%0A++++++++++++++++++++++++++%7B%0A++++++++++++++++++++++++++++auto+to_copy+%3D+std::min(buf_size+-+sz,+bar.size())%3B%0A++++++++++++++++++++++++++++memcpy(buf+%2B+sz,+bar.data(),+to_copy)%3B%0A++++++++++++++++++++++++++++return+sz+%2B+to_copy%3B+%7D)%3B%0A%0A++std::cout+%3C%3C+foo++%3C%3C+%22%5Cn%22%3B%0A%7D%0A'),l:'5',n:'0',o:'C%2B%2B+source+%231',t:'0')),k:50,l:'4',n:'0',o:'',s:0,t:'0'),(g:!((h:executor,i:(argsPanelShown:'1',compilationPanelShown:'0',compiler:g122,compilerName:'',compilerOutShown:'0',execArgs:'',execStdin:'',fontScale:14,fontUsePx:'0',j:1,lang:c%2B%2B,libs:!((name:boost,ver:'175')),options:'-std%3Dc%2B%2B23+-Wall+-Wextra+-pedantic',source:1,stdinPanelShown:'1',tree:'1',wrap:'0'),l:'5',n:'0',o:'Executor+x86-64+gcc+12.2+(C%2B%2B,+Editor+%231)',t:'0')),header:(),k:50,l:'4',n:'0',o:'',s:0,t:'0')),l:'2',n:'0',o:'',t:'0')),version:4}{\includegraphics[height=0.5cm]{input_src/play-button.png}}
\end{frame}

\begin{frame}[fragile]
	\frametitle{\lstinline|std::span|}
	\begin{itemize}
		\item Ajout de la contrainte trivialement copiable
	\end{itemize}
\end{frame}

\begin{frame}[fragile]
	\frametitle{\lstinline|std::pair|}
	\begin{itemize}
		\item Construction depuis un \textit{braced initializers}

\note[item]{Code auparavant valide mais inefficace}
	\end{itemize}

	\begin{lstlisting}[language=C++]
pair<string, vector<string>> foo("hello", {});\end{lstlisting}

	\begin{itemize}
		\item Construction depuis \lstinline|std::tuple|, \lstinline|std::array| ou un \textit{tuple-like}
	\end{itemize}

	\begin{lstlisting}[language=C++]
tuple foo{1, 3.0};
pair bar{foo};\end{lstlisting}

	\begin{lstlisting}[language=C++]
array foo{1, 3};
pair bar{foo};\end{lstlisting}
\end{frame}

\begin{frame}[fragile]
	\frametitle{\lstinline|std::stack| \& \lstinline|std::queue|}
	\begin{itemize}
		\item Création de \lstinline|std::stack| et \lstinline|std::queue| depuis une paire d'itérateurs
	\end{itemize}

	\begin{lstlisting}[language=C++]
vector<int> v{1, 3, 7, 13};
queue q(begin(v), end(v));
stack s(begin(v), end(v));\end{lstlisting}

	\vskip 10mm plus 10fill
	\hfill
	\href{https://godbolt.org/#g:!((g:!((g:!((h:codeEditor,i:(filename:'1',fontScale:14,fontUsePx:'0',j:1,lang:c%2B%2B,selection:(endColumn:1,endLineNumber:16,positionColumn:1,positionLineNumber:16,selectionStartColumn:1,selectionStartLineNumber:1,startColumn:1,startLineNumber:1),source:'%23include+%3Ciostream%3E%0A%23include+%3Cvector%3E%0A%23include+%3Cqueue%3E%0A%23include+%3Cstack%3E%0A%0Aint+main()%0A%7B%0A++std::vector%3Cint%3E+v%7B1,+3,+7,+13%7D%3B%0A%0A++std::queue+q(std::begin(v),+std::end(v))%3B%0A++std::stack+s(std::begin(v),+std::end(v))%3B%0A%0A++std::cout+%3C%3C+q.front()+%3C%3C+%22%5Cn%22%3B%0A++std::cout+%3C%3C+s.top()+%3C%3C+%22%5Cn%22%3B%0A%7D%0A'),l:'5',n:'0',o:'C%2B%2B+source+%231',t:'0')),k:50,l:'4',n:'0',o:'',s:0,t:'0'),(g:!((h:executor,i:(argsPanelShown:'1',compilationPanelShown:'0',compiler:g122,compilerName:'',compilerOutShown:'0',execArgs:'',execStdin:'',fontScale:14,fontUsePx:'0',j:1,lang:c%2B%2B,libs:!((name:boost,ver:'175')),options:'-std%3Dc%2B%2B23+-Wall+-Wextra+-pedantic',source:1,stdinPanelShown:'1',tree:'1',wrap:'0'),l:'5',n:'0',o:'Executor+x86-64+gcc+12.2+(C%2B%2B,+Editor+%231)',t:'0')),header:(),k:50,l:'4',n:'0',o:'',s:0,t:'0')),l:'2',n:'0',o:'',t:'0')),version:4}{\includegraphics[height=0.5cm]{input_src/play-button.png}}
\end{frame}

\begin{frame}[fragile]
	\frametitle{Conteneurs associatifs}
	\begin{itemize}
		\item Surcharge de \lstinline|erase()| et \lstinline|extract()| ne créant pas de clés temporaires
		\item Adaptateurs de conteneurs associatifs
		\begin{itemize}
			\item \lstinline|std::flat_map| et \lstinline|std::flat_multimap|
			\begin{itemize}
				\item Interface similaire à \lstinline|std::map| et \lstinline|std::multimap|
				\item Davantage \textit{cache-friendly}
				\item Clés et valeurs stockées dans deux conteneurs différents
			\end{itemize}
			\item \lstinline|std::flat_set| et \lstinline|std::flat_multiset|
		\end{itemize}
	\end{itemize}
\end{frame}

\begin{frame}[fragile]
	\frametitle{\lstinline|std::mdspan|}
	\begin{itemize}
		\item Vues multidimensionnelles
		\item Possibilité de fournir un \textit{layout} configurable
		\item Trois \textit{memory layouts} standards
		\begin{itemize}
			\item \lstinline|layout_right| : \textit{layout} du C et du C++, lignes puis colonnes
			\item \lstinline|layout_left| : \textit{layout} de Fortran ou Matlab, colonnes puis lignes
			\item \lstinline|layout_stride| 
		\end{itemize}
		\item Accès à un élément via \lstinline|operator[]| multi-paramètres (\lstinline|[x,y,z]|)
	\end{itemize}
\end{frame}

\subsection*{Itérateurs}
\begin{frame}[fragile]
	\frametitle{Évolutions des itérateurs}
	\begin{itemize}
		\item Correction de \lstinline|iterator_category| et \lstinline|counted_iterator|
		\item \lstinline|std::move_iterator<T*>| doit être un \textit{random access iterator}
		\item Modification des exigences sur les itérateurs des algorithmes \og{}non ranges\fg{} pour permettre l'utilisation de vues
	\end{itemize}
\end{frame}

\subsection*{Algorithmes}
\begin{frame}[fragile]
	\frametitle{\lstinline|std::byteswap()|}
	\begin{itemize}
		\item Inverse les octets d'un entier
	\end{itemize}

	\begin{lstlisting}[language=C++]
uint16_t i = 0xCAFE;
byteswap(i);	// 0xFECA

uint32_t j = 0xDEADBEEFu;
byteswap(j);	// 0xEFBEADDE\end{lstlisting}

	\vskip 10mm plus 10fill
	\hfill
	\href{https://godbolt.org/#g:!((g:!((g:!((h:codeEditor,i:(filename:'1',fontScale:14,fontUsePx:'0',j:1,lang:c%2B%2B,selection:(endColumn:1,endLineNumber:13,positionColumn:1,positionLineNumber:13,selectionStartColumn:1,selectionStartLineNumber:1,startColumn:1,startLineNumber:1),source:'%23include+%3Ciostream%3E%0A%23include+%3Ccstdint%3E%0A%23include+%3Cbit%3E%0A%0Aint+main()%0A%7B%0A++uint16_t+i+%3D+0xCAFE%3B%0A++std::cout+%3C%3C+std::hex+%3C%3C+std::byteswap(i)+%3C%3C+%22%5Cn%22%3B%0A%0A++uint32_t+j+%3D+0xDEADBEEFu%3B%0A++std::cout+%3C%3C+std::hex+%3C%3C+std::byteswap(j)+%3C%3C+%22%5Cn%22%3B%0A%7D%0A'),l:'5',n:'0',o:'C%2B%2B+source+%231',t:'0')),k:50,l:'4',n:'0',o:'',s:0,t:'0'),(g:!((h:executor,i:(argsPanelShown:'1',compilationPanelShown:'0',compiler:g122,compilerName:'',compilerOutShown:'0',execArgs:'',execStdin:'',fontScale:14,fontUsePx:'0',j:1,lang:c%2B%2B,libs:!((name:boost,ver:'175')),options:'-std%3Dc%2B%2B23+-Wall+-Wextra+-pedantic',source:1,stdinPanelShown:'1',tree:'1',wrap:'0'),l:'5',n:'0',o:'Executor+x86-64+gcc+12.2+(C%2B%2B,+Editor+%231)',t:'0')),header:(),k:50,l:'4',n:'0',o:'',s:0,t:'0')),l:'2',n:'0',o:'',t:'0')),version:4}{\includegraphics[height=0.5cm]{input_src/play-button.png}}
\end{frame}

\subsection*{Flux}
\begin{frame}[fragile]
	\frametitle{Évolutions des flux}
	\begin{itemize}
		\item \lstinline|spanstream| remplaçant de \lstinline|strstream| utilisant un \lstinline|std::span| comme buffer

\note[item]{\lstinline|strstream| déprécié depuis C++98}

		\item Support du mode exclusif à \lstinline|std::fstream|
	\end{itemize}
\end{frame}

\begin{frame}[fragile]
	\frametitle{Évolutions de \lstinline|std::format|}
	\begin{itemize}
		\item Ajout du concept formatable
		\item Vérification des chaînes de format au \textit{compile-time}
		\item Réduction de la taille du binaire de \lstinline|format_to|
		\item Support des types \lstinline|std::generator|\textit{-like} par \lstinline|std::format|
		\item Formateur de \lstinline|std::chrono| \textit{locale-independent} par défaut

\note[item]{Par cohérence avec tous les autres données}
	\end{itemize}

	\begin{lstlisting}[language=C++]
format("{:%S}", sec(4.2));
// C++20 : 04,200 / C++23 : 04.200

format("{:L%S}", sec(4.2));
// C++20 : exception / C++23 : 04,200\end{lstlisting}
\end{frame}

\begin{frame}[fragile]
	\frametitle{Évolutions de \lstinline|std::format|}
	\begin{itemize}
		\item Ajout du type \lstinline|?| pour afficher les chaînes échappées
		\item Formatage des ranges
		\item Formatage des \lstinline|std::pair| et \lstinline|std::tuple|
		\item Formateur pour \lstinline|std::vector<bool>::reference|
		\item Formatage des conteneurs (si les éléments sont formatables)
		\begin{itemize}
			\item \lstinline|std::map| et équivalent : \lstinline|{k1: v1, k2: v2}|
			\item \lstinline|std::set| et équivalent : \lstinline|{v1, v2}|
			\item \lstinline|std::vector|, \lstinline|std::list|, \ldots : \lstinline|[v1, v2]|
		\end{itemize}
		\item Formatage des \lstinline|std::thread::id|
		\item Formatage des \lstinline|std::stacktrace|
	\end{itemize}
\end{frame}

\begin{frame}[fragile]
	\frametitle{\lstinline|std::print|}
	\begin{itemize}
		\item Écriture dans \lstinline|std::cout| depuis une chaîne de format \lstinline|std::format|
	\end{itemize}

	\begin{lstlisting}[language=C++]
cout << format("Hello, {}!", name);

// Devient
print("Hello, {}!", name);\end{lstlisting}
\end{frame}

\subsection*{Pointeurs intelligents}
\begin{frame}[fragile]
	\frametitle{\lstinline|out_ptr| et \lstinline|inout_ptr|}
	\begin{itemize}
		\item Abstractions entre \textit{smart pointers} et API C modifiant un pointeur

\note[item]{Et pas uniquement le contenu pointé par le \textit{smart pointer}}

		\begin{itemize}
			\item Création d'un pointeur de pointeur temporaire depuis le \textit{smart pointer}
			\item Automatisation des appels à \lstinline|reset()| et \lstinline|release()|
			\item \textit{Exception-safe} : \textit{smart pointer} rétabli au retour de l'API C	

\note[item]{Pas d'exception dans le code C++ avant le rétablissement du \textit{smart pointer}}

			\item Permet le passage comme pointeur C \lstinline|void*| ou \lstinline|void**|
			\item Permet la conversion vers un type de pointeur arbitraire
		\end{itemize}
		\item \lstinline|out_ptr| permet à l'API C de modifier l'adresse contenu dans le smart pointer sans l'utiliser
		\item \lstinline|inout_ptr| permet à l'API C de modifier l'adresse contenu dans le smart pointer mais aussi l'utiliser
	\end{itemize}
\end{frame}

\subsection*{Debug}
\begin{frame}[fragile]
	\frametitle{Bibliothèque de \textit{Stacktrace}}
	\begin{itemize}
		\item Basée sur Boost.stacktrace
		\item \lstinline|current()| récupère la \textit{stacktrace} courante
		\item Manipulation d'une \textit{stacktrace}
		\begin{itemize}
			\item \lstinline|empty()| teste la présente d'entrée
			\item \lstinline|size()| retourne le nombre d'entrée de la \textit{stacktrace}
			\item \lstinline|begin()|, \lstinline|end()|, \ldots{} retourne les itérateurs sur les entrées
			\item \lstinline|operator[]| accède à une entrée donnée
			\item \lstinline|to_string()| retourne la description de la \textit{stacktrace}
			\item \lstinline|operator<<| affiche la \textit{stacktrace}
		\end{itemize}
		\item Manipulation des entrées de la \textit{stacktrace}
		\begin{itemize}
			\item \lstinline|description()| retourne la description de l'entrée
			\item \lstinline|source_file()| retourne le nom de la fonction
			\item \lstinline|source_line()| retourne la ligne
		\end{itemize}
	\end{itemize}
\end{frame}

\begin{frame}[fragile]
	\frametitle{\lstinline|std::unreachable()|}
	\begin{itemize}
		\item \lstinline|std::unreachable()| indique que la localisation n'est pas atteignable
		\item Permet au compilateur d'optimiser en supposant que le code ne sera pas atteint
		\item Comportement indéfini si \lstinline|std::unreachable()| est appelé
	\end{itemize}

	\vskip 10mm plus 10fill
	\hfill
	\href{https://godbolt.org/#g:!((g:!((g:!((h:codeEditor,i:(filename:'1',fontScale:14,fontUsePx:'0',j:1,lang:c%2B%2B,selection:(endColumn:1,endLineNumber:18,positionColumn:1,positionLineNumber:18,selectionStartColumn:1,selectionStartLineNumber:1,startColumn:1,startLineNumber:1),source:'%23include+%3Ciostream%3E%0A%23include+%3Cutility%3E%0A%0Aint+main()%0A%7B%0A++int+i+%3D+1%3B%0A%0A++if(i+%3E+0)%0A++%7B%0A++++std::cout+%3C%3C+%22OK%5Cn%22%3B%0A++%7D%0A++else%0A++%7B%0A++++std::unreachable()%3B%0A++++std::cout+%3C%3C+%22Unreachable%5Cn%22%3B%0A++%7D%0A%7D%0A'),l:'5',n:'0',o:'C%2B%2B+source+%231',t:'0')),k:50,l:'4',n:'0',o:'',s:0,t:'0'),(g:!((h:executor,i:(argsPanelShown:'1',compilationPanelShown:'0',compiler:g122,compilerName:'',compilerOutShown:'0',execArgs:'',execStdin:'',fontScale:14,fontUsePx:'0',j:1,lang:c%2B%2B,libs:!(),options:'-std%3Dc%2B%2B23+-Wall+-Wextra+-pedantic',source:1,stdinPanelShown:'1',wrap:'1'),l:'5',n:'0',o:'Executor+x86-64+gcc+12.2+(C%2B%2B,+Editor+%231)',t:'0')),header:(),k:50,l:'4',n:'0',o:'',s:0,t:'0')),l:'2',n:'0',o:'',t:'0')),version:4}{\includegraphics[height=0.5cm]{input_src/play-button.png}}
\end{frame}

\subsection*{Multi-threading}
\begin{frame}[fragile]
	\frametitle{Atomiques}
	\begin{itemize}
		\item Support des \lstinline|atomics| C
	\end{itemize}
\end{frame}

\subsection*{Durées et temps}
\begin{frame}[fragile]
	\frametitle{\lstinline|time_point::clock|}
	\begin{itemize}
		\item Relâchement des contraintes sur \lstinline|time_point::clock|
		\begin{itemize}
			\item Plus grande flexibilité du type d'horloge
			\item Horloges \textit{stateful}, horloges externes
			\item Représentation d'un \textit{time of day} par un \lstinline|time_point| particulier
		\end{itemize}
	\end{itemize}
\end{frame}

\subsection*{Variant}
\begin{frame}[fragile]
	\frametitle{\lstinline|std::variant|}
	\begin{itemize}
		\item Héritage possible de \lstinline|std::variant|
		\item \lstinline|std::visit()| restreints aux \lstinline|std::variant|

\note[item]{Auparavant, les variants de \lstinline|std::visit| pouvaient être formellement de n'importe quel type}
	\end{itemize}
\end{frame}

\subsection*{Optional}
\begin{frame}[fragile]
	\frametitle{Opérations monadiques de \lstinline|std::optional|}
	\begin{itemize}
		\item \lstinline|transform()| modifie la valeur contenu dans un \lstinline|std::optional|

\note[item]{Peut modifier aussi éventuellement le type}

		\begin{itemize}
			\item Retourne un \lstinline|std::optional| vide s'il n'y a pas de valeur stockée
			\item Retourne le résultat de la fonction sinon
		\end{itemize}
	\end{itemize}

	\begin{lstlisting}[language=C++]
optional<string> foo = "Abcdef", bar;

foo.transform([](auto&& s) { return s.size(); });  // 6
bar.transform([](auto&& s) { return s.size(); });  // Vide\end{lstlisting}
\end{frame}

\begin{frame}[fragile]
	\frametitle{Opérations monadiques de \lstinline|std::optional|}
	\begin{itemize}
		\item \lstinline|and_then()| dérive une fonction pour retourner un \lstinline|std::optional|
	\end{itemize}

	\begin{lstlisting}[language=C++]
auto func = [] (int i) -> optional<int> { return 2 * i; };
optional<int> foo = 42, bar;

foo.and_then(func);		// 84
bar.and_then(func);		// Vide\end{lstlisting}

	\begin{alertblock}{Retour de fonction}
		Le retour de la fonction doit être une spécialisation de \lstinline|std::optional|
	\end{alertblock}
\end{frame}

\begin{frame}[fragile]
	\frametitle{Opérations monadiques de \lstinline|std::optional|}
	\begin{itemize}
		\item \lstinline|or_else()| 
		\begin{itemize}
			\item Retourne le \lstinline|std::optional| s'il a une valeur
			\item Appelle une fonction sinon
		\end{itemize}
	\end{itemize}

	\begin{lstlisting}[language=C++]
auto func = [] -> optional<string> { return "Oups!"; };

optional<string> foo = "Abcdef", bar;

foo.or_else(func);	// Abcdef
bar.or_else(func);	// Oups!\end{lstlisting}

	\begin{alertblock}{Retour de fonction}
		Le retour de la fonction doit être une spécialisation de \lstinline|std::optional|
	\end{alertblock}

	\vskip 10mm plus 10fill
	\hfill
	\href{https://godbolt.org/#g:!((g:!((g:!((h:codeEditor,i:(filename:'1',fontScale:14,fontUsePx:'0',j:1,lang:c%2B%2B,selection:(endColumn:1,endLineNumber:30,positionColumn:1,positionLineNumber:30,selectionStartColumn:1,selectionStartLineNumber:1,startColumn:1,startLineNumber:1),source:'%23include+%3Ciostream%3E%0A%23include+%3Coptional%3E%0A%23include+%3Cstring%3E%0A%0Aint+main()%0A%7B%0A++%7B%0A++++std::optional%3Cstd::string%3E+foo+%3D+%22Abcdef%22%3B%0A++++auto+bar+%3D+foo.transform(%5B%5D(auto%26%26+s)+%7B+return+s.size()%3B+%7D)%3B%0A%0A++++std::cout+%3C%3C+(bar+%3F+*bar+:+0)+%3C%3C+%22%5Cn%22%3B%0A++%7D%0A%0A++%7B%0A++++auto+func+%3D+%5B%5D+(int+i)+-%3E+std::optional%3Cint%3E+%7B+return+2+*+i%3B+%7D%3B%0A++++std::optional%3Cint%3E+foo+%3D+42%3B%0A++++auto+bar+%3D+foo.and_then(func)%3B%0A%0A++++std::cout+%3C%3C++(bar+%3F+*bar+:+0)+%3C%3C+%22%5Cn%22%3B%0A++%7D%0A%0A++%7B%0A++++auto+func+%3D+%5B%5D+-%3E+std::optional%3Cstd::string%3E+%7B+return+%22Oups!!%22%3B+%7D%3B%0A++++std::optional%3Cstd::string%3E+foo+%3D+%22Abcdef%22%3B%0A++++auto+bar+%3D+foo.or_else(func)%3B%0A%0A++++std::cout+%3C%3C+*bar+%3C%3C+%22%5Cn%22%3B%0A++%7D%0A%7D%0A'),l:'5',n:'0',o:'C%2B%2B+source+%231',t:'0')),k:50,l:'4',n:'0',o:'',s:0,t:'0'),(g:!((h:executor,i:(argsPanelShown:'1',compilationPanelShown:'0',compiler:g122,compilerName:'',compilerOutShown:'0',execArgs:'',execStdin:'',fontScale:14,fontUsePx:'0',j:1,lang:c%2B%2B,libs:!(),options:'-std%3Dc%2B%2B23+-Wall+-Wextra+-pedantic',source:1,stdinPanelShown:'1',wrap:'1'),l:'5',n:'0',o:'Executor+x86-64+gcc+12.2+(C%2B%2B,+Editor+%231)',t:'0')),header:(),k:50,l:'4',n:'0',o:'',s:0,t:'0')),l:'2',n:'0',o:'',t:'0')),version:4}{\includegraphics[height=0.5cm]{input_src/play-button.png}}
\end{frame}

\subsection*{Expected}
\begin{frame}[fragile]
	\frametitle{\lstinline|std::expected|}
	\begin{itemize}
		\item Classe template \lstinline|std::expected<T, E>| contenant
		\begin{itemize}
			\item Soit une valeur de type \lstinline|T|, le type de la valeur attendu
			\item Soit une valeur de type \lstinline|E|, le type d'erreur
		\end{itemize}
		\item \lstinline|operator bool()| et \lstinline|has_value()| indique si l'objet contient une valeur
		\item \lstinline|operator->| et \lstinline|operator*| accède à la valeur 
		\item \lstinline|value()| retourne la valeur
		\item \lstinline|error()| retourne l'erreur
	\end{itemize}

	\begin{lstlisting}[language=C++]
expected<int, string> foo(int i) { ... }

expected<int, string> e = foo(5);
if(e)
  cout << e.value();
else
  cout << e.error(); }\end{lstlisting}

	\vskip 5mm plus 10fill
	\hfill
	\href{https://godbolt.org/#g:!((g:!((g:!((h:codeEditor,i:(filename:'1',fontScale:14,fontUsePx:'0',j:1,lang:c%2B%2B,selection:(endColumn:1,endLineNumber:30,positionColumn:1,positionLineNumber:30,selectionStartColumn:1,selectionStartLineNumber:1,startColumn:1,startLineNumber:1),source:'%23include+%3Ciostream%3E%0A%23include+%3Cexpected%3E%0A%23include+%3Cstring%3E%0A%0Astd::expected%3Cint,+std::string%3E+foo(int+i)%0A%7B%0A++if(i+%3E+0)%0A++%7B%0A++++return+i%3B%0A++%7D%0A++else%0A++%7B%0A++++return+std::unexpected%7B%22Nul%22%7D%3B%0A++%7D%0A%7D%0A%0Aint+main()%0A%7B%0A++std::expected%3Cint,+std::string%3E+e+%3D+foo(5)%3B%0A%0A++if(e)%0A++%7B%0A++++std::cout+%3C%3C+%22Val+:+%22+%3C%3C+e.value()+%3C%3C+%22%5Cn%22%3B%0A++%7D%0A++else%0A++%7B%0A++++std::cout+%3C%3C+%22Erreur+:+%22+%3C%3C+e.error()+%3C%3C+%22%5Cn%22%3B%0A++%7D%0A%7D%0A'),l:'5',n:'0',o:'C%2B%2B+source+%231',t:'0')),k:50,l:'4',n:'0',o:'',s:0,t:'0'),(g:!((h:executor,i:(argsPanelShown:'1',compilationPanelShown:'0',compiler:g122,compilerName:'',compilerOutShown:'0',execArgs:'',execStdin:'',fontScale:14,fontUsePx:'0',j:1,lang:c%2B%2B,libs:!(),options:'-std%3Dc%2B%2B23+-Wall+-Wextra+-pedantic',source:1,stdinPanelShown:'1',wrap:'1'),l:'5',n:'0',o:'Executor+x86-64+gcc+12.2+(C%2B%2B,+Editor+%231)',t:'0')),header:(),k:50,l:'4',n:'0',o:'',s:0,t:'0')),l:'2',n:'0',o:'',t:'0')),version:4}{\includegraphics[height=0.5cm]{input_src/play-button.png}}
\end{frame}

\begin{frame}[fragile]
	\frametitle{\lstinline|std::expected|}
	\begin{itemize}
		\item \lstinline|value_or()| retourne 
		\begin{itemize}
			\item La valeur si présente
			\item La valeur reçue en paramètre sinon
		\end{itemize}
		\item \lstinline|transform()| modifie la valeur contenu dans un \lstinline|std::expected|
		\item \lstinline|and_then()| dérive une fonction pour retourner un \lstinline|std::expected|
		\item \lstinline|or_else()|
		\begin{itemize}
			\item Retourne la valeur si elle est présente
			\item Appelle une fonction avec l'erreur sinon
		\end{itemize}
	\end{itemize}

	\begin{alertblock}{Retour de fonction}
		Le retour de \lstinline|and_then()| et \lstinline|or_else()| doit être  \lstinline|std::expected|
	\end{alertblock}

	\vskip 10mm plus 10fill
	\hfill
	\href{https://godbolt.org/#z:OYLghAFBqd5QCxAYwPYBMCmBRdBLAF1QCcAaPECAMzwBtMA7AQwFtMQByARg9KtQYEAysib0QXACx8BBAKoBnTAAUAHpwAMvAFYTStJg1DIApACYAQuYukl9ZATwDKjdAGFUtAK4sGe1wAyeAyYAHI%2BAEaYxCAArACcpAAOqAqETgwe3r56KWmOAkEh4SxRMQm2mPYFDEIETMQEWT5%2BXJXVGXUNBEVhkdF6CvWNzTltQ929JWUSAJS2qF7EyOwc5gDMwcjeWADUJutuTkPEmKwH2CYaAIIbWzuY%2B4eYqkmYDpjoF1e3ZpsM2y8ewObhOwWA3xuPyG6BAIBebw%2BX0OwQIpF2MLhYKMF12/FQEFRuzwsx%2BJgA7FYbrtiVRCU9sLsNKTqftKT8aTTTgQlgxiQcqdcaRSACIc3ZVJTiimCzm7bm8jEEWEgLwhV7vAifGXmMzhWi60UC6XksVQ01km5ElhMYIQFm3dmsnWsmmY%2BEapEg1Ho93YiHrRmPA4ivGoAmxUnrWWc91oLwEJ5uEH7MxmABqYl2IFTZiTKcwADoAG5iLyYAD6JAgAFouLN84dcyZYm4GIboyazbdnU6hbHlXCEZrtSjBL7ByB/bjg%2BtQ/iIJHja6lSrh16x2jV1iCMRwTO8yGJYXd4YFPxiCwIC2rLFQ4TBMSGzL5ZgecQ%2BXmAFT86NskVRoK4o0ngdKYGYDpym6k7xomIIprqma0NmuaNm4EpmCWZaYPaaHNq27Zpsu/Y0pKmDAQOKqwXhurYMQpxLChup4eBhbRMQ1bPocCFEQRHYxkaUK9jGNJMAmqB4mqyBPKGN4tveRIkrsNa4u665asiRzjtuU67vugb7CuUHGWyFivu%2Bn67D%2BeACv%2BxFympnoad62l%2BnpOIGbO87hougEUY5iLOZuE4qtOnmHnOx6GOgFYEAgjDUFJflCSRtIQOBkFQXGixwdxTaIVmObMfBTasaW3g4Vxyb5bxbb8RRZEUdBVE5TRaZ0QxxBMWmLGYexnFtWYLZ1URna9t2JoibsYlEJJAIyfssS3veaAMEMOnTmYABsEr0SQDYqQZAUjppPobe5AaMk1Jlyi%2B2UJoNADyXhJAoYBgKhJXof1XVffhI1DX%2B103XKCofspXC2UaY2pcdG5aVubl7h5QYLQuS4ww5k7qaOCMhTuyOXRhC1FiQFZkYlALJT2qWgelEHXfduXVehBXIUVPV/WV2G4X9hp8aNU2NUZOnUXz7X0ZgjEcxFLMYWxe3ELzeWs7VhGAwJFrmt2HDzLQnCxLwfgcFopCoJwyaWNYGKLMswZ/DwpAEJouvzAA1nEGj6JwkhGy7ZucLwCggF7zsm7rpBwLASAvO84lkBQEANMACjKIYVRCAgqAAO7G47aAsEkdBMDUachLQmc58bpsF0X9AxCnzCvVnaK13Q0ShKwqy8G39fPcIWe5/7sfINcxAp4HpAj3U%2BDG7w/CCCIYjsFIMiCIoKjqOHpC6G0BhGCg1jWPoeARMHkDzKgSQ1MHHA1jCIamFblh/MpADqYjITWb8vKeylvOgQwjhkBB1tisQYypghlwzoPPOvBs7ECYEkTgPA9YGz9tvc2HBsCqDjkQLqqgAAcW0axbUkLsYAyBpIQF3GqN2DYICWysJYdEuBCAkFTOsesvAw5aFmPMeKTAsAxHtKQD2sQvb6w4L7Ug1deBYKDiHJ2Lt%2BHew4GYDBpsFHKPDqo4s0Q0jOEkEAA%3D%3D%3D}{\includegraphics[height=0.5cm]{input_src/play-button.png}}
\end{frame}

\begin{frame}[fragile]
	\frametitle{\lstinline|std::expected|}
	\begin{itemize}
		\item \lstinline|error_or()| retourne
		\begin{itemize}
			\item L'erreur si la valeur n'est pas présente
			\item Le paramètre sinon
		\end{itemize}

\note[item]{\lstinline|error_or()| est le pendant de \lstinline|value_or()| sur les erreurs}

		\item \lstinline|transform_error()|
		\begin{itemize}
			\item Retourne la valeur si elle est présente
			\item Appelle une fonction avec l'erreur sinon
		\end{itemize}
	\end{itemize}

	\vskip 10mm plus 10fill
	\hfill
	\href{https://godbolt.org/#z:OYLghAFBqd5QCxAYwPYBMCmBRdBLAF1QCcAaPECAMzwBtMA7AQwFtMQByARg9KtQYEAysib0QXACx8BBAKoBnTAAUAHpwAMvAFYTStJg1DIApACYAQuYukl9ZATwDKjdAGFUtAK4sGe1wAyeAyYAHI%2BAEaYxCCSAKykAA6oCoRODB7evnrJqY4CQSHhLFEx8baY9vkMQgRMxASZPn5cFVXptfUEhWGR0XoKdQ1N2a2DXT3FpRIAlLaoXsTI7BzmAMzByN5YANQma25Og8SYrPvYJhoAguub25h7B5iqiZgOmOjnlzdmGwxbXl2%2Bzcx2CwC%2B11u/3ujzcyFBRghPz%2BAKBBzEwBIhAQLCRUNRD2BhGiTCIxDx10G6BAIGer3enwOwQIpB2VJpCPBa2wO34qAgzJ2eBm3xMAHYrNcdkKqALHjyNCKpXsJd9pdKTgRFgwhftJVdpeKACJqnaVJSm8X69U7TXatkEakgLwhF5vAgfK3mMzhWje416y1ik2Q4Oi66ClhMYIQJU3VXKr3K6Xs2luhlEwSs1Oc85mx5G3moflxEVra3q1NoLwEWHAs0AOmixBIAH0SBBvQB5LyJBTemZ1g57MxmExxNwMf3loMhm6JhMG9VMGuoXku5AFvZxKxxQsQNAMQYOp25swANjZg9NNtvt6te2Td%2BfKcdHIIxDBOwWBEDT5fd6ph%2BhgKPwxAsBAqZRMAMYKHMJ40q4kHwVBTDIAA1q2wRKA00QQD%2B8E3gBxEkeOu5GhALqpMAIToDsyAIPU9EihKtqYFqxA6kBCyJK8xAHmWFgqoWglESR6p2px341n%2BS7iYaYYzpCT6pnS7qekyWYISAubcvm%2ByFnyEClrJNqqemHqMocWk5h%2BYJ5pgZhbpgDbAUeYEsK2zYdlQG6icpckyhAjlxoBb4oD%2BQ5uCOZgAGpiDsIAxVFZpmA2ABuYheJgsYpf6E5TqOpnquamBidp1a1sC9betgxAnIsiXJdVw6OU29UdteBw1UVBXThWAahnOHBzLQnBxLwfgcFopCoJwbjWNYbILEshK/DwpAEJoI1zOhIBxBo%2BicJIk3bbNnC8AoICHVt00jaQcCwEgzxvKuZAUBA9TAAoyiGJUQgIKgADuU0bWgLCJHQpLpL9IS0ADwNTTN4OQ/QMTfcwfaAyyKN0NEoSsCsvC42jPbCIDINnS9yBXMQ30XaQ1O1PgU28PwggiGI7BSDIgiKCo6h3aQuitAYRgoItlj6HgERXZAcyoIk1RXRwAC0VIGaYljWL8OyqwA6mItB6/rzzAXrrzoIYjjIJdK3LAMjrBLD/0U6DvBA8QTCJJwPCjeNp1C3NHDYKor1kjsqgABznqr56SDswDIJuEAfi66GDhAC3a1LOy4IQJAjmsXAzLwt1aDMcwIKcWAxLGpB7QdR0cCdpBI7wweXddm3bZXzdmIHM2dz3d19%2Bl0SpM4khAA%3D}{\includegraphics[height=0.5cm]{input_src/play-button.png}}
\end{frame}

\begin{frame}[fragile]
	\frametitle{\lstinline|std::unexpected|}
	\begin{itemize}
		\item Classe template \lstinline|std::unexpected<E>| contenant une erreur
		\item \lstinline|error()| retourne l'erreur
		\item Permet de construire un \lstinline|std::expected| indiquant une erreur
	\end{itemize}

	\begin{lstlisting}[language=C++]
expected<double, int> foo = unexpected(3);

// Vrai
if (!foo) { ... }

// Vrai
if (foo == unexpected(3)) { ... }\end{lstlisting}
\end{frame}

\subsection*{Ranges}
\begin{frame}[fragile]
	\frametitle{Évolutions des ranges et vues}
	\begin{itemize}
		\item Ajout de \lstinline|starts_with()| et \lstinline|ends_with()| aux ranges
		\item Ajout de \lstinline|contains()| aux ranges
	\end{itemize}

	\begin{lstlisting}[language=C++]
auto foo = view::iota(0, 50);
auto bar = view::iota(0, 30);
if(ranges::starts_with(foo, bar)) { ... }\end{lstlisting}

	\begin{itemize}
		\item Relâchement des contraintes sur les \textit{range adaptors} pour accepter les types \textit{move-only}
		\item Relâchement des contraintes sur \lstinline|join_view| permettant le support de davantage de ranges

\note[item]{Avant seulement des ranges de \textit{glvalues ranges} (vues ou non) et des ranges de \textit{prvalues views}}
\note[item]{Maintenant aussi des ranges de \textit{prvalues non view ranges}}

		\item Suppression de la contrainte \textit{default constructible} pour les vues
		\item \lstinline|std::ranges::to<>()| construit un conteneur depuis un \lstinline|range|
	\end{itemize}
\end{frame}

\begin{frame}[fragile]
	\frametitle{Nouveaux ranges et \textit{range adaptors}}
	\begin{itemize}
		\item Amélioration de \lstinline|std::views::split()|
		\item \lstinline|std::views::lazy_split()|
		\item \lstinline|std::views::zip()| et \lstinline|std::views::zip_transform()|
	\end{itemize}

\begin{lstlisting}[language=C++]
auto x = vector{1, 2};
auto y = list<string>{"Aa", "Bb", "Cc"};
auto z = array{'A', 'B', 'C', 'D'};

// 1 Aa A
// 2 Bb B
for(tuple<int&, string&, char&> e : views::zip(x, y, z)) {
  cout << get<0>(e) << ' ' 
       << get<1>(e) << ' '
       << get<2>(e) << '\n'; }\end{lstlisting}

	\begin{itemize}
		\item \lstinline|std::views::adjacent()| et \lstinline|std::views::adjacent_transform()|
	\end{itemize}
\end{frame}

\begin{frame}[fragile]
	\frametitle{Nouveaux ranges et \textit{range adaptors}}
	\begin{itemize}
		\item \lstinline|std::ranges::iota()|
		\item \lstinline|std::ranges::shift_left()| et \lstinline|std::ranges::shift_right()|

\note[item]{Équivalents à leur contreparties \lstinline|std|}

		\item \lstinline|std::views::join_with()| transforme un range de ranges en un range
	\end{itemize}

	\begin{lstlisting}[language=C++]
vector<string> vs = {"the", "quick", "brown", "fox"};
vs | join_with('-'); // the-quick-brown-fox\end{lstlisting}

	\begin{itemize}
		\item \lstinline|std::views::chunck()| coupe un range en blocs de N éléments
		\item \lstinline|std::views::slide()| : \lstinline|std::views::adjacent()| avec une taille  \textit{run-time}
		\item \lstinline|std::views::chunck_by()| découpe un range en fonction d'un prédicat
	\end{itemize}

	\begin{lstlisting}[language=C++]
vector v = {1, 2, 2, 3, 0, 4, 5, 2};
// [[1, 2, 2, 3], [0, 4, 5], [2]]
v | chunk_by(less_equal{}));\end{lstlisting}
\end{frame}

\begin{frame}[fragile]
	\frametitle{Nouveaux ranges et \textit{range adaptors}}
	\begin{itemize}
		\item \lstinline|std::views::find_last()|
		\item \lstinline|std::ranges::stride_view()| conserve un élément sur $n$
	\end{itemize}

	\begin{lstlisting}[language=C++]
// 1 4 7 10
iota(1, 13) | stride(3);\end{lstlisting}

	\begin{itemize}
		\item \lstinline|std::ranges::fold()| équivalent range de \lstinline|std::accumulate|
	\end{itemize}

	\begin{lstlisting}[language=C++]
vector<double> v = {0.25, 0.75};
auto r = ranges::fold(v, 1, plus()); // 2\end{lstlisting}

	\begin{itemize}
		\item \lstinline|std::views::cartesian_product| construit une vue sur le produit cartésien de plusieurs conteneurs
	\end{itemize}
\end{frame}

\begin{frame}[fragile]
	\frametitle{Nouveaux ranges et \textit{range adaptors}}
	\begin{itemize}
		\item \lstinline|std::views::as_rvalue()|
		\item \lstinline|std::views::repeat()| répète $n$ fois une valeur
	\end{itemize}

	\begin{lstlisting}[language=C++]
// 17 17 17 17
views::repeat(17, 4);\end{lstlisting}

	\begin{itemize}
		\item Correction de \lstinline|std::ranges::istream_view()|
		\item \lstinline|std::views::enumerate()| range index/valeur depuis un range de valeurs
		\begin{itemize}
			\item Manipulation d'un index dans un \textit{range-based for loop} sans gestion explicite
			\item Construction de \lstinline|std::map| depuis un \lstinline|std::vector| avec l'index pour clé
		\end{itemize}
	\end{itemize}
\end{frame}

\begin{frame}[fragile]
	\frametitle{\lstinline|borrowed_range|}
	\begin{itemize}
		\item Nouveau concept de range : \lstinline|borrowed_range|
		\item Range dont les itérateurs sur celui-ci reste valide après sa destruction
		\item Des ranges inconditionnellement \textit{borrowed} : \lstinline|ref_view|, \lstinline|string_view|, \lstinline|empty_view| et \lstinline|iota_view|
		\item Des ranges conditionnellement \textit{borrowed}, selon la vue sous-jacente : \lstinline|take_view|, \lstinline|drop_view|, \ldots
	\end{itemize}
\end{frame}

\begin{frame}[fragile]
	\frametitle{\textit{Range adaptors} définis par l'utilisateur}
	\begin{itemize}
		\item Classe de base \lstinline|std::ranges::range_adaptor_closure<t>|
		\item Adaptateur de fonction \lstinline|std::bind_back()|
	\end{itemize}

	\begin{lstlisting}[language=C++]
bind_back(f, ys...)(xs...);

// Equivalent a

f(xs..., ys...);\end{lstlisting}
\end{frame}

\subsection*{Modules}
\begin{frame}[fragile]
	\frametitle{Modules}
	\begin{itemize}
		\item Module \lstinline|std| pour tout le namespace \lstinline|std| depuis les en-têtes C++ et C
		\item Module \lstinline|std.compat| : module \lstinline|std| et le namespace global des en-têtes C
	\end{itemize}
\end{frame}

\subsection*{Coroutines}
\begin{frame}[fragile]
	\frametitle{\lstinline|std::generator|}
	\begin{itemize}
		\item Générateur de coroutines synchrones
	\end{itemize}
\end{frame}
\end{document}