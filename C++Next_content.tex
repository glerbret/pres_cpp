\documentclass[C++.tex]{subfiles}
\begin{document}

\section{Et ensuite ?}
\subsection*{Présentation}
\begin{frame}[fragile]
	\frametitle{Présentation}
	\begin{itemize}
		\item C++23 ne marque pas la fin des évolutions du C++
		\item Plusieurs sujets proposés et non pris en compte dans les versions actuelles
		\item Plusieurs TS publiés et non intégrés ou en cours d'étude
	\end{itemize}
\end{frame}

\subsection*{TS}
\begin{frame}[fragile]
	\frametitle{TS - Contracts}
	\begin{itemize}
		\item Retiré du \textit{draft} C++20 et création d'un groupe d'étude en juillet 2019
		\item Support de la programmation par contrat
		\item Remplace la vérification via \lstinline|assert| 
		\item Et la documentation via commentaires \lstinline|@pre|, \lstinline|@post| et \lstinline|@invariant|
		\item Plusieurs propositions initiales concurrentes
		\item \ldots{} mais un compromis à émerger
		\item Les contrats de fonctions membres publiques peuvent utiliser des membres privés ou protégés
		\item Intégration des contrats à la bibliothèque standard
	\end{itemize}
\end{frame}

\begin{frame}[fragile]
	\frametitle{TS - Contracts}
	\begin{itemize}
		\item Diverses propositions de déclaration
		\begin{itemize}
			\item Attributs : \lstinline|[[pre:...]]|, \lstinline|[[post:...]]| et  \lstinline|[[assert:...]]|
			\item Conditions : \lstinline|precond(...)|, \lstinline|postcond(...)| et \lstinline|incond(...)|
			\item Fermetures : \lstinline|pre {...}|, \lstinline|post(...) {...}|
			\item Alternatives : \lstinline|[{pre:...}]|, \lstinline|@(pre: ...)|
		\end{itemize}
		\item Plusieurs comportements
		\begin{itemize}
			\item \lstinline|ignore| : contrat non vérifié
			\item \lstinline|enforce| : appel au \textit{handler} de violation de contrat et terminaison
			\item \lstinline|observe| : appel au \textit{handler} de violation de contrat et poursuite

\note[item]{\lstinline|enforce| : c'est un arrêt systématique, \lstinline|observe| : on laisse le \textit{handler} décider}
		\end{itemize}
	\end{itemize}
\end{frame}



\begin{frame}[fragile]
	\frametitle{TS - Networking TS}
	\begin{itemize}
		\item Publié en avril 2018
		\item Partiellement basé sur Boost.Asio
		\item Gestion de \textit{timer}
		\item Gestion de tampon et de flux orientés tampon
		\item Gestion de \textit{sockets} et de flux \textit{socket}
		\item Gestion IPv4, IPv6, TCP, UDP
		\item Manipulation d'adresses IP
		\item Pas de protocoles de plus haut niveau actuellement
		\item Demande post-TS : gestion de la sécurité (a priori pas possible)
		\item Modèle asynchrone, différent de celui déjà présent en C++
	\end{itemize}
\end{frame}

\begin{frame}[fragile]
	\frametitle{TS - Pattern matching}
	\begin{itemize}
		\item Utilisation du mot clé \lstinline|inspect| (ou \lstinline|switch|) et du \textit{wildcard} \lstinline|__|
		\item Utilisable sur
		\begin{itemize}
			\item Entiers

\note[item]{Proche dans l'esprit de \lstinline|switch/case| avec une syntaxe plus légère et sans \textit{fallthrough}}
		\end{itemize}
	\end{itemize}

	\begin{lstlisting}[language=C++]
inspect(x) {
	0  => { cout << "Aucun"; }
	1  => { cout << "Un"; }
	__ => { cout << "Plusieurs"; } }\end{lstlisting}

	\begin{itemize}
		\item[] 
		\begin{itemize}
			\item Chaînes de caractères
		\end{itemize}
	\end{itemize}

	\begin{lstlisting}[language=C++]
inspect(x) {
	"zero" => { cout << "Aucun"; }
	"un"   => { cout << "Un"; }
	__     => { cout << "Plusieurs"; } }\end{lstlisting}
\end{frame}

\begin{frame}[fragile]
	\frametitle{TS - Pattern matching}
	\begin{itemize}
		\item[] 
		\begin{itemize}
			\item \lstinline|std::tuple|, \lstinline|std::pair|, \lstinline|std::array| et tuple-like
		\end{itemize}
	\end{itemize}

	\begin{lstlisting}[language=C++]
inspect(p) {
	[0, 0] => { cout << "on origin"; }
	[0, y] => { cout << "on y-axis"; }
	[x, 0] => { cout << "on x-axis"; }
	[x, y] => { cout << x << ',' << y; } }\end{lstlisting}

	\begin{itemize}
		\item [] 
		\begin{itemize}
			\item \lstinline|std::variant| et \lstinline|std::any|
		\end{itemize}
	\end{itemize}

	\begin{lstlisting}[language=C++]
inspect(v) {
	<int> i   => { cout << "Entier " << i; }
	<float> f => { cout << "Reel " << f; } }\end{lstlisting}
\end{frame}

\begin{frame}[fragile]
	\frametitle{TS - Pattern matching}
	\begin{itemize}
		\item[] 
		\begin{itemize}
			\item Types polymorphiques
		\end{itemize}
	\end{itemize}

	\begin{lstlisting}[language=C++]
inspect(shape) {
	<Circle> [r]       => { cout << 3.14 * r * r; }
	<Rectangle> [w, h] => { cout << w * h; } }\end{lstlisting}

	\begin{itemize}
		\item Support des gardes
	\end{itemize}

	\begin{lstlisting}[language=C++]
inspect(p) {
	[x, y] if(x > y) => { cout << x << "superieur a" << y; } }\end{lstlisting}

	\begin{alertblock}{Attention}
		Prise en compte de la première correspondance et non de la meilleure
	\end{alertblock}
\end{frame}

\begin{frame}[fragile]
	\frametitle{TS - Les autres}
	\begin{itemize}
		\item \textit{Library fundamentals 2} : évolutions de la bibliothèque standard
		\begin{itemize}
			\item Partiellement intégré en C++17 et C++20
			\item Pointeurs intelligents non possédants \lstinline|observer_ptr|
			\item Nouveaux algorithmes
		\end{itemize}
		\item \textit{Library fundamentals 3}
		\begin{itemize}
			\item \textit{Scope Guard} : enregistrement d'un foncteur appelé à la sortie du scope
			\item \textit{RAII wrapper}
		\end{itemize}
		\item \textit{Parallelism} 2 : publié en juin 2018
		\item \textit{Transactional Memory} : publié
		\item \textit{Numerics}: manipulation des nombres
		\begin{itemize}
			\item Détection et gestion des débordements
			\item Gestion des arrondis
			\item Entiers larges
			\item Rationnel
			\item \textit{Proxy} pour manipuler la représentation interne des entiers
		\end{itemize}
	\end{itemize}
\end{frame}

\begin{frame}[fragile]
	\frametitle{TS - Les autres}
	\begin{itemize}
		\item \textit{2D Graphics} (io2d) : API C++ au dessus de Cairo, différé

\note[item]{Mais le projet io2d en lui-même continue}
\note[item]{Le comité ne voit pas d'intérêt à poursuivre les travaux sur les aspects graphiques pour le moment}

		\item \textit{Reflection}
		\begin{itemize}
			\item Introspection
			\item Méta-programmation et code \textit{compile-time}
			\item Injection
			\item Méta-classes
		\end{itemize}

\note[item]{Dans un premier temps, il s'agit de réflexion statique}

		\item \textit{Concurrency} 1 : publié
		\begin{itemize}
			\item Partiellement intégré à C++20
			\item \lstinline|future.then()|
		\end{itemize}
		\item \textit{Concurrency} 2
		\begin{itemize}
			\item Partiellement accepté pour C++26
		\end{itemize}
		\item \textit{Linear Algebra}
			\begin{itemize}
				\item Basé sur un sous-ensemble de \href{https://www.netlib.org/blas/}{BLAS} (vecteur, matrice, \ldots{})
			\end{itemize}
	\end{itemize}
\end{frame}

\subsection*{Dépréciation et suppression}
\begin{frame}[fragile]
	\frametitle{Dépréciation}
	\begin{itemize}
		\item Dépréciation des modes d'arrondi (\lstinline|fesetround()|)
	\end{itemize}
\end{frame}

\begin{frame}[fragile]
	\frametitle{Suppression}
	\begin{itemize}
		\item Suppression d'éléments précédemment dépréciés
		\begin{itemize}
			\item Conversion arithmétique d'énumération
			\item Comparaison de tableau C (se fait sur les adresses)
			\item \lstinline|volatile|
			\item \lstinline|std::strstream|
			\item \lstinline|std::allocator|
			\item API d'accès atomique à \lstinline|std::shared_ptr|
			\item \lstinline|std::basic_string::reserve()| sans argument
			\item \textit{Unicode Conversion Facets}
			\item \lstinline|wstring_convert|
			\item \textit{Locale Category Facets for Unicode}
		\end{itemize}
	\end{itemize}
\end{frame}

\begin{frame}[fragile]
	\frametitle{\textit{Erroneous behavior}}
	\begin{itemize}
		\item Ajout d'un nouveau type de comportement : \textit{erroneous behavior}
		\item Indique un code incorrect, mais bien défini (dont \textit{implementation-defined} et \textit{unspecified behavior})
		\item Recommandation au compilateur de lever un warning
		\item Compilateur peut rejeter le code
		\item Applicable aux lectures de variables non initialisées
	\end{itemize}
\end{frame}

\subsection*{Syntaxe}
\begin{frame}[fragile]
	\frametitle{Vérification statique}
	\begin{itemize}
		\item \textit{Procedural function interfaces} 
		\begin{itemize}
			\item Annotations de types \textit{claim} / \textit{assertion}
			\item Recouvre des points du contract TS mais plus ambitieux
		\end{itemize}
	\end{itemize}
\end{frame}

\begin{frame}[fragile]
	\frametitle{Mots-clés}
	\begin{itemize}
		\item Conversion de macros en mots-clés

\note[item]{Pour être utilisable avec les modules}

		\begin{itemize}
			\item \lstinline|assert|
			\item \lstinline|offsetof|
		\end{itemize}
	\end{itemize}
\end{frame}

\subsection*{Encodage}
\begin{frame}[fragile]
	\frametitle{Encodage}
	\begin{itemize}
		\item Ajout des algorithmes Unicode

\note[item]{Algorithmes défini par Unicode pour travailler sur les séquences de code points}

		\item Support d'Unicode (UTF-8, UTF-16 et UTF-32) dans la bibliothèque standard
	\end{itemize}
\end{frame}

\subsection*{Types}
\begin{frame}[fragile]
	\frametitle{Types}
	\begin{itemize}
		\item Relâchement des restrictions sur les \lstinline|typedef _t|
		\item Mécanismes \textit{compile-time} vérifiant que deux types ont la même représentation mémoire

\note[item]{P.ex. une structure de deux \lstinline|float| et un tableau de deux \lstinline|float|}

		\item Ajout d'un type \og{}\textit{fixed point decimal}\fg{}
		\item Entiers larges \lstinline|wide_integer<128, unsigned>|
		\item Possibilité de définir des objets \lstinline|constexpr|
		\item \textit{Zero-initialisation} des objets \textit{automatic storage duration}

\note[item]{\textit{automatic storage duration}, c'est ce qui est sur la pile, donc globalement les variables locales}

		\item Entiers non signés pour lesquels l'\textit{overflow} est un UB
		\item Rendre les \lstinline|std::initializer_list| déplaçables
	\end{itemize}
\end{frame}

\begin{frame}[fragile]
	\frametitle{Support des unités physiques}
	\begin{itemize}
		\item Gestion des quantités et dimensions
		\item Supports des unités de base, dérivées, multiples et sous-multiples
		\item Conversions et opérations entre unités
	\end{itemize}

	\begin{lstlisting}[language=C++]
static_assert(10km / 2 == 5km);

static_assert(1h == 3600s);
static_assert(1km + 1m == 1001m);

static_assert(1km / 1s == 1000mps);
static_assert(2kmph * 2h == 4km);
static_assert(2km / 2kmph == 1h);

static_assert(1000 / 1s == 1kHz);

static_assert(10km / 5km == 2);\end{lstlisting}
\end{frame}

\begin{frame}[fragile]
	\frametitle{Représentation mémoire}
	\begin{itemize}
		\item Accès aux octets sous-jacents d'un objet
		\begin{itemize}
			\item Nouvelle catégorie d'objet \textit{contiguous-layout}
			\begin{itemize}
				\item Uniquement des types scalaires et des classes sans fonction ni base virtuelle
				\item N'hérite pas d'objet non \textit{contiguous-layout}
				\item Contiguïté garantie
			\end{itemize}
			\item Représentation sous forme de tableau
			\item Obtention d'un pointeur sur la représentation via \lstinline|reinterpret_cast| vers \lstinline|char|, \lstinline|unsigned char| ou \lstinline|std::byte|
			\item Conversion pointeur sur représentation vers pointeur sur objet via \lstinline|reinterpret_cast|
		\end{itemize}
	\end{itemize}
\end{frame}

\begin{frame}[fragile]
	\frametitle{Relocate}
	\begin{itemize}
		\item Opération \textit{relocate} (déplacement puis destruction)

\note[item]{Opération généralement implémentable par un simple \lstinline|memcpy()|}
\note[item]{L'idée est de permettre certaines optimisations sur les objets correspondants}

		\begin{itemize}
			\item Définition de la notion \textit{relocatable}
			\item Concept \lstinline|Relocatable|

\note[item]{En gros, si j'ai bien compris, il s'agit d'objet sur lesquels \lstinline|swap()| peut être appelé}

			\item Définition de la notion de \textit{trivial relocatability}

\note[item]{Objets respectant la règle du 0 (pas de définition utilisateur de constructeur de copie/déplacement, d'opérateur d'affection par copie/déplacement ni de destructeur)}

			\item Traits \lstinline|std::is_relocatable|, \lstinline|std::is_nothrow_relocatable| et \lstinline|std::is_trivially_relocatable|
			\item Attribut \lstinline|[[ trivially_relocatable ]]|
			\item Algorithmes gérant cette opération : \lstinline|std::relocate_at()|, \lstinline|std::uninitialized_relocate()| et \lstinline|std::uninitialized_relocate_n()|
		\end{itemize}
	\end{itemize}
\end{frame}

\subsection*{Variables}
\begin{frame}[fragile]
	\frametitle{\textit{Shadowing}}
	\begin{itemize}
		\item Levée plusieurs restrictions
		\begin{itemize}
			\item Masquage avec un type \lstinline|void| pour empêcher l'utilisation de la variable masquée
			\item Initialisation de la nouvelle variable avec l'ancienne variable de même nom
			\item Masquage sans création d'une nouvelle portée
			\item Conversion conditionnelle
		\end{itemize}

		\begin{lstlisting}[language=C++]
auto foo = optional<string>{"Foo"};
if(foo as string) { /* foo: string& */ }
else { /* foo: optional<string> */ }\end{lstlisting}
		
		\begin{itemize}
			\item Constification d'un conteneur dans un \textit{range-based for loop}
		\end{itemize}

		\begin{lstlisting}[language=C++]
vector<string> foo{"1", "2", "3"};
cfor(auto &bar : foo) { /* foo est constant */ }\end{lstlisting}
	\end{itemize}
\end{frame}

\begin{frame}[fragile]
	\frametitle{\lstinline|std::ignore|}
	\begin{itemize}
		\item \lstinline|std::ignore| pour ignorer une valeur de retour

\note[item]{En pratique, fonctionne déjà mais n'est pas normalisé}
	\end{itemize}

	\begin{lstlisting}[language=C++]
ignore = printf("Hello\n");\end{lstlisting}
\end{frame}

\subsection*{Contrôle de flux}
\begin{frame}[fragile]
	\frametitle{Contrôle de flux}
	\begin{itemize}
		\item Ajout d'une instruction à \lstinline|break| appelé lors de la sortie de la boucle

\note[item]{Alignement sur des évolutions C en cours}

		\item Ajout d'une boucle \lstinline|do_until|
		\item Version \textit{generator-base} de \textit{for loop}
	\end{itemize}

	\begin{lstlisting}[language=C++]
struct generator { ... }

for(int i: generator())
{ ... }\end{lstlisting}
\end{frame}

\begin{frame}[fragile]
	\frametitle{do expression}
	\begin{itemize}
		\item Ajout des \og{}\textit{do expression}\fg{} : instructions traités comme une expression
	\end{itemize}

	\begin{lstlisting}[language=C++]
int x = do { do return 42; };\end{lstlisting}

	\begin{itemize}
		\item[]
		\begin{itemize}
			\item Améliorations et simplifications des coroutines, du \textit{pattern matching}, \ldots{}
			\item Introduit un nouveau scope mais pas de nouveau \textit{function scope}
			\item \lstinline|do return| pour retourner une valeur dans un \textit{do expression}
			\item Possibilité d'expliciter le type de retour
		\end{itemize}
	\end{itemize}
\end{frame}

\begin{frame}[fragile]
	\frametitle{\lstinline|static_assert|}
	\begin{itemize}
		\item Retarder à l'instanciation l'échec de \lstinline|static_assert(false)| dans des templates
	\end{itemize}

	\begin{lstlisting}[language=C++]
template<typenameT>int my_func(constT&) {
  if constexpr(is_integral_v<T>) {
    return 1; } 
  else if constexpr(is_convertible_v<string, T>) {
    return 2 ;}
  else {
    // C++20 : echec de compilation systematique
    static_assert(false); } }\end{lstlisting}

\note[item]{L'idée est de traiter le \lstinline|static_assert| à l'instanciation et donc de n'échouer que si le paramètre template n'est convertible ni en entier ni en chaîne}

	\begin{itemize}
		\item Utilisation de messages construits
	\end{itemize}

	\begin{lstlisting}[language=C++]
static_assert(sizeof(S) == 1, 
              format("Expected 1, got {}", sizeof(S)));\end{lstlisting}
\end{frame}

\subsection*{Fonctions}
\begin{frame}[fragile]
	\frametitle{Évolutions des fonctions}
	\begin{itemize}
		\item \textit{Unified Call Syntax}
		\begin{itemize}
			\item \lstinline|f(x, ...)| appelle \lstinline|x.f(...)| si \lstinline|f(x, ...)| n'est pas trouvé
			\item Généralisation de \lstinline|std::begin()| et co. dans le langage
		\end{itemize}
		\item Possibilité pour les fonctions \lstinline|va_start| de ne prendre aucun argument
		\item Élision de copie des objets de retour nommés (NRVO) garantie

\note[item]{NRVO : \textit{Named Return Value Optimization}}
\note[item]{L'élision de copie des objets retournés est déjà garantie pour des \textit{prvalues} en C++17}
\note[item]{Optimisation déjà réalisée par certains compilateurs}

		\item Paramètres \lstinline|constexpr| et \og{}\textit{maybe} \lstinline|constexpr|\fg{}
		\item Fonctions \textit{heap-free}
		\item Retour \lstinline|std::move(x)| éligible au NRVO si \lstinline|x| l'est
	\end{itemize}
\end{frame}

\subsection*{Fonctions}
\begin{frame}[fragile]
	\frametitle{Évolutions des fonctions}
	\begin{itemize}
		\item Possibilité de déterminer l'appelant
		\item Arguments nommés
	\end{itemize}
	
	\begin{lstlisting}[language=C++]
void foo(int a, int b, int c, int d, bool e = false);

foo(b: 10, a: 100, c: 640, d: 480);
foo(100, 10, d: 480, e: false, c: 640);\end{lstlisting}
	
\end{frame}

\subsection*{Opérateurs}
\begin{frame}[fragile]
	\frametitle{Opérateurs}
	\begin{itemize}
		\item Surcharge de \lstinline|operator.|
		\begin{itemize}
			\item Si l'opérateur est défini, les opérations sont transférés à son résultat
			\item \ldots{} sauf celles définies comme fonctions membres
			\item Réalisation de \textit{smart reference} (p.ex. \textit{proxy})
		\end{itemize}
		\item Surcharge de \lstinline|operator?:|
		\item \lstinline|operator??| pour tester \lstinline|std::expected|
		\item Évolutions des opérateurs de comparaison et de \lstinline|operator<=>|
		\begin{itemize}
			\item Dépréciation des conversions entre énumération et flottant
			\item Dépréciation des conversions entre énumérations
			\item Dépréciation de la comparaison \og \textit{two-way}\fg{} entre types tableaux
			\item Comparaison \textit{three-way} entre \textit{unscoped} énumération et type entier

\note[item]{\textit{unscoped} énumération, c'est à dire les \lstinline|enum| par opposition aux \lstinline|enum class|}

		\end{itemize}
		\item Interdiction de l'appel de \lstinline|operator=| sur des temporaires

\note[item]{Interdit sur les types \textit{built-in} mais possible sur les autres, avec tous les problèmes que ça peut poser}

		\item Possibilité d'utiliser \lstinline|auto| ou \lstinline|auto&| comme retour d'opérateur \lstinline|=default|
	\end{itemize}
\end{frame}

\begin{frame}[fragile]
	\frametitle{Opérateurs}
	\begin{itemize}
		\item Génération d'opérateurs à la demande via \lstinline|=default|
		\begin{itemize}
			\item \lstinline|operatorX=| à partir de \lstinline|operatorX|
			\item incrément et décrément préfixés à partir de l'addition et de la soustraction
			\item incrément et décrément postfixés à partir des versions préfixés
			\item \lstinline|operator->| et \lstinline|operator->*| à partir de \lstinline|operator*| et \lstinline|operator.|

\note[item]{A priori \lstinline|operator->| et \lstinline|operator->*| seront générés par défaut, mais cette génération sera désactivable avec \lstinline|=delete|}
\note[item]{Dans la continuité des opérateurs de comparaisons}

		\end{itemize}
		\item Ajout de \lstinline|operator[]| à \lstinline|std::initializer_list|
		\item Opérateur pipeline \lstinline!operator|>!
	\end{itemize}

	\begin{lstlisting}[language=C++]
x|>f(y);

// Equivalent a

f(x, y);\end{lstlisting}
\end{frame}

\begin{frame}[fragile]
	\frametitle{Opérateurs}
	\begin{itemize}
		\item \lstinline|operator template()| : extension du support des \textit{non-type template parameters}
		\item Opérateur d'implication \lstinline|operator=>()|
	\end{itemize}

	\begin{lstlisting}[language=C++]
p => q;
	
// Equivalent a
	
!p || q;\end{lstlisting}

	\begin{itemize}
		\item Opérateur \lstinline|nameof|

\note[item]{Similaire à l'opérateur \lstinline|nameof| de C\# ou au trait \lstinline|Debug| de Rust}

	\end{itemize}
\end{frame}

\subsection*{Structured binding}
\begin{frame}[fragile]
	\frametitle{Structured binding}
	\begin{itemize}
		\item Support du \textit{structured binding} sur \lstinline|std::extents|
	\end{itemize}
\end{frame}


\subsection*{Classes}
\begin{frame}[fragile]
	\frametitle{Classes}
	\begin{itemize}
		\item Qualifieurs autorisés sur les constructeurs
		\begin{itemize}
			\item Constructeurs \lstinline|const| pour construire systématiquement des objets constants
			\item Constructeurs non \lstinline|const| peuvent construire des objets constants ou non
		\end{itemize}
		\item Déduction template dans les constructeurs d'agrégats et les alias
		\item \textit{Layout} des classes
		\begin{itemize}
			\item Contrôle du \textit{layout} pour privilégier taille, ordre de déclaration, visibilité, vitesse, ordre alphabétique, lignes de cache ou règles d'une version antérieure du C++ ou d'un autre langage
			\item Contrôle de l'alignement (remplaçant de \lstinline|#pragma pack(N)|)
		\end{itemize}
		\item Constructeurs par déplacement \lstinline|=bitcopies|
		\item Extension de \lstinline|=delete| à d'autres construction (variables template)
		\item \lstinline|=delete| avec un message pour le diagnostic de compilation
		\item Classes de base \lstinline|std::noncopyable| et \lstinline|std::nonmovable|
		\item Mécanisme de conversion tableau de structures vers structure de tableaux

\note[item]{AoS plus lisible et facile à maintenir mais SoA souvent plus efficace}
	\end{itemize}
\end{frame}

\begin{frame}[fragile]
	\frametitle{Classes}
	\begin{itemize}
		\item Possibilité de déclaré \lstinline|friend| un \textit{parameter pack}
	\end{itemize}

	\begin{lstlisting}[language=C++]
template <typename T>
class Foo {
  friend T;	// OK en C++23 et avant
  ...};

template <typename... Ts>
class Bar {
	friend Ts...;	// Invalide en C++23
	...};\end{lstlisting}


\end{frame}

\begin{frame}[fragile]
	\frametitle{Méta-classes}
	\begin{itemize}
		\item Construction de types de classes (dont les classes elles-mêmes) ayant
		\begin{itemize}
			\item Des contraintes
			\item Des comportements par défaut
			\item Des opérations par défaut
		\end{itemize}
		\item \lstinline|class|, \lstinline|struct|, \lstinline|enum class|, \textit{interface}, \textit{value type}
	\end{itemize}
\end{frame}

\subsection*{Énumération}
\begin{frame}[fragile]
	\frametitle{Énumération}
	\begin{itemize}
		\item Ajout d'énumérations \og{}\textit{flag-only}\fg{}
		\item Fonctions membres sur les énumérations

\note[item]{Notamment des opérations de conversion}
	\end{itemize}
\end{frame}

\subsection*{Gestion d'erreur}
\begin{frame}[fragile]
	\frametitle{Gestion d'erreur}
	\begin{itemize}
		\item Exceptions légères (\textit{Zero-overhead deterministic exceptions})
		\item Objet standard pour le retour d'erreur (\lstinline|status_code| et \lstinline|error|)
		\item Récupération des informations de l'exception contenue dans un \lstinline|std::exception_ptr|
	\end{itemize}
\end{frame}

\subsection*{Conteneurs}
\begin{frame}[fragile]
	\frametitle{\textit{Initializer-list}}
	\begin{itemize}
		\item \textit{static storage} possible pour les \textit{braced-initializer-list}

\note[item]{Évite de copier les données depuis le static storage vers le tableau sous-jacent de l'\textit{initializer list} puis vers le conteneur}

	\end{itemize}
\end{frame}

\begin{frame}[fragile]
	\frametitle{Conteneurs}
	\begin{itemize}
		\item Nouveaux conteneurs
			\begin{itemize}
				\item Tableaux multidimensionnels \lstinline|std::mdarray|

\note[item]{\lstinline|std::mdarray| possède les éléments alors que \lstinline|std::mdspan| n'est qu'une vue et ne possède pas}

				\item Queue concurrente
				\item \textit{Bucket array} \lstinline|std::hive| : plusieurs blocs d'éléments liés entre eux avec un indicateur sur l'état de chaque élément (actif / effacé)
				\item Vecteur utilisant un buffer externe
				\item Conteneurs intrusifs : conteneurs non possédants

\note[item]{Intrusif car les mécanismes nécessaires au conteneur (p.ex. les pointeurs de chaînage des listes) sont dans l'objet que gère le conteneur, généralement via héritage}

				\item Conteneurs \lstinline|inplace| avec un buffer de taille fixe
			\end{itemize}
		\end{itemize}
\end{frame}

\begin{frame}[fragile]
	\frametitle{Conteneurs}
	\begin{itemize}
		\item \lstinline|std::submdspan()| retourne une vue sur un sous-ensemble d'un \lstinline|std::mdspan|
		\item \lstinline|span| de taille fixe
		\item Ajout de \lstinline|at()| à \lstinline|std::span|
		\item Relâchement des contraintes sur les tableaux C
		\begin{itemize}
			\item Initialisation des tableaux d'agrégats
			\item Copies de tableaux
			\item Tableau comme type de retour
		\end{itemize}

\note[item]{Copies des tableaux membres en particulier}

		\item Correction de dysfonctionnements de \lstinline|std::flat_map| et \lstinline|std::flat_set| 
	\end{itemize}
\end{frame}

\begin{frame}[fragile]
	\frametitle{Chaînes de caractères}
	\begin{itemize}
		\item Construction de \lstinline|std::string_view| depuis des chaînes implicites
		\item Prise en charge de \lstinline|std::string_view| par \lstinline|std::from_chars|
		\item Support de \lstinline|std::string_view| par \lstinline|std::stringstream|
		\item Concaténation de \lstinline|std::string| et \lstinline|std::string_view|
		\item Modification du constructeur de \lstinline|std::string| depuis un caractère pour interdire les autres numériques (entiers ou flottants)

\note[item]{Il s'agit d'empêcher les initialisations depuis des numériques via la conversion en char}

		\item Voire dépréciation de la construction d'un \lstinline|std::string| depuis un caractère
		\item Interfaçage de \lstinline|std::bitset| avec \lstinline|std::string_view|
	\end{itemize}
\end{frame}

\subsection*{Tuples}
\begin{frame}[fragile]
	\frametitle{Tuples}
	\begin{itemize}
		\item Récupération d'un index depuis un type pour \lstinline|std::variant| et \lstinline|std::tuple|
		\item Utilisation de tableaux C comme \textit{tuple-like}

\note[item]{Déjà le cas pour \lstinline|std::array|, extension aux tableaux C}

		\item Utilisation d'\textit{aggregates} comme \textit{tuple-like}
		\item Amélioration de l'ergonomie d'accès aux champs des \lstinline|std::tuple|
	\end{itemize}

	\begin{lstlisting}[language=C++]
t[0ic]

// Equivalent a

std::get<0>(t)\end{lstlisting}

	\begin{itemize}
		\item \lstinline|std::complex| deviennent des \textit{tuple-like}
	\end{itemize}
\end{frame}

\subsection*{Itérateurs}
\begin{frame}[fragile]
	\frametitle{Itérateurs}
	\begin{itemize}
		\item API \og itérateurs\fg{} de génération des nombres aléatoire
		\item \lstinline|std::iterator_interface| pour la définition de nouveaux itérateurs
	\end{itemize}
\end{frame}

\subsection*{Algorithmes}
\begin{frame}[fragile]
	\frametitle{Algorithmes}
	\begin{itemize}
		\item \lstinline|std::find_last()| recherche depuis la fin d'un conteneur
		\item \lstinline|std::is_uniqued| test l'absence de deux valeurs consécutives identiques
		\item Gestion des UUID
		\item Fonctions statistiques (moyenne, médiane, variance, \ldots{})
		\item Améliorations du générateur aléatoire
		\item Support de SIMD (\textit{Single Instruction on Multiple Data})
	\end{itemize}
\end{frame}

\subsection*{Ranges}
\begin{frame}[fragile]
	\frametitle{Ranges}
	\begin{itemize}
		\item Ajout d'un paramètre \og pas\fg{} à \lstinline|std::iota_view|
		\item Utilisation de \lstinline|std::get_element<>| comme point de configuration
	\end{itemize}

	\begin{lstlisting}[language=C++]
// Tri sur le premier element du tuple
vector<tuple<int, int>> v{{3,1},{2,4},{1,7}};
ranges::sort(v, less{}, get_element<0>);\end{lstlisting}

	\begin{itemize}
		\item Plusieurs nouveaux adaptateurs : \lstinline|adjacent_filter|, \lstinline|adjacent_remove_if|, \lstinline|c_str|, \lstinline|generate|, \ldots
		\item Ajout de \lstinline|ranges::size_hint| permettant aux ranges de réserver de la mémoire
	\end{itemize}
\end{frame}

\subsection*{Traits}
\begin{frame}[fragile]
	\frametitle{Traits}
	\begin{itemize}
		\item Trait \lstinline|std::is_narrowing_convertible|
		\item Traits et fonctions pour garantir des conversions sans perte
		\item Trait indiquant si un type \textit{trivially default constructible} peut être initialisé en mettant tous les octets à 0
		\item Amélioration de l'ergonomie de \lstinline|std::integral_constant<int>|
		\item Trait \lstinline|std::is_virtual_base_of| indiquant si une classe est une classe de base virtuelle d'une autre
	\end{itemize}
\end{frame}

\subsection*{Programmation fonctionnelle}
\begin{frame}[fragile]
	\frametitle{Lambdas}
	\begin{itemize}
		\item Capture mutable partielle par les lambdas

\note[item]{Par défaut, capture \lstinline|const|, possible de la rendre modifiable via \lstinline|mutable|, mais tous les données capturées sont alors mutable}
	\end{itemize}
\end{frame}

\begin{frame}[fragile]
	\frametitle{\lstinline|std::function|}
	\begin{itemize}
		\item \lstinline|std::inplace_function| : pendant de \lstinline|std::function| sans allocation

\note[item]{\lstinline|std::function| peut allouer dynamiquement de la mémoire si la taille du foncteur dépasse un seuil}

		\item \lstinline|std::function_ptr_t| : pointeur générique sur une fonction
	\end{itemize}
\end{frame}

\subsection*{Attributs}
\begin{frame}[fragile]
	\frametitle{Attributs}
	\begin{itemize}
		\item Réservation des attributs sans namespace et avec le namespace \lstinline|std|

\note[item]{Attributs sans namespace, c'est à dire dans le namespace global}

		\item Possibilité d'implémenter des attributs utilisateurs

\note[item]{Remplacement possible de certaines macros}

		\item Nouveaux attributs
		\begin{itemize}
			\item \lstinline|[[invalidate_dereferencing]]| : \lstinline|*ptr| et \lstinline|ptr->| inutilisables après l'appel

\note[item]{P.ex. sur \lstinline|realloc|}

			\item \lstinline|[[invalidate]]| : \lstinline|ptr|, \lstinline|*ptr| et \lstinline|ptr->| inutilisables

\note[item]{P.ex. sur \lstinline|free|}

			\item \lstinline|[[no_copy]]| : types et fonctions ne permettant pas la copie (mais le déplacement et le RVO)
			\item \lstinline|[[rvo]]| : fonctions utilisables uniquement dans un contexte RVO
			\item \lstinline|[[side_effect_free]]| ou \lstinline|[[pure]]|
			\item \lstinline|[[trivially_relocatable]]|
			\item \lstinline|[[discard]]| indique qu'un retour de fonction est volontairement ignoré

\note[item]{Alternative à la conversion en \lstinline|void| ou à l'affectation à \lstinline|std::ignore|}
		\end{itemize}
	\end{itemize}
\end{frame}

\subsection*{Expansion}
\begin{frame}[fragile]
	\frametitle{\textit{Expansion statement}}
	\begin{itemize}
		\item Répétition d'une expression au \textit{compile-time}
	\end{itemize}

	\begin{lstlisting}[language=C++]
auto foo = make_tuple(0, 'a', 3.14);
for... (auto elem : tup)  
  cout << elem << "\n"\end{lstlisting}

	\begin{itemize}
		\item [] 
		\begin{itemize}
			\item Duplication de l'expression pour chaque élément (pas de boucle)
			\item Éléments de type différent
			\item Utilisable sur \lstinline|std::tuple|, \lstinline|std::array|, classes destructurables, \ldots{}
		\end{itemize}
	\end{itemize}
\end{frame}

\subsection*{Parameters pack}
\begin{frame}[fragile]
	\frametitle{Parameters pack}
	\begin{itemize}
		\item Généralisation et simplification des \textit{parameters pack}
		\begin{itemize}
			\item Déclaration possible partout où une variable peut être déclarée
		\end{itemize}
	\end{itemize}

	\begin{lstlisting}[language=C++]
template <typename... Ts>
struct Foo { Ts... elems; };\end{lstlisting}

	\begin{itemize}
		\item [] 
		\begin{itemize}
			\item Indexation des \textit{packs}
		\end{itemize}
	\end{itemize}

	\begin{lstlisting}[language=C++]
struct tuple_element<I, tuple<Ts...>>
  { using type = Ts...[I]; };\end{lstlisting}

	\begin{itemize}
		\item [] 
		\begin{itemize}
			\item \textit{Slicing} de \textit{packs}
		\end{itemize}
	\end{itemize}

	\begin{lstlisting}[language=C++]
auto x = Foo(a1, [:]t1..., [3:]t2..., a2);
bar([1:]t1..., a3, [0]t1);\end{lstlisting}
\end{frame}

\begin{frame}[fragile]
	\frametitle{Parameters pack}
	\begin{itemize}
		\item []
		\begin{itemize}
			\item \textit{Pack} de taille fixe
		\end{itemize}
	\end{itemize}

	\begin{lstlisting}[language=C++]
template<unsigned int N> struct my_vector {
  my_vector(int...[N] v) : values{v...} {}};\end{lstlisting}

	\begin{itemize}
		\item [] 
		\begin{itemize}
			\item \textit{Variadic function} homogène
		\end{itemize}
	\end{itemize}

	\begin{lstlisting}[language=C++]
template <class T>
void f(T... vs);\end{lstlisting}

\note[item]{La fonction prend un nombre quelconque de paramètres, mais tous du type T}

	\begin{itemize}
		\item [] 
		\begin{itemize}
			\item \textit{Unpack} de \lstinline|std::tuple| à la volée
		\end{itemize}
	\end{itemize}

	\begin{lstlisting}[language=C++]
int sum(int x, int y, int z) { return x + y + z; }

tuple<int, int, int> point{1, 2, 3};
int s = sum(point.elems...);\end{lstlisting}
\end{frame}

\begin{frame}[fragile]
	\frametitle{Structured bindings}
	\begin{itemize}
		\item Utilisation de \textit{parameters pack} dans les \textit{structures bindings}
	\end{itemize}

	\begin{lstlisting}[language=C++]
tuple<X, Y, Z> f();

auto [...xs] = f();
auto [x, ...rest] = f();
auto [x,y,z, ...rest] = f();
auto [x, ...rest, z] = f();
auto [...a, ...b] = f();  // ill-formed\end{lstlisting}
\end{frame}

\subsection*{Flux}
\begin{frame}[fragile]
	\frametitle{Flux}
	\begin{itemize}
		\item Redéfinition de \lstinline|std::to_string| en terme de \lstinline|std::format|
	\end{itemize}
\end{frame}

\begin{frame}[fragile]
	\frametitle{\lstinline|std::format|}
	\begin{itemize}
		\item Amélioration du support de \lstinline|std::filesystem::path|
		\begin{itemize}
			\item Présence de caractères d'échappement (p.ex. \lstinline|\n|)
			\item Support de caractère UTF-8
		\end{itemize}
		\item Amélioration du support de \lstinline|std::chrono::time_point|
		\item Ajout de formateurs
		\begin{itemize}
			\item Valeurs atomiques
			\item Générateurs aléatoires et distributions
			\item Pointeurs et \textit{smart pointers}
			\item \lstinline|std::optional|, \lstinline|std::variant|, \lstinline|std::any| et \lstinline|std::expected|
			\item \lstinline|std::mdspan|, \lstinline|std::flat_map| et \lstinline|std::flat_set|
		\end{itemize}
		\item Possibilité de fournir une chaîne de format au \textit{runtime}
	\end{itemize}
\end{frame}

\begin{frame}[fragile]
	\frametitle{\lstinline|std::scan|}
	\begin{itemize}
		\item Pendant du formatage de texte introduit en C++20
		\item Alternative sûre et robuste à \lstinline|sscanf()|
		\item Extensible aux types utilisateurs
		\item Compatible avec les itérateurs et les ranges
	\end{itemize}

	\begin{lstlisting}[language=C++]
string key;
int value;
scan("answer = 42", "{} = {}", key, value);
//    ~~~~~~~~~~~~~  ~~~~~~~~~  ~~~~~~~~~~
//        entree       format    arguments
// key : "answer", value : 42\end{lstlisting}

	\begin{lstlisting}[language=C++]
string key;
chrono::seconds time;
scan("start = 10:30", "{0} = {1:%H:%M}", key, time);\end{lstlisting}
\end{frame}

\subsection*{Templates}
\begin{frame}[fragile]
	\frametitle{Templates}
	\begin{itemize}
		\item Instanciation possible de templates au \textit{runtime} (JIT limité aux templates)

\note[item]{P.ex. pour des matrices dont la taille n'est pas connue à la compilation}

		\item Paramètre template universel

\note[item]{Utile pour la création de méta-fonctions template \textit{high-order}, pour avoir des \lstinline|static_assert(false)| dépendants et pour certains tests sur les types}

		\item Templates dans les classes locales
		\item Rendre les \lstinline|<>| vides optionnels
	\end{itemize}
\end{frame}

\subsection*{Concepts}
\begin{frame}[fragile]
	\frametitle{Concepts}
	\begin{itemize}
		\item Concept pour les algorithmes numériques
	\end{itemize}
\end{frame}

\subsection*{Polymorphismes}
\begin{frame}[fragile]
	\frametitle{Type erasure}
	\begin{itemize}
		\item Programmation polymorphique via \textit{type erasure} : \textit{Proxy}, \textit{Facade}, \textit{Addresser}

\note[item]{Alternative à la POO et programmation fonctionnelle éliminant certaines de leurs limites}
	\end{itemize}
\end{frame}

\subsection*{Gestion mémoire}
\begin{frame}[fragile]
	\frametitle{Références}
	\begin{itemize}
		\item Ajout de références possédantes, \lstinline|T~|, gérant la destruction de l'objet référencé
		\item \textit{Reallocation constructor} transférant la responsabilité de l'objet initial à l'objet créé : \lstinline|T::T(T~)|
	\end{itemize}
\end{frame}

\begin{frame}[fragile]
	\frametitle{Pointeurs}
	\begin{itemize}
		\item Suppression de \lstinline|NULL| et interdiction de \lstinline|0| comme pointeur nul
		\item Surcharge de \lstinline|new| retournant la taille réellement allouée
	\end{itemize}
\end{frame}

\begin{frame}[fragile]
	\frametitle{Pointeurs intelligents}
	\begin{itemize}
		\item \lstinline|std::retain_ptr| pointeur intrusif manipulant le comptage de référence interne
		\item Création de pointeurs intelligents avec une valeur par défaut
		\item Comparaison entre pointeurs intelligents et pointeurs nus
		\item Retour covariant avec \lstinline|std::unique_ptr<T>| (comme \lstinline|T*|)
	\end{itemize}
\end{frame}

\begin{frame}[fragile]
	\frametitle{Contrôle mémoire}
	\begin{itemize}
		\item Ajout de mécanismes de sécurité de l'usage mémoire

\note[item]{Pour éviter l'usage d'un objet après libération ou suppression de ces dépendances}

		\begin{itemize}
			\item \textit{Aliasing}
			\item Suivi des dépendances
			\item Annotation de types
			\item Gestion de \textit{lifetime}
			\item \ldots
		\end{itemize}

		\item Accès à la taille réellement allouée
		\item Spécificateur de stockage des temporaires 
		\begin{itemize}
			\item \lstinline|constinit|
			\item \lstinline|variable_scope|
			\item \lstinline|block_scope| : durée de vie des littéraux C
			\item \lstinline|statement_scope| : durée de vie des temporaires en C++
		\end{itemize}
		\item Seuils d'allocation SOO (\textit{Small Object Optimization})
	\end{itemize}
\end{frame}

\subsection*{Concurrence}
\begin{frame}[fragile]
	\frametitle{Concurrence}
	\begin{itemize}
		\item Version \lstinline|atomic| de minimum et maximum
		\item Obtention de l'adresse de l'objet référencé par \lstinline|std::atomic_ref| via \lstinline|data()|
		\item Invocation concurrente
		\item Configuration du nom et de la taille de la pile des threads
		\item \lstinline|std::volatile_load<T>| et \lstinline|std::volatile_store<T>|
		\item Gestion des processus, de la communication avec ceux-ci et des \textit{pipes}
		\item \lstinline|std::fiber_context| : changement de contexte \textit{stackfull} sans besoin de \textit{scheduler}
		\item Ajout d'un nom aux threads et mutex
		\item Contrôle de la priorité et de la taille de pile des threads
	\end{itemize}
\end{frame}

\begin{frame}[fragile]
	\frametitle{Coroutines}
	\begin{itemize}
		\item Bibliothèques de support des coroutines
		\item \lstinline|std::lazy<T>| : coroutines permettant l'évaluation différée
		\item Unification et amélioration des API asynchrones

\note[item]{Qu'il s'agisse de coroutines, d'algorithmes parallèles, d'\textit{executor} ou autre}
	\end{itemize}
\end{frame}

\subsection*{Durées et temps}
\begin{frame}[fragile]
	\frametitle{Durées et temps}
	\begin{itemize}
		\item Spécialisation de \lstinline|std::hash| pour \lstinline|std::chrono|
	\end{itemize}
\end{frame}

\subsection*{Regex}
\begin{frame}[fragile]
	\frametitle{Regex}
	\begin{itemize}
		\item Ajout de regex \textit{compile-time}
	\end{itemize}
\end{frame}

\subsection*{Interface utilisateur}
\begin{frame}[fragile]
	\frametitle{Interface utilisateur}
	\begin{itemize}
		\item Support des entrées/sorties audio
		\item \lstinline|std::web_view| API fournissant une fenêtre dans laquelle le programme peut injecter des composants web (ou être appelé via \textit{callback})
	\end{itemize}
\end{frame}

\begin{frame}[fragile]
	\frametitle{Debug}
	\begin{itemize}
		\item \lstinline|std::breakpoint()| : point d'arrêt dans le programme
		\item \lstinline|std::breakpoint_if_debugging| : point d'arrêt si l'exécution se fait dans un debugger
		\item \lstinline|std::is_debugger_present()| permet de savoir si l'exécution se fait dans un debugger
	\end{itemize}
\end{frame}

\subsection*{Compilation \& implémentation}
\begin{frame}[fragile]
	\frametitle{Module}
	\begin{itemize}
		\item Communication d'informations aux outils de \textit{build} par les modules
		\item Gestion de la compatibilité ascendante via la configuration d'un \textit{epoch} au niveau d'un module pour activer des évolutions brisant la compatibilité

\note[item]{P.ex. si le standard C++2z supprime les conversions implicites, il serait possible d'appliquer ce changement uniquement sur les modules déclarant \lstinline|epoch 202z|}
	\end{itemize}
\end{frame}

\begin{frame}[fragile]
	\frametitle{Compilation \& implémentation}
	\begin{itemize}
		\item \lstinline|std::embed()| ressources externes disponibles au \textit{runtime} 

\note[item]{P.ex. pour embarquer une image dans l'application sans aller lire au \textit{runtime} le fichier contenant la ressource}
\note[item]{Et sans passer par des conversions externes vers des tableaux de \textit{bytes} injectés dans un fichier d'en-tête}

		\item Implémentations \textit{freestanding} : plus grand sous-ensemble possible de la bibliothèque standard sans \textit{memory overhead} ni support par l'OS

\note[item]{Actuellement le sous-ensemble de la bibliothèque standard proposé est \textit{implementation-defined}}

		\item Remplaçant à \lstinline|#ifdef| ... \lstinline|#endif|
		\item API d'interaction avec le système de build et le compilateur
	\end{itemize}
\end{frame}
\end{document}