\documentclass[C++.tex]{subfiles}
\begin{document}

\section{C++26}
\subsection*{Présentation}
\begin{frame}
	\frametitle{Présentation}
	\begin{itemize}
		\item Début formel des travaux en juin 2023
%		\item Dernier \textit{Working Draft} : \href{https://github.com/cplusplus/draft/releases/download/n4917/n4917.pdf}{n4917}
	\end{itemize}
\end{frame}

\subsection*{Syntaxe}
\begin{frame}[fragile]
	\frametitle{Vérification statique}
	\begin{itemize}
		\item Support de messages construits par \lstinline|static_assert|
	\end{itemize}

	\begin{lstlisting}[language=C++]
static_assert(sizeof(S) == 1,
format("Attendu 1, obtenu {}", sizeof(S)));\end{lstlisting}
\end{frame}

\begin{frame}[fragile]
	\frametitle{\textit{Lexer}}
	\begin{itemize}
		\item Suppression de comportements indéfinis
		\begin{itemize}
			\item \textit{Universal characters} sur plusieurs lignes autorisés
		\end{itemize}
	\end{itemize}
	
	\begin{lstlisting}[language=C++]
int \\
u\
0\
3\
9\
1 = 0;\end{lstlisting}

	\begin{itemize}
		\item [] \begin{itemize}
			\item Construction possible d'\textit{universal characters} par des macros
		\end{itemize}
	\end{itemize}

	\begin{lstlisting}[language=C++]
#define CONCAT(x, y) x ## y
int CONCAT(\, u0393) = 0;\end{lstlisting}

	\begin{itemize}
		\item [] \begin{itemize}
			\item Une chaîne non terminée est une erreur
		\end{itemize}
	\end{itemize}
\end{frame}

\subsection*{Encodage}
\begin{frame}[fragile]
	\frametitle{Encodage}
	\begin{itemize}
		\item Ajout de \lstinline|@|, \lstinline|$| et \lstinline|`| au jeu de caractères de base
	
\note[item]{Ajoutés en C (C23)}
\note[item]{Supportés par tous les encodages communément utilisés}
	\end{itemize}
\end{frame}

\subsection*{Variables}
\begin{frame}[fragile]
	\frametitle{\textit{Placeholders}}
	\begin{itemize}
		\item Joker \lstinline|_| pour des variables inutilisées
	\end{itemize}
	
	\begin{lstlisting}[language=C++]
auto _ = foo();
// Equivalent a [[maybe_unused]] auto _ = foo();\end{lstlisting}

	\begin{lstlisting}[language=C++]
std::lock_guard _(mutex);
auto  [x, y, _] = f();
inspect(foo) { 
	_ => bar; };\end{lstlisting}
\end{frame}

\subsection*{constexpr}
\begin{frame}[fragile]
	\frametitle{\lstinline|constexpr|}
	\begin{itemize}
		\item Davantage de \lstinline|constexpr| dans la bibliothèque standard
		\item Conversion depuis \lstinline|void*| dans des contextes \lstinline|constexpr|
		\begin{itemize}
			\item \lstinline|std::format()| au compile-time
			\item \lstinline|std::function_ref|, \lstinline|std::function| et \lstinline|std::any| \lstinline|constexpr|
		\end{itemize}
	\end{itemize}
\end{frame}

\subsection*{Gestion mémoire}
\begin{frame}[fragile]
	\frametitle{\textit{lifetime}}
	\begin{itemize}
		\item \lstinline|std::is_within_lifetime()| indique si l'objet pointé est vivant
		\item \ldots{} en particulier si un membre d'une union est active
	\end{itemize}
\end{frame}

\subsection*{Multithreading}
\begin{frame}[fragile]
	\frametitle{Gestion mémoire}
	\begin{itemize}
		\item \textit{hazard pointers} : unique écrivain, multiples lecteurs
		\item Structure de donnée \textit{read-copy update} : planification d'actions (p.ex. suppression) à réaliser plus tard
	\end{itemize}
\end{frame}

\subsection*{Programmation fonctionnelle}
\begin{frame}[fragile]
	\frametitle{Type appelable}
	\begin{itemize}
		\item Ajout de \lstinline|std::copiable_function| pour les fonctions copiables

\note[item]{Pendant de \lstinline|std::move_only_function|}

		\item Ajout de \lstinline|std::function_ref|
		\begin{itemize}
			\item Type référence pour le passage d'appelable à une fonction
			\item Plus générique et moins gourmand que \lstinline|std::function| et équivalents
	
\note[item]{Les fonctions n'ont pas besoin d'être copiables ni déplaçables}
		\end{itemize}
	\end{itemize}
\end{frame}

\subsection*{Flux}
\begin{frame}[fragile]
	\frametitle{\lstinline|std::format|}
	\begin{itemize}
		\item Davantage de vérifications \textit{compile-time} du type des arguments
		\begin{itemize}
			\item Déjà le cas de la majorité des erreurs
			\item \ldots{} mais pas de toutes
		\end{itemize}
	\end{itemize}

	\begin{lstlisting}[language=C++]
// Erreur run-time
format("{:>{}}", "hello", "10");\end{lstlisting}

	\begin{itemize}
		\item Formatage des pointeurs
	\end{itemize}
\end{frame}

\subsection*{Système de fichiers}
\begin{frame}[fragile]
	\frametitle{Accès bas-niveaux aux IO}
	\begin{itemize}
		\item Alias sur le descripteur de fichier de la plateforme : \lstinline|native_handle_type|
		
\note[item]{\lstinline|int| sous POSIX, \lstinline|HANDLE| (\lstinline|void*|) sous Windows}
		 
		\item \lstinline|native_handle()| retourne ce descripteur
	\end{itemize}
\end{frame}

\end{document}