\documentclass[C++.tex]{subfiles}
\begin{document}

\section{C++26}
\subsection*{Présentation}
\begin{frame}
	\frametitle{Présentation}
	\begin{itemize}
		\item Début formel des travaux en juin 2023
%		\item Dernier \textit{Working Draft} : \href{https://github.com/cplusplus/draft/releases/download/n4917/n4917.pdf}{n4917\linklogo}
	\end{itemize}
\end{frame}

\subsection*{Dépréciation et suppression}
\begin{frame}[fragile]
	\frametitle{Suppression}
	\begin{itemize}
		\item Suppression d'éléments précédemment dépréciés
		\begin{itemize}
			\item Conversion arithmétique d'énumération
			\item \mintinline{cpp}|strstream|
			\item API d'accès atomique à \mintinline{cpp}|std::shared_ptr|
			\item \mintinline{cpp}|wstring_convert|
		\end{itemize}
	\end{itemize}

	\addproposal{P2864}{https://wg21.link/P2864R2}
	\addproposal{P2867}{https://wg21.link/P2867R1}
	\addproposal{P2869}{https://wg21.link/P2869R3}
	\addproposal{P2872}{https://wg21.link/P2872R2}
\end{frame}

\subsection*{Compatibilité C}
\begin{frame}[fragile]
	\frametitle{En-têtes C23}
	\begin{itemize}
		\item Support des en-têtes C23 \mintinline{cpp}|<stdbit.h>| et \mintinline{cpp}|<stdckdint.h>|
	\end{itemize}

	\addproposal{p3370}{https://wg21.link/p3370r1}
\end{frame}

\subsection*{Comportement}
\begin{frame}[fragile]
	\frametitle{Erroneous Behavior}
	\begin{itemize}
		\item Ajout d'un nouveau type de comportement : \textit{Erroneous Behavior}
		\item Indique un code incorrect, mais bien défini (dont \textit{Implementation Defined} et \textit{Unspecified Behavior})
		\item Recommandation au compilateur de lever un warning
		\item Compilateur peut rejeter le code
		\item Applicable aux lectures de variables non initialisées
		\begin{itemize}
			\item Doit retourner une valeur \og{}erronée\fg{}
			\item \ldots{}et non la valeur d'une autre variable récemment libérée
		\end{itemize}
	\end{itemize}

	\addproposal{P2795}{https://wg21.link/P2795R5}
\end{frame}

\begin{frame}[fragile]
	\frametitle{Boucles infinies}
	\begin{itemize}
		\item Les boucles infinies triviales ne sont plus des \textit{Undefined Behavior}
		\item Alignement avec le comportement du C
	\end{itemize}

	\begin{minted}{cpp}
		// Comportement indefini en C++23
		while (true)
		{}
	\end{minted}

	\addproposal{P2809}{https://wg21.link/P2809R3}
\end{frame}

\subsection*{Syntaxe}
\begin{frame}[fragile]
	\frametitle{Vérification statique}
	\begin{itemize}
		\item Support de messages construits par \mintinline{cpp}|static_assert|
	\end{itemize}

	\begin{minted}{cpp}
		static_assert(sizeof(Foo) == 1,
		              format("Attendu 1, obtenu {}", sizeof(Foo)));
	\end{minted}

	\begin{alertblock}{\textit{Compile-time}}
		\begin{itemize}
			\item Uniquement des valeurs connues au \textit{compile-time}
		\end{itemize}
	\end{alertblock}

	\begin{alertblock}{Dépendance}
		\begin{itemize}
			\item Nécessite que \mintinline{cpp}|std::format| devienne \mintinline{cpp}|constexpr|
		\end{itemize}
	\end{alertblock}

	\addproposal{P2741}{https://wg21.link/P2741R3}
\end{frame}

\begin{frame}[fragile]
	\frametitle{Lexer}

	\begin{itemize}
		\item Suppression de comportements indéfinis
		\begin{itemize}
			\item \textit{Universal characters} sur plusieurs lignes autorisés
		\end{itemize}
	\end{itemize}

	\begin{minted}[escapeinside=||]{cpp}
		int |\textbackslash|\
		u\
		0\
		3\
		9\
		1 = 0;
	\end{minted}

	\begin{itemize}
		\item [] \begin{itemize}
			\item Construction possible d'\textit{universal characters} par des macros
		\end{itemize}
	\end{itemize}

	\begin{minted}[escapeinside=||]{cpp}
		#define CONCAT(x, y) x ## y
		int CONCAT(|\textbackslash|, u0393) = 0;
	\end{minted}

	\begin{itemize}
		\item [] \begin{itemize}
			\item Une chaîne non terminée est une erreur
		\end{itemize}
	\end{itemize}

	\addproposal{P2621}{https://wg21.link/P2621R2}
\end{frame}

\subsection*{Encodage}
\begin{frame}[fragile]
	\frametitle{Encodage}
	\begin{itemize}
		\item Ajout de \mintinline[escapeinside=||]{cpp}{|@|}, \mintinline{cpp}|$| et \mintinline[escapeinside=||]{cpp}{|`|} au jeu de caractères de base

\note[item]{Ajoutés en C (C23)}
\note[item]{Supportés par tous les encodages communément utilisés}

		\item Caractères non-encodables sont mal formés

		\item Notion de \textit{literal encoding} et \textit{environnement encoding} et API d'interrogation
	\end{itemize}

	\addproposal{P2558}{https://wg21.link/P2558R2}
	\addproposal{P1854}{https://wg21.link/P1854R4}
\end{frame}

\subsection*{Types}
\begin{frame}[fragile]
	\frametitle{Saturation arithmétic}
	\begin{itemize}
		\item Fonctions \mintinline{cpp}|std::add_sat()|, \mintinline{cpp}|std::sub_sat()|, \mintinline{cpp}|std::mul_sat()|, \mintinline{cpp}|std::div_sat()| et \mintinline{cpp}|std::saturate_cast()|
		\item Les calculs dont le résultat est hors borne retournent les plus grandes ou plus petites valeurs représentables
	\end{itemize}

	\begin{minted}{cpp}
		add_sat(3, 4);                  // 7
		sub_sat(INT_MIN, 1);            // INT_MIN
		add_sat<unsigned char>(255, 4); // 255
	\end{minted}

	\begin{codesample}
		\sample{https://godbolt.org/#g:!((g:!((g:!((h:codeEditor,i:(filename:'1',fontScale:14,fontUsePx:'0',j:1,lang:c%2B%2B,selection:(endColumn:1,endLineNumber:1,positionColumn:1,positionLineNumber:1,selectionStartColumn:1,selectionStartLineNumber:1,startColumn:1,startLineNumber:1),source:'%23include+%3Ciostream%3E%0A%23include+%3Cnumeric%3E%0A%23include+%3Cclimits%3E%0A%0Aint+main()%0A%7B%0A++std::cout+%3C%3C+std::add_sat(3,+4)+%3C%3C+%22%5Cn%22%3B%0A++std::cout+%3C%3C+INT_MIN+%3C%3C+%22+%22+%3C%3Cstd::sub_sat(INT_MIN,+1)+%3C%3C+%22%5Cn%22%3B%0A++std::cout+%3C%3C+static_cast%3Cint%3E(std::add_sat%3Cunsigned+char%3E(255,+4))+%3C%3C+%22%5Cn%22%3B%0A%7D%0A'),l:'5',n:'0',o:'C%2B%2B+source+%231',t:'0')),k:50,l:'4',n:'0',o:'',s:0,t:'0'),(g:!((h:executor,i:(argsPanelShown:'1',compilationPanelShown:'0',compiler:gsnapshot,compilerName:'',compilerOutShown:'0',execArgs:'',execStdin:'',fontScale:14,fontUsePx:'0',j:1,lang:c%2B%2B,libs:!(),options:'-std%3Dc%2B%2B26+-Wall+-Wextra+-pedantic',overrides:!(),runtimeTools:!(),source:1,stdinPanelShown:'1',wrap:'1'),l:'5',n:'0',o:'Executor+x86-64+gcc+(trunk)+(C%2B%2B,+Editor+%231)',t:'0')),header:(),k:50,l:'4',n:'0',o:'',s:0,t:'0')),l:'2',n:'0',o:'',t:'0')),version:4}
	\end{codesample}

	\addproposal{P0543}{https://wg21.link/P0543R3}
\end{frame}

\subsection*{Variables}
\begin{frame}[fragile]
	\frametitle{Placeholders}
	\begin{itemize}
		\item Joker \mintinline{cpp}|_| pour des variables inutilisées
	\end{itemize}

	\begin{minted}{cpp}
		auto _ = foo();  // Equivalent a [[maybe_unused]] auto _ = foo();
	\end{minted}

	\begin{minted}{cpp}
		std::lock_guard _(mutex);
	\end{minted}

	\begin{minted}{cpp}
		auto  [x, y, _] = f();
	\end{minted}

	\begin{itemize}
		\item \mintinline{cpp}|std::ignore| pour ignorer un retour de fonction
	\end{itemize}

	\begin{minted}{cpp}
		std::ignore = f();
	\end{minted}

	\begin{codesample}
		\sample{https://godbolt.org/#g:!((g:!((g:!((h:codeEditor,i:(filename:'1',fontScale:14,fontUsePx:'0',j:1,lang:c%2B%2B,selection:(endColumn:6,endLineNumber:12,positionColumn:6,positionLineNumber:12,selectionStartColumn:6,selectionStartLineNumber:12,startColumn:6,startLineNumber:12),source:'%23include+%3Ciostream%3E%0A%23include+%3Cnumeric%3E%0A%23include+%3Cclimits%3E%0A%0A%5B%5Bnodiscard%5D%5Dint+foo()%0A%7B%0A++return+42%3B%0A%7D%0A%0Aint+main()%0A%7B%0A%23if+1%0A++foo()%3B%0A%23else%0A++std::ignore+%3D+foo()%3B%0A%23endif%0A%7D%0A'),l:'5',n:'1',o:'C%2B%2B+source+%231',t:'0')),k:50,l:'4',n:'0',o:'',s:0,t:'0'),(g:!((h:executor,i:(argsPanelShown:'1',compilationPanelShown:'0',compiler:gsnapshot,compilerName:'',compilerOutShown:'0',execArgs:'',execStdin:'',fontScale:14,fontUsePx:'0',j:1,lang:c%2B%2B,libs:!(),options:'-std%3Dc%2B%2B26+-Wall+-Wextra+-pedantic',overrides:!(),runtimeTools:!(),source:1,stdinPanelShown:'1',wrap:'1'),l:'5',n:'0',o:'Executor+x86-64+gcc+(trunk)+(C%2B%2B,+Editor+%231)',t:'0')),header:(),k:50,l:'4',n:'0',o:'',s:0,t:'0')),l:'2',n:'0',o:'',t:'0')),version:4}

		\sample{https://godbolt.org/#g:!((g:!((g:!((h:codeEditor,i:(filename:'1',fontScale:14,fontUsePx:'0',j:1,lang:c%2B%2B,selection:(endColumn:1,endLineNumber:1,positionColumn:1,positionLineNumber:1,selectionStartColumn:1,selectionStartLineNumber:1,startColumn:1,startLineNumber:1),source:'%23include+%3Ciostream%3E%0A%23include+%3Cnumeric%3E%0A%23include+%3Cclimits%3E%0A%0Aint+foo()%0A%7B%0A++return+42%3B%0A%7D%0A%0Aint+main()%0A%7B%0A%23if+1%0A++auto+bar+%3D+foo()%3B%0A%23else%0A++auto+_+%3D+foo()%3B%0A%23endif%0A%7D%0A'),l:'5',n:'0',o:'C%2B%2B+source+%231',t:'0')),k:50,l:'4',n:'0',o:'',s:0,t:'0'),(g:!((h:executor,i:(argsPanelShown:'1',compilationPanelShown:'0',compiler:gsnapshot,compilerName:'',compilerOutShown:'0',execArgs:'',execStdin:'',fontScale:14,fontUsePx:'0',j:1,lang:c%2B%2B,libs:!(),options:'-std%3Dc%2B%2B26+-Wall+-Wextra+-pedantic',overrides:!(),runtimeTools:!(),source:1,stdinPanelShown:'1',wrap:'1'),l:'5',n:'0',o:'Executor+x86-64+gcc+(trunk)+(C%2B%2B,+Editor+%231)',t:'0')),header:(),k:50,l:'4',n:'0',o:'',s:0,t:'0')),l:'2',n:'0',o:'',t:'0')),version:4}
	\end{codesample}

	\addproposal{P2169}{https://wg21.link/P2169R4}
	\addproposal{p2968}{https://wg21.link/p2968r2}
\end{frame}

\subsection*{Structured binding}
\begin{frame}[fragile]
	\frametitle{Structured binding}
	\begin{itemize}
		\item Utilisable comme condition dans les \mintinline{cpp}|if|, \mintinline{cpp}|while|, \mintinline{cpp}|for| et \mintinline{cpp}|switch|
	\end{itemize}

	\addproposal{p0963}{https://wg21.link/p0963r3}
\end{frame}

\subsection*{Classes}
\begin{frame}[fragile]
	\frametitle{\mintinline[style=white]{cpp}|delete|}
	\begin{itemize}
		\item Ajout d'un message à \mintinline{cpp}|=delete|
		\item Permet d'indiquer la raison de la suppression
		\item Et d'obtenir de meilleures erreurs de compilation
	\end{itemize}

	\addproposal{P2573}{https://wg21.link/P2573R2}
\end{frame}

\begin{frame}[fragile]
	\frametitle{Variadic friends}
	\begin{itemize}
		\item Possibilité de déclaré \mintinline{cpp}|friend| un \textit{parameter pack}
	\end{itemize}

	\begin{minted}{cpp}
		template <typename... Ts>
		class Bar {
		  friend Ts...;  // Invalide en C++23
		  ...
		};
	\end{minted}

	\addproposal{P2893}{https://wg21.link/P2893R3}
\end{frame}

\subsection*{Conteneurs}
\begin{frame}[fragile]
	\frametitle{Conteneurs}
	\begin{itemize}
		\item Nouveaux conteneurs
		\begin{itemize}
			\item Vecteur de capacité fixée en \textit{compile-time} \mintinline{cpp}|std::inplace_vector|

\note[item]{Contrairement à \mintinline{cpp}|std::array| la taille n'est pas fixée, seule la capacité l'est, et donc utilisable pour des éléments \og sans valeur par défaut\fg{}}
		\end{itemize}
		\item Possibilité d'utiliser \mintinline{cpp}|std::weak_ptr| en tant que clé de conteneur associatif
		\item \mintinline{cpp}|std::submdspan()| retourne une vue sur un sous-ensemble d'un \mintinline{cpp}|std::mdspan|
		\item Nouveaux layouts pour \mintinline{cpp}|std::mdspan| : \mintinline{cpp}|layout_left_padded| et \mintinline{cpp}|layout_right_padded|
		\item Ajout de \mintinline{cpp}|at()| à \mintinline{cpp}|std::span|
	\end{itemize}

	\addproposal{p0843}{https://wg21.link/p0843r11}
	\addproposal{P1901}{https://wg21.link/P1901R2}
	\addproposal{P2630}{https://wg21.link/P2630R4}
	\addproposal{P2642}{https://wg21.link/P2642R5}
	\addproposal{P2821}{https://wg21.link/p2821r4}
\end{frame}

\begin{frame}[fragile]
	\frametitle{Chaînes de caractères}
	\begin{itemize}
		\item Support de \mintinline{cpp}|std::string_view| par \mintinline{cpp}|std::stringstream|
		\item Interfaçage de \mintinline{cpp}|std::bitset| avec \mintinline{cpp}|std::string_view|
		\item Concaténation de \mintinline{cpp}|std::string| et \mintinline{cpp}|std::string_view|
	\end{itemize}

	\addproposal{P2495}{https://wg21.link/P2495R3}
	\addproposal{P2697}{https://wg21.link/P2697R1}
	\addproposal{P2591}{https://wg21.link/P2591R4}
\end{frame}

\begin{frame}[fragile]
	\frametitle{Initializer-list}
	\begin{itemize}
		\item \textit{static storage} possible pour les \textit{braced-initializer-list}

\note[item]{Évite de copier les données depuis le static storage vers le tableau sous-jacent de l'\textit{initializer list} puis vers le conteneur}

		\item \mintinline{cpp}|std::span| sur les \textit{braced-initializer-list}
	\end{itemize}

	\addproposal{P2752}{https://wg21.link/P2752R2}
	\addproposal{P2447}{https://wg21.link/P2447R6}
\end{frame}

\begin{frame}[fragile]
	\frametitle{\mintinline{cpp}|reference_wrapper|}
	\begin{itemize}
		\item Comparaison de \mintinline{cpp}|std::reference_wrapper|
	\end{itemize}

	\addproposal{P2944}{https://wg21.link/P2944R2}
\end{frame}

\subsection*{Tuples}
\begin{frame}[fragile]
	\frametitle{Tuples}
	\begin{itemize}
		\item \mintinline{cpp}|std::complex| deviennent des \textit{tuple-like}
	\end{itemize}

	\addproposal{p2819}{https://wg21.link/p2819r2}
\end{frame}

\subsection*{Algorithme}
\begin{frame}[fragile]
	\frametitle{Algèbre linéaire}
	\begin{itemize}
		\item Basé sur un sous-ensemble de \href{https://www.netlib.org/blas/}{BLAS\linklogo}
		\item Multiples opérations
		\begin{itemize}
			\item Somme de vecteurs
			\item Multiplication de vecteurs ou de matrices par un scalaire
			\item Produit de vecteurs et de matrices
			\item Triangularisation de matrices
			\item Rotation de plans
		\end{itemize}
		\item Plusieurs formats de stockage des matrices
	\end{itemize}

	\begin{minted}{cpp}
		vector<double> x_vec{1., 2., 3., 4., 5.};
		mdspan x(x_vec.data(), 5);

		linalg::scale(2.0, x); // x = 2.0 * x
	\end{minted}

	\addproposal{P1673}{https://wg21.link/P1673R12}
\end{frame}

\begin{frame}[fragile]
	\frametitle{Algorithmes}
	\begin{itemize}
		\item Algorithmes appelables avec des \textit{list-initialization}
	\end{itemize}

	\begin{minted}{cpp}
		struct Foo { int x; int y; };
		vector<Foo> v{ ... };

		find(begin(v), end(v), {3, 4}); // Foo{3, 4} en C++23
	\end{minted}

	\addproposal{P2248}{https://wg21.link/P2248R8}
\end{frame}


\begin{frame}[fragile]
	\frametitle{\mintinline[style=white]{cpp}|std::visit()|}
	\begin{itemize}
		\item Versions membres de \mintinline{cpp}|std::visit()| et \mintinline{cpp}|std::visit_format_arg()|
	\end{itemize}

	\addproposal{P2637}{https://wg21.link/P2637R3}
\end{frame}

\subsection*{Ranges}
\begin{frame}[fragile]
	\frametitle{Ranges}
	\begin{itemize}
		\item \mintinline{cpp}|std::views::concat| crée une vue concaténant plusieurs ranges
	\end{itemize}

	\begin{minted}{cpp}
		vector<int> v1{1,2,3}, v2{4,5}, v3{};
		array  a{6,7,8};

		// 1, 2, 3, 4, 5, 6, 7, 8
		views::concat(v1, v2, v3, a);
	\end{minted}

	\begin{itemize}
		\item Api \og{}vector\fg{} pour la génération de nombres aléatoires
	\end{itemize}

	\begin{minted}{cpp}
		array<int, 10> a;
		mt19937 g(777);

		ranges::generate_random(a, g);
	\end{minted}

	\addproposal{P2542}{https://wg21.link/P2542R7}
	\addproposal{P1068}{https://wg21.link/P1068R10}
\end{frame}

\begin{frame}[fragile]
	\frametitle{Ranges}
	\begin{itemize}
		\item Traitement de \mintinline{cpp}|std::optional| comme un range similaire à \mintinline{cpp}|single_view|
	\end{itemize}

	\begin{minted}{cpp}
		optional<int> empty;
		for(int i: empty) { std::cout << i; } // Vide

		optional<int> not_empty;
		for(int i: not_empty) { std::cout << i; } // Un element
	\end{minted}

	\addproposal{p3168}{https://wg21.link/p3168r2}
\end{frame}

\subsection*{Ratio}
\begin{frame}[fragile]
	\frametitle{Ratio}
	\begin{itemize}
		\item Ajout des préfixes \mintinline{cpp}|quecto|, \mintinline{cpp}|ronto|, \mintinline{cpp}|ronna| et \mintinline{cpp}|quetta|
	\end{itemize}

	\addproposal{P2734}{https://wg21.link/P2734R0}
\end{frame}

\subsection*{Évaluation \textit{compile-time}}
\begin{frame}[fragile]
	\frametitle{\mintinline[style=white]{cpp}|constexpr|}
	\begin{itemize}
		\item Davantage de \mintinline{cpp}|constexpr| dans la bibliothèque standard
		\item Conversion depuis \mintinline{cpp}|void*| dans des contextes \mintinline{cpp}|constexpr|
		\begin{itemize}
			\item \mintinline{cpp}|std::format()| au compile-time
			\item \mintinline{cpp}|std::function_ref|, \mintinline{cpp}|std::function| et \mintinline{cpp}|std::any| \mintinline{cpp}|constexpr|
		\end{itemize}
		\item \textit{Structured bindings} \mintinline{cpp}|constexpr|
		\item \mintinline{cpp}|atomic| \mintinline{cpp}|constexpr|
	\end{itemize}

	\addproposal{P2738}{https://wg21.link/P2738R1}
	\addproposal{P2686}{https://wg21.link/P2686R4}
	\addproposal{P3309}{https://wg21.link/P3309R2}
\end{frame}

\begin{frame}[fragile]
	\frametitle{Exceptions}
	\begin{itemize}
		\item Possibilité de lancer des exceptions dans des fonctions \mintinline{cpp}|consteval|
		\begin{itemize}
			\item Erreur de compilation si l'exception est lancé lors d'une évaluation \textit{compile-time}
		\end{itemize}
	\end{itemize}

	\addproposal{P3068}{https://wg21.link/P3068R4}
\end{frame}

\subsection*{Parameters pack}
\begin{frame}[fragile]
	\frametitle{Parameters pack}
	\begin{itemize}
		\item Indexation des \textit{packs}
	\end{itemize}

	\begin{minted}{cpp}
		template <typename... T>
		constexpr auto first_plus_last(T... values) -> T...[0] {
		  return T...[0](values...[0] + values...[sizeof...(values)-1]);
		}

		first_plus_last(1, 2, 10);  // 11
	\end{minted}

	\addproposal{P2662}{https://wg21.link/P2662R3}
\end{frame}

\subsection*{Gestion mémoire}
\begin{frame}[fragile]
	\frametitle{lifetime}
	\begin{itemize}
		\item \mintinline{cpp}|std::is_within_lifetime()| indique si l'objet pointé est vivant
		\item \ldots{} en particulier si un membre d'une union est active
	\end{itemize}

	\addproposal{P2641}{https://wg21.link/P2641R4}
\end{frame}

\subsection*{Parallélisme}
\begin{frame}[fragile]
	\frametitle{Gestion mémoire}
	\begin{itemize}
		\item \textit{hazard pointers} : unique écrivain, multiples lecteurs
		\item Structure de donnée \textit{read-copy update} : planification d'actions (p.ex. suppression) à réaliser plus tard
	\end{itemize}

	\addproposal{P2530}{https://wg21.link/P2530R3}
	\addproposal{P2545}{https://wg21.link/P2545R4}
\end{frame}

\begin{frame}[fragile]
	\frametitle{SIMD (Single Instruction on Multiple Data)}
	\begin{itemize}
		\item Intégration de \mintinline{cpp}|basic_simd|
	\end{itemize}

	\addproposal{p1928}{https://wg21.link/p1928r12}
\end{frame}

\subsection*{Programmation fonctionnelle}
\begin{frame}[fragile]
	\frametitle{Type appelable}
	\begin{itemize}
		\item Ajout de \mintinline{cpp}|std::copiable_function| pour les fonctions copiables
		\item Ajout de \mintinline{cpp}|std::function_ref|
		\begin{itemize}
			\item Type référence pour le passage d'appelable à une fonction
			\item Plus générique et moins gourmand que \mintinline{cpp}|std::function| et équivalents

\note[item]{Les fonctions n'ont pas besoin d'être copiables ni déplaçables}
		\end{itemize}
	\end{itemize}

	\addproposal{P2548}{https://wg21.link/P2548R6}
	\addproposal{P0792}{https://wg21.link/P0792R14}
\end{frame}

\subsection*{Attributs}
\begin{frame}[fragile]
	\frametitle{Attributs}
	\begin{itemize}
		\item Attributs sur les structured binding
	\end{itemize}

	\begin{minted}{cpp}
		auto [a, b [[attribute]], c] = foo();
	\end{minted}

	\addproposal{P0609}{https://wg21.link/P0609R3}
\end{frame}

\subsection*{Flux}
\begin{frame}[fragile]
	\frametitle{\mintinline[style=white]{cpp}|std::format|}
	\begin{itemize}
		\item Possibilité de fournir une chaîne de format au \textit{runtime}
		\item Amélioration du support de \mintinline{cpp}|std::filesystem::path|
		\begin{itemize}
			\item Présence de caractères d'échappement (p.ex. \mintinline[escapeinside=||]{cpp}{|\textbackslash|n})
			\item Support de caractère UTF-8
		\end{itemize}
	\end{itemize}

	\begin{minted}{cpp}
		string str = "{}";
		format(runtime_format(str), 42);
	\end{minted}

	\begin{itemize}
		\item Redéfinition de \mintinline{cpp}|std::to_string| en terme de \mintinline{cpp}|std::format|
		\item Davantage de vérifications \textit{compile-time} du type des arguments
		\begin{itemize}
			\item Déjà le cas de la majorité des erreurs
			\item \ldots{} mais pas de toutes
		\end{itemize}
	\end{itemize}

	\begin{minted}{cpp}
		format("{:>{}}", "hello", "10");
		// Erreur run-time
	\end{minted}

	\begin{codesample}
		\sample{https://godbolt.org/#g:!((g:!((g:!((h:codeEditor,i:(filename:'1',fontScale:14,fontUsePx:'0',j:1,lang:c%2B%2B,selection:(endColumn:1,endLineNumber:1,positionColumn:1,positionLineNumber:1,selectionStartColumn:1,selectionStartLineNumber:1,startColumn:1,startLineNumber:1),source:'%23include+%3Ciostream%3E%0A%23include+%3Cformat%3E%0A%0Aint+main()%0A%7B%0A++int+i+%3D+0%3B%0A++std::cout+%3C%3C+std::format(%22%7B:%23018X%7D%22,+reinterpret_cast%3Cuintptr_t%3E(%26i))%3B%0A%7D%0A'),l:'5',n:'0',o:'C%2B%2B+source+%231',t:'0')),k:50,l:'4',n:'0',o:'',s:0,t:'0'),(g:!((h:executor,i:(argsPanelShown:'1',compilationPanelShown:'0',compiler:gsnapshot,compilerName:'',compilerOutShown:'0',execArgs:'',execStdin:'',fontScale:14,fontUsePx:'0',j:1,lang:c%2B%2B,libs:!(),options:'-std%3Dc%2B%2B26+-Wall+-Wextra+-pedantic',overrides:!(),runtimeTools:!(),source:1,stdinPanelShown:'1',wrap:'1'),l:'5',n:'0',o:'Executor+x86-64+gcc+(trunk)+(C%2B%2B,+Editor+%231)',t:'0')),header:(),k:50,l:'4',n:'0',o:'',s:0,t:'0')),l:'2',n:'0',o:'',t:'0')),version:4}
	\end{codesample}

	\addproposal{P2918}{https://wg21.link/P2918R2}
	\addproposal{P2845}{https://wg21.link/P2845R6}
	\addproposal{P2587}{https://wg21.link/P2587R3}
	\addproposal{P2757}{https://wg21.link/P2757R3}
\end{frame}

\begin{frame}[fragile]
	\frametitle{\mintinline[style=white]{cpp}|std::format|}
	\begin{itemize}
		\item Formatage des pointeurs
	\end{itemize}

	\begin{minted}{cpp}
		format("{:#018X}", reinterpret_cast<uintptr_t>(&i));
		// 0X00007FFE0325C4E4
	\end{minted}

	\begin{codesample}
		\sample{https://godbolt.org/#g:!((g:!((g:!((h:codeEditor,i:(filename:'1',fontScale:14,fontUsePx:'0',j:1,lang:c%2B%2B,selection:(endColumn:1,endLineNumber:1,positionColumn:1,positionLineNumber:1,selectionStartColumn:1,selectionStartLineNumber:1,startColumn:1,startLineNumber:1),source:'%23include+%3Ciostream%3E%0A%23include+%3Cformat%3E%0A%23include+%3Cstring%3E%0A%0Aint+main()%0A%7B%0A++std::string+str+%3D+%22%7B%7D%22%3B%0A++std::cout+%3C%3C+std::format(std::runtime_format(str),+42)%3B%0A%7D%0A'),l:'5',n:'0',o:'C%2B%2B+source+%231',t:'0')),k:50,l:'4',n:'0',o:'',s:0,t:'0'),(g:!((h:executor,i:(argsPanelShown:'1',compilationPanelShown:'0',compiler:gsnapshot,compilerName:'',compilerOutShown:'0',execArgs:'',execStdin:'',fontScale:14,fontUsePx:'0',j:1,lang:c%2B%2B,libs:!(),options:'-std%3Dc%2B%2B26+-Wall+-Wextra+-pedantic',overrides:!(),runtimeTools:!(),source:1,stdinPanelShown:'1',wrap:'1'),l:'5',n:'0',o:'Executor+x86-64+gcc+(trunk)+(C%2B%2B,+Editor+%231)',t:'0')),header:(),k:50,l:'4',n:'0',o:'',s:0,t:'0')),l:'2',n:'0',o:'',t:'0')),version:4}
	\end{codesample}

	\addproposal{P2510}{https://wg21.link/P2510R3}
\end{frame}

\begin{frame}[fragile]
	\frametitle{\mintinline[style=white]{cpp}|std::print|}
	\begin{itemize}
		\item Impression de ligne vide
	\end{itemize}

	\begin{minted}{cpp}
		println();
		// println("") en C++23
	\end{minted}

	\begin{itemize}
		\item Optimisation de \mintinline{cpp}|std::print()|
	\end{itemize}

	\addproposal{P3142}{https://wg21.link/P3142R0}
	\addproposal{p3235}{https://wg21.link/p3235r0}
\end{frame}

\subsection*{Durées et temps}
\begin{frame}[fragile]
	\frametitle{Durées et temps}
	\begin{itemize}
		\item Spécialisation de \mintinline{cpp}|std::hash| pour \mintinline{cpp}|std::chrono|
	\end{itemize}

	\addproposal{P2592}{https://wg21.link/P2592R3}
\end{frame}

\subsection*{Système de fichiers}
\begin{frame}[fragile]
	\frametitle{Accès bas-niveaux aux IO}
	\begin{itemize}
		\item Alias \mintinline{cpp}|native_handle_type| sur le descripteur de fichier de la plateforme
		\item \mintinline{cpp}|native_handle()| retourne ce descripteur
	\end{itemize}

	\addproposal{P1759}{https://wg21.link/P1759R6}
\end{frame}

\subsection*{Concurrence}
\begin{frame}[fragile]
	\frametitle{Concurrence}
	\begin{itemize}
		\item Version \mintinline{cpp}|atomic| de minimum et maximum
		\item \mintinline{cpp}|std::execution| : gestion d'exécution asynchrone
		\begin{itemize}
			\item Basé sur des \textit{schedulers}, \textit{senders} et \textit{receivers}
			\item Et un ensemble d'algorithmes asynchrones
		\end{itemize}
	\end{itemize}

	\addproposal{P0493}{https://wg21.link/P0493R5}
	\addproposal{p2300}{https://wg21.link/p2300r9}
\end{frame}

\subsection*{Compilation et implémentation}
\begin{frame}[fragile]
	\frametitle{Modules}
	\begin{itemize}
		\item Suppression de l'expansion de macros dans les déclarations de module
	\end{itemize}

	\addproposal{P3034}{https://wg21.link/P3034R1}
\end{frame}

\begin{frame}[fragile]
	\frametitle{Debug}
	\begin{itemize}
		\item \mintinline{cpp}|std::breakpoint()| : point d'arrêt dans le programme
		\item \mintinline{cpp}|std::breakpoint_if_debugging| : point d'arrêt si l'exécution se fait dans un debugger
		\item \mintinline{cpp}|std::is_debugger_present()| permet de savoir si l'exécution se fait dans un debugger
	\end{itemize}

	\addproposal{P2546}{https://wg21.link/P2546R5}
\end{frame}
\end{document}