\documentclass[C++.tex]{subfiles}
\begin{document}

\section{C++20}
\subsection*{Présentation}
\begin{frame}
	\frametitle{Présentation}
	\begin{itemize}
		\item Approuvé en décembre 2020
		\item Dernier Working Draft : \href{https://github.com/cplusplus/draft/releases/download/n4861/n4861.pdf}{N4861\linklogo}
	\end{itemize}
\end{frame}

\subsection*{Organisation}
\begin{frame}
	\frametitle{Changements d'organisation du comité}
	\begin{itemize}
		\item Création d'un \textit{Direction Group}
		\item Création d'un \textit{Study Group} pour l'éducation (SG20)
	\end{itemize}

	\centering
	\adjincludegraphics[height=0.7\textheight]{input_src/wg21-structure.png}
\end{frame}

\subsection*{Dépréciations et suppressions}
\begin{frame}[fragile]
	\frametitle{Dépréciations et suppressions}
	\begin{itemize}
		\item Dépréciation du terme POD et de \mintinline{cpp}|std::is_pod()|
		\item Dépréciation partielle de \mintinline{cpp}|volatile|
		\item Dépréciation de l'usage de l'opérateur virgule dans les expressions d'indiçage

\note[item]{C'est à dire de ne plus pouvoir écrire \mintinline{cpp}|[x,y]| comme indice d'un conteneur classique}
\note[item]{L'expression \mintinline{cpp}|[x,y]| revient à utiliser \mintinline{cpp}|y| comme indice mais en ayant évalué \mintinline{cpp}|x|}

		\item Dépréciation de \mintinline{cpp}|std::rel_ops|
		\item Suppression des membres dépréciés de \mintinline{cpp}|std::reference_wrapper| : \mintinline{cpp}|result_type|, \mintinline{cpp}|argument_type|, \mintinline{cpp}|first_argument_type| et \mintinline{cpp}|second_argument_type|
	\end{itemize}
\end{frame}

\subsection*{compilation conditionnelle et macro}
\begin{frame}[fragile]
	\frametitle{Fonctionnalités}
	\begin{itemize}
		\item \mintinline{cpp}|__has_cpp_attribute| teste le support d'un attribut
		\begin{itemize}
			\item Similaire à \mintinline{cpp}|__has_include| pour la présence d'entête
			\item Extensible aux attributs propriétaires d'une implémentation
		\end{itemize}
		\item Macros testant le support de fonctionnalité du langage
		\begin{itemize}
			\item \mintinline{cpp}|__cpp_decltype| : support de \mintinline{cpp}|decltype|
			\item \mintinline{cpp}|__cpp_range_based_for| : support du \textit{range-based for loop}
			\item \mintinline{cpp}|__cpp_static_assert| : support de \mintinline{cpp}|static_assert|
			\item \ldots{}
		\end{itemize}
		\item Macros testant le support de fonctionnalités par la bibliothèque standard
		\begin{itemize}
			\item \mintinline{cpp}|__cpp_lib_any| : support de \mintinline{cpp}|std::any|
			\item \mintinline{cpp}|__cpp_lib_chrono| : support de \mintinline{cpp}|std::chrono|
			\item \mintinline{cpp}|__cpp_lib_gcd_lcm| : support des fonctions \mintinline{cpp}|std::gcd()| et \mintinline{cpp}|std::lcm()|
			\item \ldots{}
		\end{itemize}
	\end{itemize}

	\begin{block}{Valorisation}
		\begin{itemize}
			\item Année et mois de l'acceptation dans le standard ou de l'évolution

\note[item]{Ainsi en testant la valeur, on peut savoir l'état d'implémentation de la fonctionnalité si celle-ci a évolué}
		\end{itemize}
	\end{block}

	\begin{codesample}
		\sample{https://godbolt.org/#g:!((g:!((g:!((h:codeEditor,i:(filename:'1',fontScale:14,fontUsePx:'0',j:1,lang:c%2B%2B,selection:(endColumn:1,endLineNumber:1,positionColumn:1,positionLineNumber:1,selectionStartColumn:1,selectionStartLineNumber:1,startColumn:1,startLineNumber:1),source:'%23include+%3Ciostream%3E%0A%0A%23if+1%0A%23include+%3Cnumeric%3E%0A%23endif%0A%0Aint+main()%0A%7B%0A%23if+__cpp_lib_gcd_lcm%0A++std::cout+%3C%3C+%22__cpp_lib_gcd_lcm+%22+%3C%3C+__cpp_lib_gcd_lcm+%3C%3C+%22%5Cn%22%3B%0A%23endif%0A%7D%0A'),l:'5',n:'0',o:'C%2B%2B+source+%231',t:'0')),k:50,l:'4',n:'0',o:'',s:0,t:'0'),(g:!((h:executor,i:(argsPanelShown:'1',compilationPanelShown:'0',compiler:gsnapshot,compilerName:'',compilerOutShown:'0',execArgs:'',execStdin:'',fontScale:14,fontUsePx:'0',j:1,lang:c%2B%2B,libs:!(),options:'-std%3Dc%2B%2B20+-Wall+-Wextra+-pedantic',overrides:!(),runtimeTools:!(),source:1,stdinPanelShown:'1',tree:'1',wrap:'0'),l:'5',n:'0',o:'Executor+x86-64+gcc+(trunk)+(C%2B%2B,+Editor+%231)',t:'0')),header:(),k:50,l:'4',n:'0',o:'',s:0,t:'0')),l:'2',n:'0',o:'',t:'0')),version:4}
	\end{codesample}

	\begin{codesample}
		\sample{https://godbolt.org/#g:!((g:!((g:!((h:codeEditor,i:(filename:'1',fontScale:14,fontUsePx:'0',j:1,lang:c%2B%2B,selection:(endColumn:1,endLineNumber:1,positionColumn:1,positionLineNumber:1,selectionStartColumn:1,selectionStartLineNumber:1,startColumn:1,startLineNumber:1),source:'%23include+%3Ciostream%3E%0A%0Aint+main()%0A%7B%0A%23if+__cpp_attributes%0A++std::cout+%3C%3C+%22__cpp_attributes+%22+%3C%3C+__cpp_attributes+%3C%3C+%22%5Cn%22%3B%0A%23endif%0A%7D%0A'),l:'5',n:'0',o:'C%2B%2B+source+%231',t:'0')),k:50,l:'4',n:'0',o:'',s:0,t:'0'),(g:!((h:executor,i:(argsPanelShown:'1',compilationPanelShown:'0',compiler:gsnapshot,compilerName:'',compilerOutShown:'0',execArgs:'',execStdin:'',fontScale:14,fontUsePx:'0',j:1,lang:c%2B%2B,libs:!(),options:'-std%3Dc%2B%2B20+-Wall+-Wextra+-pedantic',overrides:!(),runtimeTools:!(),source:1,stdinPanelShown:'1',tree:'1',wrap:'0'),l:'5',n:'0',o:'Executor+x86-64+gcc+(trunk)+(C%2B%2B,+Editor+%231)',t:'0')),header:(),k:50,l:'4',n:'0',o:'',s:0,t:'0')),l:'2',n:'0',o:'',t:'0')),version:4}
	\end{codesample}

	\begin{codesample}
		\sample{https://godbolt.org/#g:!((g:!((g:!((h:codeEditor,i:(filename:'1',fontScale:14,fontUsePx:'0',j:1,lang:c%2B%2B,selection:(endColumn:1,endLineNumber:1,positionColumn:1,positionLineNumber:1,selectionStartColumn:1,selectionStartLineNumber:1,startColumn:1,startLineNumber:1),source:'%23include+%3Ciostream%3E%0A%0Aint+main()%0A%7B%0A%23if+__has_cpp_attribute(deprecated)%0A++std::cout+%3C%3C+%22__has_cpp_attribute(deprecated)+%22+%3C%3C+__has_cpp_attribute(deprecated)+%3C%3C+%22%5Cn%22%3B%0A%23endif%0A%0A%23if+__has_cpp_attribute(toto)%0A++std::cout+%3C%3C+%22__has_cpp_attribute(toto)+%22+%3C%3C+__has_cpp_attribute(toto)+%3C%3C+%22%5Cn%22%3B%0A%23endif%0A%7D%0A'),l:'5',n:'0',o:'C%2B%2B+source+%231',t:'0')),k:50,l:'4',n:'0',o:'',s:0,t:'0'),(g:!((h:executor,i:(argsPanelShown:'1',compilationPanelShown:'0',compiler:gsnapshot,compilerName:'',compilerOutShown:'0',execArgs:'',execStdin:'',fontScale:14,fontUsePx:'0',j:1,lang:c%2B%2B,libs:!(),options:'-std%3Dc%2B%2B20+-Wall+-Wextra+-pedantic',overrides:!(),runtimeTools:!(),source:1,stdinPanelShown:'1',tree:'1',wrap:'0'),l:'5',n:'0',o:'Executor+x86-64+gcc+(trunk)+(C%2B%2B,+Editor+%231)',t:'0')),header:(),k:50,l:'4',n:'0',o:'',s:0,t:'0')),l:'2',n:'0',o:'',t:'0')),version:4}
	\end{codesample}

	\addproposal{P0941}{https://wg21.link/P0941R2}
\end{frame}

\begin{frame}[fragile]
	\frametitle{Information à la compilation}
	\begin{itemize}
		\item Entête \mintinline{cpp}|<version>| : informations de version
		\begin{itemize}
			\item Contenu \textit{implementation-dependent}
			\item Version du standard, de la bibliothèque, \textit{release date}, copyright, \ldots{}
		\end{itemize}
		\item \mintinline{cpp}|source_location| : position dans le code source
		\begin{itemize}
			\item Fichier, ligne, colonne et fonction courante
			\item Contenu \textit{implementation-dependent}

\note[item]{En particulier, la numérotation des lignes et colonnes peut commencer à 0 ou à 1}
\note[item]{Il est toutefois encouragé de numéroter à partir de 1 et de retourner 0 lorsque le numéro n'est pas connu}

			\item Remplaçant de \mintinline{cpp}|__LINE__|, \mintinline{cpp}|__FILE__|, \mintinline{cpp}|__func__| et autres macros propriétaires
		\end{itemize}
	\end{itemize}

	\begin{codesample}
		\sample{https://godbolt.org/#g:!((g:!((g:!((h:codeEditor,i:(filename:'1',fontScale:14,fontUsePx:'0',j:1,lang:c%2B%2B,selection:(endColumn:1,endLineNumber:1,positionColumn:1,positionLineNumber:1,selectionStartColumn:1,selectionStartLineNumber:1,startColumn:1,startLineNumber:1),source:'%23include+%3Ciostream%3E%0A%23include+%3Cstring_view%3E%0A%23include+%3Csource_location%3E%0A%0Avoid+log(const+std::string_view%26+message,+const+std::source_location%26+location+%3D+std::source_location::current())%0A%7B%0A++std::cout+%3C%3C+location.file_name()+%3C%3C+%22:%22%0A++++++++++++%3C%3C+location.line()+%3C%3C+%22+%22%0A++++++++++++%3C%3C+location.function_name()+%3C%3C+%22+%22%0A++++++++++++%3C%3C+message+%3C%3C+!'%5Cn!'%3B%0A%7D%0A+%0Aint+main()%0A%7B%0A++log(%22Hello+world!!%22)%3B%0A%7D%0A'),l:'5',n:'0',o:'C%2B%2B+source+%231',t:'0')),k:50,l:'4',n:'0',o:'',s:0,t:'0'),(g:!((h:executor,i:(argsPanelShown:'1',compilationPanelShown:'0',compiler:gsnapshot,compilerName:'',compilerOutShown:'0',execArgs:'',execStdin:'',fontScale:14,fontUsePx:'0',j:1,lang:c%2B%2B,libs:!(),options:'-std%3Dc%2B%2B20+-Wall+-Wextra+-pedantic',overrides:!(),runtimeTools:!(),source:1,stdinPanelShown:'1',tree:'1',wrap:'0'),l:'5',n:'0',o:'Executor+x86-64+gcc+(trunk)+(C%2B%2B,+Editor+%231)',t:'0')),header:(),k:50,l:'4',n:'0',o:'',s:0,t:'0')),l:'2',n:'0',o:'',t:'0')),version:4}
	\end{codesample}

	\addproposal{P0754}{https://wg21.link/P0754R2}
	\addproposal{P1208}{https://wg21.link/P1208R6}
\end{frame}

\begin{frame}[fragile]
	\frametitle{Compilation conditionnelle}
	\begin{itemize}
		\item Ajout d'un paramètre booléen, optionnel, à \mintinline{cpp}|explicit|
		\begin{itemize}
			\item Pilotage de \mintinline{cpp}|explicit| via un paramètre booléen \textit{compile-time}
			\item Possibilité de rendre des constructeurs templates explicites ou non en fonction de l'instanciation
			\item Alternative à des constructions à base de macros de compilation ou de SFINAE
		\end{itemize}
	\end{itemize}

	\addproposal{P0892}{https://wg21.link/P0892R2}
\end{frame}

\subsection*{Types}
\begin{frame}[fragile]
	\frametitle{Types entiers}
	\begin{itemize}
		\item Types entiers signés obligatoirement en compléments à 2
	\end{itemize}

	\begin{block}{Situation pré-C++20}
		\begin{itemize}
			\item Pas de contrainte en C++
			\item 3 choix en C : signe + mantisse, complément à 1 et complément à 2
		\end{itemize}
	\end{block}

\note[item]{A priori, le même changement va avoir lieu en C2x}
\note[item]{Complément à 1 : négation de tous les bits}
\note[item]{Complément à 2 : négation de tous les bits, puis ajout de 1}

	\begin{alertblock}{Compatibilité}
		\begin{itemize}
			\item En pratique, toutes les implémentations actuelles sont en complément à 2
		\end{itemize}
	\end{alertblock}

	\begin{itemize}
		\item Précision de comportements sur des types entiers signés
		\begin{itemize}
			\item Conversion vers non signé est toujours bien définie

\note[item]{C'est l'unique valeur de la destination congruent à la source modulo $2^N$ où $N$ est le nombre de bits de la destination}

			\item Décalage à gauche : même résultat que celui du type non signé correspondant
			\item Décalage à droite : décalage arithmétique avec extension du signe
		\end{itemize}
	\end{itemize}

	\addproposal{P1236}{https://wg21.link/P1236R1}
\end{frame}

\begin{frame}[fragile]
	\frametitle{Caractères}
	\begin{itemize}
		\item Contraintes de \mintinline{cpp}|char16_t| et \mintinline{cpp}|char32_t| : caractères UTF-16 et UTF-32

\note[item]{Auparavant, ils pouvaient représenter ces caractères, mais ce n'était pas une obligation et pouvaient représenter d'autres codages de caractères larges}
\note[item]{Et les \textit{Universal character name} doivent correspondre à des \textit{code points} ISO/IEC 10646}

		\item \mintinline{cpp}|char8_t| pour les caractères UTF-8
		\begin{itemize}
			\item Pendant UTF-8 de \mintinline{cpp}|char16_t| et \mintinline{cpp}|char32_t|
			\item Similaire en terme de taille, d'alignement, de conversion à \mintinline{cpp}|unsigned char|
			\item Pas un alias sur un autre type
			\item Prise en compte dans la bibliothèque standard
		\end{itemize}
		\item Type \mintinline{cpp}|u8string| pour les chaînes UTF-8
	\end{itemize}

	\begin{block}{Motivation}
		\begin{itemize}
			\item Suppression de l’ambiguïté caractère UTF-8 / littéral
			\item Suppression d’ambiguïté sur les surcharges et spécialisation de template
		\end{itemize}
	\end{block}

	\addproposal{P0482}{https://wg21.link/P0482R6}
	\addproposal{P1041}{https://wg21.link/P1041R4}
\end{frame}

\begin{frame}[fragile]
	\frametitle{Définition d'agrégat}
	\begin{itemize}
		\item Modification de la définition d'agrégat :
		\begin{itemize}
			\item C++17 : pas de constructeur \textit{user-provided}
			\item C++20 : pas de constructeur \textit{user-declared}

\note[item]{Ni \mintinline{cpp}|==default|, ni \mintinline{cpp}|==delete|}
\note[item]{Y compris dans les classes de base}
		\end{itemize}
	\end{itemize}

	\begin{minted}{cpp}
		// Agregat en C++17 pas en C++20
		struct S {
		  S() = default;
		};
	\end{minted}

	\addproposal{P1008}{https://wg21.link/P1008R1}
\end{frame}

\begin{frame}[fragile]
	\frametitle{Initialisation des agrégats}
	\begin{itemize}
		\item Initialisation nommée des membres d'un agrégat ou d'une union
	\end{itemize}

	\begin{minted}{cpp}
		struct S { int a; int b; int c; };
		S s{.a = 1, .c = 2};

		union U { int a; char* b };
		U u{.b = "foo"};
	\end{minted}

	\begin{alertblock}{Restrictions}
		\begin{itemize}
			\item Uniquement sur les agrégats et les unions
			\item Initialisation des champs dans leur ordre de déclaration
			\item Initialisation d'un unique membre d'une union

\note[item]{Par contre, il est permis de ne pas initialiser des champs d'un agrégat}
		\end{itemize}
	\end{alertblock}

	\begin{codesample}
		\sample{https://godbolt.org/#g:!((g:!((g:!((h:codeEditor,i:(filename:'1',fontScale:14,fontUsePx:'0',j:1,lang:c%2B%2B,selection:(endColumn:1,endLineNumber:1,positionColumn:1,positionLineNumber:1,selectionStartColumn:1,selectionStartLineNumber:1,startColumn:1,startLineNumber:1),source:'%23include+%3Ciostream%3E%0A%0Aunion+U%0A%7B%0A++int+a%3B%0A++const+char*+b%3B%0A%7D%3B%0A%0Aint+main()%0A%7B%0A++%7B%0A++++U+u+%3D+%7B+.b+%3D+%22foo%22+%7D%3B%0A++++std::cout+%3C%3C+u.b+%3C%3C+%22%5Cn%22%3B%0A++%7D%0A%0A%23if+0%0A++%7B%0A++++U+u+%3D+%7B+.a+%3D+1,+.b+%3D+%22asdf%22+%7D%3B%0A++%7D%0A%23endif%0A%7D%0A'),l:'5',n:'0',o:'C%2B%2B+source+%231',t:'0')),k:50,l:'4',n:'0',o:'',s:0,t:'0'),(g:!((h:executor,i:(argsPanelShown:'1',compilationPanelShown:'0',compiler:gsnapshot,compilerName:'',compilerOutShown:'0',execArgs:'',execStdin:'',fontScale:14,fontUsePx:'0',j:1,lang:c%2B%2B,libs:!(),options:'-std%3Dc%2B%2B20+-Wall+-Wextra+-pedantic',overrides:!(),runtimeTools:!(),source:1,stdinPanelShown:'1',tree:'1',wrap:'0'),l:'5',n:'0',o:'Executor+x86-64+gcc+(trunk)+(C%2B%2B,+Editor+%231)',t:'0')),header:(),k:50,l:'4',n:'0',o:'',s:0,t:'0')),l:'2',n:'0',o:'',t:'0')),version:4}
	\end{codesample}

	\begin{codesample}
		\sample{https://godbolt.org/#g:!((g:!((g:!((h:codeEditor,i:(filename:'1',fontScale:14,fontUsePx:'0',j:1,lang:c%2B%2B,selection:(endColumn:1,endLineNumber:1,positionColumn:1,positionLineNumber:1,selectionStartColumn:1,selectionStartLineNumber:1,startColumn:1,startLineNumber:1),source:'%23include+%3Ciostream%3E%0A%0Astruct+S%0A%7B%0A++int+a%3B%0A++int+b%3B%0A++int+c%7B0%7D%3B%0A%7D%3B%0A%0Aint+main()%0A%7B%0A++%7B%0A++++S+s%7B.a+%3D+1,+.b+%3D+2%7D%3B%0A++++std::cout+%3C%3C+s.a+%3C%3C+%22+%22+%3C%3C+s.b+%3C%3C+%22+%22+%3C%3C+s.c+%3C%3C+%22%5Cn%22%3B%0A++%7D%0A%0A++%7B%0A++++S+s%7B.a+%3D+1,+.b+%3D+2,+.c+%3D+3%7D%3B%0A++++std::cout+%3C%3C+s.a+%3C%3C+%22+%22+%3C%3C+s.b+%3C%3C+%22+%22+%3C%3C+s.c+%3C%3C+%22%5Cn%22%3B%0A++%7D%0A%0A%23if+0%0A++%7B%0A++++S+s%7B.b+%3D+2,+.a+%3D+1%7D%3B%0A++%7D%0A%23endif%0A%7D%0A'),l:'5',n:'0',o:'C%2B%2B+source+%231',t:'0')),k:50,l:'4',n:'0',o:'',s:0,t:'0'),(g:!((h:executor,i:(argsPanelShown:'1',compilationPanelShown:'0',compiler:gsnapshot,compilerName:'',compilerOutShown:'0',execArgs:'',execStdin:'',fontScale:14,fontUsePx:'0',j:1,lang:c%2B%2B,libs:!(),options:'-std%3Dc%2B%2B20+-Wall+-Wextra+-pedantic',overrides:!(),runtimeTools:!(),source:1,stdinPanelShown:'1',tree:'1',wrap:'0'),l:'5',n:'0',o:'Executor+x86-64+gcc+(trunk)+(C%2B%2B,+Editor+%231)',t:'0')),header:(),k:50,l:'4',n:'0',o:'',s:0,t:'0')),l:'2',n:'0',o:'',t:'0')),version:4}
	\end{codesample}

	\addproposal{P0329}{https://wg21.link/P0329R4}
\end{frame}

\begin{frame}[fragile]
	\frametitle{Initialisation des agrégats}
	\begin{itemize}
		\item Initialisation des agrégats via des données parenthésées
	\end{itemize}

	\begin{block}{\mintinline[style=white,escapeinside=||]{cpp}{|\{\}|} ou \mintinline[style=white]{cpp}|()|}
		\begin{itemize}
			\item \mintinline{cpp}|{}| permet l'utilisation d'\textit{initializer list}
			\item \mintinline{cpp}|()| permet les conversions avec perte de précision
		\end{itemize}
	\end{block}

	\begin{block}{Motivations}
		\begin{itemize}
			\item Fonctions transférant les arguments à un constructeur sur des agrégats
		\end{itemize}
	\end{block}

\note[item]{Comme \mintinline{cpp}|std::make_unique()| ou \mintinline{cpp}|emplace_back()|}

	\begin{itemize}
		\item Initialisation par défaut des champs de bits
	\end{itemize}

	\begin{minted}{cpp}
		struct Foo {
		  unsigned int a : 1 {0};
		  unsigned int b : 1 = 1;
		};
	\end{minted}

\note[item]{Avant C++20, il fallait définir un constructeur}

	\begin{codesample}
		\sample{https://godbolt.org/#g:!((g:!((g:!((h:codeEditor,i:(filename:'1',fontScale:14,fontUsePx:'0',j:1,lang:c%2B%2B,selection:(endColumn:1,endLineNumber:1,positionColumn:1,positionLineNumber:1,selectionStartColumn:1,selectionStartLineNumber:1,startColumn:1,startLineNumber:1),source:'%23include+%3Ciostream%3E%0A%0Astruct+Foo%0A%7B%0A++unsigned+int+a+:+1+%7B0%7D%3B%0A++unsigned+int+b+:+1+%3D+1%3B%0A%7D%3B%0A%0Aint+main()%0A%7B%0A++Foo+foo%3B%0A++std::cout+%3C%3C+foo.a+%3C%3C+%22+%22+%3C%3C+foo.b+%3C%3C+%22%5Cn%22%3B%0A%7D%0A'),l:'5',n:'0',o:'C%2B%2B+source+%231',t:'0')),k:50,l:'4',n:'0',o:'',s:0,t:'0'),(g:!((h:executor,i:(argsPanelShown:'1',compilationPanelShown:'0',compiler:gsnapshot,compilerName:'',compilerOutShown:'0',execArgs:'',execStdin:'',fontScale:14,fontUsePx:'0',j:1,lang:c%2B%2B,libs:!(),options:'-std%3Dc%2B%2B20+-Wall+-Wextra+-pedantic',overrides:!(),runtimeTools:!(),source:1,stdinPanelShown:'1',tree:'1',wrap:'0'),l:'5',n:'0',o:'Executor+x86-64+gcc+(trunk)+(C%2B%2B,+Editor+%231)',t:'0')),header:(),k:50,l:'4',n:'0',o:'',s:0,t:'0')),l:'2',n:'0',o:'',t:'0')),version:4}
	\end{codesample}

	\addproposal{P0960}{https://wg21.link/P0960R3}
\end{frame}

\begin{frame}[fragile]
	\frametitle{Endianess}
	\begin{itemize}
		\item Énumération \mintinline{cpp}|std::endian|
		\begin{itemize}
			\item \mintinline{cpp}|little| : \textit{little-endian}
			\item \mintinline{cpp}|big| : \textit{big-endian}
			\item \mintinline{cpp}|native| : \textit{endianess} du système
		\end{itemize}
	\end{itemize}

	\begin{minted}{cpp}
		if(endian::native == endian::big)
		  cout << "big-endian\n";
		else if(endian::native == endian::little)
		  cout << "little-endian\n";
		else
		  cout << "mixed-endian\n";
	\end{minted}

	\begin{codesample}
		\sample{https://godbolt.org/#g:!((g:!((g:!((h:codeEditor,i:(filename:'1',fontScale:14,fontUsePx:'0',j:1,lang:c%2B%2B,selection:(endColumn:1,endLineNumber:1,positionColumn:1,positionLineNumber:1,selectionStartColumn:1,selectionStartLineNumber:1,startColumn:1,startLineNumber:1),source:'%23include+%3Cbit%3E%0A%23include+%3Ciostream%3E%0A+%0Aint+main()%0A%7B%0A++if(std::endian::native+%3D%3D+std::endian::big)%0A++++std::cout+%3C%3C+%22big-endian%5Cn%22%3B%0A++else+if(std::endian::native+%3D%3D+std::endian::little)%0A++++std::cout+%3C%3C+%22little-endian%5Cn%22%3B%0A++else%0A++++std::cout+%3C%3C+%22mixed-endian%5Cn%22%3B%0A%7D%0A'),l:'5',n:'0',o:'C%2B%2B+source+%231',t:'0')),k:50,l:'4',n:'0',o:'',s:0,t:'0'),(g:!((h:executor,i:(argsPanelShown:'1',compilationPanelShown:'0',compiler:gsnapshot,compilerName:'',compilerOutShown:'0',execArgs:'',execStdin:'',fontScale:14,fontUsePx:'0',j:1,lang:c%2B%2B,libs:!(),options:'-std%3Dc%2B%2B20+-Wall+-Wextra+-pedantic',overrides:!(),runtimeTools:!(),source:1,stdinPanelShown:'1',tree:'1',wrap:'0'),l:'5',n:'0',o:'Executor+x86-64+gcc+(trunk)+(C%2B%2B,+Editor+%231)',t:'0')),header:(),k:50,l:'4',n:'0',o:'',s:0,t:'0')),l:'2',n:'0',o:'',t:'0')),version:4}
	\end{codesample}

	\addproposal{P0463}{https://wg21.link/P0463R1}
\end{frame}

\begin{frame}[fragile]
	\frametitle{\mintinline[style=white]{cpp}|using enum|}
	\begin{itemize}
		\item Utilisation d'\mintinline{cpp}|using| sur une \mintinline{cpp}|enum class|
	\end{itemize}

	\begin{minted}{cpp}
		enum class Foo { val1, val2, val3 };
		using enum Foo;

		if(foo == val2) { ... }
	\end{minted}

	\begin{itemize}
		\item Sur une valeur de l'énumération
	\end{itemize}

	\begin{minted}{cpp}
		enum class Foo { val1, val2, val3 };
		using Foo::val2;

		if(foo == val2) { ... }
	\end{minted}

	\begin{itemize}
		\item Sur une \textit{unscoped} \mintinline{cpp}|enum|
	\end{itemize}

	\begin{codesample}
		\sample{https://godbolt.org/#g:!((g:!((g:!((h:codeEditor,i:(filename:'1',fontScale:14,fontUsePx:'0',j:1,lang:c%2B%2B,selection:(endColumn:1,endLineNumber:1,positionColumn:1,positionLineNumber:1,selectionStartColumn:1,selectionStartLineNumber:1,startColumn:1,startLineNumber:1),source:'%23include+%3Cbit%3E%0A%23include+%3Ciostream%3E%0A+%0Aenum+class+Foo%0A%7B%0A++val1,%0A++val2,%0A++val3,%0A%7D%3B%0A%0Ausing+Foo::val2%3B%0A%0Aint+main()%0A%7B%0A++Foo+foo+%3D+Foo::val2%3B%0A%0A++if(foo+%3D%3D+val2)%0A++%7B%0A++++std::cout+%3C%3C+%22val2%5Cn%22%3B%0A++%7D%0A%0A%7D%0A'),l:'5',n:'0',o:'C%2B%2B+source+%231',t:'0')),k:50,l:'4',n:'0',o:'',s:0,t:'0'),(g:!((h:executor,i:(argsPanelShown:'1',compilationPanelShown:'0',compiler:gsnapshot,compilerName:'',compilerOutShown:'0',execArgs:'',execStdin:'',fontScale:14,fontUsePx:'0',j:1,lang:c%2B%2B,libs:!(),options:'-std%3Dc%2B%2B20+-Wall+-Wextra+-pedantic',overrides:!(),runtimeTools:!(),source:1,stdinPanelShown:'1',tree:'1',wrap:'0'),l:'5',n:'0',o:'Executor+x86-64+gcc+(trunk)+(C%2B%2B,+Editor+%231)',t:'0')),header:(),k:50,l:'4',n:'0',o:'',s:0,t:'0')),l:'2',n:'0',o:'',t:'0')),version:4}
	\end{codesample}

	\begin{codesample}
		\sample{https://godbolt.org/#g:!((g:!((g:!((h:codeEditor,i:(filename:'1',fontScale:14,fontUsePx:'0',j:1,lang:c%2B%2B,selection:(endColumn:1,endLineNumber:1,positionColumn:1,positionLineNumber:1,selectionStartColumn:1,selectionStartLineNumber:1,startColumn:1,startLineNumber:1),source:'%23include+%3Cbit%3E%0A%23include+%3Ciostream%3E%0A+%0Aenum+class+Foo%0A%7B%0A++val1,%0A++val2,%0A++val3,%0A%7D%3B%0A%0Ausing+enum+Foo%3B%0A%0Aint+main()%0A%7B%0A++Foo+foo+%3D+Foo::val2%3B%0A%0A++if(foo+%3D%3D+val2)%0A++%7B%0A++++std::cout+%3C%3C+%22val2%5Cn%22%3B%0A++%7D%0A%0A%7D%0A'),l:'5',n:'0',o:'C%2B%2B+source+%231',t:'0')),k:50,l:'4',n:'0',o:'',s:0,t:'0'),(g:!((h:executor,i:(argsPanelShown:'1',compilationPanelShown:'0',compiler:gsnapshot,compilerName:'',compilerOutShown:'0',execArgs:'',execStdin:'',fontScale:14,fontUsePx:'0',j:1,lang:c%2B%2B,libs:!(),options:'-std%3Dc%2B%2B20+-Wall+-Wextra+-pedantic',overrides:!(),runtimeTools:!(),source:1,stdinPanelShown:'1',tree:'1',wrap:'0'),l:'5',n:'0',o:'Executor+x86-64+gcc+(trunk)+(C%2B%2B,+Editor+%231)',t:'0')),header:(),k:50,l:'4',n:'0',o:'',s:0,t:'0')),l:'2',n:'0',o:'',t:'0')),version:4}
	\end{codesample}

	\addproposal{P1099}{https://wg21.link/P1099R5}
\end{frame}

\begin{frame}[fragile]
	\frametitle{Conversion pointeur-booléen}
	\begin{itemize}
		\item Conversion pointeur vers booléen devient \textit{narrowing}

\note[item]{Et donc interdite dans tous les contextes où les conversion \textit{narrowing} ne sont pas autorisées}

		\item \mintinline{cpp}|nullptr| reste autorisé dans les initialisations directes
	\end{itemize}

	\begin{minted}{cpp}
		struct Foo {
		  int i;
		  bool b;
		};

		void* p;
		Foo foo{1, p};       // erreur
		bool b1{p};          // erreur
		bool b2 = p;         // OK
		bool b3{nullptr};    // OK
		bool b4 = nullptr;   // erreur
		bool b5 = {nullptr}; // erreur
		if(p) { ... }        // OK
	\end{minted}

	\begin{codesample}
		\sample{https://godbolt.org/#g:!((g:!((g:!((h:codeEditor,i:(filename:'1',fontScale:14,fontUsePx:'0',j:1,lang:c%2B%2B,selection:(endColumn:1,endLineNumber:1,positionColumn:1,positionLineNumber:1,selectionStartColumn:1,selectionStartLineNumber:1,startColumn:1,startLineNumber:1),source:'%23include+%3Cbit%3E%0A%23include+%3Ciostream%3E%0A%0Astruct+Foo%0A%7B%0A++int+i%3B%0A++bool+b%3B%0A%7D%3B%0A%0Aint+main()%0A%7B%0A++void*+p+%3D+nullptr%3B%0A%0A%23if+0%0A++Foo+foo%7B1,+p%7D%3B%0A%23endif%0A%23if+0%0A++bool+b1%7Bp%7D%3B%0A%23endif%0A++bool+b2+%3D+p%3B%0A++bool+b3%7Bnullptr%7D%3B%0A%23if+0%0A++bool+b4+%3D+nullptr%3B%0A%23endif%0A%23if+0%0A++bool+b5+%3D+%7Bnullptr%7D%3B%0A%23endif%0A++if(p)+%7B+%7D%0A%7D%0A'),l:'5',n:'0',o:'C%2B%2B+source+%231',t:'0')),k:50,l:'4',n:'0',o:'',s:0,t:'0'),(g:!((h:executor,i:(argsPanelShown:'1',compilationPanelShown:'0',compiler:gsnapshot,compilerName:'',compilerOutShown:'0',execArgs:'',execStdin:'',fontScale:14,fontUsePx:'0',j:1,lang:c%2B%2B,libs:!(),options:'-std%3Dc%2B%2B20+-Wall+-Wextra+-pedantic+-Wno-unused-variable',overrides:!(),runtimeTools:!(),source:1,stdinPanelShown:'1',tree:'1',wrap:'0'),l:'5',n:'0',o:'Executor+x86-64+gcc+(trunk)+(C%2B%2B,+Editor+%231)',t:'0')),header:(),k:50,l:'4',n:'0',o:'',s:0,t:'0')),l:'2',n:'0',o:'',t:'0')),version:4}
	\end{codesample}

	\addproposal{P1957}{https://wg21.link/P1957R2}
\end{frame}

\subsection*{Classe}
\begin{frame}[fragile]
	\frametitle{Spécifications d'exception et \mintinline[style=white]{cpp}|=default|}
	\begin{itemize}
		\item Définition possible de spécifications d'exception des fonctions \mintinline{cpp}|=default| différentes de celles de la fonction implicite
	\end{itemize}

	\begin{minted}{cpp}
		struct S {
		  // Valide en C++20
		  // Invalide en C++17 (constructeur implicite noexcept)
		  S() noexcept(false) = default;
		};
	\end{minted}

	\addproposal{P1286}{https://wg21.link/P1286R2}
\end{frame}

\subsection*{Sémantique de déplacement}
\begin{frame}[fragile]
	\frametitle{Sémantique de déplacement}
	\begin{itemize}
		\item Davantage de déplacements possibles
	\end{itemize}

	\begin{minted}{cpp}
		unique_ptr<T> f0(unique_ptr<T> && ptr) { return ptr; }

		string f1(string && x) { return x; }

		struct Foo{};

		void f2(Foo w) { throw w; }

		struct Bar { B(Foo); };

		Bar f3() {
		  Foo w;
		  return w;
		}
	\end{minted}

\note[item]{Tous ces exemples sont des cas où il y a une copie en C++17 et un déplacement possible en C++20}

	\addproposal{P1825}{https://wg21.link/P1825R0}
\end{frame}

\subsection*{Opérateurs}
\begin{frame}[fragile]
	\frametitle{spaceship operator -- \mintinline[style=white]{cpp}|operator<=>|}
	\begin{itemize}
		\item Effectue une \og \textit{Three-way comparison}\fg{}
		\item Génère les opérateurs d'ordre (\mintinline{cpp}|<|, \mintinline{cpp}|<=|, \mintinline{cpp}|>| et \mintinline{cpp}|>=|)
		\item Réécrit \mintinline[escapeinside=||]{cpp}{a|@|b} en \mintinline[escapeinside=||]{cpp}{a<=>b|@|0} ou \mintinline[escapeinside=||]{cpp}{0|@|b<=>a}
	\end{itemize}

	\begin{exampleblock}{Comparaison hétérogène}
		\begin{itemize}
			\item Une unique version à écrire (\mintinline{cpp}|A<=>B| ou \mintinline{cpp}|B<=>A|)
		\end{itemize}
	\end{exampleblock}

	\begin{itemize}
		\item Peut être déclaré \mintinline{cpp}|=default| et généré par
		\begin{itemize}
			\item \mintinline{cpp}|operator<=>| des bases et membres
			\item \mintinline{cpp}|operator==| et \mintinline{cpp}|operator<|
		\end{itemize}
	\end{itemize}

	\begin{alertblock}{Attention}
		\begin{itemize}
			\item Uniquement pour des comparaisons homogènes
		\end{itemize}
	\end{alertblock}

	\begin{itemize}
		\item Utilisation de l'opérateur binaire déclaré s'il existe
		\item Supporté par la bibliothèque standard
	\end{itemize}

	\addproposal{P1185}{https://wg21.link/P1185R2}
	\addproposal{P1186}{https://wg21.link/P1186R3}
	\addproposal{P0768}{https://wg21.link/P0768R1}
\end{frame}

\begin{frame}[fragile]
	\frametitle{spaceship operator -- \mintinline[style=white]{cpp}|operator<=>|}
	\begin{itemize}
		\item Trois types de retour possibles
		\begin{itemize}
			\item \mintinline{cpp}|std::strong_ordering| : ordre total et égalité
			\begin{itemize}
				\item \mintinline{cpp}|less|, \mintinline{cpp}|equivalent|/\mintinline{cpp}|equal| et \mintinline{cpp}|greater|
			\end{itemize}
			\item \mintinline{cpp}|std::weak_ordering| : ordre total et équivalence
			\begin{itemize}
				\item \mintinline{cpp}|less|, \mintinline{cpp}|equivalent| et \mintinline{cpp}|greater|
			\end{itemize}
			\item \mintinline{cpp}|std::partial_ordering| : ordre partiel
			\begin{itemize}
				\item \mintinline{cpp}|less|, \mintinline{cpp}|equivalent|, \mintinline{cpp}|greater| et \mintinline{cpp}|unordered|
			\end{itemize}
		\end{itemize}
		\item Conversion \mintinline{cpp}|strong_ordering| $\rightarrow$ \mintinline{cpp}|weak_ordering| $\rightarrow$ \mintinline{cpp}|partial_ordering|
		\item Comparable uniquement avec \mintinline{cpp}|0|
	\end{itemize}

	\addproposal{P1185}{https://wg21.link/P1185R2}
	\addproposal{P1186}{https://wg21.link/P1186R3}
	\addproposal{P0768}{https://wg21.link/P0768R1}
\end{frame}

\begin{frame}[fragile]
	\frametitle{spaceship operator -- \mintinline[style=white]{cpp}|operator==|}
	\begin{itemize}
		\item Génère l'opérateur \mintinline{cpp}|!=|
		\item Peut être déclaré \mintinline{cpp}|=default| et généré par \mintinline{cpp}|operator==| des bases et membres
	\end{itemize}

	\begin{alertblock}{Génération de \mintinline[style=white]{cpp}|operator==|}
		\begin{itemize}
			\item Pas de génération depuis \mintinline{cpp}|operator<=>|

\note[item]{Initialement, \mintinline{cpp}|operator==| et \mintinline{cpp}|operator!=| aussi résolu via \mintinline{cpp}|operator<=>|. Modifié pour ne pas avoir des tests d'égalité non optimaux \og par erreur\fg{} via le \textit{fallback} sur \mintinline{cpp}|operator<=>| (p.ex. court-circuit sur la taille de conteneurs)}
		\end{itemize}
	\end{alertblock}

	\begin{exampleblock}{\mintinline[style=white]{cpp}|=default| implicite}
		\begin{itemize}
			\item Implicitement \mintinline{cpp}|=default| lorsque \mintinline{cpp}|operator<=>| est \mintinline{cpp}|=default|
		\end{itemize}
	\end{exampleblock}

	\begin{codesample}
		\sample{https://godbolt.org/#g:!((g:!((g:!((h:codeEditor,i:(filename:'1',fontScale:14,fontUsePx:'0',j:1,lang:c%2B%2B,selection:(endColumn:1,endLineNumber:1,positionColumn:1,positionLineNumber:1,selectionStartColumn:1,selectionStartLineNumber:1,startColumn:1,startLineNumber:1),source:'%23include+%3Ccompare%3E%0A%23include+%3Ciostream%3E%0A%0Astruct+Foo%0A%7B%0A++int+value%3B%0A%0A%23if+1%0A++auto+operator%3C%3D%3E(const+Foo%26+rhs)+const%0A++%7B%0A++++return+value+%3C%3D%3E+rhs.value%3B%0A++%7D%0A%23else%0A++auto+operator%3C%3D%3E(const+Foo%26+rhs)+const+%3D+default%3B%0A%23endif%0A%0A%23if+0%0A++bool+operator%3D%3D(const+Foo%26+rhs)+const+%3D+default%3B%0A%23endif%0A%0A%7D%3B%0A%0Aint+main()%0A%7B%0A++Foo+foo1%7B2011%7D%3B%0A++Foo+foo2%7B2014%7D%3B%0A%0A++std::cout+%3C%3C+std::boolalpha%3B%0A++std::cout+%3C%3C+(foo1+%3C+foo2)+%3C%3C+%22%5Cn%22%3B%0A++std::cout+%3C%3C+(foo1+%3E+foo2)+%3C%3C+%22%5Cn%22%3B%0A%23if+0%0A++std::cout+%3C%3C+(foo1+%3D%3D+foo2)+%3C%3C+%22%5Cn%22%3B%0A++std::cout+%3C%3C+(foo1+!!%3D+foo2)+%3C%3C+%22%5Cn%22%3B%0A%23endif%0A%7D%0A'),l:'5',n:'0',o:'C%2B%2B+source+%231',t:'0')),k:50,l:'4',n:'0',o:'',s:0,t:'0'),(g:!((h:executor,i:(argsPanelShown:'1',compilationPanelShown:'0',compiler:gsnapshot,compilerName:'',compilerOutShown:'0',execArgs:'',execStdin:'',fontScale:14,fontUsePx:'0',j:1,lang:c%2B%2B,libs:!(),options:'-std%3Dc%2B%2B20+-Wall+-Wextra+-pedantic',overrides:!(),runtimeTools:!(),source:1,stdinPanelShown:'1',tree:'1',wrap:'0'),l:'5',n:'0',o:'Executor+x86-64+gcc+(trunk)+(C%2B%2B,+Editor+%231)',t:'0')),header:(),k:50,l:'4',n:'0',o:'',s:0,t:'0')),l:'2',n:'0',o:'',t:'0')),version:4}
	\end{codesample}

	\addproposal{P1185}{https://wg21.link/P1185R2}
	\addproposal{P1186}{https://wg21.link/P1186R3}
	\addproposal{P0768}{https://wg21.link/P0768R1}
\end{frame}

\begin{frame}[fragile]
	\frametitle{spaceship operator -- Conclusion}
	\begin{exampleblock}{Do}
		\begin{itemize}
			\item Privilégiez \mintinline{cpp}|operator<=>| aux opérateurs \mintinline{cpp}|<|, \mintinline{cpp}|<=|, \mintinline{cpp}|>| et \mintinline{cpp}|>=|
			\item Déclarez \mintinline{cpp}|operator<=>| et \mintinline{cpp}|operator==| \mintinline{cpp}|=default| si possible
		\end{itemize}
	\end{exampleblock}

	\begin{alertblock}{Don't}
		\begin{itemize}
			\item Ne mélangez pas \mintinline{cpp}|operator<=>| et opérateurs d'ordre dans une même classe
		\end{itemize}
	\end{alertblock}
\end{frame}

\subsection*{Namespace}
\begin{frame}[fragile]
	\frametitle{Nested namespace}
	\begin{itemize}
		\item Extension des \textit{nested namespaces} aux \textit{inline namespaces}
	\end{itemize}

	\begin{minted}{cpp}
		namespace A::inline B::C {
		  int i;
		}

		// Equivalent a

		namespace A {
		  inline namespace B {
		    namespace C {
		      int i;
		} } }
	\end{minted}

	\begin{codesample}
		\sample{https://godbolt.org/#g:!((g:!((g:!((h:codeEditor,i:(filename:'1',fontScale:14,fontUsePx:'0',j:1,lang:c%2B%2B,selection:(endColumn:1,endLineNumber:1,positionColumn:1,positionLineNumber:1,selectionStartColumn:1,selectionStartLineNumber:1,startColumn:1,startLineNumber:1),source:'%23include+%3Ciostream%3E%0A%0Anamespace+A::inline+B::C%0A%7B%0A++int+i+%3D+5%3B%0A%7D%0A%0Aint+main()%0A%7B%0A++std::cout+%3C%3C+A::B::C::i+%3C%3C+%22%5Cn%22%3B%0A++std::cout+%3C%3C+A::C::i+%3C%3C+%22%5Cn%22%3B%0A%7D%0A'),l:'5',n:'0',o:'C%2B%2B+source+%231',t:'0')),k:50,l:'4',n:'0',o:'',s:0,t:'0'),(g:!((h:executor,i:(argsPanelShown:'1',compilationPanelShown:'0',compiler:gsnapshot,compilerName:'',compilerOutShown:'0',execArgs:'',execStdin:'',fontScale:14,fontUsePx:'0',j:1,lang:c%2B%2B,libs:!(),options:'-std%3Dc%2B%2B20+-Wall+-Wextra+-pedantic',overrides:!(),runtimeTools:!(),source:1,stdinPanelShown:'1',tree:'1',wrap:'0'),l:'5',n:'0',o:'Executor+x86-64+gcc+(trunk)+(C%2B%2B,+Editor+%231)',t:'0')),header:(),k:50,l:'4',n:'0',o:'',s:0,t:'0')),l:'2',n:'0',o:'',t:'0')),version:4}
	\end{codesample}

	\addproposal{P1094}{https://wg21.link/P1094R2}
\end{frame}

\subsection*{Modules}
\begin{frame}[fragile]
	\frametitle{Modules -- Présentation}
	\begin{itemize}
\note[item]{Fusion du TS de mai 2018 et de la proposition concurrente de Clang (ATOM)}

		\item Alternative au mécanisme d'inclusion
	\end{itemize}

	\begin{alertblock}{Modules et \mintinline[style=white]{cpp}|namespace|}
		\begin{itemize}
			\item Ne replace pas les \mintinline{cpp}|namespace|
		\end{itemize}
	\end{alertblock}

	\begin{itemize}
		\item [] \begin{itemize}
			\item Réduction des temps de compilation
			\item Nouveau niveau d'encapsulation
			\item Plus grande robustesse (isolation des effets des macros)
			\item Meilleurs prises en charge des bibliothèques par l'analyse statique, les optimiseurs, \ldots{}
			\item Gestion des inclusions multiples sans garde
			\item Compatible avec le système actuel d'inclusion
		\end{itemize}
	\end{itemize}

	\begin{alertblock}{Bibliothèque standard}
		\begin{itemize}
			\item En C++20, la bibliothèque standard n'utilise pas les modules
		\end{itemize}
	\end{alertblock}

	\addproposal{P1103}{https://wg21.link/P1103R3}
\end{frame}

\begin{frame}[fragile]
	\frametitle{Modules -- Interface Unit}
	\begin{itemize}
		\item L'\textit{Interface Unit} commence par un préambule
		\begin{itemize}
			\item Nom du module à exporter
			\item Suivi de l'import d'autres modules
			\item Éventuellement ré-exportés par le module
		\end{itemize}
	\end{itemize}

	\begin{minted}{cpp}
		export module foo;
		import a;
		export import b;
	\end{minted}

	\begin{itemize}
		\item Suivi du corps exportant des symboles via le mot-clé \mintinline{cpp}|export|
	\end{itemize}

\note[item]{Les symboles \og internal linkage\fg{} ne sont bien entendu pas exportables}

	\begin{minted}{cpp}
		export int i;
		export void bar(int j);
		export {
		  void baz();
		  long l;
		}
	\end{minted}

	\addproposal{P1103}{https://wg21.link/P1103R3}
\end{frame}

\begin{frame}[fragile]
	\frametitle{Modules -- Implementation Unit}
	\begin{itemize}
		\item L'\textit{Implementation Unit} commence par un préambule
		\begin{itemize}
			\item Nom du module implémenté
			\item Suivi de l'import d'autres modules
		\end{itemize}
		\item Suivi du corps contenant les détails d'implémentation
	\end{itemize}

	\begin{minted}{cpp}
		module foo;
		void bar(int j) { return 3 * j; }
	\end{minted}

	\begin{block}{Note}
		\begin{itemize}
			\item \textit{Implementation Unit} a accès aux déclarations non exportées du module
		\end{itemize}

\note[item]{Les déclarations non exportées sont visibles de l'ensemble du module}
	\end{block}

	\begin{alertblock}{Mais \ldots{}}
		\begin{itemize}
			\item Mais pas les autres unités de compilation même si elles importent le module
		\end{itemize}
	\end{alertblock}

	\addproposal{P1103}{https://wg21.link/P1103R3}
\end{frame}

\begin{frame}[fragile]
	\frametitle{Modules -- Partitions}
	\begin{itemize}
		\item Les modules peuvent être partitionnés sur plusieurs unités
		\item Les partitions fournissent alors un nom de partition
	\end{itemize}

	\begin{minted}{cpp}
		// Interface Unit
		export module foo:part;
	\end{minted}

	\begin{minted}{cpp}
		// Implementation Unit
		module foo:part;
	\end{minted}

	\begin{alertblock}{\textit{Primary Module Interface Unit}}
		\begin{itemize}
			\item Une et une seule \textit{Interface Unit} sans nom de partition par module
		\end{itemize}
	\end{alertblock}

	\begin{itemize}
		\item Un élément peut être déclaré dans une partition et défini dans une autre
	\end{itemize}

	\addproposal{P1103}{https://wg21.link/P1103R3}
\end{frame}

\begin{frame}[fragile]
	\frametitle{Modules -- Partitions}
	\begin{itemize}
		\item Les partitions sont un détail d'implémentation non visibles hors du module

\note[item]{Concrètement, hors du module on importe le module dans son ensemble}

		\item Une partition peut être importée dans une \textit{Implementation Unit}
		\item \ldots{} en important uniquement le nom de la partition
	\end{itemize}

	\begin{minted}{cpp}
		module foo;
		import :part;     // Importe foo:part
		import foo:part;  // Erreur
	\end{minted}

	\begin{itemize}
		\item Le \textit{Primary Module Interface Unit} peut exporter les partitions
	\end{itemize}

	\begin{minted}{cpp}
		export module foo;
		export :part1;
		export :part2;
	\end{minted}

	\addproposal{P1103}{https://wg21.link/P1103R3}
\end{frame}

\begin{frame}[fragile]
	\frametitle{Modules -- Export de namespace}
	\begin{itemize}
		\item Un namespace est exporté s'il est déclaré \mintinline{cpp}|export|
		\item \ldots{} ou implicitement si un de ses éléments est exporté
	\end{itemize}

	\begin{minted}{cpp}
		export namespace A {  // A est exporte
		  int n;              // A::n est exporte
		}

		namespace B {
		  export int n;       // B::n et B sont exportes
		  int m;              // B::m n'est pas exporte
		}
	\end{minted}

	\addproposal{P1103}{https://wg21.link/P1103R3}
\end{frame}

\begin{frame}[fragile]
	\frametitle{Modules -- Export de namespace}
	\begin{itemize}
		\item Les éléments d'une partie exportée d'un namespace sont exportés
	\end{itemize}

	\begin{minted}{cpp}
		namespace C { int n; }         // C::m est exporte

		export namespace C { int m; }  // mais pas C::n
	\end{minted}

	\addproposal{P1103}{https://wg21.link/P1103R3}
\end{frame}

\begin{frame}[fragile]
	\frametitle{Modules -- Implémentation inline}
	\begin{itemize}
		\item Interface et implémentation dans un unique fichier
		\item Implémentation dans un fragment \mintinline{cpp}|private|
	\end{itemize}

	\begin{minted}{cpp}
		export module m;
		struct s;
		export using s_ptr = s*;

		module :private;
		struct s {};
	\end{minted}

	\begin{alertblock}{Restriction}
		\begin{itemize}
			\item Uniquement dans une \textit{Primary Module Interface Unit}
			\item Qui doit être la seule unité du module
		\end{itemize}
	\end{alertblock}

	\addproposal{P1103}{https://wg21.link/P1103R3}
\end{frame}

\begin{frame}[fragile]
	\frametitle{Modules -- Utilisation}
	\begin{itemize}
		\item Import des modules via la directive \mintinline{cpp}|import|
	\end{itemize}

	\begin{minted}{cpp}
		import foo;

		// Utilisation des symboles exportes de foo
	\end{minted}

	\begin{itemize}
		\item Cohabitation possible avec des inclusions
	\end{itemize}

	\begin{minted}{cpp}
		#include <vector>
		import foo;
		#include "bar.h"
	\end{minted}

	\addproposal{P1103}{https://wg21.link/P1103R3}
\end{frame}

\begin{frame}[fragile]
	\frametitle{Modules -- Code non-modulaire}
	\begin{itemize}
		\item Inclusion d'en-têtes avant le préambule du module
	\end{itemize}

\note[item]{Seules des directives \mintinline{cpp}|include| peuvent apparaître}

	\begin{minted}{cpp}
		module;
		#include "bar.h"
		export module foo;
	\end{minted}

	\begin{itemize}
		\item Ou import des en-têtes
	\end{itemize}

	\begin{minted}{cpp}
		export module foo;
		import "bar.h";
		import <version>;
	\end{minted}

	\addproposal{P1103}{https://wg21.link/P1103R3}
\end{frame}

\begin{frame}[fragile]
	\frametitle{Modules -- Code non-modulaire}
	\begin{itemize}
		\item Export possible des symboles inclus
	\end{itemize}

	\begin{minted}{cpp}
		module;
		#include "bar.h" // Definit X
		export module foo;
		export using X = ::X;
	\end{minted}

	\begin{itemize}
		\item Ou de l'en-tête dans son ensemble
	\end{itemize}

	\begin{minted}{cpp}
		export module foo;
		export import "bar.h";
	\end{minted}

	\addproposal{P1103}{https://wg21.link/P1103R3}
\end{frame}

\subsection*{Conteneurs}
\begin{frame}[fragile]
	\frametitle{Chaînes de caractères}
	\begin{itemize}
		\item \mintinline{cpp}|std::basic_string::reserve()| ne peut plus réduire la capacité
		\begin{itemize}
			\item Appel avec une capacité inférieure sans effet

\note[item]{Auparavant, la norme permettait de réduire effectivement la capacité mais ne l'imposait pas (idem \mintinline{cpp}|shrink_to_fit()|)}
\note[item]{Si la capacité demandée était inférieur à la taille effective, cet appel était équivalent à un appel à \mintinline{cpp}|shrink_to_fit()|}

			\item Comportement similaire à \mintinline{cpp}|std::vector::reserve()|
		\end{itemize}
	\end{itemize}

	\begin{block}{Rappel}
		\begin{itemize}
			\item Après \mintinline{cpp}|reserve()|, la capacité est supérieure ou égale à la capacité demandée
		\end{itemize}
	\end{block}

	\begin{itemize}
		\item Dépréciation de \mintinline{cpp}|reserve()| sans paramètre
	\end{itemize}

	\begin{exampleblock}{Réduction à la capacité utile}
		\begin{itemize}
			\item Utilisez \mintinline{cpp}|shrink_to_fit()| et non \mintinline{cpp}|reserve()|
		\end{itemize}
	\end{exampleblock}

	\addproposal{P0966}{https://wg21.link/P0966R1}
\end{frame}

\begin{frame}[fragile]
	\frametitle{Chaînes de caractères}
	\begin{itemize}
		\item Ajout à \mintinline{cpp}|std::basic_string| et \mintinline{cpp}|std::string_view|
		\begin{itemize}
			\item \mintinline{cpp}|starts_with()| teste si la chaîne commence par une sous-chaîne
			\item \mintinline{cpp}|ends_with()| teste si la chaîne termine par une sous-chaîne
		\end{itemize}
	\end{itemize}

	\begin{minted}{cpp}
		string foo = "Hello world";

		foo.starts_with("Hello");   // true
		foo.ends_with("monde");     // false
	\end{minted}

	\begin{itemize}
		\item \mintinline{cpp}|std::string_view| constructible depuis une paire d'itérateurs
	\end{itemize}

	\begin{codesample}
		\sample{https://godbolt.org/#g:!((g:!((g:!((h:codeEditor,i:(filename:'1',fontScale:14,fontUsePx:'0',j:1,lang:c%2B%2B,selection:(endColumn:1,endLineNumber:1,positionColumn:1,positionLineNumber:1,selectionStartColumn:1,selectionStartLineNumber:1,startColumn:1,startLineNumber:1),source:'%23include+%3Cstring%3E%0A%23include+%3Ciostream%3E%0A%0Aint+main()%0A%7B%0A++std::string+foo+%3D+%22Hello+world%22%3B%0A%0A++std::cout+%3C%3C+std::boolalpha%3B%0A++std::cout+%3C%3C+foo.starts_with(%22Hello%22)+%3C%3C+%22%5Cn%22%3B%0A++std::cout+%3C%3C+foo.ends_with(%22monde%22)+%3C%3C+%22%5Cn%22%3B%0A%7D%0A'),l:'5',n:'0',o:'C%2B%2B+source+%231',t:'0')),k:50,l:'4',n:'0',o:'',s:0,t:'0'),(g:!((h:executor,i:(argsPanelShown:'1',compilationPanelShown:'0',compiler:gsnapshot,compilerName:'',compilerOutShown:'0',execArgs:'',execStdin:'',fontScale:14,fontUsePx:'0',j:1,lang:c%2B%2B,libs:!(),options:'-std%3Dc%2B%2B20+-Wall+-Wextra+-pedantic',overrides:!(),runtimeTools:!(),source:1,stdinPanelShown:'1',tree:'1',wrap:'0'),l:'5',n:'0',o:'Executor+x86-64+gcc+(trunk)+(C%2B%2B,+Editor+%231)',t:'0')),header:(),k:50,l:'4',n:'0',o:'',s:0,t:'0')),l:'2',n:'0',o:'',t:'0')),version:4}
	\end{codesample}

	\addproposal{P0457}{https://wg21.link/P0457R2}
\end{frame}

\begin{frame}[fragile]
	\frametitle{Conteneurs associatifs}
	\begin{itemize}
		\item \mintinline{cpp}|contains()| teste la présence d'une clé
	\end{itemize}

	\begin{minted}{cpp}
		map<int, string> foo{{1, "foo"}, {42, "bar"}};

		foo.contains(42);  // true
		foo.contains(38);  // false
	\end{minted}

	\begin{codesample}
		\sample{https://godbolt.org/#g:!((g:!((g:!((h:codeEditor,i:(filename:'1',fontScale:14,fontUsePx:'0',j:1,lang:c%2B%2B,selection:(endColumn:1,endLineNumber:1,positionColumn:1,positionLineNumber:1,selectionStartColumn:1,selectionStartLineNumber:1,startColumn:1,startLineNumber:1),source:'%23include+%3Cmap%3E%0A%23include+%3Cstring%3E%0A%23include+%3Ciostream%3E%0A%0Aint+main()%0A%7B%0A++std::map%3Cint,+std::string%3E+foo%7B%7B1,+%22foo%22%7D,+%7B42,+%22bar%22%7D%7D%3B%0A%0A++std::cout+%3C%3C+std::boolalpha%3B%0A++std::cout+%3C%3C+foo.contains(42)+%3C%3C+%22%5Cn%22%3B%0A++std::cout+%3C%3C+foo.contains(38)+%3C%3C+%22%5Cn%22%3B%0A%7D%0A'),l:'5',n:'0',o:'C%2B%2B+source+%231',t:'0')),k:50,l:'4',n:'0',o:'',s:0,t:'0'),(g:!((h:executor,i:(argsPanelShown:'1',compilationPanelShown:'0',compiler:gsnapshot,compilerName:'',compilerOutShown:'0',execArgs:'',execStdin:'',fontScale:14,fontUsePx:'0',j:1,lang:c%2B%2B,libs:!(),options:'-std%3Dc%2B%2B20+-Wall+-Wextra+-pedantic',overrides:!(),runtimeTools:!(),source:1,stdinPanelShown:'1',tree:'1',wrap:'0'),l:'5',n:'0',o:'Executor+x86-64+gcc+(trunk)+(C%2B%2B,+Editor+%231)',t:'0')),header:(),k:50,l:'4',n:'0',o:'',s:0,t:'0')),l:'2',n:'0',o:'',t:'0')),version:4}
	\end{codesample}

	\addproposal{P0458}{https://wg21.link/P0458R2}
\end{frame}

\begin{frame}[fragile]
	\frametitle{Conteneurs associatifs}
	\begin{itemize}
		\item Optimisation de la recherche hétérogène dans des conteneurs non-ordonnés
		\begin{itemize}
			\item Fourniture d'une classe exposant
			\begin{itemize}
				\item Différents foncteurs de calcul du hash
				\item Tag \mintinline{cpp}|transparent_key_equal| sur une comparaison transparente
			\end{itemize}
			\item Suppression de conversions inutiles
		\end{itemize}
	\end{itemize}

	\begin{minted}{cpp}
		struct string_hash {
		  using transparent_key_equal = equal_to<>;
		  size_t operator()(string_view txt) const {
		    return hash_type{}(txt); }
		  size_t operator()(const string& txt) const {
		    return hash_type{}(txt); }
		  size_t operator()(const char* txt) const {
		    return hash_type{}(txt); } };

		unordered_map<string, int, string_hash> foo = ...;
		foo.find("abc");
		foo.find("def"sv);
	\end{minted}

	\addproposal{P0919}{https://wg21.link/P0919R3}
	\addproposal{P1690}{https://wg21.link/P1690R1}
\end{frame}

\begin{frame}[fragile]
	\frametitle{\mintinline[style=white]{cpp}|std::list| et \mintinline[style=white]{cpp}|forward_list|}
	\begin{itemize}
		\item \mintinline{cpp}|remove()|, \mintinline{cpp}|remove_if()| et \mintinline{cpp}|unique()| retournent le nombre d'éléments supprimés
	\end{itemize}

	\begin{codesample}
		\sample{https://godbolt.org/#g:!((g:!((g:!((h:codeEditor,i:(filename:'1',fontScale:14,fontUsePx:'0',j:1,lang:c%2B%2B,selection:(endColumn:1,endLineNumber:1,positionColumn:1,positionLineNumber:1,selectionStartColumn:1,selectionStartLineNumber:1,startColumn:1,startLineNumber:1),source:'%23include+%3Ciostream%3E%0A%23include+%3Calgorithm%3E%0A%23include+%3Clist%3E%0A%0Aint+main()%0A%7B%0A++std::list%3Cint%3E+foo%7B1,+5,+12,+8,+13%7D%3B%0A%0A++std::cout+%3C%3C+foo.remove_if(%5B%5D+(int+i)+%7B+return+i+%3E+10%3B+%7D)+%3C%3C+%22%5Cn%22%3B%0A%7D%0A'),l:'5',n:'0',o:'C%2B%2B+source+%231',t:'0')),k:50,l:'4',n:'0',o:'',s:0,t:'0'),(g:!((h:executor,i:(argsPanelShown:'1',compilationPanelShown:'0',compiler:gsnapshot,compilerName:'',compilerOutShown:'0',execArgs:'',execStdin:'',fontScale:14,fontUsePx:'0',j:1,lang:c%2B%2B,libs:!(),options:'-std%3Dc%2B%2B20+-Wall+-Wextra+-pedantic',overrides:!(),runtimeTools:!(),source:1,stdinPanelShown:'1',tree:'1',wrap:'0'),l:'5',n:'0',o:'Executor+x86-64+gcc+(trunk)+(C%2B%2B,+Editor+%231)',t:'0')),header:(),k:50,l:'4',n:'0',o:'',s:0,t:'0')),l:'2',n:'0',o:'',t:'0')),version:4}
	\end{codesample}

	\addproposal{P0646}{https://wg21.link/P0646R1}
\end{frame}

\begin{frame}[fragile]
	\frametitle{\mintinline[style=white]{cpp}|std::array|}
	\begin{itemize}
		\item \mintinline{cpp}|std::to_array()| construit un \mintinline{cpp}|std::array| depuis un tableau C
	\end{itemize}

	\begin{minted}{cpp}
		auto foo = to_array({1, 2, 5, 42});

		long foo[] = {1, 2, 5, 42};
		auto bar = to_array(foo);

		auto foo = to_array<long>({1, 2, 5, 42});
	\end{minted}

	\begin{itemize}
		\item Y compris une chaîne C
	\end{itemize}

	\begin{minted}{cpp}
		auto foo = to_array("foo");
	\end{minted}

	\begin{alertblock}{\mintinline[style=white]{cpp}|0| terminal}
		\begin{itemize}
			\item Le \mintinline[escapeinside=||]{cpp}{|\textbackslash|0} terminal est un élément du tableau
		\end{itemize}
	\end{alertblock}

	\begin{codesample}
		\sample{https://godbolt.org/#g:!((g:!((g:!((h:codeEditor,i:(filename:'1',fontScale:14,fontUsePx:'0',j:1,lang:c%2B%2B,selection:(endColumn:1,endLineNumber:1,positionColumn:1,positionLineNumber:1,selectionStartColumn:1,selectionStartLineNumber:1,startColumn:1,startLineNumber:1),source:'%23include+%3Ciostream%3E%0A%23include+%3Carray%3E%0A%0Aint+main()%0A%7B%0A++auto+foo+%3D+std::to_array(%22foo%22)%3B%0A++std::cout+%3C%3C+foo.size()+%3C%3C+%22%5Cn%22%3B%0A++for(const+auto+c+:+foo)%0A++%7B%0A++++std::cout+%3C%3C+c+%3C%3C+%22+%22%3B%0A++%7D%0A++std::cout+%3C%3C+%22%5Cn%22%3B%0A%7D%0A'),l:'5',n:'0',o:'C%2B%2B+source+%231',t:'0')),k:50,l:'4',n:'0',o:'',s:0,t:'0'),(g:!((h:executor,i:(argsPanelShown:'1',compilationPanelShown:'0',compiler:gsnapshot,compilerName:'',compilerOutShown:'0',execArgs:'',execStdin:'',fontScale:14,fontUsePx:'0',j:1,lang:c%2B%2B,libs:!(),options:'-std%3Dc%2B%2B20+-Wall+-Wextra+-pedantic',overrides:!(),runtimeTools:!(),source:1,stdinPanelShown:'1',tree:'1',wrap:'0'),l:'5',n:'0',o:'Executor+x86-64+gcc+(trunk)+(C%2B%2B,+Editor+%231)',t:'0')),header:(),k:50,l:'4',n:'0',o:'',s:0,t:'0')),l:'2',n:'0',o:'',t:'0')),version:4}
	\end{codesample}

	\begin{codesample}
		\sample{https://godbolt.org/#g:!((g:!((g:!((h:codeEditor,i:(filename:'1',fontScale:14,fontUsePx:'0',j:1,lang:c%2B%2B,selection:(endColumn:1,endLineNumber:1,positionColumn:1,positionLineNumber:1,selectionStartColumn:1,selectionStartLineNumber:1,startColumn:1,startLineNumber:1),source:'%23include+%3Ciostream%3E%0A%23include+%3Carray%3E%0A%0Aint+main()%0A%7B%0A++%7B%0A++++auto+foo+%3D+std::to_array(%7B1,+2,+5,+42,+58%7D)%3B%0A++++std::cout+%3C%3C+foo.size()+%3C%3C+%22%5Cn%22%3B%0A++++for(const+auto+c+:+foo)%0A++++%7B%0A++++++std::cout+%3C%3C+c+%3C%3C+%22+%22%3B%0A++++%7D%0A++++std::cout+%3C%3C+%22%5Cn%22%3B%0A++%7D%0A%0A++%7B%0A++++long+foo%5B%5D+%3D+%7B1,+2,+5,+42,+33,+12%7D%3B%0A++++auto+bar+%3D+std::to_array(foo)%3B%0A++++std::cout+%3C%3C+bar.size()+%3C%3C+%22%5Cn%22%3B%0A++++for(const+auto+c+:+bar)%0A++++%7B%0A++++++std::cout+%3C%3C+c+%3C%3C+%22+%22%3B%0A++++%7D%0A++++std::cout+%3C%3C+%22%5Cn%22%3B%0A++%7D%0A%7D%0A'),l:'5',n:'0',o:'C%2B%2B+source+%231',t:'0')),k:50,l:'4',n:'0',o:'',s:0,t:'0'),(g:!((h:executor,i:(argsPanelShown:'1',compilationPanelShown:'0',compiler:gsnapshot,compilerName:'',compilerOutShown:'0',execArgs:'',execStdin:'',fontScale:14,fontUsePx:'0',j:1,lang:c%2B%2B,libs:!(),options:'-std%3Dc%2B%2B20+-Wall+-Wextra+-pedantic',overrides:!(),runtimeTools:!(),source:1,stdinPanelShown:'0',tree:'1',wrap:'0'),l:'5',n:'0',o:'Executor+x86-64+gcc+(trunk)+(C%2B%2B,+Editor+%231)',t:'0')),header:(),k:50,l:'4',n:'0',o:'',s:0,t:'0')),l:'2',n:'0',o:'',t:'0')),version:4}
	\end{codesample}

	\addproposal{P0325}{https://wg21.link/P0325R4}
\end{frame}

\begin{frame}[fragile]
	\frametitle{Suppression d'éléments}
	\begin{itemize}
		\item \mintinline{cpp}|std::erase()| supprime les éléments égaux à la valeur fournie
		\item \mintinline{cpp}|std::erase_if()| supprime les éléments satisfaisant le prédicat fourni
	\end{itemize}

	\begin{minted}{cpp}
		vector<int> foo {5, 12, 2, 56, 18, 33};

		erase_if(foo, [](int i) { return i > 20; }); // 5 12 2 18
	\end{minted}

	\begin{itemize}
		\item Remplacement de l'idiome \og \textit{Erase-remove}\fg{}
		\item Remplacement de la fonction membre \mintinline{cpp}|erase()|
	\end{itemize}

	\begin{codesample}
		\sample{https://godbolt.org/#g:!((g:!((g:!((h:codeEditor,i:(filename:'1',fontScale:14,fontUsePx:'0',j:1,lang:c%2B%2B,selection:(endColumn:1,endLineNumber:26,positionColumn:1,positionLineNumber:26,selectionStartColumn:1,selectionStartLineNumber:1,startColumn:1,startLineNumber:1),source:'%23include+%3Cvector%3E%0A%23include+%3Cmap%3E%0A%23include+%3Cstring%3E%0A%23include+%3Ciostream%3E%0A%0Aint+main()%0A%7B%0A++std::vector%3Cint%3E+foo+%7B5,+12,+2,+56,+18,+33%7D%3B%0A++std::erase_if(foo,+%5B%5D(int+i)+%7Breturn+i+%3E+20%3B%7D)%3B%0A%0A++for(int+i+:+foo)%0A++%7B%0A++++std::cout+%3C%3C+i+%3C%3C+%22+%22%3B%0A++%7D%0A++std::cout+%3C%3C+%22%5Cn%22%3B%0A%0A++std::map%3Cint,+std::string%3E+bar%7B%7B5,+%22a%22%7D,+%7B12,+%22b%22%7D,+%7B2,+%22c%22%7D,+%7B42,+%22d%22%7D%7D%3B%0A++std::erase_if(bar,+%5B%5D(std::pair%3Cint,+std::string%3E+i)+%7Breturn+i.first+%3E+20%3B%7D)%3B%0A%0A++for(auto+i+:+bar)%0A++%7B%0A++++std::cout+%3C%3C+i.first+%3C%3C+%22+:+%22+%3C%3C+i.second+%3C%3C+%22+%22%3B%0A++%7D%0A++std::cout+%3C%3C+%22%5Cn%22%3B%0A%7D%0A'),l:'5',n:'0',o:'C%2B%2B+source+%231',t:'0')),k:50,l:'4',n:'0',o:'',s:0,t:'0'),(g:!((h:executor,i:(argsPanelShown:'1',compilationPanelShown:'0',compiler:gsnapshot,compilerName:'',compilerOutShown:'0',execArgs:'',execStdin:'',fontScale:14,fontUsePx:'0',j:1,lang:c%2B%2B,libs:!(),options:'-std%3Dc%2B%2B20+-Wall+-Wextra+-pedantic',overrides:!(),runtimeTools:!(),source:1,stdinPanelShown:'1',tree:'1',wrap:'0'),l:'5',n:'0',o:'Executor+x86-64+gcc+(trunk)+(C%2B%2B,+Editor+%231)',t:'0')),header:(),k:50,l:'4',n:'0',o:'',s:0,t:'0')),l:'2',n:'0',o:'',t:'0')),version:4}
	\end{codesample}

	\addproposal{P1209}{https://wg21.link/P1209R0}
	\addproposal{P1115}{https://wg21.link/P1115R3}
\end{frame}

\begin{frame}[fragile]
	\frametitle{\mintinline[style=white]{cpp}|std::span|}
	\begin{itemize}
		\item Vue sur un conteneur contigu
		\item Similaire à \mintinline{cpp}|std::string_view|
		\item Constructible depuis
		\begin{itemize}
			\item Conteneur
			\item Couple début / taille
			\item Couple début / fin
			\item Range
			\item Autre \mintinline{cpp}|std::span|
		\end{itemize}
	\end{itemize}

	\begin{minted}{cpp}
		array<int, 5> foo = {0, 1, 2, 3, 4};
		span<int> s1{foo};
		span<int> s2(foo.data(), 3);
	\end{minted}

	\addproposal{P0122}{https://wg21.link/P0122R7}
	\addproposal{P1024}{https://wg21.link/P1024R3}
	\addproposal{P1227}{https://wg21.link/P1227R2}
\end{frame}

\begin{frame}[fragile]
	\frametitle{\mintinline[style=white]{cpp}|std::span|}
	\begin{itemize}
		\item \mintinline{cpp}|begin()|, \mintinline{cpp}|end()|, \ldots{} : itérateurs sur le \mintinline{cpp}|std::span|
		\item \mintinline{cpp}|size()|, \mintinline{cpp}|empty()| : taille et vacuité
		\item \mintinline{cpp}|operator[]|, \mintinline{cpp}|front()|, \mintinline{cpp}|back()| : accès à un élément
	\end{itemize}

	\begin{minted}{cpp}
		array<int, 5> foo = {0, 1, 2, 3, 4};
		span<int> bar{ foo.data(), 4 };

		bar.front();  // 0
	\end{minted}

	\begin{itemize}
		\item \mintinline{cpp}|first()|, \mintinline{cpp}|last()| : construction de \textit{sous-span}
	\end{itemize}

	\begin{minted}{cpp}
		span<int> baz = bar.first(2);  // 0, 1
	\end{minted}

	\begin{itemize}
		\item \textit{structured binding} sur des \mintinline{cpp}|std::span| de taille fixe
	\end{itemize}

	\begin{codesample}
		\sample{https://godbolt.org/#g:!((g:!((g:!((h:codeEditor,i:(filename:'1',fontScale:14,fontUsePx:'0',j:1,lang:c%2B%2B,selection:(endColumn:1,endLineNumber:1,positionColumn:1,positionLineNumber:1,selectionStartColumn:1,selectionStartLineNumber:1,startColumn:1,startLineNumber:1),source:'%23include+%3Carray%3E%0A%23include+%3Cspan%3E%0A%23include+%3Ciostream%3E%0A%0Aint+main()%0A%7B%0A++std::array%3Cint,+5%3E+foo+%3D+%7B7,+12,+28,+3,+9%7D%3B%0A++std::span%3Cint%3E+bar(foo.data(),+3)%3B%0A%0A++std::cout+%3C%3C+bar.size()+%3C%3C+%22%5Cn%22%3B%0A++std::cout+%3C%3C+bar.front()+%3C%3C+%22%5Cn%22%3B+%0A++std::cout+%3C%3C+bar%5B2%5D+%3C%3C+%22%5Cn%22%3B+%0A%0A++std::span%3Cint%3E+baz+%3D+bar.first(2)%3B%0A++for(const+auto+i+:+baz)%0A++%7B%0A++++std::cout+%3C%3C+i+%3C%3C+%22+%22%3B%0A++%7D%0A++std::cout+%3C%3C+%22%5Cn%22%3B%0A%7D%0A'),l:'5',n:'0',o:'C%2B%2B+source+%231',t:'0')),k:50,l:'4',n:'0',o:'',s:0,t:'0'),(g:!((h:executor,i:(argsPanelShown:'1',compilationPanelShown:'0',compiler:gsnapshot,compilerName:'',compilerOutShown:'0',execArgs:'',execStdin:'',fontScale:14,fontUsePx:'0',j:1,lang:c%2B%2B,libs:!(),options:'-std%3Dc%2B%2B20+-Wall+-Wextra+-pedantic',overrides:!(),runtimeTools:!(),source:1,stdinPanelShown:'1',tree:'1',wrap:'0'),l:'5',n:'0',o:'Executor+x86-64+gcc+(trunk)+(C%2B%2B,+Editor+%231)',t:'0')),header:(),k:50,l:'4',n:'0',o:'',s:0,t:'0')),l:'2',n:'0',o:'',t:'0')),version:4}
	\end{codesample}

	\addproposal{P0122}{https://wg21.link/P0122R7}
	\addproposal{P1024}{https://wg21.link/P1024R3}
	\addproposal{P1227}{https://wg21.link/P1227R2}
\end{frame}

\subsection*{Algorithmes}
\begin{frame}[fragile]
	\frametitle{Décalages d'éléments}
	\begin{itemize}
		\item \mintinline{cpp}|std::shift_left()| décale les éléments vers le début de l'ensemble
		\item \mintinline{cpp}|std::shift_right()| décale les éléments vers la fin de l'ensemble
		\item \ldots{} retournent un itérateur vers la fin (resp. début) du nouvel ensemble
	\end{itemize}

	\begin{block}{Taille et décalage}
		\begin{itemize}
			\item Opération sans effet si le décalage est supérieur la taille de l'ensemble
		\end{itemize}
	\end{block}

	\begin{minted}{cpp}
		vector<int> foo{5, 10, 15, 20};
		shift_left(foo.begin(), foo.end(), 2);   // 15, 20

		vector<int> bar{5, 10, 15, 20};
		shift_right(bar.begin(), bar.end(), 1);  // 5, 10, 15
	\end{minted}

	\begin{codesample}
		\sample{https://godbolt.org/#g:!((g:!((g:!((h:codeEditor,i:(filename:'1',fontScale:14,fontUsePx:'0',j:1,lang:c%2B%2B,selection:(endColumn:1,endLineNumber:1,positionColumn:1,positionLineNumber:1,selectionStartColumn:1,selectionStartLineNumber:1,startColumn:1,startLineNumber:1),source:'%23include+%3Cvector%3E%0A%23include+%3Calgorithm%3E%0A%23include+%3Ciostream%3E%0A%0Aint+main()%0A%7B%0A++std::vector%3Cint%3E+foo%7B5,+10,+15,+20%7D%3B%0A++std::shift_left(foo.begin(),+foo.end(),+2)%3B%0A++//+%7B15,+20,+%3F,+%3F%7D%0A++for(int+i+:+foo)%0A++%7B%0A++++std::cout+%3C%3C+i+%3C%3C+%22+%22%3B%0A++%7D%0A++std::cout+%3C%3C+%22%5Cn%22%3B%0A%0A++std::vector%3Cint%3E+bar%7B5,+10,+15,+20%7D%3B%0A++std::shift_right(bar.begin(),+bar.end(),+1)%3B%0A++//+%7B%3F,+5,+10,+15%7D%0A++for(int+i+:+bar)%0A++%7B%0A++++std::cout+%3C%3C+i+%3C%3C+%22+%22%3B%0A++%7D%0A++std::cout+%3C%3C+%22%5Cn%22%3B%0A%7D%0A'),l:'5',n:'0',o:'C%2B%2B+source+%231',t:'0')),k:50,l:'4',n:'0',o:'',s:0,t:'0'),(g:!((h:executor,i:(argsPanelShown:'1',compilationPanelShown:'0',compiler:gsnapshot,compilerName:'',compilerOutShown:'0',execArgs:'',execStdin:'',fontScale:14,fontUsePx:'0',j:1,lang:c%2B%2B,libs:!(),options:'-std%3Dc%2B%2B20+-Wall+-Wextra+-pedantic',overrides:!(),runtimeTools:!(),source:1,stdinPanelShown:'1',tree:'1',wrap:'0'),l:'5',n:'0',o:'Executor+x86-64+gcc+(trunk)+(C%2B%2B,+Editor+%231)',t:'0')),header:(),k:50,l:'4',n:'0',o:'',s:0,t:'0')),l:'2',n:'0',o:'',t:'0')),version:4}
	\end{codesample}

	\addproposal{P0769}{https://wg21.link/P0769R2}
\end{frame}

\begin{frame}[fragile]
	\frametitle{Manipulation de puissances de deux}
	\begin{itemize}
		\item \mintinline{cpp}|std::has_single_bit()| teste si un entier est une puissance de deux
		\item \mintinline{cpp}|std::bit_ceil()| plus petite puissance de deux non strictement inférieure
		\item \mintinline{cpp}|std::bit_floor()| plus grande puissance de deux non strictement supérieure
		\item \mintinline{cpp}|std::bit_width()| plus petit nombre de bits pour représenter un entier
	\end{itemize}

	\begin{minted}{cpp}
		has_single_bit(4u);  // true
		has_single_bit(7u);  // false
		bit_ceil(7u);        // 8
		bit_ceil(8u);        // 8
		bit_floor(7u);       // 4
		bit_width(7u);       // 3
	\end{minted}

	\begin{alertblock}{Restriction}
		\begin{itemize}
			\item Uniquement sur des entiers non signés
		\end{itemize}
	\end{alertblock}

	\begin{codesample}
		\sample{https://godbolt.org/#g:!((g:!((g:!((h:codeEditor,i:(filename:'1',fontScale:14,fontUsePx:'0',j:1,lang:c%2B%2B,selection:(endColumn:1,endLineNumber:1,positionColumn:1,positionLineNumber:1,selectionStartColumn:1,selectionStartLineNumber:1,startColumn:1,startLineNumber:1),source:'%23include+%3Ciostream%3E%0A%23include+%3Cbit%3E%0A%0Aint+main()%0A%7B%0A++std::cout+%3C%3C+std::has_single_bit(4u)+%3C%3C+%22%5Cn%22%3B%0A++std::cout+%3C%3C+std::has_single_bit(7u)+%3C%3C+%22%5Cn%22%3B%0A++std::cout+%3C%3C+std::bit_ceil(7u)++%3C%3C+%22%5Cn%22%3B%0A++std::cout+%3C%3C+std::bit_ceil(8u)++%3C%3C+%22%5Cn%22%3B%0A++std::cout+%3C%3C+std::bit_floor(7u)+%3C%3C+%22%5Cn%22%3B%0A++std::cout+%3C%3C+std::bit_width(7u)+%3C%3C+%22%5Cn%22%3B%0A%7D%0A'),l:'5',n:'0',o:'C%2B%2B+source+%231',t:'0')),k:50,l:'4',n:'0',o:'',s:0,t:'0'),(g:!((h:executor,i:(argsPanelShown:'1',compilationPanelShown:'0',compiler:gsnapshot,compilerName:'',compilerOutShown:'0',execArgs:'',execStdin:'',fontScale:14,fontUsePx:'0',j:1,lang:c%2B%2B,libs:!(),options:'-std%3Dc%2B%2B20+-Wall+-Wextra+-pedantic',overrides:!(),runtimeTools:!(),source:1,stdinPanelShown:'1',tree:'1',wrap:'0'),l:'5',n:'0',o:'Executor+x86-64+gcc+(trunk)+(C%2B%2B,+Editor+%231)',t:'0')),header:(),k:50,l:'4',n:'0',o:'',s:0,t:'0')),l:'2',n:'0',o:'',t:'0')),version:4}
	\end{codesample}

	\addproposal{P0556}{https://wg21.link/P0556R3}
	\addproposal{P1956}{https://wg21.link/P1956R1}
\end{frame}

\begin{frame}[fragile]
	\frametitle{Manipulation binaire}
	\begin{itemize}
		\item \mintinline{cpp}|std::rotl()| et \mintinline{cpp}|std::rotr()| rotations binaires
		\item \mintinline{cpp}|std::countl_zero| nombre consécutif de bits à zéro depuis le plus significatif
		\item \mintinline{cpp}|std::countl_one| nombre consécutif de bits à un depuis le plus significatif
		\item \mintinline{cpp}|std::countr_zero| nombre consécutif de bits à zéro depuis le moins significatif
		\item \mintinline{cpp}|std::countr_one| nombre consécutif de bits à un depuis le moins significatif
		\item \mintinline{cpp}|std::popcount| nombre de bit à un
	\end{itemize}

	\begin{minted}{cpp}
		rotl(6u, 2);   // 24
		rotr(6u, 1);   // 3
		popcount(6u);  // 2
	\end{minted}

	\begin{alertblock}{Restriction}
		\begin{itemize}
			\item Uniquement sur des entiers non signés
		\end{itemize}
	\end{alertblock}

	\begin{codesample}
		\sample{https://godbolt.org/#g:!((g:!((g:!((h:codeEditor,i:(filename:'1',fontScale:14,fontUsePx:'0',j:1,lang:c%2B%2B,selection:(endColumn:1,endLineNumber:1,positionColumn:1,positionLineNumber:1,selectionStartColumn:1,selectionStartLineNumber:1,startColumn:1,startLineNumber:1),source:'%23include+%3Ciostream%3E%0A%23include+%3Cbit%3E%0A%0Aint+main()%0A%7B%0A++std::cout+%3C%3C+std::rotl(6u,+2)+%3C%3C+%22%5Cn%22%3B%0A++std::cout+%3C%3C+std::rotr(6u,+1)+%3C%3C+%22%5Cn%22%3B%0A++std::cout+%3C%3C+std::popcount(6u)+%3C%3C+%22%5Cn%22%3B%0A%7D%0A'),l:'5',n:'0',o:'C%2B%2B+source+%231',t:'0')),k:50,l:'4',n:'0',o:'',s:0,t:'0'),(g:!((h:executor,i:(argsPanelShown:'1',compilationPanelShown:'0',compiler:gsnapshot,compilerName:'',compilerOutShown:'0',execArgs:'',execStdin:'',fontScale:14,fontUsePx:'0',j:1,lang:c%2B%2B,libs:!(),options:'-std%3Dc%2B%2B20+-Wall+-Wextra+-pedantic',overrides:!(),runtimeTools:!(),source:1,stdinPanelShown:'1',tree:'1',wrap:'0'),l:'5',n:'0',o:'Executor+x86-64+gcc+(trunk)+(C%2B%2B,+Editor+%231)',t:'0')),header:(),k:50,l:'4',n:'0',o:'',s:0,t:'0')),l:'2',n:'0',o:'',t:'0')),version:4}
	\end{codesample}

	\addproposal{P0553}{https://wg21.link/P0553R4}
\end{frame}

\begin{frame}[fragile]
	\frametitle{Conversion binaire}
	\begin{itemize}
		\item \mintinline{cpp}|std::bit_cast| ré-interprète une représentation binaire en un autre type
		\begin{itemize}
			\item Conversion bit-à-bit
			\item Alternative plus sûre à \mintinline{cpp}|reinterpret_cast| ou \mintinline{cpp}|memcpy()|
			\item Conversion \mintinline{cpp}|constexpr| si possible
		\end{itemize}
	\end{itemize}

	\begin{alertblock}{Restriction}
		\begin{itemize}
			\item Uniquement sur des types \textit{trivially copyable}
		\end{itemize}
	\end{alertblock}

	\addproposal{P0476}{https://wg21.link/P0476R2}
\end{frame}

\begin{frame}[fragile]
	\frametitle{Comparaison d'entiers}
	\begin{itemize}
		\item Ajout de fonctions de comparaison d'entier : \mintinline{cpp}|std::cmp_equal()|, \mintinline{cpp}|std::cmp_not_equal()|, \mintinline{cpp}|std::cmp_less()|, \mintinline{cpp}|std::cmp_greated()|, \mintinline{cpp}|std::cmp_less_equal()| et \mintinline{cpp}|std::cmp_greater_equal()|
		\item Permettent la comparaison signé / non signé sans promotion
	\end{itemize}

	\begin{codesample}
		\sample{https://godbolt.org/#g:!((g:!((g:!((h:codeEditor,i:(filename:'1',fontScale:14,fontUsePx:'0',j:1,lang:c%2B%2B,selection:(endColumn:1,endLineNumber:1,positionColumn:1,positionLineNumber:1,selectionStartColumn:1,selectionStartLineNumber:1,startColumn:1,startLineNumber:1),source:'%23include+%3Ciostream%3E%0A%23include+%3Cutility%3E%0A%0Aint+main()%0A%7B%0A++std::cout+%3C%3C+std::boolalpha%3B%0A++std::cout+%3C%3C+(-1+%3E+1U)+%3C%3C+%22%5Cn%22%3B%0A++std::cout+%3C%3C+std::cmp_greater(-1,+1U)+%3C%3C+%22%5Cn%22%3B%0A%7D%0A'),l:'5',n:'0',o:'C%2B%2B+source+%231',t:'0')),k:50,l:'4',n:'0',o:'',s:0,t:'0'),(g:!((h:executor,i:(argsPanelShown:'1',compilationPanelShown:'0',compiler:gsnapshot,compilerName:'',compilerOutShown:'0',execArgs:'',execStdin:'',fontScale:14,fontUsePx:'0',j:1,lang:c%2B%2B,libs:!(),options:'-std%3Dc%2B%2B20+-Wall+-Wextra+-pedantic',overrides:!(),runtimeTools:!(),source:1,stdinPanelShown:'1',tree:'1',wrap:'0'),l:'5',n:'0',o:'Executor+x86-64+gcc+(trunk)+(C%2B%2B,+Editor+%231)',t:'0')),header:(),k:50,l:'4',n:'0',o:'',s:0,t:'0')),l:'2',n:'0',o:'',t:'0')),version:4}
	\end{codesample}

	\addproposal{P0586}{https://wg21.link/P0586R2}
\end{frame}

\begin{frame}[fragile]
	\frametitle{Mathématiques}
	\begin{itemize}
		\item Définition des constantes mathématiques $e$, $\log_2 e$, $\log_{10} e$, $\pi$, $\dfrac{1}{\pi}$, $\dfrac{1}{\sqrt{\pi}}$, $\ln{2}$, $\ln{10}$, $\sqrt{2}$, $\sqrt{3}$, $\dfrac{1}{\sqrt{3}}$, $\gamma$, $\varphi$

\note[item]{$\gamma$ est la constante de Euler-Mascheroni et $\varphi$ le nombre d'or}

		\item \mintinline{cpp}|std::midpoint()| : demi-somme de deux valeurs (entières ou flottantes)
	\end{itemize}

	\begin{block}{Règle d'arrondi}
		\begin{itemize}
			\item La demi-somme d'entiers est entière et arrondie vers le premier paramètre
		\end{itemize}

		\begin{minted}{cpp}
			midpoint(2, 4);  // 3
			midpoint(2, 5);  // 3
			midpoint(5, 2);  // 4
		\end{minted}
	\end{block}

	\begin{codesample}
		\sample{https://godbolt.org/#g:!((g:!((g:!((h:codeEditor,i:(filename:'1',fontScale:14,fontUsePx:'0',j:1,lang:c%2B%2B,selection:(endColumn:1,endLineNumber:1,positionColumn:1,positionLineNumber:1,selectionStartColumn:1,selectionStartLineNumber:1,startColumn:1,startLineNumber:1),source:'%23include+%3Ciostream%3E%0A%23include+%3Cnumeric%3E%0A%0Aint+main()%0A%7B%0A++std::cout+%3C%3C+std::midpoint(2,+4)+%3C%3C+%22%5Cn%22%3B%0A++std::cout+%3C%3C+std::midpoint(2,+5)+%3C%3C+%22%5Cn%22%3B%0A++std::cout+%3C%3C+std::midpoint(5,+2)+%3C%3C+%22%5Cn%22%3B+%0A%7D%0A'),l:'5',n:'0',o:'C%2B%2B+source+%231',t:'0')),k:50,l:'4',n:'0',o:'',s:0,t:'0'),(g:!((h:executor,i:(argsPanelShown:'1',compilationPanelShown:'0',compiler:gsnapshot,compilerName:'',compilerOutShown:'0',execArgs:'',execStdin:'',fontScale:14,fontUsePx:'0',j:1,lang:c%2B%2B,libs:!(),options:'-std%3Dc%2B%2B20+-Wall+-Wextra+-pedantic',overrides:!(),runtimeTools:!(),source:1,stdinPanelShown:'1',tree:'1',wrap:'0'),l:'5',n:'0',o:'Executor+x86-64+gcc+(trunk)+(C%2B%2B,+Editor+%231)',t:'0')),header:(),k:50,l:'4',n:'0',o:'',s:0,t:'0')),l:'2',n:'0',o:'',t:'0')),version:4}
	\end{codesample}

	\addproposal{P0631}{https://wg21.link/P0631R8}
	\addproposal{P0811}{https://wg21.link/P0811R3}
\end{frame}

\begin{frame}[fragile]
\frametitle{Mathématiques}
	\begin{itemize}
		\item \mintinline{cpp}|std::lerp()| : interpolation linéaire entre deux valeurs flottantes
	\end{itemize}

	\begin{minted}{cpp}
		lerp(10, 20, 0);    // 10
		lerp(10, 20, 0.1);  // 11
		lerp(10, 20, 0.2);  // 12
		lerp(10, 20, 0.3);  // 13
		lerp(10, 20, 0.4);  // 14
		lerp(10, 20, 0.5);  // 15
		lerp(10, 20, 0.6);  // 16
		lerp(10, 20, 0.7);  // 17
		lerp(10, 20, 0.8);  // 18
		lerp(10, 20, 0.9);  // 19
		lerp(10, 20, 1);    // 20
	\end{minted}

	\begin{codesample}
		\sample{https://godbolt.org/#z:OYLghAFBqd5QCxAYwPYBMCmBRdBLAF1QCcAaPECAMzwBtMA7AQwFtMQByARg9KtQYEAysib0QXACx8BBAKoBnTAAUAHpwAMvAFYTStJg1DIApACYAQuYukl9ZATwDKjdAGFUtAK4sGe1wAyeAyYAHI%2BAEaYxHoADqgKhE4MHt6%2BcQlJAkEh4SxRMVy2mPaOAkIETMQEqT5%2BRXaYDskVVQQ5YZHRegqV1bXpDX3twZ353VwAlLaoXsTI7BzmAMzByN5YANQmy25OvcSYrDvYJhoAgitrG5jbu8gsTAQIJ2eXF8EEm4/BEJNvJgA7FYLptNr10CAULMvjs3HDwQRISB6MRYhAuBpSJszFjNhpJnd4bttmYzCYAKxuBjmcnLEHnMEQqFoLyw3YI5ko6LozHY3HYjQAOimRIRtMp1Il9LeTKRLJhYpJXNRvLxAvxQrMhLh4rJkpp%2BploMRyNZ7OJblNUNVGPVeOFyx1HJJEqphrpDLlZsVuuV8u5aLt/IdQskzstpPJ7ulXut0LZSqtKp5wZxoYpEb10alRrjXPNSfjtr56cFQoAbFnXfqY3nZfHC37kwGS/by4Dq1a3bnPQ2C76XS3kW2Q%2BWABxdqMG2P9gNNofF1OljXCgCcU57Hp2%2Bfng8jKaDK7xoub07rfYuQIAIm8ONNaJwKbw/BwtKRUJx4ZZrODZvNbhWHhSAITR72mABrEAKSxR8OEkF8wI/TheAUEAsVAt971IOBYCQTBVCaNkSHISgqmABRlEMEohAQVAAHdX2AtAWFiOgnmSKiQloWiGNfd8WLY1EQAo5hYgUOiCFIQS6GiUJWEWXgZNRAB5NleMYpCCKac5iAolDSG05AKnwV9eH4QQRDEdgpBkQRFBUdQsNIXQigMIwUGsax9DwCI0MgaZUFiMoGDQjgAFoIR2a9TB/SxcU2cKAHUxFoRKkoIghiCYRLYkwdBDEcZBeFQAA3aJiDwLB/L%2BUhiC8QQ8DYAAVVBPBq6YFH/BYeiRYIuJoujNO4XgsswRZgPo7LYjAh8n0Q5zPw4bBCOQYjiE2VRxwrcKK0kTZgGQZBNggLKGogwkIG/KxLGxXBCBIUllimUbZumBAjiwGJaqgmD9E4BDSH4kqDLQjC3v%2BjgzAW98lterDJmmcriESZxJCAA%3D}
	\end{codesample}

	\addproposal{P0811}{https://wg21.link/P0811R3}
\end{frame}

\begin{frame}[fragile]
	\frametitle{Évolutions de la bibliothèque standard}
	\begin{itemize}
		\item Utilisation de l'attribut \mintinline{cpp}|[[ nodiscard ]]|
		\item Davantage de \mintinline{cpp}|noexcept|
		\item Optimisation d'algorithmes numériques via \mintinline{cpp}|std::move()|
	\end{itemize}
\end{frame}

\subsection*{Ranges}
\begin{frame}
	\frametitle{Ranges -- Présentation}
	\begin{itemize}
		\item Abstraction de plus haut niveau que les itérateurs
		\item Manipulation d'ensemble au travers d'algorithmes et de \textit{range adaptators}
		\item Vivent dans le \mintinline{cpp}|namespace std::ranges|

\note[item]{Pour être précis, les \textit{range adaptors} manipulent seulement les \textit{viewable ranges}}
	\end{itemize}

	\begin{block}{Pour aller plus loin}
		\begin{itemize}
			\item \href{https://accu.org/content/conf2009/AndreiAlexandrescu_iterators-must-go.pdf}{Iterators Must Go\linklogo (Andrei Alexandrescu)}
			\item \href{http://ericniebler.com/}{Blog d'Eric Niebler\linklogo}
		\end{itemize}
	\end{block}

	\addproposal{P0896}{https://wg21.link/P0896R4}
	\addproposal{P1035}{https://wg21.link/P1035R7}
\end{frame}

\begin{frame}[fragile]
	\frametitle{Ranges -- Concepts}
	\begin{itemize}
		\item \mintinline{cpp}|Range|
		\begin{itemize}
			\item Abstraction pour manipuler une séquence d'éléments
			\item Itérateur de début
			\item Sentinelle de fin
			\begin{itemize}
				\item Itérateur
				\item Valeur particulière
				\item \mintinline{cpp}|std::default_sentinel_t| : itérateurs gérant la limite du range

\note[item]{\mintinline{cpp}|std::default_sentinel_t| utilisable avec \mintinline{cpp}|std::counted_iterator|}
\note[item]{Appelé \textit{counted range}, range représenté par un itérateur de début et un nombre d'éléments}
			\end{itemize}
		\end{itemize}

		\item Conteneur : \mintinline{cpp}|range| possédant ses éléments
		\item \mintinline{cpp}|View|
		\begin{itemize}
			\item \mintinline{cpp}|range| ne possédant pas les éléments pointés par \mintinline{cpp}|begin()| et \mintinline{cpp}|end()|
			\item Copie, déplacement et affectation en temps constant
		\end{itemize}
		\item \mintinline{cpp}|SizedRange| : taille en temps constant
		\item \mintinline{cpp}|ViewableRange| : \mintinline{cpp}|range| convertible en \mintinline{cpp}|View|
		\item \mintinline{cpp}|CommonRange| : itérateur et sentinelle de même type
	\end{itemize}

	\addproposal{P0896}{https://wg21.link/P0896R4}
	\addproposal{P1035}{https://wg21.link/P1035R7}
\end{frame}

\begin{frame}[fragile]
	\frametitle{Ranges -- Concepts}
	\begin{itemize}
		\item \mintinline{cpp}|InputRange| : fournit des \mintinline{cpp}|input_iterator|
		\item \mintinline{cpp}|OutputRange| : fournit des \mintinline{cpp}|output_iterator|
		\item \mintinline{cpp}|ForwardRange| : fournit \mintinline{cpp}|forward_iterator|
		\item \mintinline{cpp}|BidirectionalRange| : fournit \mintinline{cpp}|bidirectional_iterator|
		\item \mintinline{cpp}|RandomAccessRange| : fournit \mintinline{cpp}|random_access_iterator|
		\item \mintinline{cpp}|ContiguousRange| : fournit \mintinline{cpp}|contiguous_iterator|
	\end{itemize}

	\begin{block}{En résumé}
		\begin{itemize}
			\item Conteneurs : possession, copie profonde
			\item Vues : référence, copie superficielle
		\end{itemize}
	\end{block}

	\addproposal{P0896}{https://wg21.link/P0896R4}
	\addproposal{P1035}{https://wg21.link/P1035R7}
\end{frame}

\begin{frame}[fragile]
	\frametitle{Ranges -- Itérateurs}
	\begin{itemize}
		\item \mintinline{cpp}|std::common_iterator| : adaptateur d'itérateurs/sentinelles permettant de représenter un \textit{non-common} \mintinline{cpp}|range| comme un \mintinline{cpp}|CommonRange|

\note[item]{Passe de itérateur/sentinelle de types différents à itérateur/sentinelle de même type}
\note[item]{Grâce à l'opérateur de comparaison adéquate}

		\item \mintinline{cpp}|std::counted_iterator| : adaptateur d'itérateurs reprenant le fonctionnement de l'itérateur sous-jacent mais conservant la distance à la fin du \mintinline{cpp}|range|
	\end{itemize}

	\addproposal{P0896}{https://wg21.link/P0896R4}
	\addproposal{P1035}{https://wg21.link/P1035R7}
\end{frame}

\begin{frame}
	\frametitle{Ranges -- Opérations}
	\begin{itemize}
		\item \mintinline{cpp}|begin()|, \mintinline{cpp}|end()|, \mintinline{cpp}|cbegin()|, \mintinline{cpp}|cend()|, \ldots{} retournent itérateurs et sentinelles
		\item \mintinline{cpp}|size()| retourne la taille du \mintinline{cpp}|range|
		\item \mintinline{cpp}|empty()| teste la vacuité
		\item \mintinline{cpp}|data()| et \mintinline{cpp}|cdata()| retournent l'adresse de début du \mintinline{cpp}|range|
	\end{itemize}

	\begin{alertblock}{Restrictions}
		\begin{itemize}
			\item \mintinline{cpp}|data()| et \mintinline{cpp}|cdata()| uniquement sur des \mintinline{cpp}|ContiguousRange|
		\end{itemize}
	\end{alertblock}

	\begin{itemize}
		\item Surcharges de différents algorithmes prenant un \mintinline{cpp}|range| en paramètre
	\end{itemize}

	\addproposal{P0896}{https://wg21.link/P0896R4}
	\addproposal{P1035}{https://wg21.link/P1035R7}
\end{frame}

\begin{frame}[fragile]
	\frametitle{Ranges -- Factory}
	\begin{itemize}
		\item \mintinline{cpp}|std::views::empty| crée une vue vide
		\item \mintinline{cpp}|std::views::single| crée une vue d'un unique élément
		\item \mintinline{cpp}|std::views::iota| crée une vue en incrémentant une valeur initiale
	\end{itemize}

	\begin{minted}{cpp}
		for(int i : views::iota(1, 10))
		  cout << i << ' ';   // 1 2 3 4 5 6 7 8 9
  	\end{minted}

	\begin{itemize}
		\item \mintinline{cpp}|std::views::counted| crée un \mintinline{cpp}|range| depuis un itérateur et un nombre d'éléments
	\end{itemize}

	\begin{minted}{cpp}
		int a[] = {1, 2, 3, 4, 5, 6, 7, 8, 9, 10};
		for(int i : views::counted(a, 3))
		  cout << i << ' ';   // 1 2 3
  	\end{minted}

	\begin{codesample}
		\sample{https://godbolt.org/#g:!((g:!((g:!((h:codeEditor,i:(filename:'1',fontScale:14,fontUsePx:'0',j:1,lang:c%2B%2B,selection:(endColumn:1,endLineNumber:1,positionColumn:1,positionLineNumber:1,selectionStartColumn:1,selectionStartLineNumber:1,startColumn:1,startLineNumber:1),source:'%23include+%3Cranges%3E%0A%23include+%3Ciostream%3E%0A%0Aint+main()%0A%7B%0A++int+a%5B%5D+%3D+%7B1,+2,+3,+4,+5,+6,+7,+8,+9,+10%7D%3B%0A++for(int+i+:+std::views::counted(a,+3))%0A++%7B%0A++++std::cout+%3C%3C+i+%3C%3C+!'+!'%3B%0A++%7D%0A%7D%0A'),l:'5',n:'0',o:'C%2B%2B+source+%231',t:'0')),k:50,l:'4',n:'0',o:'',s:0,t:'0'),(g:!((h:executor,i:(argsPanelShown:'1',compilationPanelShown:'0',compiler:gsnapshot,compilerName:'',compilerOutShown:'0',execArgs:'',execStdin:'',fontScale:14,fontUsePx:'0',j:1,lang:c%2B%2B,libs:!(),options:'-std%3Dc%2B%2B20+-Wall+-Wextra+-pedantic',overrides:!(),runtimeTools:!(),source:1,stdinPanelShown:'1',tree:'1',wrap:'0'),l:'5',n:'0',o:'Executor+x86-64+gcc+(trunk)+(C%2B%2B,+Editor+%231)',t:'0')),header:(),k:50,l:'4',n:'0',o:'',s:0,t:'0')),l:'2',n:'0',o:'',t:'0')),version:4}
	\end{codesample}

	\begin{codesample}
		\sample{https://godbolt.org/#g:!((g:!((g:!((h:codeEditor,i:(filename:'1',fontScale:14,fontUsePx:'0',j:1,lang:c%2B%2B,selection:(endColumn:1,endLineNumber:1,positionColumn:1,positionLineNumber:1,selectionStartColumn:1,selectionStartLineNumber:1,startColumn:1,startLineNumber:1),source:'%23include+%3Cranges%3E%0A%23include+%3Ciostream%3E%0A%0Aint+main()%0A%7B%0A++for(int+i+:+std::views::iota(1,+10))%0A++%7B%0A++++std::cout+%3C%3C+i+%3C%3C+!'+!'%3B%0A++%7D%0A%7D%0A'),l:'5',n:'0',o:'C%2B%2B+source+%231',t:'0')),k:50,l:'4',n:'0',o:'',s:0,t:'0'),(g:!((h:executor,i:(argsPanelShown:'1',compilationPanelShown:'0',compiler:gsnapshot,compilerName:'',compilerOutShown:'0',execArgs:'',execStdin:'',fontScale:14,fontUsePx:'0',j:1,lang:c%2B%2B,libs:!(),options:'-std%3Dc%2B%2B20+-Wall+-Wextra+-pedantic',overrides:!(),runtimeTools:!(),source:1,stdinPanelShown:'1',tree:'1',wrap:'0'),l:'5',n:'0',o:'Executor+x86-64+gcc+(trunk)+(C%2B%2B,+Editor+%231)',t:'0')),header:(),k:50,l:'4',n:'0',o:'',s:0,t:'0')),l:'2',n:'0',o:'',t:'0')),version:4}
	\end{codesample}

	\addproposal{P0896}{https://wg21.link/P0896R4}
	\addproposal{P1035}{https://wg21.link/P1035R7}
\end{frame}

\begin{frame}[fragile]
	\frametitle{Ranges -- Range adaptators}
	\begin{itemize}
		\item Appliquent filtres et transformations aux \mintinline{cpp}|range|
		\item Associés, pour certains, à un \textit{range adaptor closure object}
		\begin{itemize}
			\item Prends un unique paramètre \mintinline{cpp}|viewable_range|
			\item Retourne une \mintinline{cpp}|view|
		\end{itemize}
		\item Évaluation paresseuse des \mintinline{cpp}|view|
	\end{itemize}

	\addproposal{P0896}{https://wg21.link/P0896R4}
	\addproposal{P1035}{https://wg21.link/P1035R7}
\end{frame}

\begin{frame}[fragile]
	\frametitle{Ranges -- Range adaptators}
	\begin{itemize}
		\item Peuvent être chaînés avec une syntaxe \og appel de fonction\fg{}
	\end{itemize}

	\begin{minted}{cpp}
		D(C(R));
	\end{minted}

	\begin{itemize}
		\item Ou une syntaxe \og pipeline\fg{}
	\end{itemize}

	\begin{minted}{cpp}
		R | C | D;
	\end{minted}

	\begin{itemize}
		\item Peuvent prendre plusieurs arguments
	\end{itemize}

	\begin{minted}{cpp}
		adaptor(range, args...);
		adaptor(args...)(range);
		range | adaptor(args...);
	\end{minted}

\note[item]{\mintinline{cpp}|adaptor(args...)| est forcément une \textit{range adaptor closure object}}

	\addproposal{P0896}{https://wg21.link/P0896R4}
	\addproposal{P1035}{https://wg21.link/P1035R7}
\end{frame}

\begin{frame}[fragile]
	\frametitle{Ranges -- Range adaptators}
	\begin{itemize}
		\item Plusieurs adaptateurs fournis par la bibliothèque standard

\note[item]{Les adapteurs \mintinline{cpp}|XXX_view| ont des \textit{closure object} de la forme \mintinline{cpp}|views::XXX|}

		\begin{itemize}
			\item \mintinline{cpp}|all_view| : tous les éléments du \mintinline{cpp}|range|
			\item \mintinline{cpp}|ref_view| : références sur les éléments du \mintinline{cpp}|range|
			\item \mintinline{cpp}|filter_view| : tous les éléments satisfaisants un prédicat
		\end{itemize}
	\end{itemize}

	\begin{minted}{cpp}
		vector<int> ints{0, 1, 2, 3, 4, 5};
		auto even = [](int i){ return (i % 2) == 0; };

		ints | views::filter(even);       // 0, 2, 4
	\end{minted}

	\begin{itemize}
		\item [] \begin{itemize}
			\item \mintinline{cpp}|transform_view| : les éléments transformés par l'application d'une fonction
		\end{itemize}
	\end{itemize}

	\begin{minted}{cpp}
		vector<int> ints{0, 1, 2, 3, 4, 5};
		auto foo = [](int i){ return 2 * i; };

		ints | views::transform(foo);  // 0, 2, 4, 6, 8, 10
	\end{minted}

	\begin{codesample}
		\sample{https://godbolt.org/#g:!((g:!((g:!((h:codeEditor,i:(filename:'1',fontScale:14,fontUsePx:'0',j:1,lang:c%2B%2B,selection:(endColumn:1,endLineNumber:1,positionColumn:1,positionLineNumber:1,selectionStartColumn:1,selectionStartLineNumber:1,startColumn:1,startLineNumber:1),source:'%23include+%3Ciostream%3E%0A%23include+%3Cvector%3E%0A%23include+%3Cranges%3E%0A%0Aint+main()%0A%7B%0A++std::vector%3Cint%3E+ints%7B0,+1,+2,+3,+4,+5%7D%3B%0A++auto+foo+%3D+%5B%5D(int+i)%7B+return+2+*+i%3B+%7D%3B%0A%0A++for(int+i+:+ints+%7C+std::views::transform(foo))%0A++%7B%0A++++std::cout+%3C%3C+i+%3C%3C+!'+!'%3B%0A++%7D%0A%7D%0A'),l:'5',n:'0',o:'C%2B%2B+source+%231',t:'0')),k:50,l:'4',n:'0',o:'',s:0,t:'0'),(g:!((h:executor,i:(argsPanelShown:'1',compilationPanelShown:'0',compiler:gsnapshot,compilerName:'',compilerOutShown:'0',execArgs:'',execStdin:'',fontScale:14,fontUsePx:'0',j:1,lang:c%2B%2B,libs:!(),options:'-std%3Dc%2B%2B20+-Wall+-Wextra+-pedantic',overrides:!(),runtimeTools:!(),source:1,stdinPanelShown:'1',tree:'1',wrap:'0'),l:'5',n:'0',o:'Executor+x86-64+gcc+(trunk)+(C%2B%2B,+Editor+%231)',t:'0')),header:(),k:50,l:'4',n:'0',o:'',s:0,t:'0')),l:'2',n:'0',o:'',t:'0')),version:4}
	\end{codesample}

	\begin{codesample}
		\sample{https://godbolt.org/#g:!((g:!((g:!((h:codeEditor,i:(filename:'1',fontScale:14,fontUsePx:'0',j:1,lang:c%2B%2B,selection:(endColumn:1,endLineNumber:1,positionColumn:1,positionLineNumber:1,selectionStartColumn:1,selectionStartLineNumber:1,startColumn:1,startLineNumber:1),source:'%23include+%3Ciostream%3E%0A%23include+%3Cvector%3E%0A%23include+%3Cranges%3E%0A%0Aint+main()%0A%7B%0A++std::vector%3Cint%3E+ints%7B0,+1,+2,+3,+4,+5%7D%3B%0A++auto+even+%3D+%5B%5D(int+i)%7B+return+(i+%25+2)+%3D%3D+0%3B+%7D%3B%0A%0A++for(int+i+:+ints+%7C+std::views::filter(even))%0A++%7B%0A++++std::cout+%3C%3C+i+%3C%3C+!'+!'%3B%0A++%7D%0A%7D%0A'),l:'5',n:'0',o:'C%2B%2B+source+%231',t:'0')),k:50,l:'4',n:'0',o:'',s:0,t:'0'),(g:!((h:executor,i:(argsPanelShown:'1',compilationPanelShown:'0',compiler:gsnapshot,compilerName:'',compilerOutShown:'0',execArgs:'',execStdin:'',fontScale:14,fontUsePx:'0',j:1,lang:c%2B%2B,libs:!(),options:'-std%3Dc%2B%2B20+-Wall+-Wextra+-pedantic',overrides:!(),runtimeTools:!(),source:1,stdinPanelShown:'1',tree:'1',wrap:'0'),l:'5',n:'0',o:'Executor+x86-64+gcc+(trunk)+(C%2B%2B,+Editor+%231)',t:'0')),header:(),k:50,l:'4',n:'0',o:'',s:0,t:'0')),l:'2',n:'0',o:'',t:'0')),version:4}
	\end{codesample}

	\addproposal{P0896}{https://wg21.link/P0896R4}
	\addproposal{P1035}{https://wg21.link/P1035R7}
\end{frame}

\begin{frame}[fragile]
	\frametitle{Ranges -- Range adaptators}
	\begin{itemize}
		\item  [] \begin{itemize}
			\item \mintinline{cpp}|take_view| : les $N$ premiers éléments
			\item \mintinline{cpp}|take_while_view| : les éléments jusqu'au premier ne satisfaisant pas un prédicat
			\item \mintinline{cpp}|drop_view| : tous les éléments sauf les $N$ premiers
			\item \mintinline{cpp}|drop_while_view| : tous les éléments depuis le premier ne satisfaisant pas un prédicat
			\item \mintinline{cpp}|common_view| convertit une vue en \mintinline{cpp}|common_range|
			\item \mintinline{cpp}|reverse_view| : éléments en sens inverse
			\item \mintinline{cpp}|istream_view| : vue par application successive de \mintinline{cpp}|operator>>| sur un flux
		\end{itemize}
	\end{itemize}

	\addproposal{P0896}{https://wg21.link/P0896R4}
	\addproposal{P1035}{https://wg21.link/P1035R7}
\end{frame}

\begin{frame}[fragile]
	\frametitle{Ranges -- Range adaptators}
	\begin{itemize}
		\item [] \begin{itemize}
			\item \mintinline{cpp}|join_view| \og aplati\fg{} les éléments d'un \mintinline{cpp}|range|
		\end{itemize}
	\end{itemize}

	\begin{minted}{cpp}
		vector<string> foo{"hello", " ", "world", "!"};
		ranges::join_view bar{foo};  // hello world!
  	\end{minted}

	\begin{itemize}
		\item  [] \begin{itemize}
			\item \mintinline{cpp}|split_view| sépare un \mintinline{cpp}|range| en élément sur un délimiteur donné
		\end{itemize}
	\end{itemize}

	\begin{minted}{cpp}
		string foo{"the quick brown fox"};
		ranges::split_view bar{foo, ' '};
		// { the, quick, brown, fox }
  	\end{minted}

	\begin{codesample}
		\sample{https://godbolt.org/#g:!((g:!((g:!((h:codeEditor,i:(filename:'1',fontScale:14,fontUsePx:'0',j:1,lang:c%2B%2B,selection:(endColumn:1,endLineNumber:1,positionColumn:1,positionLineNumber:1,selectionStartColumn:1,selectionStartLineNumber:1,startColumn:1,startLineNumber:1),source:'%23include+%3Ciostream%3E%0A%23include+%3Cvector%3E%0A%23include+%3Cstring%3E%0A%23include+%3Cranges%3E%0A%0Aint+main()%0A%7B%0A++std::string+foo%7B%22the+quick+brown+fox%22%7D%3B%0A++std::ranges::split_view+bar%7Bfoo,+!'+!'%7D%3B%0A++for(auto+baz+:+bar)%0A++%7B%0A++++for(char+c+:+baz)%0A++++%7B%0A++++++std::cout+%3C%3C+c%3B%0A++++%7D%0A++++std::cout+%3C%3C+%22%5Cn%22%3B%0A++%7D%0A%7D%0A'),l:'5',n:'0',o:'C%2B%2B+source+%231',t:'0')),k:50,l:'4',n:'0',o:'',s:0,t:'0'),(g:!((h:executor,i:(argsPanelShown:'1',compilationPanelShown:'0',compiler:gsnapshot,compilerName:'',compilerOutShown:'0',execArgs:'',execStdin:'',fontScale:14,fontUsePx:'0',j:1,lang:c%2B%2B,libs:!(),options:'-std%3Dc%2B%2B20+-Wall+-Wextra+-pedantic',overrides:!(),runtimeTools:!(),source:1,stdinPanelShown:'1',tree:'1',wrap:'0'),l:'5',n:'0',o:'Executor+x86-64+gcc+(trunk)+(C%2B%2B,+Editor+%231)',t:'0')),header:(),k:50,l:'4',n:'0',o:'',s:0,t:'0')),l:'2',n:'0',o:'',t:'0')),version:4}
	\end{codesample}

	\begin{codesample}
		\sample{https://godbolt.org/#g:!((g:!((g:!((h:codeEditor,i:(filename:'1',fontScale:14,fontUsePx:'0',j:1,lang:c%2B%2B,selection:(endColumn:1,endLineNumber:1,positionColumn:1,positionLineNumber:1,selectionStartColumn:1,selectionStartLineNumber:1,startColumn:1,startLineNumber:1),source:'%23include+%3Ciostream%3E%0A%23include+%3Cvector%3E%0A%23include+%3Cstring%3E%0A%23include+%3Cranges%3E%0A%0Aint+main()%0A%7B%0A++std::vector%3Cstd::string%3E+foo%7B%22hello%22,+%22+%22,+%22world%22,+%22!!%22%7D%3B%0A++std::ranges::join_view+bar%7Bfoo%7D%3B%0A++for(char+c+:+bar)%0A++%7B%0A++++std::cout+%3C%3C+c%3B%0A++%7D%0A%7D%0A'),l:'5',n:'0',o:'C%2B%2B+source+%231',t:'0')),k:50,l:'4',n:'0',o:'',s:0,t:'0'),(g:!((h:executor,i:(argsPanelShown:'1',compilationPanelShown:'0',compiler:gsnapshot,compilerName:'',compilerOutShown:'0',execArgs:'',execStdin:'',fontScale:14,fontUsePx:'0',j:1,lang:c%2B%2B,libs:!(),options:'-std%3Dc%2B%2B20+-Wall+-Wextra+-pedantic',overrides:!(),runtimeTools:!(),source:1,stdinPanelShown:'1',tree:'1',wrap:'0'),l:'5',n:'0',o:'Executor+x86-64+gcc+(trunk)+(C%2B%2B,+Editor+%231)',t:'0')),header:(),k:50,l:'4',n:'0',o:'',s:0,t:'0')),l:'2',n:'0',o:'',t:'0')),version:4}
	\end{codesample}

	\addproposal{P0896}{https://wg21.link/P0896R4}
	\addproposal{P1035}{https://wg21.link/P1035R7}
\end{frame}

\begin{frame}[fragile]
	\frametitle{Ranges -- Range adaptators}
	\begin{itemize}
		\item [] \begin{itemize}
			\item \mintinline{cpp}|elements_view| : vue des $N\textsuperscript{e}$ éléments de chaque \textit{tuple} d'une vue de \textit{tuple-likes}
		\end{itemize}
	\end{itemize}

	\begin{minted}{cpp}
		map<std::string, int> foo {
		  {"Lovelace"s, 1815}, {"Turing"s,   1912},
		  {"Babbage"s,  1791}, {"Hamilton"s, 1936}
		};

		foo | views::elements<1>;  // 1791 1936 1815 1912
	\end{minted}

	\begin{itemize}
		\item [] \begin{itemize}
			\item \mintinline{cpp}|keys_view| : vue des clés de chaque \mintinline{cpp}|std::pair| d'une vue de \mintinline{cpp}|std::pair|
			\item \mintinline{cpp}|values_view| : vue des valeurs de chaque \mintinline{cpp}|std::pair| d'une vue de \mintinline{cpp}|std::pair|
		\end{itemize}
	\item Possible d'utiliser les algorithmes opérants sur les \mintinline{cpp}|range|
	\end{itemize}

	\begin{codesample}
		\sample{https://godbolt.org/#g:!((g:!((g:!((h:codeEditor,i:(filename:'1',fontScale:14,fontUsePx:'0',j:1,lang:c%2B%2B,selection:(endColumn:1,endLineNumber:1,positionColumn:1,positionLineNumber:1,selectionStartColumn:1,selectionStartLineNumber:1,startColumn:1,startLineNumber:1),source:'%23include+%3Ciostream%3E%0A%23include+%3Cmap%3E%0A%23include+%3Cstring%3E%0A%23include+%3Cranges%3E%0A%0Ausing+namespace+std::literals%3B%0A%0Aint+main()%0A%7B%0A++std::map%3Cstd::string,+int%3E+foo+%3D+%0A++%7B%0A++++%7B%22Lovelace%22s,+1815%7D,%0A++++%7B%22Turing%22s,+++1912%7D,%0A++++%7B%22Babbage%22s,++1791%7D,%0A++++%7B%22Hamilton%22s,+1936%7D,%0A++%7D%3B%0A%0A++for(auto+i+:+foo+%7C+std::views::elements%3C1%3E)%0A++%7B%0A++++std::cout+%3C%3C+i+%3C%3C+%22+%22%3B%0A++%7D%0A%7D%0A'),l:'5',n:'0',o:'C%2B%2B+source+%231',t:'0')),k:50,l:'4',n:'0',o:'',s:0,t:'0'),(g:!((h:executor,i:(argsPanelShown:'1',compilationPanelShown:'0',compiler:gsnapshot,compilerName:'',compilerOutShown:'0',execArgs:'',execStdin:'',fontScale:14,fontUsePx:'0',j:1,lang:c%2B%2B,libs:!(),options:'-std%3Dc%2B%2B20+-Wall+-Wextra+-pedantic',overrides:!(),runtimeTools:!(),source:1,stdinPanelShown:'1',tree:'1',wrap:'0'),l:'5',n:'0',o:'Executor+x86-64+gcc+(trunk)+(C%2B%2B,+Editor+%231)',t:'0')),header:(),k:50,l:'4',n:'0',o:'',s:0,t:'0')),l:'2',n:'0',o:'',t:'0')),version:4}
	\end{codesample}

	\addproposal{P0896}{https://wg21.link/P0896R4}
	\addproposal{P1035}{https://wg21.link/P1035R7}
\end{frame}

\begin{frame}[fragile]
	\frametitle{Ranges -- Projections}
	\begin{itemize}
		\item Paramètres des algorithmes pré-traitant les éléments du \mintinline{cpp}|range|
	\end{itemize}

	\begin{minted}{cpp}
		vector foo{-1, 2, -3, 4, -5, 6};

		ranges::sort(foo, {}, [](int i) { return abs(i); });  // -1 2 -3 4 -5 6
	\end{minted}

\note[item]{La paire d'accolade vide indique d'utiliser la fonction de comparaison par défaut (\mintinline{cpp}|ranges::less|)}

	\begin{itemize}
		\item Ou extrayant une données des éléments du \mintinline{cpp}|range|
	\end{itemize}

	\begin{minted}{cpp}
		struct Task { string desc;
		              unsigned int priority { 0 }; };

		vector<Task> foo { {"AAA", 10}, {"BBB", 5},
		                   {"CCC", 8}, {"DDD", 12} };

		sort(foo, greater{}, &Task::priority);  // DDD, AAA, CCC, BBB
	\end{minted}

	\begin{codesample}
		\sample{https://godbolt.org/#g:!((g:!((g:!((h:codeEditor,i:(filename:'1',fontScale:14,fontUsePx:'0',j:1,lang:c%2B%2B,selection:(endColumn:1,endLineNumber:1,positionColumn:1,positionLineNumber:1,selectionStartColumn:1,selectionStartLineNumber:1,startColumn:1,startLineNumber:1),source:'%23include+%3Ciostream%3E%0A%23include+%3Cranges%3E%0A%23include+%3Calgorithm%3E%0A%23include+%3Cvector%3E%0A%0Astruct+Task%0A%7B%0A++std::string+desc%3B%0A++unsigned+int+priority+%7B+0+%7D%3B%0A%7D%3B%0A%0Aint+main()%0A%7B%0A++std::vector%3CTask%3E+foo%0A++%7B%0A++++%7B%22AAA%22,+10%7D,%0A++++%7B%22BBB%22,+5%7D,%0A++++%7B%22CCC%22,+8%7D,%0A++++%7B%22DDD%22,+12%7D,%0A++%7D%3B%0A++%0A++std::ranges::sort(foo,+std::greater%7B%7D,+%26Task::priority)%3B%0A++for(auto+i+:+foo)%0A++%7B%0A++++std::cout+%3C%3C+i.desc+%3C%3C+!'+!'%3B%0A++%7D%0A++std::cout+%3C%3C+%22%5Cn%22%3B%0A%7D%0A'),l:'5',n:'0',o:'C%2B%2B+source+%231',t:'0')),k:50,l:'4',n:'0',o:'',s:0,t:'0'),(g:!((h:executor,i:(argsPanelShown:'1',compilationPanelShown:'0',compiler:gsnapshot,compilerName:'',compilerOutShown:'0',execArgs:'',execStdin:'',fontScale:14,fontUsePx:'0',j:1,lang:c%2B%2B,libs:!(),options:'-std%3Dc%2B%2B20+-Wall+-Wextra+-pedantic',overrides:!(),runtimeTools:!(),source:1,stdinPanelShown:'1',tree:'1',wrap:'0'),l:'5',n:'0',o:'Executor+x86-64+gcc+(trunk)+(C%2B%2B,+Editor+%231)',t:'0')),header:(),k:50,l:'4',n:'0',o:'',s:0,t:'0')),l:'2',n:'0',o:'',t:'0')),version:4}
	\end{codesample}

	\begin{codesample}
		\sample{https://godbolt.org/#g:!((g:!((g:!((h:codeEditor,i:(filename:'1',fontScale:14,fontUsePx:'0',j:1,lang:c%2B%2B,selection:(endColumn:1,endLineNumber:1,positionColumn:1,positionLineNumber:1,selectionStartColumn:1,selectionStartLineNumber:1,startColumn:1,startLineNumber:1),source:'%23include+%3Ciostream%3E%0A%23include+%3Cranges%3E%0A%23include+%3Calgorithm%3E%0A%23include+%3Cvector%3E%0A%0Aint+main()%0A%7B%0A++std::vector+foo%7B-1,+2,+-3,+4,+-5,+6%7D%3B%0A%0A++std::ranges::sort(foo,+%7B%7D,+%5B%5D(int+i)+%7B+return+abs(i)%3B+%7D)%3B%0A++for(int+i+:+foo)%0A++%7B%0A++++std::cout+%3C%3C+i+%3C%3C+!'+!'%3B%0A++%7D%0A++std::cout+%3C%3C+%22%5Cn%22%3B%0A%7D%0A'),l:'5',n:'0',o:'C%2B%2B+source+%231',t:'0')),k:50,l:'4',n:'0',o:'',s:0,t:'0'),(g:!((h:executor,i:(argsPanelShown:'1',compilationPanelShown:'0',compiler:gsnapshot,compilerName:'',compilerOutShown:'0',execArgs:'',execStdin:'',fontScale:14,fontUsePx:'0',j:1,lang:c%2B%2B,libs:!(),options:'-std%3Dc%2B%2B20+-Wall+-Wextra+-pedantic',overrides:!(),runtimeTools:!(),source:1,stdinPanelShown:'1',tree:'1',wrap:'0'),l:'5',n:'0',o:'Executor+x86-64+gcc+(trunk)+(C%2B%2B,+Editor+%231)',t:'0')),header:(),k:50,l:'4',n:'0',o:'',s:0,t:'0')),l:'2',n:'0',o:'',t:'0')),version:4}
	\end{codesample}

	\addproposal{P0896}{https://wg21.link/P0896R4}
	\addproposal{P1035}{https://wg21.link/P1035R7}
\end{frame}

\subsection*{Flux}
\begin{frame}[fragile]
	\frametitle{Gestion des flux}
	\begin{itemize}
		\item Flux synchrones
		\begin{itemize}
			\item Classe tampon synchrone : \mintinline{cpp}|std::basic_syncbuf|
			\item Classe flux bufferisé synchrone : \mintinline{cpp}|std::basic_osyncstream|
			\item \mintinline{cpp}|emit()| transfère le tampon vers le flux de sortie
		\end{itemize}
	\end{itemize}

	\begin{minted}{cpp}
		{ osyncstream s(cout);
		  s << "Hello," << '\n'; // no flush
		  s.emit(); // characters transferred, cout not flushed
		  s << "World!" << endl; // flush noted, cout not flushed
		  s.emit(); // characters transferred, cout flushed
		  s << "Greetings." << '\n'; // no flush
		} // characters transferred, cout not flushed
	\end{minted}

	\begin{itemize}
		\item Limitation de la taille lue dans les flux avec \mintinline{cpp}|std::setw()|
	\end{itemize}

	\begin{minted}{cpp}
		// Seuls 24 caracteres sont lus
		cin >> setw(24) >> a;
	\end{minted}

	\addproposal{P0053}{https://wg21.link/P0053R7}
\end{frame}

\begin{frame}[fragile]
	\frametitle{\mintinline[style=white]{cpp}|std::format| -- Présentation}
	\begin{itemize}
		\item API de formatage inspiré de la bibliothèque \href{https://github.com/fmtlib/fmt}{\mintinline[escapeinside=||]{cpp}{|\{|fmt|\}|}}
	\end{itemize}

	\begin{block}{Motivations}
		\begin{itemize}
			\item Formatage \og à la C\fg{} non extensible et peu sûr
			\item Flux complexes, peu performants, peu propices à l'internationalisation et la localisation, formateurs globaux

\note[item]{Peu propices à la localisation car le format et l'ordre des éléments est dans le code lui-même et ne peut pas être sortie facilement}
\note[item]{Les formateurs sont globaux, ainsi l'injection de \mintinline{cpp}|std::hex| dans un flux va passer tous les affichages des entiers en hexadécimal jusqu'à un changement explicite et pas uniquement celui concerné}
		\end{itemize}
	\end{block}

	\begin{itemize}
		\item Formatage \textit{locale-specific} ou \textit{locale-independent}

\note[item]{Par défaut, la locale n'est pas prise en compte, mais peut l'être si souhaité}

		\item Format sous forme de chaînes utilisant \mintinline{cpp}|{}| comme \textit{placeholder}
	\end{itemize}

	\begin{block}{En attendant C++20}
		\begin{itemize}
			\item Utilisez \mintinline{cpp}|{fmt}| ou \mintinline{cpp}|Boost.Format|
		\end{itemize}
	\end{block}

	\begin{block}{Voir aussi}
		\begin{itemize}
			\item Overload 166
		\end{itemize}
	\end{block}

	\addproposal{P0645}{https://wg21.link/P0645R10}
\end{frame}

\begin{frame}[fragile]
	\frametitle{\mintinline[style=white]{cpp}|std::format| -- API}
	\begin{itemize}
		\item \mintinline{cpp}|format()| retourne une chaîne
	\end{itemize}

	\begin{minted}{cpp}
		format("{}", "a");  // "a"
	\end{minted}

	\begin{itemize}
		\item \mintinline{cpp}|format_to()| formate dans un \mintinline{cpp}|output_iterator|
	\end{itemize}

	\begin{minted}{cpp}
		vector<char> foo;

		format_to(back_inserter(foo), "{}", "a");
	\end{minted}

\note[item]{\mintinline{cpp}|output_iterator| est retourné par \mintinline{cpp}|format_to()|}

	\begin{itemize}
		\item \mintinline{cpp}|format_to_n()| formate dans un \mintinline{cpp}|output_iterator| avec une taille limite
	\end{itemize}

	\begin{minted}{cpp}
		array<char, 4> foo;

		format_to_n(foo.data(), foo.size(), "{}", "a");
	\end{minted}

\note[item]{Troncature à la taille indiquée si nécessaire}

	\begin{codesample}
		\sample{https://godbolt.org/#g:!((g:!((g:!((h:codeEditor,i:(filename:'1',fontScale:14,fontUsePx:'0',j:1,lang:c%2B%2B,selection:(endColumn:1,endLineNumber:1,positionColumn:1,positionLineNumber:1,selectionStartColumn:1,selectionStartLineNumber:1,startColumn:1,startLineNumber:1),source:'%23include+%3Ciostream%3E%0A%23include+%3Carray%3E%0A%23include+%3Cformat%3E%0A%0Aint+main()%0A%7B%0A++std::array%3Cchar,+4%3E+foo%3B%0A%0A++std::format_to_n(foo.data(),+foo.size(),+%22%7B%7D%22,+%22a%22)%3B%0A++for(auto+c:+foo)%0A++%7B%0A++++std::cout+%3C%3C+c+%3C%3C+%22+%22%3B%0A++%7D%0A%7D%0A'),l:'5',n:'0',o:'C%2B%2B+source+%231',t:'0')),k:50,l:'4',n:'0',o:'',s:0,t:'0'),(g:!((h:executor,i:(argsPanelShown:'1',compilationPanelShown:'0',compiler:gsnapshot,compilerName:'',compilerOutShown:'0',execArgs:'',execStdin:'',fontScale:14,fontUsePx:'0',j:1,lang:c%2B%2B,libs:!(),options:'-std%3Dc%2B%2B20+-Wall+-Wextra+-pedantic',overrides:!(),runtimeTools:!(),source:1,stdinPanelShown:'1',tree:'1',wrap:'0'),l:'5',n:'0',o:'Executor+x86-64+gcc+(trunk)+(C%2B%2B,+Editor+%231)',t:'0')),header:(),k:50,l:'4',n:'0',o:'',s:0,t:'0')),l:'2',n:'0',o:'',t:'0')),version:4}
	\end{codesample}

	\begin{codesample}
		\sample{https://godbolt.org/#g:!((g:!((g:!((h:codeEditor,i:(filename:'1',fontScale:14,fontUsePx:'0',j:1,lang:c%2B%2B,selection:(endColumn:1,endLineNumber:1,positionColumn:1,positionLineNumber:1,selectionStartColumn:1,selectionStartLineNumber:1,startColumn:1,startLineNumber:1),source:'%23include+%3Ciostream%3E%0A%23include+%3Cvector%3E%0A%23include+%3Cformat%3E%0A%0Aint+main()%0A%7B%0A++std::vector%3Cchar%3E+foo%3B%0A%0A++std::format_to(std::back_inserter(foo),+%22%7B%7D%22,+%22a%22)%3B%0A++for(auto+c:+foo)%0A++%7B%0A++++std::cout+%3C%3C+c+%3C%3C+%22+%22%3B%0A++%7D%0A%7D%0A'),l:'5',n:'0',o:'C%2B%2B+source+%231',t:'0')),k:50,l:'4',n:'0',o:'',s:0,t:'0'),(g:!((h:executor,i:(argsPanelShown:'1',compilationPanelShown:'0',compiler:gsnapshot,compilerName:'',compilerOutShown:'0',execArgs:'',execStdin:'',fontScale:14,fontUsePx:'0',j:1,lang:c%2B%2B,libs:!(),options:'-std%3Dc%2B%2B20+-Wall+-Wextra+-pedantic',overrides:!(),runtimeTools:!(),source:1,stdinPanelShown:'1',tree:'1',wrap:'0'),l:'5',n:'0',o:'Executor+x86-64+gcc+(trunk)+(C%2B%2B,+Editor+%231)',t:'0')),header:(),k:50,l:'4',n:'0',o:'',s:0,t:'0')),l:'2',n:'0',o:'',t:'0')),version:4}
	\end{codesample}

	\begin{codesample}
		\sample{https://godbolt.org/#g:!((g:!((g:!((h:codeEditor,i:(filename:'1',fontScale:14,fontUsePx:'0',j:1,lang:c%2B%2B,selection:(endColumn:1,endLineNumber:1,positionColumn:1,positionLineNumber:1,selectionStartColumn:1,selectionStartLineNumber:1,startColumn:1,startLineNumber:1),source:'%23include+%3Ciostream%3E%0A%23include+%3Cformat%3E%0A%0Aint+main()%0A%7B%0A++std::cout+%3C%3C+std::format(%22%7B%7D%22,+%22a%22)%3B%0A%7D%0A'),l:'5',n:'0',o:'C%2B%2B+source+%231',t:'0')),k:50,l:'4',n:'0',o:'',s:0,t:'0'),(g:!((h:executor,i:(argsPanelShown:'1',compilationPanelShown:'0',compiler:gsnapshot,compilerName:'',compilerOutShown:'0',execArgs:'',execStdin:'',fontScale:14,fontUsePx:'0',j:1,lang:c%2B%2B,libs:!(),options:'-std%3Dc%2B%2B20+-Wall+-Wextra+-pedantic',overrides:!(),runtimeTools:!(),source:1,stdinPanelShown:'1',tree:'1',wrap:'0'),l:'5',n:'0',o:'Executor+x86-64+gcc+(trunk)+(C%2B%2B,+Editor+%231)',t:'0')),header:(),k:50,l:'4',n:'0',o:'',s:0,t:'0')),l:'2',n:'0',o:'',t:'0')),version:4}
	\end{codesample}

	\addproposal{P0645}{https://wg21.link/P0645R10}
\end{frame}

\begin{frame}[fragile]
	\frametitle{\mintinline[style=white]{cpp}|std::format| -- API}
	\begin{itemize}
		\item \mintinline{cpp}|formatted_size()| retourne la taille nécessaire au formatage
	\end{itemize}

	\begin{minted}{cpp}
		formatted_size("{}", "a");  // 1
	\end{minted}

	\begin{itemize}
		\item \mintinline{cpp}|vformat()| et \mintinline{cpp}|vformat_to()| arguments regroupés dans un \textit{tuple-like}
	\end{itemize}

	\begin{minted}{cpp}
		vformat("{}", make_format_args("a"));
	\end{minted}

	\begin{itemize}
		\item Variantes \mintinline{cpp}|wchar| et \mintinline{cpp}|locale|
	\end{itemize}

	\begin{codesample}
		\sample{https://godbolt.org/#g:!((g:!((g:!((h:codeEditor,i:(filename:'1',fontScale:14,fontUsePx:'0',j:1,lang:c%2B%2B,selection:(endColumn:1,endLineNumber:1,positionColumn:1,positionLineNumber:1,selectionStartColumn:1,selectionStartLineNumber:1,startColumn:1,startLineNumber:1),source:'%23include+%3Ciostream%3E%0A%23include+%3Cformat%3E%0A%0Aint+main()%0A%7B%0A++std::cout+%3C%3C+std::vformat(%22%7B%7D%22,+std::make_format_args(%22a%22))+%3C%3C+%22%5Cn%22%3B%0A%7D%0A'),l:'5',n:'0',o:'C%2B%2B+source+%231',t:'0')),k:50,l:'4',n:'0',o:'',s:0,t:'0'),(g:!((h:executor,i:(argsPanelShown:'1',compilationPanelShown:'0',compiler:gsnapshot,compilerName:'',compilerOutShown:'0',execArgs:'',execStdin:'',fontScale:14,fontUsePx:'0',j:1,lang:c%2B%2B,libs:!(),options:'-std%3Dc%2B%2B20+-Wall+-Wextra+-pedantic',overrides:!(),runtimeTools:!(),source:1,stdinPanelShown:'1',tree:'1',wrap:'0'),l:'5',n:'0',o:'Executor+x86-64+gcc+(trunk)+(C%2B%2B,+Editor+%231)',t:'0')),header:(),k:50,l:'4',n:'0',o:'',s:0,t:'0')),l:'2',n:'0',o:'',t:'0')),version:4}
	\end{codesample}

	\begin{codesample}
		\sample{https://godbolt.org/#g:!((g:!((g:!((h:codeEditor,i:(filename:'1',fontScale:14,fontUsePx:'0',j:1,lang:c%2B%2B,selection:(endColumn:1,endLineNumber:1,positionColumn:1,positionLineNumber:1,selectionStartColumn:1,selectionStartLineNumber:1,startColumn:1,startLineNumber:1),source:'%23include+%3Ciostream%3E%0A%23include+%3Cformat%3E%0A%0Aint+main()%0A%7B%0A++std::cout+%3C%3C+std::formatted_size(%22%7B%7D%22,+%22a%22)+%3C%3C+%22%5Cn%22%3B%0A%7D%0A'),l:'5',n:'0',o:'C%2B%2B+source+%231',t:'0')),k:50,l:'4',n:'0',o:'',s:0,t:'0'),(g:!((h:executor,i:(argsPanelShown:'1',compilationPanelShown:'0',compiler:gsnapshot,compilerName:'',compilerOutShown:'0',execArgs:'',execStdin:'',fontScale:14,fontUsePx:'0',j:1,lang:c%2B%2B,libs:!(),options:'-std%3Dc%2B%2B20+-Wall+-Wextra+-pedantic',overrides:!(),runtimeTools:!(),source:1,stdinPanelShown:'1',tree:'1',wrap:'0'),l:'5',n:'0',o:'Executor+x86-64+gcc+(trunk)+(C%2B%2B,+Editor+%231)',t:'0')),header:(),k:50,l:'4',n:'0',o:'',s:0,t:'0')),l:'2',n:'0',o:'',t:'0')),version:4}
	\end{codesample}

	\addproposal{P0645}{https://wg21.link/P0645R10}
\end{frame}

\begin{frame}[fragile]
	\frametitle{\mintinline[style=white]{cpp}|std::format| -- Placeholder}
	\begin{itemize}
		\item Format général : \mintinline{cpp}|{[arg-id][:format-spec]}|
		\begin{itemize}
			\item \mintinline{cpp}|arg-id| : index, optionnel, de l'argument de la liste de paramètres
			\item \mintinline{cpp}|format-spec| : spécifications, optionnelles, du format
		\end{itemize}
	\end{itemize}

	\begin{block}{Séquences d'échappement}
		\begin{itemize}
			\item \mintinline{cpp}|{{| affiche \mintinline{cpp}|{|
			\item \mintinline{cpp}|}}| affiche \mintinline{cpp}|}|
		\end{itemize}
	\end{block}

	\addproposal{P0645}{https://wg21.link/P0645R10}
\end{frame}

\begin{frame}[fragile]
	\frametitle{\mintinline[style=white]{cpp}|std::format| -- Identifiant d'arguments}
	\begin{itemize}
		\item Valeur optionnelle indiquant l'index du paramètre à afficher
		\item Débute à 0
	\end{itemize}

	\begin{minted}{cpp}
		format("{1} et {0}", "a", "b"); // "b et a"
		format("{0} et {0}", "a");      // "a et a"
	\end{minted}

	\begin{itemize}
		\item En cas d'absence, les paramètres sont pris dans l'ordre d'apparition
	\end{itemize}

	\begin{minted}{cpp}
		format("{} et {}", "a", "b");   // "a et b"
	\end{minted}

	\begin{alertblock}{Limite}
		\begin{itemize}
			\item Impossible d'en n'omettre que certains
		\end{itemize}
	\end{alertblock}

	\begin{codesample}
		\sample{https://godbolt.org/#g:!((g:!((g:!((h:codeEditor,i:(filename:'1',fontScale:14,fontUsePx:'0',j:1,lang:c%2B%2B,selection:(endColumn:1,endLineNumber:1,positionColumn:1,positionLineNumber:1,selectionStartColumn:1,selectionStartLineNumber:1,startColumn:1,startLineNumber:1),source:'%23include+%3Ciostream%3E%0A%23include+%3Cformat%3E%0A%0Aint+main()%0A%7B%0A++std::cout+%3C%3C+std::format(%22%7B%7D+et+%7B%7D%22,+%22a%22,+%22b%22)%3B%0A%7D%0A'),l:'5',n:'0',o:'C%2B%2B+source+%231',t:'0')),k:50,l:'4',n:'0',o:'',s:0,t:'0'),(g:!((h:executor,i:(argsPanelShown:'1',compilationPanelShown:'0',compiler:gsnapshot,compilerName:'',compilerOutShown:'0',execArgs:'',execStdin:'',fontScale:14,fontUsePx:'0',j:1,lang:c%2B%2B,libs:!(),options:'-std%3Dc%2B%2B20+-Wall+-Wextra+-pedantic',overrides:!(),runtimeTools:!(),source:1,stdinPanelShown:'1',tree:'1',wrap:'0'),l:'5',n:'0',o:'Executor+x86-64+gcc+(trunk)+(C%2B%2B,+Editor+%231)',t:'0')),header:(),k:50,l:'4',n:'0',o:'',s:0,t:'0')),l:'2',n:'0',o:'',t:'0')),version:4}
	\end{codesample}

	\begin{codesample}
		\sample{https://godbolt.org/#g:!((g:!((g:!((h:codeEditor,i:(filename:'1',fontScale:14,fontUsePx:'0',j:1,lang:c%2B%2B,selection:(endColumn:1,endLineNumber:1,positionColumn:1,positionLineNumber:1,selectionStartColumn:1,selectionStartLineNumber:1,startColumn:1,startLineNumber:1),source:'%23include+%3Ciostream%3E%0A%23include+%3Cformat%3E%0A%0Aint+main()%0A%7B%0A++std::cout+%3C%3C+std::format(%22%7B1%7D+et+%7B0%7D%5Cn%22,+%22a%22,+%22b%22)%3B%0A++std::cout+%3C%3C+std::format(%22%7B0%7D+et+%7B0%7D%5Cn%22,+%22a%22)%3B%0A%7D%0A'),l:'5',n:'0',o:'C%2B%2B+source+%231',t:'0')),k:50,l:'4',n:'0',o:'',s:0,t:'0'),(g:!((h:executor,i:(argsPanelShown:'1',compilationPanelShown:'0',compiler:gsnapshot,compilerName:'',compilerOutShown:'0',execArgs:'',execStdin:'',fontScale:14,fontUsePx:'0',j:1,lang:c%2B%2B,libs:!(),options:'-std%3Dc%2B%2B20+-Wall+-Wextra+-pedantic',overrides:!(),runtimeTools:!(),source:1,stdinPanelShown:'1',tree:'1',wrap:'0'),l:'5',n:'0',o:'Executor+x86-64+gcc+(trunk)+(C%2B%2B,+Editor+%231)',t:'0')),header:(),k:50,l:'4',n:'0',o:'',s:0,t:'0')),l:'2',n:'0',o:'',t:'0')),version:4}
	\end{codesample}

	\addproposal{P0645}{https://wg21.link/P0645R10}
\end{frame}

\begin{frame}[fragile]
	\frametitle{\mintinline[style=white]{cpp}|std::format| -- Spécification de format}
	\begin{itemize}
		\item Format général : \mintinline[escapeinside=||]{cpp}{[[fill]align][sign][|#|][|0|][width][prec][L][type]}
		\begin{itemize}
			\item \mintinline{cpp}|fill| et \mintinline{cpp}|align| : gestion de l'alignement
			\item \mintinline{cpp}|sign| : gestion du signe
			\item \mintinline[escapeinside=||]{cpp}{|#|} : forme alternative
			\item \mintinline[escapeinside=||]{cpp}{|0|} : gestion des zéros non significatifs
			\item \mintinline{cpp}|width| : taille minimal du champ
			\item \mintinline{cpp}|prec| : précision du champ
			\item \mintinline{cpp}|L| : prise en compte de la locale
			\item \mintinline{cpp}|type| : type à afficher
		\end{itemize}
	\end{itemize}

	\addproposal{P0645}{https://wg21.link/P0645R10}
\end{frame}

\begin{frame}[fragile]
	\frametitle{\mintinline[style=white]{cpp}|std::format| -- Alignement}
	\begin{itemize}
		\item Alignement par défaut dépendant du type
	\end{itemize}

	\begin{minted}{cpp}
		format("{:6}", 42);     // "    42"
		format("{:6}", 'x');    // "x     "
	\end{minted}

	\begin{itemize}
		\item Fourniture du caractère de \textit{padding}
	\end{itemize}

	\begin{minted}{cpp}
		format("{:06}", 42);    // "000042"
	\end{minted}

	\begin{itemize}
		\item Choix de l'alignement
	\end{itemize}

	\begin{minted}{cpp}
		format("{:*<6}", 'x');  // "x*****"
		format("{:*>6}", 'x');  // "*****x"
		format("{:*^6}", 'x');  // "**x***"
	\end{minted}

	\begin{codesample}
		\sample{https://godbolt.org/#g:!((g:!((g:!((h:codeEditor,i:(filename:'1',fontScale:14,fontUsePx:'0',j:1,lang:c%2B%2B,selection:(endColumn:1,endLineNumber:1,positionColumn:1,positionLineNumber:1,selectionStartColumn:1,selectionStartLineNumber:1,startColumn:1,startLineNumber:1),source:'%23include+%3Ciostream%3E%0A%23include+%3Cformat%3E%0A%0Aint+main()%0A%7B%0A++std::cout+%3C%3C+std::format(%22%7B:*%3C6%7D%22,+!'x!')+%3C%3C+%22%5Cn%22%3B%0A++std::cout+%3C%3C+std::format(%22%7B:*%3E6%7D%22,+!'x!')+%3C%3C+%22%5Cn%22%3B%0A++std::cout+%3C%3C+std::format(%22%7B:*%5E6%7D%22,+!'x!')+%3C%3C+%22%5Cn%22%3B%0A%7D%0A'),l:'5',n:'0',o:'C%2B%2B+source+%231',t:'0')),k:50,l:'4',n:'0',o:'',s:0,t:'0'),(g:!((h:executor,i:(argsPanelShown:'1',compilationPanelShown:'0',compiler:gsnapshot,compilerName:'',compilerOutShown:'0',execArgs:'',execStdin:'',fontScale:14,fontUsePx:'0',j:1,lang:c%2B%2B,libs:!(),options:'-std%3Dc%2B%2B20+-Wall+-Wextra+-pedantic',overrides:!(),runtimeTools:!(),source:1,stdinPanelShown:'1',tree:'1',wrap:'0'),l:'5',n:'0',o:'Executor+x86-64+gcc+(trunk)+(C%2B%2B,+Editor+%231)',t:'0')),header:(),k:50,l:'4',n:'0',o:'',s:0,t:'0')),l:'2',n:'0',o:'',t:'0')),version:4}
	\end{codesample}

	\begin{codesample}
		\sample{https://godbolt.org/#g:!((g:!((g:!((h:codeEditor,i:(filename:'1',fontScale:14,fontUsePx:'0',j:1,lang:c%2B%2B,selection:(endColumn:1,endLineNumber:1,positionColumn:1,positionLineNumber:1,selectionStartColumn:1,selectionStartLineNumber:1,startColumn:1,startLineNumber:1),source:'%23include+%3Ciostream%3E%0A%23include+%3Cformat%3E%0A%0Aint+main()%0A%7B%0A++std::cout+%3C%3C+std::format(%22%7B:06%7D%22,+42)+%3C%3C+%22%5Cn%22%3B%0A%7D%0A'),l:'5',n:'0',o:'C%2B%2B+source+%231',t:'0')),k:50,l:'4',n:'0',o:'',s:0,t:'0'),(g:!((h:executor,i:(argsPanelShown:'1',compilationPanelShown:'0',compiler:gsnapshot,compilerName:'',compilerOutShown:'0',execArgs:'',execStdin:'',fontScale:14,fontUsePx:'0',j:1,lang:c%2B%2B,libs:!(),options:'-std%3Dc%2B%2B20+-Wall+-Wextra+-pedantic',overrides:!(),runtimeTools:!(),source:1,stdinPanelShown:'1',tree:'1',wrap:'0'),l:'5',n:'0',o:'Executor+x86-64+gcc+(trunk)+(C%2B%2B,+Editor+%231)',t:'0')),header:(),k:50,l:'4',n:'0',o:'',s:0,t:'0')),l:'2',n:'0',o:'',t:'0')),version:4}
	\end{codesample}

	\begin{codesample}
		\sample{https://godbolt.org/#g:!((g:!((g:!((h:codeEditor,i:(filename:'1',fontScale:14,fontUsePx:'0',j:1,lang:c%2B%2B,selection:(endColumn:1,endLineNumber:1,positionColumn:1,positionLineNumber:1,selectionStartColumn:1,selectionStartLineNumber:1,startColumn:1,startLineNumber:1),source:'%23include+%3Ciostream%3E%0A%23include+%3Cformat%3E%0A%0Aint+main()%0A%7B%0A++std::cout+%3C%3C+std::format(%22%7B:6%7D%22,+42)+%3C%3C+%22%5Cn%22%3B%0A++std::cout+%3C%3C+std::format(%22%7B:6%7D%22,+!'x!')+%3C%3C+%22%5Cn%22%3B%0A%7D%0A'),l:'5',n:'0',o:'C%2B%2B+source+%231',t:'0')),k:50,l:'4',n:'0',o:'',s:0,t:'0'),(g:!((h:executor,i:(argsPanelShown:'1',compilationPanelShown:'0',compiler:gsnapshot,compilerName:'',compilerOutShown:'0',execArgs:'',execStdin:'',fontScale:14,fontUsePx:'0',j:1,lang:c%2B%2B,libs:!(),options:'-std%3Dc%2B%2B20+-Wall+-Wextra+-pedantic',overrides:!(),runtimeTools:!(),source:1,stdinPanelShown:'1',tree:'1',wrap:'0'),l:'5',n:'0',o:'Executor+x86-64+gcc+(trunk)+(C%2B%2B,+Editor+%231)',t:'0')),header:(),k:50,l:'4',n:'0',o:'',s:0,t:'0')),l:'2',n:'0',o:'',t:'0')),version:4}
	\end{codesample}

	\addproposal{P0645}{https://wg21.link/P0645R10}
\end{frame}

\begin{frame}[fragile]
	\frametitle{\mintinline[style=white]{cpp}|std::format| -- Taille minimale}
	\begin{itemize}
		\item Fournit la taille minimal du champ
		\item Si le champ est plus long, il n'est pas tronqué
	\end{itemize}

	\begin{minted}{cpp}
		// "|  10| |          10|"
		format("|{0:4}| |{0:12}|", 10);
		// "|10000000| |    10000000|"
		format("|{0:4}| |{0:12}|", 1000000);
	\end{minted}

	\begin{itemize}
		\item Possible de fournir la taille en paramètre via un \textit{placeholder}
	\end{itemize}

	\begin{minted}{cpp}
		// "|  10| |          10|"
		format("|{0:{1}}| |{0:{2}}|", 10, 4, 12);
	\end{minted}

	\begin{codesample}
		\sample{https://godbolt.org/#g:!((g:!((g:!((h:codeEditor,i:(filename:'1',fontScale:14,fontUsePx:'0',j:1,lang:c%2B%2B,selection:(endColumn:1,endLineNumber:1,positionColumn:1,positionLineNumber:1,selectionStartColumn:1,selectionStartLineNumber:1,startColumn:1,startLineNumber:1),source:'%23include+%3Ciostream%3E%0A%23include+%3Cformat%3E%0A%0Aint+main()%0A%7B%0A++std::cout+%3C%3C+std::format(%22%7C%7B0:%7B1%7D%7D%7C+%7C%7B0:%7B2%7D%7D%7C%22,+10,+4,+12)+%3C%3C+%22%5Cn%22%3B%0A%7D%0A'),l:'5',n:'0',o:'C%2B%2B+source+%231',t:'0')),k:50,l:'4',n:'0',o:'',s:0,t:'0'),(g:!((h:executor,i:(argsPanelShown:'1',compilationPanelShown:'0',compiler:gsnapshot,compilerName:'',compilerOutShown:'0',execArgs:'',execStdin:'',fontScale:14,fontUsePx:'0',j:1,lang:c%2B%2B,libs:!(),options:'-std%3Dc%2B%2B20+-Wall+-Wextra+-pedantic',overrides:!(),runtimeTools:!(),source:1,stdinPanelShown:'1',tree:'1',wrap:'0'),l:'5',n:'0',o:'Executor+x86-64+gcc+(trunk)+(C%2B%2B,+Editor+%231)',t:'0')),header:(),k:50,l:'4',n:'0',o:'',s:0,t:'0')),l:'2',n:'0',o:'',t:'0')),version:4}
	\end{codesample}

	\begin{codesample}
		\sample{https://godbolt.org/#g:!((g:!((g:!((h:codeEditor,i:(filename:'1',fontScale:14,fontUsePx:'0',j:1,lang:c%2B%2B,selection:(endColumn:1,endLineNumber:1,positionColumn:1,positionLineNumber:1,selectionStartColumn:1,selectionStartLineNumber:1,startColumn:1,startLineNumber:1),source:'%23include+%3Ciostream%3E%0A%23include+%3Cformat%3E%0A%0Aint+main()%0A%7B%0A++std::cout+%3C%3C+std::format(%22%7C%7B0:4%7D%7C+%7C%7B0:12%7D%7C%22,+10)+%3C%3C+%22%5Cn%22%3B%0A++std::cout+%3C%3C+std::format(%22%7C%7B0:4%7D%7C+%7C%7B0:12%7D%7C%22,+1000000)+%3C%3C+%22%5Cn%22%3B%0A%7D%0A'),l:'5',n:'0',o:'C%2B%2B+source+%231',t:'0')),k:50,l:'4',n:'0',o:'',s:0,t:'0'),(g:!((h:executor,i:(argsPanelShown:'1',compilationPanelShown:'0',compiler:gsnapshot,compilerName:'',compilerOutShown:'0',execArgs:'',execStdin:'',fontScale:14,fontUsePx:'0',j:1,lang:c%2B%2B,libs:!(),options:'-std%3Dc%2B%2B20+-Wall+-Wextra+-pedantic',overrides:!(),runtimeTools:!(),source:1,stdinPanelShown:'1',tree:'1',wrap:'0'),l:'5',n:'0',o:'Executor+x86-64+gcc+(trunk)+(C%2B%2B,+Editor+%231)',t:'0')),header:(),k:50,l:'4',n:'0',o:'',s:0,t:'0')),l:'2',n:'0',o:'',t:'0')),version:4}
	\end{codesample}

	\addproposal{P0645}{https://wg21.link/P0645R10}
\end{frame}

\begin{frame}[fragile]
	\frametitle{\mintinline[style=white]{cpp}|std::format| -- Précision}
	\begin{itemize}
		\item Introduit par un \mintinline{cpp}|.|
		\item Uniquement sur
		\begin{itemize}
			\item Les nombres flottants
		\end{itemize}
	\end{itemize}

	\begin{minted}{cpp}
		format("{:.6f}", 392.65);       // "392.650000"
	\end{minted}

	\begin{itemize}
		\item [] \begin{itemize}
			\item Les chaînes de caractères : troncature
		\end{itemize}
	\end{itemize}

	\begin{minted}{cpp}
		format("{:.6}", "azertyuiop");  // "azerty"
	\end{minted}

	\begin{itemize}
		\item Possible de fournir la taille en paramètre via un \textit{placeholder}
	\end{itemize}

	\begin{codesample}
		\sample{https://godbolt.org/#g:!((g:!((g:!((h:codeEditor,i:(filename:'1',fontScale:14,fontUsePx:'0',j:1,lang:c%2B%2B,selection:(endColumn:1,endLineNumber:1,positionColumn:1,positionLineNumber:1,selectionStartColumn:1,selectionStartLineNumber:1,startColumn:1,startLineNumber:1),source:'%23include+%3Ciostream%3E%0A%23include+%3Cformat%3E%0A%0Aint+main()%0A%7B%0A++std::cout+%3C%3C+std::format(%22%7B:.6%7D%22,+%22azertyuiop%22)+%3C%3C+%22%5Cn%22%3B%0A%7D%0A'),l:'5',n:'0',o:'C%2B%2B+source+%231',t:'0')),k:50,l:'4',n:'0',o:'',s:0,t:'0'),(g:!((h:executor,i:(argsPanelShown:'1',compilationPanelShown:'0',compiler:gsnapshot,compilerName:'',compilerOutShown:'0',execArgs:'',execStdin:'',fontScale:14,fontUsePx:'0',j:1,lang:c%2B%2B,libs:!(),options:'-std%3Dc%2B%2B20+-Wall+-Wextra+-pedantic',overrides:!(),runtimeTools:!(),source:1,stdinPanelShown:'1',tree:'1',wrap:'0'),l:'5',n:'0',o:'Executor+x86-64+gcc+(trunk)+(C%2B%2B,+Editor+%231)',t:'0')),header:(),k:50,l:'4',n:'0',o:'',s:0,t:'0')),l:'2',n:'0',o:'',t:'0')),version:4}
	\end{codesample}

	\begin{codesample}
		\sample{https://godbolt.org/#g:!((g:!((g:!((h:codeEditor,i:(filename:'1',fontScale:14,fontUsePx:'0',j:1,lang:c%2B%2B,selection:(endColumn:1,endLineNumber:1,positionColumn:1,positionLineNumber:1,selectionStartColumn:1,selectionStartLineNumber:1,startColumn:1,startLineNumber:1),source:'%23include+%3Ciostream%3E%0A%23include+%3Cformat%3E%0A%0Aint+main()%0A%7B%0A++std::cout+%3C%3C+std::format(%22%7B:.6f%7D%22,+392.65)+%3C%3C+%22%5Cn%22%3B%0A%7D%0A'),l:'5',n:'0',o:'C%2B%2B+source+%231',t:'0')),k:50,l:'4',n:'0',o:'',s:0,t:'0'),(g:!((h:executor,i:(argsPanelShown:'1',compilationPanelShown:'0',compiler:gsnapshot,compilerName:'',compilerOutShown:'0',execArgs:'',execStdin:'',fontScale:14,fontUsePx:'0',j:1,lang:c%2B%2B,libs:!(),options:'-std%3Dc%2B%2B20+-Wall+-Wextra+-pedantic',overrides:!(),runtimeTools:!(),source:1,stdinPanelShown:'1',tree:'1',wrap:'0'),l:'5',n:'0',o:'Executor+x86-64+gcc+(trunk)+(C%2B%2B,+Editor+%231)',t:'0')),header:(),k:50,l:'4',n:'0',o:'',s:0,t:'0')),l:'2',n:'0',o:'',t:'0')),version:4}
	\end{codesample}

	\addproposal{P0645}{https://wg21.link/P0645R10}
\end{frame}

\begin{frame}[fragile]
	\frametitle{\mintinline[style=white]{cpp}|std::format| -- Signe}
	\begin{itemize}
		\item Uniquement sur les négatifs : '\mintinline{cpp}|-|'
		\item Sur toutes les valeurs : '\mintinline{cpp}|+|'
		\item Uniquement sur les négatifs en réservant l'espace : '\mintinline[escapeinside=||]{cpp}{| |}'
	\end{itemize}

	\begin{minted}{cpp}
		format("{0:},{0:+},{0:-},{0: }", 1);   // "1,+1,1, 1"
		format("{0:},{0:+},{0:-},{0: }", -1);  // "-1,-1,-1,-1"
	\end{minted}

	\begin{codesample}
		\sample{https://godbolt.org/#g:!((g:!((g:!((h:codeEditor,i:(filename:'1',fontScale:14,fontUsePx:'0',j:1,lang:c%2B%2B,selection:(endColumn:1,endLineNumber:1,positionColumn:1,positionLineNumber:1,selectionStartColumn:1,selectionStartLineNumber:1,startColumn:1,startLineNumber:1),source:'%23include+%3Ciostream%3E%0A%23include+%3Cformat%3E%0A%0Aint+main()%0A%7B%0A++std::cout+%3C%3C+std::format(%22%7B0:%7D,%7B0:%2B%7D,%7B0:-%7D,%7B0:+%7D%22,+1)+%3C%3C+%22%5Cn%22%3B%0A++std::cout+%3C%3C+std::format(%22%7B0:%7D,%7B0:%2B%7D,%7B0:-%7D,%7B0:+%7D%22,+-1)+%3C%3C+%22%5Cn%22%3B%0A%7D%0A'),l:'5',n:'0',o:'C%2B%2B+source+%231',t:'0')),k:50,l:'4',n:'0',o:'',s:0,t:'0'),(g:!((h:executor,i:(argsPanelShown:'1',compilationPanelShown:'0',compiler:gsnapshot,compilerName:'',compilerOutShown:'0',execArgs:'',execStdin:'',fontScale:14,fontUsePx:'0',j:1,lang:c%2B%2B,libs:!(),options:'-std%3Dc%2B%2B20+-Wall+-Wextra+-pedantic',overrides:!(),runtimeTools:!(),source:1,stdinPanelShown:'1',tree:'1',wrap:'0'),l:'5',n:'0',o:'Executor+x86-64+gcc+(trunk)+(C%2B%2B,+Editor+%231)',t:'0')),header:(),k:50,l:'4',n:'0',o:'',s:0,t:'0')),l:'2',n:'0',o:'',t:'0')),version:4}
	\end{codesample}

	\addproposal{P0645}{https://wg21.link/P0645R10}
\end{frame}

\begin{frame}[fragile]
	\frametitle{\mintinline[style=white]{cpp}|std::format| -- Zéros non significatifs}
	\begin{itemize}
		\item Affichage des zéros non significatifs
	\end{itemize}

	\begin{minted}{cpp}
		format("{:+06d}", 120);  // "+00120"
	\end{minted}

	\begin{codesample}
		\sample{https://godbolt.org/#g:!((g:!((g:!((h:codeEditor,i:(filename:'1',fontScale:14,fontUsePx:'0',j:1,lang:c%2B%2B,selection:(endColumn:1,endLineNumber:1,positionColumn:1,positionLineNumber:1,selectionStartColumn:1,selectionStartLineNumber:1,startColumn:1,startLineNumber:1),source:'%23include+%3Ciostream%3E%0A%23include+%3Cformat%3E%0A%0Aint+main()%0A%7B%0A++std::cout+%3C%3C+std::format(%22%7B:%2B06d%7D%22,+120)+%3C%3C+%22%5Cn%22%3B%0A%7D%0A'),l:'5',n:'0',o:'C%2B%2B+source+%231',t:'0')),k:50,l:'4',n:'0',o:'',s:0,t:'0'),(g:!((h:executor,i:(argsPanelShown:'1',compilationPanelShown:'0',compiler:gsnapshot,compilerName:'',compilerOutShown:'0',execArgs:'',execStdin:'',fontScale:14,fontUsePx:'0',j:1,lang:c%2B%2B,libs:!(),options:'-std%3Dc%2B%2B20+-Wall+-Wextra+-pedantic',overrides:!(),runtimeTools:!(),source:1,stdinPanelShown:'1',tree:'1',wrap:'0'),l:'5',n:'0',o:'Executor+x86-64+gcc+(trunk)+(C%2B%2B,+Editor+%231)',t:'0')),header:(),k:50,l:'4',n:'0',o:'',s:0,t:'0')),l:'2',n:'0',o:'',t:'0')),version:4}
	\end{codesample}

	\addproposal{P0645}{https://wg21.link/P0645R10}
\end{frame}

\begin{frame}[fragile]
	\frametitle{\mintinline[style=white]{cpp}|std::format| -- Format}
	\begin{itemize}
		\item Entiers : décimal, octal, binaire ou hexadécimal
	\end{itemize}

	\begin{minted}{cpp}
		format("{:d}", 42);           // "42"
		format("{:x} {:X}", 42, 42);  // "2a 2A"
		format("{:b}", 42);           // "101010"
		format("{:o}", 42);           // "52"
	\end{minted}

	\begin{itemize}
		\item Caractères : valeur numérique ou caractère
	\end{itemize}

	\begin{minted}{cpp}
		format("{:X}", 'A');          // "41"
		format("{:c}", 'A');          // "A"
	\end{minted}

	\begin{itemize}
		\item Booléens : chaîne ou nombre
	\end{itemize}

	\begin{minted}{cpp}
		format("{:d}", true);         // "1"
		format("{:s}", true);         // "true"
	\end{minted}

	\begin{codesample}
		\sample{https://godbolt.org/#g:!((g:!((g:!((h:codeEditor,i:(filename:'1',fontScale:14,fontUsePx:'0',j:1,lang:c%2B%2B,selection:(endColumn:1,endLineNumber:1,positionColumn:1,positionLineNumber:1,selectionStartColumn:1,selectionStartLineNumber:1,startColumn:1,startLineNumber:1),source:'%23include+%3Ciostream%3E%0A%23include+%3Cformat%3E%0A%0Aint+main()%0A%7B%0A++std::cout+%3C%3C+std::format(%22%7B:d%7D%22,+true)+%3C%3C+%22%5Cn%22%3B%0A++std::cout+%3C%3C+std::format(%22%7B:s%7D%22,+true)+%3C%3C+%22%5Cn%22%3B%0A%7D%0A'),l:'5',n:'0',o:'C%2B%2B+source+%231',t:'0')),k:50,l:'4',n:'0',o:'',s:0,t:'0'),(g:!((h:executor,i:(argsPanelShown:'1',compilationPanelShown:'0',compiler:gsnapshot,compilerName:'',compilerOutShown:'0',execArgs:'',execStdin:'',fontScale:14,fontUsePx:'0',j:1,lang:c%2B%2B,libs:!(),options:'-std%3Dc%2B%2B20+-Wall+-Wextra+-pedantic',overrides:!(),runtimeTools:!(),source:1,stdinPanelShown:'1',tree:'1',wrap:'0'),l:'5',n:'0',o:'Executor+x86-64+gcc+(trunk)+(C%2B%2B,+Editor+%231)',t:'0')),header:(),k:50,l:'4',n:'0',o:'',s:0,t:'0')),l:'2',n:'0',o:'',t:'0')),version:4}
	\end{codesample}

	\begin{codesample}
		\sample{https://godbolt.org/#g:!((g:!((g:!((h:codeEditor,i:(filename:'1',fontScale:14,fontUsePx:'0',j:1,lang:c%2B%2B,selection:(endColumn:1,endLineNumber:1,positionColumn:1,positionLineNumber:1,selectionStartColumn:1,selectionStartLineNumber:1,startColumn:1,startLineNumber:1),source:'%23include+%3Ciostream%3E%0A%23include+%3Cformat%3E%0A%0Aint+main()%0A%7B%0A++std::cout+%3C%3C+std::format(%22%7B:X%7D%22,+!'A!')+%3C%3C+%22%5Cn%22%3B%0A++std::cout+%3C%3C+std::format(%22%7B:c%7D%22,+!'A!')+%3C%3C+%22%5Cn%22%3B%0A%7D%0A'),l:'5',n:'0',o:'C%2B%2B+source+%231',t:'0')),k:50,l:'4',n:'0',o:'',s:0,t:'0'),(g:!((h:executor,i:(argsPanelShown:'1',compilationPanelShown:'0',compiler:gsnapshot,compilerName:'',compilerOutShown:'0',execArgs:'',execStdin:'',fontScale:14,fontUsePx:'0',j:1,lang:c%2B%2B,libs:!(),options:'-std%3Dc%2B%2B20+-Wall+-Wextra+-pedantic',overrides:!(),runtimeTools:!(),source:1,stdinPanelShown:'1',tree:'1',wrap:'0'),l:'5',n:'0',o:'Executor+x86-64+gcc+(trunk)+(C%2B%2B,+Editor+%231)',t:'0')),header:(),k:50,l:'4',n:'0',o:'',s:0,t:'0')),l:'2',n:'0',o:'',t:'0')),version:4}
	\end{codesample}

	\begin{codesample}
		\sample{https://godbolt.org/#g:!((g:!((g:!((h:codeEditor,i:(filename:'1',fontScale:14,fontUsePx:'0',j:1,lang:c%2B%2B,selection:(endColumn:1,endLineNumber:1,positionColumn:1,positionLineNumber:1,selectionStartColumn:1,selectionStartLineNumber:1,startColumn:1,startLineNumber:1),source:'%23include+%3Ciostream%3E%0A%23include+%3Cformat%3E%0A%0Aint+main()%0A%7B%0A++std::cout+%3C%3C+std::format(%22%7B:d%7D%22,+42)+%3C%3C+%22%5Cn%22%3B%0A++std::cout+%3C%3C+std::format(%22%7B:x%7D+%7B:X%7D%22,+42,+42)+%3C%3C+%22%5Cn%22%3B%0A++std::cout+%3C%3C+std::format(%22%7B:b%7D%22,+42)+%3C%3C+%22%5Cn%22%3B%0A++std::cout+%3C%3C+std::format(%22%7B:o%7D%22,+42)+%3C%3C+%22%5Cn%22%3B%0A%7D%0A'),l:'5',n:'0',o:'C%2B%2B+source+%231',t:'0')),k:50,l:'4',n:'0',o:'',s:0,t:'0'),(g:!((h:executor,i:(argsPanelShown:'1',compilationPanelShown:'0',compiler:gsnapshot,compilerName:'',compilerOutShown:'0',execArgs:'',execStdin:'',fontScale:14,fontUsePx:'0',j:1,lang:c%2B%2B,libs:!(),options:'-std%3Dc%2B%2B20+-Wall+-Wextra+-pedantic',overrides:!(),runtimeTools:!(),source:1,stdinPanelShown:'1',tree:'1',wrap:'0'),l:'5',n:'0',o:'Executor+x86-64+gcc+(trunk)+(C%2B%2B,+Editor+%231)',t:'0')),header:(),k:50,l:'4',n:'0',o:'',s:0,t:'0')),l:'2',n:'0',o:'',t:'0')),version:4}
	\end{codesample}

	\addproposal{P0645}{https://wg21.link/P0645R10}
\end{frame}

\begin{frame}[fragile]
	\frametitle{\mintinline[style=white]{cpp}|std::format| -- Format}
	\begin{itemize}
		\item Flottants : fixe, court, scientifique ou hexadécimal
	\end{itemize}

	\begin{minted}{cpp}
		format("{:.6f}", 392.65);   // "392.650000"
		format("{:.6g}", 392.65);   // "392.65"
		format("{:.6e}", 392.65);   // "3.9265e+02"
		format("{:.6E}", 392.65);   // "3.9265E+02"
		format("{:.6a}", 42.);      // "1.500000p+5"
	\end{minted}

\note[item]{Format court : fixe sur les petites nombres, scientifique sur les grands}
\note[item]{Par défaut court : .6g}

	\begin{itemize}
		\item Chaîne de caractère
	\end{itemize}

	\begin{minted}{cpp}
		format("{:s}", "azerty");   // "azerty"
	\end{minted}

	\begin{codesample}
		\sample{https://godbolt.org/#g:!((g:!((g:!((h:codeEditor,i:(filename:'1',fontScale:14,fontUsePx:'0',j:1,lang:c%2B%2B,selection:(endColumn:1,endLineNumber:1,positionColumn:1,positionLineNumber:1,selectionStartColumn:1,selectionStartLineNumber:1,startColumn:1,startLineNumber:1),source:'%23include+%3Ciostream%3E%0A%23include+%3Cformat%3E%0A%0Aint+main()%0A%7B%0A++std::cout+%3C%3C+std::format(%22%7B:s%7D%22,+%22azerty%22)+%3C%3C+%22%5Cn%22%3B%0A%7D%0A'),l:'5',n:'0',o:'C%2B%2B+source+%231',t:'0')),k:50,l:'4',n:'0',o:'',s:0,t:'0'),(g:!((h:executor,i:(argsPanelShown:'1',compilationPanelShown:'0',compiler:gsnapshot,compilerName:'',compilerOutShown:'0',execArgs:'',execStdin:'',fontScale:14,fontUsePx:'0',j:1,lang:c%2B%2B,libs:!(),options:'-std%3Dc%2B%2B20+-Wall+-Wextra+-pedantic',overrides:!(),runtimeTools:!(),source:1,stdinPanelShown:'1',tree:'1',wrap:'0'),l:'5',n:'0',o:'Executor+x86-64+gcc+(trunk)+(C%2B%2B,+Editor+%231)',t:'0')),header:(),k:50,l:'4',n:'0',o:'',s:0,t:'0')),l:'2',n:'0',o:'',t:'0')),version:4}
	\end{codesample}

	\begin{codesample}
		\sample{https://godbolt.org/#g:!((g:!((g:!((h:codeEditor,i:(filename:'1',fontScale:14,fontUsePx:'0',j:1,lang:c%2B%2B,selection:(endColumn:1,endLineNumber:1,positionColumn:1,positionLineNumber:1,selectionStartColumn:1,selectionStartLineNumber:1,startColumn:1,startLineNumber:1),source:'%23include+%3Ciostream%3E%0A%23include+%3Cformat%3E%0A%0Aint+main()%0A%7B%0A++std::cout+%3C%3C+std::format(%22%7B:.6f%7D%22,+392.65)+%3C%3C+%22%5Cn%22%3B%0A++std::cout+%3C%3C+std::format(%22%7B:.6g%7D%22,+392.65)+%3C%3C+%22%5Cn%22%3B%0A++std::cout+%3C%3C+std::format(%22%7B:.6e%7D%22,+392.65)+%3C%3C+%22%5Cn%22%3B%0A++std::cout+%3C%3C+std::format(%22%7B:.6E%7D%22,+392.65)+%3C%3C+%22%5Cn%22%3B%0A++std::cout+%3C%3C+std::format(%22%7B:.6a%7D%22,+42.)+%3C%3C+%22%5Cn%22%3B%0A%7D%0A'),l:'5',n:'0',o:'C%2B%2B+source+%231',t:'0')),k:50,l:'4',n:'0',o:'',s:0,t:'0'),(g:!((h:executor,i:(argsPanelShown:'1',compilationPanelShown:'0',compiler:gsnapshot,compilerName:'',compilerOutShown:'0',execArgs:'',execStdin:'',fontScale:14,fontUsePx:'0',j:1,lang:c%2B%2B,libs:!(),options:'-std%3Dc%2B%2B20+-Wall+-Wextra+-pedantic',overrides:!(),runtimeTools:!(),source:1,stdinPanelShown:'1',tree:'1',wrap:'0'),l:'5',n:'0',o:'Executor+x86-64+gcc+(trunk)+(C%2B%2B,+Editor+%231)',t:'0')),header:(),k:50,l:'4',n:'0',o:'',s:0,t:'0')),l:'2',n:'0',o:'',t:'0')),version:4}
	\end{codesample}

	\addproposal{P0645}{https://wg21.link/P0645R10}
\end{frame}

\begin{frame}[fragile]
	\frametitle{\mintinline[style=white]{cpp}|std::format| -- Forme alternative}
	\begin{itemize}
		\item Affichage de la base des entiers
	\end{itemize}

	\begin{minted}{cpp}
		format("{:#x}", 42);        // "0x2a"
		format("{:#X}", 42);        // "0X2A"
	\end{minted}

	\begin{itemize}
		\item Affichage du point décimal et de l'ensemble de la précision des flottants
	\end{itemize}

	\begin{minted}{cpp}
		format("{:.6g}", 392.65);   // "392.65"
		format("{:#.6g}", 392.65);  // "392.650"
	\end{minted}

	\begin{codesample}
		\sample{https://godbolt.org/#g:!((g:!((g:!((h:codeEditor,i:(filename:'1',fontScale:14,fontUsePx:'0',j:1,lang:c%2B%2B,selection:(endColumn:1,endLineNumber:1,positionColumn:1,positionLineNumber:1,selectionStartColumn:1,selectionStartLineNumber:1,startColumn:1,startLineNumber:1),source:'%23include+%3Ciostream%3E%0A%23include+%3Cformat%3E%0A%0Aint+main()%0A%7B%0A++std::cout+%3C%3C+std::format(%22%7B:.6g%7D%22,+392.65)+%3C%3C+%22%5Cn%22%3B%0A++std::cout+%3C%3C+std::format(%22%7B:%23.6g%7D%22,+392.65)+%3C%3C+%22%5Cn%22%3B%0A%7D%0A'),l:'5',n:'0',o:'C%2B%2B+source+%231',t:'0')),k:50,l:'4',n:'0',o:'',s:0,t:'0'),(g:!((h:executor,i:(argsPanelShown:'1',compilationPanelShown:'0',compiler:gsnapshot,compilerName:'',compilerOutShown:'0',execArgs:'',execStdin:'',fontScale:14,fontUsePx:'0',j:1,lang:c%2B%2B,libs:!(),options:'-std%3Dc%2B%2B20+-Wall+-Wextra+-pedantic',overrides:!(),runtimeTools:!(),source:1,stdinPanelShown:'1',tree:'1',wrap:'0'),l:'5',n:'0',o:'Executor+x86-64+gcc+(trunk)+(C%2B%2B,+Editor+%231)',t:'0')),header:(),k:50,l:'4',n:'0',o:'',s:0,t:'0')),l:'2',n:'0',o:'',t:'0')),version:4}
	\end{codesample}

	\begin{codesample}
		\sample{https://godbolt.org/#g:!((g:!((g:!((h:codeEditor,i:(filename:'1',fontScale:14,fontUsePx:'0',j:1,lang:c%2B%2B,selection:(endColumn:1,endLineNumber:1,positionColumn:1,positionLineNumber:1,selectionStartColumn:1,selectionStartLineNumber:1,startColumn:1,startLineNumber:1),source:'%23include+%3Ciostream%3E%0A%23include+%3Cformat%3E%0A%0Aint+main()%0A%7B%0A++std::cout+%3C%3C+std::format(%22%7B:%23x%7D%22,+42)+%3C%3C+%22%5Cn%22%3B%0A++std::cout+%3C%3C+std::format(%22%7B:%23X%7D%22,+42)+%3C%3C+%22%5Cn%22%3B%0A%7D%0A'),l:'5',n:'0',o:'C%2B%2B+source+%231',t:'0')),k:50,l:'4',n:'0',o:'',s:0,t:'0'),(g:!((h:executor,i:(argsPanelShown:'1',compilationPanelShown:'0',compiler:gsnapshot,compilerName:'',compilerOutShown:'0',execArgs:'',execStdin:'',fontScale:14,fontUsePx:'0',j:1,lang:c%2B%2B,libs:!(),options:'-std%3Dc%2B%2B20+-Wall+-Wextra+-pedantic',overrides:!(),runtimeTools:!(),source:1,stdinPanelShown:'1',tree:'1',wrap:'0'),l:'5',n:'0',o:'Executor+x86-64+gcc+(trunk)+(C%2B%2B,+Editor+%231)',t:'0')),header:(),k:50,l:'4',n:'0',o:'',s:0,t:'0')),l:'2',n:'0',o:'',t:'0')),version:4}
	\end{codesample}

	\addproposal{P0645}{https://wg21.link/P0645R10}
\end{frame}

\begin{frame}[fragile]
	\frametitle{\mintinline[style=white]{cpp}|std::format| -- Dates et heures}
	\begin{itemize}
		\item Format basé sur \mintinline{cpp}|strftime|
		\begin{itemize}
			\item \mintinline{cpp}|%y| : année sur deux digits
			\item \mintinline{cpp}|%m| : mois
			\item \mintinline{cpp}|%d| : jour dans le mois
			\item \mintinline{cpp}|%u|, \mintinline{cpp}|%w| : jour dans la semaine

\note[item]{1 à 7 avec 1 lundi pour u, 0 à 6 avec 0 pour dimanche pour w}

			\item \mintinline{cpp}|%H|, \mintinline{cpp}|%I| : heure (format 24h ou 12h)
			\item \mintinline{cpp}|%M| : minutes
			\item \mintinline{cpp}|%S| : secondes
			\item \mintinline{cpp}|%Z| : timezone
			\item \ldots{}
		\end{itemize}
	\end{itemize}

	\begin{minted}{cpp}
		format("{:%F %T}", chrono::system_clock::now());
		// AAAA-MM-JJ HH:mm:ss
	\end{minted}

	\begin{codesample}
		\sample{https://godbolt.org/#g:!((g:!((g:!((h:codeEditor,i:(filename:'1',fontScale:14,fontUsePx:'0',j:1,lang:c%2B%2B,selection:(endColumn:1,endLineNumber:1,positionColumn:1,positionLineNumber:1,selectionStartColumn:1,selectionStartLineNumber:1,startColumn:1,startLineNumber:1),source:'%23include+%3Ciostream%3E%0A%23include+%3Cchrono%3E%0A%23include+%3Cformat%3E%0A%0Aint+main()%0A%7B%0A++std::cout+%3C%3C+std::format(%22%7B:%25F+%25T%7D%22,+std::chrono::system_clock::now())%3B%0A%7D%0A'),l:'5',n:'0',o:'C%2B%2B+source+%231',t:'0')),k:50,l:'4',n:'0',o:'',s:0,t:'0'),(g:!((h:executor,i:(argsPanelShown:'1',compilationPanelShown:'0',compiler:gsnapshot,compilerName:'',compilerOutShown:'0',execArgs:'',execStdin:'',fontScale:14,fontUsePx:'0',j:1,lang:c%2B%2B,libs:!(),options:'-std%3Dc%2B%2B20+-Wall+-Wextra+-pedantic',overrides:!(),runtimeTools:!(),source:1,stdinPanelShown:'1',tree:'1',wrap:'0'),l:'5',n:'0',o:'Executor+x86-64+gcc+(trunk)+(C%2B%2B,+Editor+%231)',t:'0')),header:(),k:50,l:'4',n:'0',o:'',s:0,t:'0')),l:'2',n:'0',o:'',t:'0')),version:4}
	\end{codesample}

	\addproposal{P0645}{https://wg21.link/P0645R10}
\end{frame}

\begin{frame}[fragile]
	\frametitle{\mintinline[style=white]{cpp}|std::format| -- Gestion des erreurs}
	\begin{itemize}
		\item Exception \mintinline{cpp}|std::format_error|
		\begin{itemize}
			\item Chaîne de format invalide
			\item Spécificateurs non cohérents avec le type fournit
			\item Absence de valeur
			\item Exception levée par un formateur
		\end{itemize}
	\end{itemize}

	\begin{block}{Valeur surnuméraire}
		\begin{itemize}
			\item Les valeurs surnuméraires ne sont pas des erreurs et sont ignorées
		\end{itemize}
	\end{block}

	\addproposal{P0645}{https://wg21.link/P0645R10}
\end{frame}

\begin{frame}[fragile]
	\frametitle{\mintinline[style=white]{cpp}|std::format| -- Types utilisateur}
	\begin{itemize}
		\item Par spécialisation de \mintinline{cpp}|std::formatter<>|
	\end{itemize}

	\begin{minted}{cpp}
		template<>
		struct formatter<T> {
		  template <class ParseContext>
		  auto parse(ParseContext& parse_ctx);

		  template <class FormatContext>
		  auto format(const T& value, FormatContext& fmt_ctx);
		};
	\end{minted}

	\addproposal{P0645}{https://wg21.link/P0645R10}
\end{frame}

\begin{frame}[fragile]
	\frametitle{\mintinline[style=white]{cpp}|std::format| -- Types utilisateur}
	\begin{minted}{cpp}
		struct MyComplex { double real; double imag; };

		template <>
		struct formatter<MyComplex> {
		  constexpr auto parse(format_parse_context& ctx) {
		    return ctx.begin();
		  }

		  auto format(const MyComplex& value, format_context& ctx) const {
		    return format_to(ctx.out(), "{}+{}i", value.real, value.imag);
		  }
		};

		format("{}", MyComplex{1, 2});  // "1+2i"
	\end{minted}

	\begin{codesample}
		\sample{https://godbolt.org/#g:!((g:!((g:!((h:codeEditor,i:(filename:'1',fontScale:14,fontUsePx:'0',j:1,lang:c%2B%2B,selection:(endColumn:1,endLineNumber:1,positionColumn:1,positionLineNumber:1,selectionStartColumn:1,selectionStartLineNumber:1,startColumn:1,startLineNumber:1),source:'%23include+%3Ciostream%3E%0A%23include+%3Cformat%3E%0A%0Astruct+MyComplex%0A%7B%0A++double+real%3B%0A++double+imag%3B%0A%7D%3B%0A%0Atemplate%3C%3E%0Astruct+std::formatter%3CMyComplex%3E%0A%7B%0A++++constexpr+auto+parse(std::format_parse_context%26+ctx)%0A++++%7B%0A++++++++return+ctx.begin()%3B%0A++++%7D%0A+%0A++++auto+format(const+MyComplex%26+value,+std::format_context%26+ctx)+const%0A++++%7B%0A++++++++return+std::format_to(ctx.out(),+%22%7B%7D%2B%7B%7Di%22,+value.real,+value.imag)%3B%0A++++%7D%0A%7D%3B%0A%0Aint+main()%0A%7B%0A++++std::cout+%3C%3C+std::format(%22%7B%7D%22,+MyComplex%7B1,+2%7D)+%3C%3C+%22%5Cn%22%3B%0A%7D'),l:'5',n:'0',o:'C%2B%2B+source+%231',t:'0')),k:50,l:'4',n:'0',o:'',s:0,t:'0'),(g:!((h:executor,i:(argsPanelShown:'1',compilationPanelShown:'0',compiler:gsnapshot,compilerName:'',compilerOutShown:'0',execArgs:'',execStdin:'',fontScale:14,fontUsePx:'0',j:1,lang:c%2B%2B,libs:!(),options:'-std%3Dc%2B%2B20+-Wall+-Wextra+-pedantic',overrides:!(),runtimeTools:!(),source:1,stdinPanelShown:'1',tree:'1',wrap:'0'),l:'5',n:'0',o:'Executor+x86-64+gcc+(trunk)+(C%2B%2B,+Editor+%231)',t:'0')),header:(),k:50,l:'4',n:'0',o:'',s:0,t:'0')),l:'2',n:'0',o:'',t:'0')),version:4}
	\end{codesample}

	\addproposal{P0645}{https://wg21.link/P0645R10}
\end{frame}

\subsection*{Mémoire}
\begin{frame}[fragile]
	\frametitle{Tableaux}
	\begin{itemize}
		\item Support des tableaux par \mintinline{cpp}|std::make_shared()|
	\end{itemize}

	\begin{minted}{cpp}
		shared_ptr<double[]> foo = make_shared<double[]>(1024);
	\end{minted}

	\begin{itemize}
		\item Déduction de la taille des tableaux par \mintinline{cpp}|new()|
	\end{itemize}

	\begin{minted}{cpp}
		double* a = new double[]{1, 2, 3};
	\end{minted}

	\begin{codesample}
		\sample{https://godbolt.org/#g:!((g:!((g:!((h:codeEditor,i:(filename:'1',fontScale:14,fontUsePx:'0',j:1,lang:c%2B%2B,selection:(endColumn:1,endLineNumber:1,positionColumn:1,positionLineNumber:1,selectionStartColumn:1,selectionStartLineNumber:1,startColumn:1,startLineNumber:1),source:'%23include+%3Ciostream%3E%0A%0Aint+main()%0A%7B%0A++double*+foo+%3D+new+double%5B%5D%7B1,+2,+3%7D%3B%0A++delete%5B%5D+foo%3B%0A%7D%0A'),l:'5',n:'0',o:'C%2B%2B+source+%231',t:'0')),k:50,l:'4',n:'0',o:'',s:0,t:'0'),(g:!((h:executor,i:(argsPanelShown:'1',compilationPanelShown:'0',compiler:gsnapshot,compilerName:'',compilerOutShown:'0',execArgs:'',execStdin:'',fontScale:14,fontUsePx:'0',j:1,lang:c%2B%2B,libs:!(),options:'-std%3Dc%2B%2B20+-Wall+-Wextra+-pedantic',overrides:!(),runtimeTools:!(),source:1,stdinPanelShown:'1',tree:'1',wrap:'0'),l:'5',n:'0',o:'Executor+x86-64+gcc+(trunk)+(C%2B%2B,+Editor+%231)',t:'0')),header:(),k:50,l:'4',n:'0',o:'',s:0,t:'0')),l:'2',n:'0',o:'',t:'0')),version:4}
	\end{codesample}

	\begin{codesample}
		\sample{https://godbolt.org/#g:!((g:!((g:!((h:codeEditor,i:(filename:'1',fontScale:14,fontUsePx:'0',j:1,lang:c%2B%2B,selection:(endColumn:1,endLineNumber:1,positionColumn:1,positionLineNumber:1,selectionStartColumn:1,selectionStartLineNumber:1,startColumn:1,startLineNumber:1),source:'%23include+%3Ciostream%3E%0A%23include+%3Cmemory%3E%0A%0Aint+main()%0A%7B%0A++std::shared_ptr%3Cdouble%5B%5D%3E+foo+%3D+std::make_shared%3Cdouble%5B%5D%3E(1024)%3B%0A%7D%0A'),l:'5',n:'0',o:'C%2B%2B+source+%231',t:'0')),k:50,l:'4',n:'0',o:'',s:0,t:'0'),(g:!((h:executor,i:(argsPanelShown:'1',compilationPanelShown:'0',compiler:gsnapshot,compilerName:'',compilerOutShown:'0',execArgs:'',execStdin:'',fontScale:14,fontUsePx:'0',j:1,lang:c%2B%2B,libs:!(),options:'-std%3Dc%2B%2B20+-Wall+-Wextra+-pedantic',overrides:!(),runtimeTools:!(),source:1,stdinPanelShown:'1',tree:'1',wrap:'0'),l:'5',n:'0',o:'Executor+x86-64+gcc+(trunk)+(C%2B%2B,+Editor+%231)',t:'0')),header:(),k:50,l:'4',n:'0',o:'',s:0,t:'0')),l:'2',n:'0',o:'',t:'0')),version:4}
	\end{codesample}

	\addproposal{P0674}{https://wg21.link/P0674R1}
	\addproposal{P1009}{https://wg21.link/P1009R2}
\end{frame}

\begin{frame}[fragile]
	\frametitle{Destruction}
	\begin{itemize}
		\item \mintinline{cpp}|std::destroying_delete_t| : pas de destruction avant l'appel à \mintinline{cpp}|delete()|
	\end{itemize}

	\begin{block}{Intérêt}
		\begin{itemize}
			\item Conserver des informations nécessaire à la libération
		\end{itemize}
	\end{block}

	\begin{minted}{cpp}
		struct Foo {
		  void operator delete(Foo* ptr, destroying_delete_t) {
		    const size_t realSize = ...;
		    ptr->~Foo();
		    ::operator delete(ptr, realSize);
		  }
		};
	\end{minted}

	\begin{alertblock}{Ne pas oublier}
		\begin{itemize}
			\item La destruction doit être appelée explicitement
		\end{itemize}
	\end{alertblock}

	\addproposal{P0722}{https://wg21.link/P0722R3}
\end{frame}

\subsection*{Date et heure}
\begin{frame}[fragile]
	\frametitle{Horloges}
	\begin{itemize}
		\item Nouvelles horloges
		\begin{itemize}
			\shorthandoff{:}
			\item \mintinline{cpp}|std::chrono::utc_clock|
			\begin{itemize}
				\item Temps universel coordonné
				\item Epoch : 1 janvier 1970 00:00:00
				\item Support des secondes intercalaires
			\end{itemize}
			\item \mintinline{cpp}|std::chrono::gps_clock|
			\begin{itemize}
				\item Epoch : 6 janvier 1980 00:00:00 UTC
				\item Pas de seconde intercalaire
			\end{itemize}
			\item \mintinline{cpp}|std::chrono::tai_clock|
			\begin{itemize}
				\item Temps atomique universel
				\item Epoch : 31 décembre 1957 23:59:50 UTC
				\item Pas de seconde intercalaire
			\end{itemize}
			\item \mintinline{cpp}|std::chrono::file_clock| : alias vers le temps du système de fichier
			\shorthandon{:}
		\end{itemize}
	\end{itemize}

	\begin{codesample}
		\sample{https://godbolt.org/#g:!((g:!((g:!((h:codeEditor,i:(filename:'1',fontScale:14,fontUsePx:'0',j:1,lang:c%2B%2B,selection:(endColumn:1,endLineNumber:1,positionColumn:1,positionLineNumber:1,selectionStartColumn:1,selectionStartLineNumber:1,startColumn:1,startLineNumber:1),source:'%23include+%3Ciostream%3E%0A%23include+%3Cchrono%3E%0A%23include+%3Cformat%3E%0A%0Aint+main()%0A%7B%0A++std::cout+%3C%3C+std::format(%22%7B:%25F+%25T%7D%22,+std::chrono::utc_clock::now())+%3C%3C+%22%5Cn%22%3B%0A++std::cout+%3C%3C+std::format(%22%7B:%25F+%25T%7D%22,+std::chrono::gps_clock::now())+%3C%3C+%22%5Cn%22%3B%0A++std::cout+%3C%3C+std::format(%22%7B:%25F+%25T%7D%22,+std::chrono::tai_clock::now())+%3C%3C+%22%5Cn%22%3B%0A++std::cout+%3C%3C+std::format(%22%7B:%25F+%25T%7D%22,+std::chrono::file_clock::now())+%3C%3C+%22%5Cn%22%3B%0A%7D%0A'),l:'5',n:'0',o:'C%2B%2B+source+%231',t:'0')),k:50,l:'4',n:'0',o:'',s:0,t:'0'),(g:!((h:executor,i:(argsPanelShown:'1',compilationPanelShown:'0',compiler:gsnapshot,compilerName:'',compilerOutShown:'0',execArgs:'',execStdin:'',fontScale:14,fontUsePx:'0',j:1,lang:c%2B%2B,libs:!(),options:'-std%3Dc%2B%2B20+-Wall+-Wextra+-pedantic',overrides:!(),runtimeTools:!(),source:1,stdinPanelShown:'1',tree:'1',wrap:'0'),l:'5',n:'0',o:'Executor+x86-64+gcc+(trunk)+(C%2B%2B,+Editor+%231)',t:'0')),header:(),k:50,l:'4',n:'0',o:'',s:0,t:'0')),l:'2',n:'0',o:'',t:'0')),version:4}
	\end{codesample}

	\addproposal{P0355}{https://wg21.link/P0355R7}
\end{frame}

\begin{frame}[fragile]
	\frametitle{Horloges}
	\begin{itemize}
		\item Conversion des horloges vers et depuis UTC
		\item Conversion de \mintinline{cpp}|std::chrono::utc_clock| vers et depuis le temps système
		\item Conversion des horloges entre-elles
	\end{itemize}

	\begin{alertblock}{Conversion de \mintinline{cpp}|std::chrono::file_clock|}
		\begin{itemize}
			\item Support optionnel des conversions entre \mintinline{cpp}|std::chrono::file_clock| et \mintinline{cpp}|std::chrono::utc_clock| ou \mintinline{cpp}|std::chrono::system_clock|
		\end{itemize}
	\end{alertblock}

	\begin{itemize}
		\item Pseudo-horloge \mintinline{cpp}|std::chrono::local_t| temps dans la \textit{timezone} locale
	\end{itemize}

	\addproposal{P0355}{https://wg21.link/P0355R7}
\end{frame}

\begin{frame}[fragile]
	\frametitle{Évolution de \mintinline[style=white]{cpp}|std::chrono::duration|}
	\begin{itemize}
		\item \textit{Helper} pour le jour, la semaine, le mois ou l'année
		\item \mintinline{cpp}|from_stream()| lit une \mintinline{cpp}|std::chrono::duration|
		\item Utilisation de chaîne de format utilisant des séquences préfixées par \mintinline{cpp}|%|
		\begin{itemize}
			\item \mintinline{cpp}|%H|,\mintinline{cpp}|%I| : heure (format 24h ou 12h)
			\item \mintinline{cpp}|%M| : minutes
			\item \mintinline{cpp}|%S| : secondes
			\item \mintinline{cpp}|%Y|, \mintinline{cpp}|%y| : année (4 ou 2 chiffres)
			\item \mintinline{cpp}|%m| : numéro du mois
			\item \mintinline{cpp}|%b|, \mintinline{cpp}|%B| : nom du mois dans la locale (abrégé ou complet)
			\item \mintinline{cpp}|%d| : numéro du jour dans le mois
			\item \mintinline{cpp}|%U| : numéro de la semaine
			\item \mintinline{cpp}|%Z| : abréviation de la \textit{timezone}
			\item \ldots{}

\note[item]{Similaire à \mintinline{cpp}|strftime()|}
\note[item]{Et identique à ceux utilisés dans \mintinline{cpp}|std::format|}
		\end{itemize}
	\end{itemize}

	\addproposal{P0355}{https://wg21.link/P0355R7}
\end{frame}

\begin{frame}[fragile]
	\frametitle{Calendrier}
	\begin{itemize}
		\item Gestion du calendrier grégorien
		\begin{itemize}
			\item Différentes représentations
			\begin{itemize}
				\item Année, mois
				\item Jour dans l'année, dans le mois
				\item Dernier jour du mois
				\item Jour dans la semaine, $n\textsuperscript{e}$ jour de la semaine dans le mois
			\end{itemize}
		\end{itemize}
	\end{itemize}

	\begin{alertblock}{Convention anglo-saxonne}
		\begin{itemize}
			\item Le premier jour de la semaine est le dimanche
		\end{itemize}
	\end{alertblock}

	\begin{itemize}
		\item [] \begin{itemize}
			\item [] \begin{itemize}
				\item Et différentes combinaisons permettant de construire une date complète
			\end{itemize}
		\end{itemize}
	\end{itemize}

	\addproposal{P0355}{https://wg21.link/P0355R7}
\end{frame}

\begin{frame}[fragile]
	\frametitle{Calendrier}
	\begin{itemize}
		\item [] \begin{itemize}
			\item Constantes représentant les jours de la semaine et les mois
			\item Suffixes littéraux \mintinline{cpp}|y| et \mintinline{cpp}|d| pour les années et les jours
			\item \mintinline{cpp}|operator/| pour construire une date depuis un format humain
		\end{itemize}
	\end{itemize}

	\begin{minted}{cpp}
auto date1 = 2016y/May/29d;
auto date2 = Sunday[3]/May/2016y;
	\end{minted}

	\begin{codesample}
		\sample{https://godbolt.org/#g:!((g:!((g:!((h:codeEditor,i:(filename:'1',fontScale:14,fontUsePx:'0',j:1,lang:c%2B%2B,selection:(endColumn:1,endLineNumber:1,positionColumn:1,positionLineNumber:1,selectionStartColumn:1,selectionStartLineNumber:1,startColumn:1,startLineNumber:1),source:'%23include+%3Ciostream%3E%0A%23include+%3Cchrono%3E%0A%23include+%3Cformat%3E%0A%0Ausing+namespace+std::literals::chrono_literals%3B%0A%0Aint+main()%0A%7B%0A++auto+date1+%3D+2016y/std::chrono::May/29d%3B%0A++auto+date2+%3D+std::chrono::Sunday%5B3%5D/std::chrono::May/2016y%3B%0A%0A++std::cout+%3C%3C+std::format(%22%7B:%25F%7D%22,+date1)+%3C%3C+%22%5Cn%22%3B%0A++std::cout+%3C%3C+std::format(%22%7B:%25F%7D%22,+date2)+%3C%3C+%22%5Cn%22%3B%0A%7D%0A'),l:'5',n:'0',o:'C%2B%2B+source+%231',t:'0')),k:50,l:'4',n:'0',o:'',s:0,t:'0'),(g:!((h:executor,i:(argsPanelShown:'1',compilationPanelShown:'0',compiler:gsnapshot,compilerName:'',compilerOutShown:'0',execArgs:'',execStdin:'',fontScale:14,fontUsePx:'0',j:1,lang:c%2B%2B,libs:!(),options:'-std%3Dc%2B%2B20+-Wall+-Wextra+-pedantic',overrides:!(),runtimeTools:!(),source:1,stdinPanelShown:'1',tree:'1',wrap:'0'),l:'5',n:'0',o:'Executor+x86-64+gcc+(trunk)+(C%2B%2B,+Editor+%231)',t:'0')),header:(),k:50,l:'4',n:'0',o:'',s:0,t:'0')),l:'2',n:'0',o:'',t:'0')),version:4}
	\end{codesample}

	\addproposal{P0355}{https://wg21.link/P0355R7}
\end{frame}

\begin{frame}[fragile]
	\frametitle{Timezone}
	\begin{itemize}
		\item Gestion des \textit{timezones}
		\begin{itemize}
			\item Gestion de la base de \textit{timezones} de l'IANA

\note[item]{IANA : \textit{Internet Assigned Numbers Authority}}

			\item Récupération de la \textit{timezone} courante
			\item Recherche d'une \textit{timezone} depuis son nom
			\item Caractéristiques d'une \textit{timezone}
			\item Informations sur les secondes intercalaires
			\item Récupération du nom d'une \textit{timezone}
			\item Conversion entre \textit{timezone}
			\item Gestion des ambiguïté de conversion
		\end{itemize}
	\end{itemize}

	\begin{minted}{cpp}
		// 2016-05-29 07:30:06.153 UTC
		auto tp = sys_days{2016y/may/29d} + 7h + 30min + 6s + 153ms;
		// 2016-05-29 16:30:06.153 JST
		zoned_time zt = {"Asia/Tokyo", tp};
	\end{minted}

	\begin{codesample}
		\sample{https://godbolt.org/#g:!((g:!((g:!((h:codeEditor,i:(filename:'1',fontScale:14,fontUsePx:'0',j:1,lang:c%2B%2B,selection:(endColumn:1,endLineNumber:1,positionColumn:1,positionLineNumber:1,selectionStartColumn:1,selectionStartLineNumber:1,startColumn:1,startLineNumber:1),source:'%23include+%3Ciostream%3E%0A%23include+%3Cchrono%3E%0A%23include+%3Cformat%3E%0A%0Ausing+namespace+std::literals::chrono_literals%3B%0A%0Aint+main()%0A%7B%0A++auto+tp+%3D+std::chrono::sys_days%7B2016y/std::chrono::May/29d%7D+%2B+7h+%2B+30min+%2B+6s+%2B+153ms%3B+%0A++std::chrono::zoned_time+zt+%3D+%7B%22Asia/Tokyo%22,+tp%7D%3B%0A%0A++std::cout+%3C%3C+std::format(%22%7B:%25F+%25T+%25Z%7D%22,+tp)+%3C%3C+%22%5Cn%22%3B%0A++std::cout+%3C%3C+std::format(%22%7B:%25F+%25T+%25Z%7D%22,+zt)+%3C%3C+%22%5Cn%22%3B%0A%7D%0A'),l:'5',n:'0',o:'C%2B%2B+source+%231',t:'0')),k:50,l:'4',n:'0',o:'',s:0,t:'0'),(g:!((h:executor,i:(argsPanelShown:'1',compilationPanelShown:'0',compiler:gsnapshot,compilerName:'',compilerOutShown:'0',execArgs:'',execStdin:'',fontScale:14,fontUsePx:'0',j:1,lang:c%2B%2B,libs:!(),options:'-std%3Dc%2B%2B20+-Wall+-Wextra+-pedantic',overrides:!(),runtimeTools:!(),source:1,stdinPanelShown:'1',tree:'1',wrap:'0'),l:'5',n:'0',o:'Executor+x86-64+gcc+(trunk)+(C%2B%2B,+Editor+%231)',t:'0')),header:(),k:50,l:'4',n:'0',o:'',s:0,t:'0')),l:'2',n:'0',o:'',t:'0')),version:4}
	\end{codesample}

	\addproposal{P0355}{https://wg21.link/P0355R7}
\end{frame}

\begin{frame}[fragile]
	\frametitle{Date et heure}
	\begin{block}{En attendant C++20}
		\begin{itemize}
			\item Utilisez \mintinline{cpp}|Boost.DateTime|
		\end{itemize}
	\end{block}

	\begin{block}{Pour aller plus loin}
		\begin{itemize}
			\item \href{http://site.icu-project.org/home}{\mintinline{cpp}|ICU|\linklogo} supporte de nombreux calendriers et mécanismes de localisation
		\end{itemize}

\note[item]{ICU : \textit{International Components for Unicode}}
	\end{block}
\end{frame}

\subsection*{Range-based for loop}
\begin{frame}[fragile]
	\frametitle{Range-based for loop}
	\begin{itemize}
		\item Initialisation dans les range-based for loop
	\end{itemize}

	\begin{minted}{cpp}
		vector<int> foo{1, 8, 5, 56, 42};
		for(size_t i = 0; const auto& bar : foo) {
		  cout << bar << " " << i << "\n";
		  ++i;
		}
	\end{minted}

	\begin{itemize}
		\item Seuls des couples \mintinline{cpp}|begin()|, \mintinline{cpp}|end()| cohérents sont utilisés
		\begin{itemize}
			\item \og Début\fg{} et \og début + taille\fg{}
			\item fonctions membres \mintinline{cpp}|begin()| et \mintinline{cpp}|end()|
			\item fonctions libres \mintinline{cpp}|std::begin()| et \mintinline{cpp}|std::end()|
		\end{itemize}
	\end{itemize}

	\begin{block}{Intérêt}
		\begin{itemize}
			\item Itération (via des fonctions libres) d'éléments ayant une fonction membre \mintinline{cpp}|begin()| ou \mintinline{cpp}|end()| mais pas les deux
		\end{itemize}
	\end{block}

\note[item]{Fonction membre qui n'est probablement pas une fonction d'itération}
\note[item]{Auparavant la fonction membre était utilisé bien qu'incohérente}

	\begin{codesample}
		\sample{https://godbolt.org/#g:!((g:!((g:!((h:codeEditor,i:(filename:'1',fontScale:14,fontUsePx:'0',j:1,lang:c%2B%2B,selection:(endColumn:1,endLineNumber:1,positionColumn:1,positionLineNumber:1,selectionStartColumn:1,selectionStartLineNumber:1,startColumn:1,startLineNumber:1),source:'%23include+%3Ciostream%3E%0A%23include+%3Cvector%3E%0A%0Aint+main()%0A%7B%0A++std::vector%3Cint%3E+foo%7B1,+8,+5,+56,+42%7D%3B%0A++for(size_t+i+%3D+0%3B+const+auto%26+bar+:+foo)+%0A++%7B%0A++++std::cout+%3C%3C+bar+%3C%3C+%22+%22+%3C%3C+i+%3C%3C+%22%5Cn%22%3B%0A++++%2B%2Bi%3B+%0A++%7D%0A%7D%0A'),l:'5',n:'0',o:'C%2B%2B+source+%231',t:'0')),k:50,l:'4',n:'0',o:'',s:0,t:'0'),(g:!((h:executor,i:(argsPanelShown:'1',compilationPanelShown:'0',compiler:gsnapshot,compilerName:'',compilerOutShown:'0',execArgs:'',execStdin:'',fontScale:14,fontUsePx:'0',j:1,lang:c%2B%2B,libs:!(),options:'-std%3Dc%2B%2B20+-Wall+-Wextra+-pedantic',overrides:!(),runtimeTools:!(),source:1,stdinPanelShown:'1',tree:'1',wrap:'0'),l:'5',n:'0',o:'Executor+x86-64+gcc+(trunk)+(C%2B%2B,+Editor+%231)',t:'0')),header:(),k:50,l:'4',n:'0',o:'',s:0,t:'0')),l:'2',n:'0',o:'',t:'0')),version:4}
	\end{codesample}

	\addproposal{P0614}{https://wg21.link/P0614R1}
	\addproposal{P0962}{https://wg21.link/P0962R1}
\end{frame}

\subsection*{Compile-time}
\begin{frame}[fragile]
	\frametitle{\mintinline[style=white]{cpp}|consteval|}
	\begin{itemize}
		\item \mintinline{cpp}|consteval| impose une évaluation \textit{compile-time}

\note[item]{\mintinline{cpp}|constexpr| permet une évaluation \textit{compile-time} mais ne l'impose pas}

		\begin{itemize}
			\item \mintinline{cpp}|consteval| implique \mintinline{cpp}|inline|
		\end{itemize}
	\end{itemize}

	\begin{minted}{cpp}
		consteval int sqr(int n) { return n * n; }
		sqr(100);    // OK
		int x = 100;
		sqr(x);      // Erreur
	\end{minted}

	\begin{alertblock}{Restriction}
		\begin{itemize}
			\item Pas de pointeur dans des contextes \mintinline{cpp}|consteval|
		\end{itemize}
	\end{alertblock}

	\begin{codesample}
		\sample{https://godbolt.org/#g:!((g:!((g:!((h:codeEditor,i:(filename:'1',fontScale:14,fontUsePx:'0',j:1,lang:c%2B%2B,selection:(endColumn:1,endLineNumber:1,positionColumn:1,positionLineNumber:1,selectionStartColumn:1,selectionStartLineNumber:1,startColumn:1,startLineNumber:1),source:'%23include+%3Ciostream%3E%0A%0Aconsteval+int+sqr(int+n)%0A%7B%0A++return+n+*+n%3B%0A%7D%0A%0Aconstexpr+int+sqr2(int+n)%0A%7B%0A++return+n+*+n%3B%0A%7D%0A%0Aint+main()%0A%7B%0A++std::cout+%3C%3C+sqr(100)+%3C%3C+%22%5Cn%22%3B%0A%0A++constexpr+int+x+%3D+10%3B%0A++std::cout+%3C%3C+sqr(x)+%3C%3C+%22%5Cn%22%3B%0A%0A++int+y+%3D+100%3B%0A%23if+0%0A++std::cout+%3C%3C+sqr(y)%3B+%3C%3C+%22%5Cn%22%3B%0A%23endif%0A%0A%23if+0%0A++std::cout+%3C%3C+sqr2(y)+%3C%3C+%22%5Cn%22%3B%0A%23endif%0A%7D%0A'),l:'5',n:'0',o:'C%2B%2B+source+%231',t:'0')),k:50,l:'4',n:'0',o:'',s:0,t:'0'),(g:!((h:executor,i:(argsPanelShown:'1',compilationPanelShown:'0',compiler:gsnapshot,compilerName:'',compilerOutShown:'0',execArgs:'',execStdin:'',fontScale:14,fontUsePx:'0',j:1,lang:c%2B%2B,libs:!(),options:'-std%3Dc%2B%2B20+-Wall+-Wextra+-pedantic+-Wno-unused-variable',overrides:!(),runtimeTools:!(),source:1,stdinPanelShown:'1',tree:'1',wrap:'0'),l:'5',n:'0',o:'Executor+x86-64+gcc+(trunk)+(C%2B%2B,+Editor+%231)',t:'0')),header:(),k:50,l:'4',n:'0',o:'',s:0,t:'0')),l:'2',n:'0',o:'',t:'0')),version:4}
	\end{codesample}

	\addproposal{P1073}{https://wg21.link/P1073R3}
\end{frame}

\begin{frame}[fragile]
	\frametitle{\mintinline[style=white]{cpp}|constinit|}
	\begin{itemize}
		\item \mintinline{cpp}|constinit| impose une initialisation durant la phase \textit{static initialization}
		\begin{itemize}
			\item Uniquement sur des objets dont la \textit{storage duration} est \textit{static} ou \textit{thread}
			\item Mal-formé en cas d'initialisation dynamique
			\item Adresse le \textit{static initialization order fiasco}
		\end{itemize}
	\end{itemize}

	\addproposal{P1143}{https://wg21.link/P1143R2}
\end{frame}

\begin{frame}[fragile]
	\frametitle{\mintinline[style=white]{cpp}|constexpr|}
	\begin{itemize}
		\item Initialisation triviale dans des contextes \mintinline{cpp}|constexpr|
		\item \mintinline{cpp}|std::is_constant_evaluated()| pour savoir si l'évaluation est \textit{compile-time}
		\item Prise en compte accrue dans la bibliothèque standard
		\item Assouplissement des restrictions
		\begin{itemize}
			\item Fonctions virtuelles \mintinline{cpp}|constexpr|
			\item Utilisation d'\mintinline{cpp}|union|
			\item Utilisation de \mintinline{cpp}|try {} catch()|
			\begin{itemize}
				\item Se comporte comme \textit{no-ops} en \textit{compile-time}
				\item Ne peut pas lancer d'exception \textit{compile-time}
			\end{itemize}
			\item Utilisation de \mintinline{cpp}|dynamic_cast| et \mintinline{cpp}|typeid|
			\item Utilisation de \mintinline{cpp}|asm|
			\begin{itemize}
				\item Si le code \mintinline{cpp}|asm| n'est pas évalué en \textit{compile-time}
			\end{itemize}
		\end{itemize}
	\end{itemize}

	\addproposal{P1064}{https://wg21.link/P1064R0}
	\addproposal{P1002}{https://wg21.link/P1002R1}
	\addproposal{P1330}{https://wg21.link/P1330R0}
	\addproposal{P1331}{https://wg21.link/P1331R2}
\end{frame}

\subsection*{structured binding}
\begin{frame}[fragile]
	\frametitle{Structured binding}
	\begin{itemize}
		\item Extension à tous les membres visibles

\note[item]{\textit{structured binding} sur les membres privés depuis une fonction membre ou amis}

		\item Plus proche de variables classiques
		\begin{itemize}
			\item Capture par les lambdas (copie et référence)
		\end{itemize}
	\end{itemize}

\note[item]{En C++17, les lambdas ne peuvent pas capturer de \textit{structured binding}}

	\begin{minted}{cpp}
		tuple foo{5, 42};

		auto[a, b] = foo;
		auto f1 = [a] { return a; };
		auto f2 = [=] { return b; };
	\end{minted}

	\begin{itemize}
		\item [] \begin{itemize}
			\item Déclaration \mintinline{cpp}|inline|, \mintinline{cpp}|extern|, \mintinline{cpp}|static|, \mintinline{cpp}|thread_local| ou \mintinline{cpp}|constexpr| possible
			\item Possibilité de marquer \mintinline{cpp}|[[ maybe_unused ]]|
		\end{itemize}
	\end{itemize}

	\begin{codesample}
		\sample{https://godbolt.org/#g:!((g:!((g:!((h:codeEditor,i:(filename:'1',fontScale:14,fontUsePx:'0',j:1,lang:c%2B%2B,selection:(endColumn:1,endLineNumber:1,positionColumn:1,positionLineNumber:1,selectionStartColumn:1,selectionStartLineNumber:1,startColumn:1,startLineNumber:1),source:'%23include+%3Ciostream%3E%0A%23include+%3Ctuple%3E%0A%0Aint+main()%0A%7B%0A++std::tuple+foo%7B5,+42%7D%3B%0A%0A++auto%5Ba,+b%5D+%3D+foo%3B%0A++auto+f1+%3D+%5Ba%5D+%7B+return+a%3B+%7D%3B%0A++auto+f2+%3D+%5B%3D%5D+%7B+return+b%3B+%7D%3B%0A%0A++std::cout+%3C%3C+f1()+%3C%3C+%22%5Cn%22%3B%0A++std::cout+%3C%3C+f2()+%3C%3C+%22%5Cn%22%3B%0A%7D%0A'),l:'5',n:'0',o:'C%2B%2B+source+%231',t:'0')),k:50,l:'4',n:'0',o:'',s:0,t:'0'),(g:!((h:executor,i:(argsPanelShown:'1',compilationPanelShown:'0',compiler:gsnapshot,compilerName:'',compilerOutShown:'0',execArgs:'',execStdin:'',fontScale:14,fontUsePx:'0',j:1,lang:c%2B%2B,libs:!(),options:'-std%3Dc%2B%2B20+-Wall+-Wextra+-pedantic',overrides:!(),runtimeTools:!(),source:1,stdinPanelShown:'1',tree:'1',wrap:'0'),l:'5',n:'0',o:'Executor+x86-64+gcc+(trunk)+(C%2B%2B,+Editor+%231)',t:'0')),header:(),k:50,l:'4',n:'0',o:'',s:0,t:'0')),l:'2',n:'0',o:'',t:'0')),version:4}
	\end{codesample}

	\addproposal{P0961}{https://wg21.link/P0961R1}
	\addproposal{P1091}{https://wg21.link/P1091R3}
	\addproposal{P1381}{https://wg21.link/P1381R1}
\end{frame}

\begin{frame}[fragile]
	\frametitle{Structured binding}
	\begin{itemize}
		\item Recherche de \mintinline{cpp}|get()| : seules les fonctions membres templates dont le premier paramètre template n'est pas un type sont retenues
	\end{itemize}

	\begin{block}{Motivation}
		\begin{itemize}
			\item Utiliser des classes possédant un \mintinline{cpp}|get()| indépendant de l'interface \textit{tuple-like}
		\end{itemize}
	\end{block}

	\begin{minted}{cpp}
		struct X : shared_ptr<int> { string foo; };

		template<int N> string& get(X& x) {
		  if constexpr(N==0) return x.foo; }
		template<> class tuple_size<X> :
		  public integral_constant<int, 1> {};
		template<> class tuple_element<0, X> {
		  public: using type = string; };

		X x;
		auto& [y] = x;
	\end{minted}

\note[item]{Exemple invalide en C++17 à cause du \mintinline{cpp}|get()| de \mintinline{cpp}|shared_ptr| qui n'était pas utilisable mais empêchait de rechercher dans le \mintinline{cpp}|namespace| englobant. En C++20 cette fonction n'est pas retenue et on recherche dans le \mintinline{cpp}|namespace|}

	\addproposal{P0969}{https://wg21.link/P0969R0}
\end{frame}

\subsection*{Template}
\begin{frame}[fragile]
	\frametitle{Non-Type Template Parameters}
	\begin{itemize}
		\item Utilisation possible de classes
		\begin{itemize}
			\item \textit{strong structural equality}
			\begin{itemize}
				\item Classes de base et membres non statiques avec une \textit{defaulted} \mintinline{cpp}|operator==|
				\item Pas de référence
				\item Pas de type flottant
			\end{itemize}
			\item Pas d'union
		\end{itemize}
	\end{itemize}

	\begin{minted}{cpp}
		template<chrono::seconds seconds>
		class fixed_timer { ... };
	\end{minted}

\note[item]{En C++17, il fallait utiliser un type entier, p.ex. \mintinline{cpp}|size_t|}

	\begin{minted}{cpp}
		template<fixed_string Id>
		class entity { ... };

		entity<"hello"> e;
	\end{minted}

\note[item]{En C++17, il faut utiliser un ensemble de \mintinline{cpp}|char| : \mintinline{cpp}|template<char... Id>|}

	\addproposal{P0732}{https://wg21.link/P0732R2}
\end{frame}

\begin{frame}[fragile]
	\frametitle{Templates}
	\begin{itemize}
		\item \mintinline{cpp}|typename| optionnel lorsque seul un nom de type est possible

\note[item]{Il ne sert qu'à lever des ambigüités}

		\item Spécialisation possible sur des classes internes privées ou protégées
		\item \mintinline{cpp}|std::type_identity<>| désactive la déduction de type
	\end{itemize}

	\begin{minted}{cpp}
		template<class T>
		void foo(T, T);

		foo(4.2, 0); // erreur, int ou double
	\end{minted}

	\begin{minted}{cpp}
		template<class T>
		void foo(T, type_identity_t<T>);

		foo(4.2, 0); // OK, g<double>
	\end{minted}

	\begin{codesample}
		\sample{https://godbolt.org/#g:!((g:!((g:!((h:codeEditor,i:(filename:'1',fontScale:14,fontUsePx:'0',j:1,lang:c%2B%2B,selection:(endColumn:1,endLineNumber:1,positionColumn:1,positionLineNumber:1,selectionStartColumn:1,selectionStartLineNumber:1,startColumn:1,startLineNumber:1),source:'%23include+%3Ciostream%3E%0A%23include+%3Ctype_traits%3E%0A%0Atemplate%3Cclass+T%3E%0A%23if+1%0AT+foo(T+a,+std::type_identity_t%3CT%3E+b)%0A%23else%0AT+foo(T+a,+T+b)%0A%23endif%0A%7B%0A++return+a+%2B+b%3B%0A%7D%0A%0Aint+main()%0A%7B%0A++std::cout+%3C%3C+foo(4.2,+3)+%3C%3C+%22%5Cn%22%3B%0A%7D%0A'),l:'5',n:'0',o:'C%2B%2B+source+%231',t:'0')),k:50,l:'4',n:'0',o:'',s:0,t:'0'),(g:!((h:executor,i:(argsPanelShown:'1',compilationPanelShown:'0',compiler:gsnapshot,compilerName:'',compilerOutShown:'0',execArgs:'',execStdin:'',fontScale:14,fontUsePx:'0',j:1,lang:c%2B%2B,libs:!(),options:'-std%3Dc%2B%2B20+-Wall+-Wextra+-pedantic',overrides:!(),runtimeTools:!(),source:1,stdinPanelShown:'1',tree:'1',wrap:'0'),l:'5',n:'0',o:'Executor+x86-64+gcc+(trunk)+(C%2B%2B,+Editor+%231)',t:'0')),header:(),k:50,l:'4',n:'0',o:'',s:0,t:'0')),l:'2',n:'0',o:'',t:'0')),version:4}
	\end{codesample}

	\addproposal{P0846}{https://wg21.link/P0846R0}
	\addproposal{P0887}{https://wg21.link/P0887R1}
\end{frame}

\begin{frame}[fragile]
	\frametitle{Templates}
	\begin{itemize}
		\item Déduction de type sur les alias de template
	\end{itemize}

	\begin{minted}{cpp}
		template<typename T>
		using IntPair = std::pair<int, T>;

		// C++ 17
		IntPair<double> p0{1, 2.0};

		// C++ 20
		IntPair p1{1, 2.0};   // std::pair<int, double>
	\end{minted}

	\begin{codesample}
		\sample{https://godbolt.org/#g:!((g:!((g:!((h:codeEditor,i:(filename:'1',fontScale:14,fontUsePx:'0',j:1,lang:c%2B%2B,selection:(endColumn:1,endLineNumber:1,positionColumn:1,positionLineNumber:1,selectionStartColumn:1,selectionStartLineNumber:1,startColumn:1,startLineNumber:1),source:'%23include+%3Ciostream%3E%0A%23include+%3Cutility%3E%0A%0Atemplate%3Ctypename+T%3E%0Ausing+IntPair+%3D+std::pair%3Cint,+T%3E%3B%0A%0Aint+main()%0A%7B%0A++IntPair%3Cdouble%3E+p0%7B1,+2.0%7D%3B+%0A++IntPair+p1%7B1,+2.0%7D%3B+%0A%7D%0A'),l:'5',n:'0',o:'C%2B%2B+source+%231',t:'0')),k:50,l:'4',n:'0',o:'',s:0,t:'0'),(g:!((h:executor,i:(argsPanelShown:'1',compilationPanelShown:'0',compiler:gsnapshot,compilerName:'',compilerOutShown:'0',execArgs:'',execStdin:'',fontScale:14,fontUsePx:'0',j:1,lang:c%2B%2B,libs:!(),options:'-std%3Dc%2B%2B20+-Wall+-Wextra+-pedantic',overrides:!(),runtimeTools:!(),source:1,stdinPanelShown:'1',tree:'1',wrap:'0'),l:'5',n:'0',o:'Executor+x86-64+gcc+(trunk)+(C%2B%2B,+Editor+%231)',t:'0')),header:(),k:50,l:'4',n:'0',o:'',s:0,t:'0')),l:'2',n:'0',o:'',t:'0')),version:4}
	\end{codesample}

	\addproposal{P1814}{https://wg21.link/P1814R0}
\end{frame}

\begin{frame}[fragile]
	\frametitle{Paramètres \mintinline[style=white]{cpp}|auto|}
	\begin{itemize}
		\item Création de fonctions templates via \mintinline{cpp}|auto|
	\end{itemize}

	\begin{minted}{cpp}
		void foo(auto a, auto b) { ... };
	\end{minted}

	\begin{itemize}
		\item Similaire	à la création de lambdas polymorphiques
	\end{itemize}

	\begin{codesample}
		\sample{https://godbolt.org/#g:!((g:!((g:!((h:codeEditor,i:(filename:'1',fontScale:14,fontUsePx:'0',j:1,lang:c%2B%2B,selection:(endColumn:1,endLineNumber:1,positionColumn:1,positionLineNumber:1,selectionStartColumn:1,selectionStartLineNumber:1,startColumn:1,startLineNumber:1),source:'%23include+%3Ciostream%3E%0A%0Aauto+foo(auto+a,+auto+b)%0A%7B%0A++return+a+%2B+b%3B%0A%7D%3B%0A%0Aint+main()%0A%7B%0A++std::cout+%3C%3C+foo(4,+3)+%3C%3C+%22%5Cn%22%3B%0A%7D%0A'),l:'5',n:'0',o:'C%2B%2B+source+%231',t:'0')),k:50,l:'4',n:'0',o:'',s:0,t:'0'),(g:!((h:executor,i:(argsPanelShown:'1',compilationPanelShown:'0',compiler:gsnapshot,compilerName:'',compilerOutShown:'0',execArgs:'',execStdin:'',fontScale:14,fontUsePx:'0',j:1,lang:c%2B%2B,libs:!(),options:'-std%3Dc%2B%2B20+-Wall+-Wextra+-pedantic',overrides:!(),runtimeTools:!(),source:1,stdinPanelShown:'1',tree:'1',wrap:'0'),l:'5',n:'0',o:'Executor+x86-64+gcc+(trunk)+(C%2B%2B,+Editor+%231)',t:'0')),header:(),k:50,l:'4',n:'0',o:'',s:0,t:'0')),l:'2',n:'0',o:'',t:'0')),version:4}
	\end{codesample}
\end{frame}

\subsection*{Concepts}
\begin{frame}[fragile]
	\frametitle{Concepts -- Présentation}
	\begin{itemize}
		\item Histoire ancienne et mouvementée
		\begin{itemize}
			\item Prévu initialement pour C++0x
			\item \ldots{} et cause des décalages successifs
			\item Retrait à grand bruit de C++11
			\item Finalement Concept lite TS publié en 2015
			\item Intégration du TS acceptée en juillet 2017

\note[item]{Mais pas en C++17, il fallait le support par les compilateurs avant pour valider le concept}

		\end{itemize}
		\item Définition de contraintes sur les paramètres templates et l'inférence de type
		\begin{itemize}
			\item Meilleurs diagnostics
			\item Meilleure documentation du code
			\item Aide à la déduction de type
			\item Aide à la résolution de spécialisation

\note[item]{Basé du coup sur un \og nom\fg{} et non sur la structure (typage nominal/structurel)}

		\end{itemize}
		\item Propositions abandonnées / mises de côté
		\begin{itemize}
			\item \textit{Axiom} : spécification de propriétés sémantiques d'un concept

\note[item]{Les concepts actuels se basent sur la structure et des propriétés syntaxiques (p.ex. l'opération d'addition est présent)}
\note[item]{Exemple d'\textit{axiom} : associativité ou commutativité}
\note[item]{Intérêt des \textit{axiom} : meilleure documentation, informations au compilateur (meilleurs avertissement et diagnostics, meilleures optimisations)}

			\item \textit{Concept map} : transformation d'un type non-compatible vers un concept

\note[item]{Exemple de \textit{concept map} : Utilisation de type définissant l'inégalité stricte et la comparaison vers un concept attendant l'inégalité large, implémentation du concept de pile (\textit{stack}) par \mintinline{cpp}|std::vector|}
		\end{itemize}
	\end{itemize}

	\addproposal{P0734}{https://wg21.link/P0734R0}
	\addproposal{P0898}{https://wg21.link/P0898R3}
\end{frame}

\begin{frame}[fragile]
	\frametitle{Concepts -- Utilisation template}
	\begin{itemize}
		\item Utilisable via une \textit{Requires clause}
	\end{itemize}

	\begin{minted}{cpp}
		template<typename T> requires incrementable<T>
		void foo(T);
	\end{minted}

	\begin{itemize}
		\item \ldots{} via une \textit{Trailing requires clause}
	\end{itemize}

	\begin{minted}{cpp}
		template<typename T>
		void foo(T) requires incrementable<T>;
	\end{minted}

	\begin{itemize}
		\item \ldots{} via des paramètres templates contraints
	\end{itemize}

	\begin{minted}{cpp}
		template<incrementable T>
		void foo(T);
	\end{minted}

	\begin{itemize}
		\item \ldots{} ou via des combinaisons de ces syntaxes
	\end{itemize}

	\begin{codesample}
		\sample{https://godbolt.org/#g:!((g:!((g:!((h:codeEditor,i:(filename:'1',fontScale:14,fontUsePx:'0',j:1,lang:c%2B%2B,selection:(endColumn:1,endLineNumber:1,positionColumn:1,positionLineNumber:1,selectionStartColumn:1,selectionStartLineNumber:1,startColumn:1,startLineNumber:1),source:'%23include+%3Ciostream%3E%0A%23include+%3Cvector%3E%0A%0Atemplate%3Cstd::incrementable+T%3E%0Avoid+foo(T)%0A%7B%7D%0A%0Aint+main()%0A%7B%0A++foo(5)%3B%0A%23if+0%0A++foo(std::vector%3Cint%3E%7B1,+2,+3%7D)%3B%0A%23endif%0A%7D%0A'),l:'5',n:'0',o:'C%2B%2B+source+%231',t:'0')),k:50,l:'4',n:'0',o:'',s:0,t:'0'),(g:!((h:executor,i:(argsPanelShown:'1',compilationPanelShown:'0',compiler:gsnapshot,compilerName:'',compilerOutShown:'0',execArgs:'',execStdin:'',fontScale:14,fontUsePx:'0',j:1,lang:c%2B%2B,libs:!(),options:'-std%3Dc%2B%2B20+-Wall+-Wextra+-pedantic',overrides:!(),runtimeTools:!(),source:1,stdinPanelShown:'1',tree:'1',wrap:'0'),l:'5',n:'0',o:'Executor+x86-64+gcc+(trunk)+(C%2B%2B,+Editor+%231)',t:'0')),header:(),k:50,l:'4',n:'0',o:'',s:0,t:'0')),l:'2',n:'0',o:'',t:'0')),version:4}
	\end{codesample}

	\begin{codesample}
		\sample{https://godbolt.org/#g:!((g:!((g:!((h:codeEditor,i:(filename:'1',fontScale:14,fontUsePx:'0',j:1,lang:c%2B%2B,selection:(endColumn:1,endLineNumber:1,positionColumn:1,positionLineNumber:1,selectionStartColumn:1,selectionStartLineNumber:1,startColumn:1,startLineNumber:1),source:'%23include+%3Ciostream%3E%0A%23include+%3Cvector%3E%0A%0Atemplate%3Ctypename+T%3E%0Avoid+foo(T)+requires+std::incrementable%3CT%3E%0A%7B%7D%0A%0Aint+main()%0A%7B%0A++foo(5)%3B%0A%23if+0%0A++foo(std::vector%3Cint%3E%7B1,+2,+3%7D)%3B%0A%23endif%0A%7D%0A'),l:'5',n:'0',o:'C%2B%2B+source+%231',t:'0')),k:50,l:'4',n:'0',o:'',s:0,t:'0'),(g:!((h:executor,i:(argsPanelShown:'1',compilationPanelShown:'0',compiler:gsnapshot,compilerName:'',compilerOutShown:'0',execArgs:'',execStdin:'',fontScale:14,fontUsePx:'0',j:1,lang:c%2B%2B,libs:!(),options:'-std%3Dc%2B%2B20+-Wall+-Wextra+-pedantic',overrides:!(),runtimeTools:!(),source:1,stdinPanelShown:'1',tree:'1',wrap:'0'),l:'5',n:'0',o:'Executor+x86-64+gcc+(trunk)+(C%2B%2B,+Editor+%231)',t:'0')),header:(),k:50,l:'4',n:'0',o:'',s:0,t:'0')),l:'2',n:'0',o:'',t:'0')),version:4}
	\end{codesample}

	\begin{codesample}
		\sample{https://godbolt.org/#g:!((g:!((g:!((h:codeEditor,i:(filename:'1',fontScale:14,fontUsePx:'0',j:1,lang:c%2B%2B,selection:(endColumn:1,endLineNumber:1,positionColumn:1,positionLineNumber:1,selectionStartColumn:1,selectionStartLineNumber:1,startColumn:1,startLineNumber:1),source:'%23include+%3Ciostream%3E%0A%23include+%3Cvector%3E%0A%0Atemplate%3Ctypename+T%3E+requires+std::incrementable%3CT%3E%0Avoid+foo(T)%0A%7B%7D%0A%0Aint+main()%0A%7B%0A++foo(5)%3B%0A%23if+0%0A++foo(std::vector%3Cint%3E%7B1,+2,+3%7D)%3B%0A%23endif%0A%7D%0A'),l:'5',n:'0',o:'C%2B%2B+source+%231',t:'0')),k:50,l:'4',n:'0',o:'',s:0,t:'0'),(g:!((h:executor,i:(argsPanelShown:'1',compilationPanelShown:'0',compiler:gsnapshot,compilerName:'',compilerOutShown:'0',execArgs:'',execStdin:'',fontScale:14,fontUsePx:'0',j:1,lang:c%2B%2B,libs:!(),options:'-std%3Dc%2B%2B20+-Wall+-Wextra+-pedantic',overrides:!(),runtimeTools:!(),source:1,stdinPanelShown:'1',tree:'1',wrap:'0'),l:'5',n:'0',o:'Executor+x86-64+gcc+(trunk)+(C%2B%2B,+Editor+%231)',t:'0')),header:(),k:50,l:'4',n:'0',o:'',s:0,t:'0')),l:'2',n:'0',o:'',t:'0')),version:4}
	\end{codesample}

	\addproposal{P0734}{https://wg21.link/P0734R0}
	\addproposal{P0898}{https://wg21.link/P0898R3}
\end{frame}

\begin{frame}[fragile]
	\frametitle{Concepts -- Utilisation template}
	\begin{itemize}
		\item Utilisable depuis un concept nommé
	\end{itemize}

	\begin{minted}{cpp}
		template<typename T> requires incrementable<T>
		void foo(T);
	\end{minted}

	\begin{itemize}
		\item \ldots{} ou depuis des expressions
	\end{itemize}

	\begin{minted}{cpp}
		template<typename T> requires requires (T x) { ++x; }
		void foo(T);
	\end{minted}

	\begin{minted}{cpp}
		template<typename T> requires (sizeof(T) > 1)
		void foo(T);
	\end{minted}

\note[item]{Oui, \mintinline{cpp}|requires| est bien en double}

	\begin{codesample}
		\sample{https://godbolt.org/#g:!((g:!((g:!((h:codeEditor,i:(filename:'1',fontScale:14,fontUsePx:'0',j:1,lang:c%2B%2B,selection:(endColumn:1,endLineNumber:1,positionColumn:1,positionLineNumber:1,selectionStartColumn:1,selectionStartLineNumber:1,startColumn:1,startLineNumber:1),source:'%23include+%3Ciostream%3E%0A%0Atemplate%3Ctypename+T%3E+requires+(sizeof(T)+%3E+1)%0Avoid+foo(T)%0A%7B%7D%0A%0Aint+main()%0A%7B%0A++foo(5)%3B%0A%23if+0%0A++foo(!'a!')%3B%0A%23endif%0A%7D%0A'),l:'5',n:'0',o:'C%2B%2B+source+%231',t:'0')),k:50,l:'4',n:'0',o:'',s:0,t:'0'),(g:!((h:executor,i:(argsPanelShown:'1',compilationPanelShown:'0',compiler:gsnapshot,compilerName:'',compilerOutShown:'0',execArgs:'',execStdin:'',fontScale:14,fontUsePx:'0',j:1,lang:c%2B%2B,libs:!(),options:'-std%3Dc%2B%2B20+-Wall+-Wextra+-pedantic',overrides:!(),runtimeTools:!(),source:1,stdinPanelShown:'1',tree:'1',wrap:'0'),l:'5',n:'0',o:'Executor+x86-64+gcc+(trunk)+(C%2B%2B,+Editor+%231)',t:'0')),header:(),k:50,l:'4',n:'0',o:'',s:0,t:'0')),l:'2',n:'0',o:'',t:'0')),version:4}
	\end{codesample}

	\begin{codesample}
		\sample{https://godbolt.org/#g:!((g:!((g:!((h:codeEditor,i:(filename:'1',fontScale:14,fontUsePx:'0',j:1,lang:c%2B%2B,selection:(endColumn:1,endLineNumber:1,positionColumn:1,positionLineNumber:1,selectionStartColumn:1,selectionStartLineNumber:1,startColumn:1,startLineNumber:1),source:'%23include+%3Ciostream%3E%0A%23include+%3Cvector%3E%0A%0Atemplate%3Ctypename+T%3E+requires+requires+(T+x)+%7B+x%2B%2B%3B+%7D%0Avoid+foo(T)%0A%7B%7D%0A%0Aint+main()%0A%7B%0A++foo(5)%3B%0A%23if+0%0A++foo(std::vector%3Cint%3E%7B0,+1%7D)%3B%0A%23endif%0A%7D%0A'),l:'5',n:'0',o:'C%2B%2B+source+%231',t:'0')),k:50,l:'4',n:'0',o:'',s:0,t:'0'),(g:!((h:executor,i:(argsPanelShown:'1',compilationPanelShown:'0',compiler:gsnapshot,compilerName:'',compilerOutShown:'0',execArgs:'',execStdin:'',fontScale:14,fontUsePx:'0',j:1,lang:c%2B%2B,libs:!(),options:'-std%3Dc%2B%2B20+-Wall+-Wextra+-pedantic',overrides:!(),runtimeTools:!(),source:1,stdinPanelShown:'1',tree:'1',wrap:'0'),l:'5',n:'0',o:'Executor+x86-64+gcc+(trunk)+(C%2B%2B,+Editor+%231)',t:'0')),header:(),k:50,l:'4',n:'0',o:'',s:0,t:'0')),l:'2',n:'0',o:'',t:'0')),version:4}
	\end{codesample}

	\begin{codesample}
		\sample{https://godbolt.org/#g:!((g:!((g:!((h:codeEditor,i:(filename:'1',fontScale:14,fontUsePx:'0',j:1,lang:c%2B%2B,selection:(endColumn:1,endLineNumber:1,positionColumn:1,positionLineNumber:1,selectionStartColumn:1,selectionStartLineNumber:1,startColumn:1,startLineNumber:1),source:'%23include+%3Ciostream%3E%0A%23include+%3Cvector%3E%0A%0Atemplate%3Ctypename+T%3E+requires+std::incrementable%3CT%3E%0Avoid+foo(T)%0A%7B%7D%0A%0Aint+main()%0A%7B%0A++foo(5)%3B%0A%23if+0%0A++foo(std::vector%3Cint%3E%7B0,+1%7D)%3B%0A%23endif%0A%7D%0A'),l:'5',n:'0',o:'C%2B%2B+source+%231',t:'0')),k:50,l:'4',n:'0',o:'',s:0,t:'0'),(g:!((h:executor,i:(argsPanelShown:'1',compilationPanelShown:'0',compiler:gsnapshot,compilerName:'',compilerOutShown:'0',execArgs:'',execStdin:'',fontScale:14,fontUsePx:'0',j:1,lang:c%2B%2B,libs:!(),options:'-std%3Dc%2B%2B20+-Wall+-Wextra+-pedantic',overrides:!(),runtimeTools:!(),source:1,stdinPanelShown:'1',tree:'1',wrap:'0'),l:'5',n:'0',o:'Executor+x86-64+gcc+(trunk)+(C%2B%2B,+Editor+%231)',t:'0')),header:(),k:50,l:'4',n:'0',o:'',s:0,t:'0')),l:'2',n:'0',o:'',t:'0')),version:4}
	\end{codesample}

	\addproposal{P0734}{https://wg21.link/P0734R0}
	\addproposal{P0898}{https://wg21.link/P0898R3}
\end{frame}

\begin{frame}[fragile]
	\frametitle{Concepts -- Utilisation template}
	\begin{itemize}
		\item Peuvent être composés
	\end{itemize}

	\begin{minted}{cpp}
		template<typename T>
		requires (sizeof(T) > 1 && sizeof(T) <= 4)
		void foo(T);
	\end{minted}

	\begin{minted}{cpp}
		template<typename T>
		requires (sizeof(T) == 2 || sizeof(T) == 4)
		void foo(T);
	\end{minted}

	\begin{codesample}
		\sample{https://godbolt.org/#g:!((g:!((g:!((h:codeEditor,i:(filename:'1',fontScale:14,fontUsePx:'0',j:1,lang:c%2B%2B,selection:(endColumn:1,endLineNumber:1,positionColumn:1,positionLineNumber:1,selectionStartColumn:1,selectionStartLineNumber:1,startColumn:1,startLineNumber:1),source:'%23include+%3Ciostream%3E%0A%0Atemplate%3Ctypename+T%3E%0Arequires+(sizeof(T)+%3D%3D+2+%7C%7C+sizeof(T)+%3D%3D+4)%0Avoid+foo(T)%0A%7B%7D%0A%0Aint+main()%0A%7B%0A++foo(5)%3B%0A%23if+0%0A++foo(!'a!')%3B%0A%23endif%0A%7D%0A'),l:'5',n:'0',o:'C%2B%2B+source+%231',t:'0')),k:50,l:'4',n:'0',o:'',s:0,t:'0'),(g:!((h:executor,i:(argsPanelShown:'1',compilationPanelShown:'0',compiler:gsnapshot,compilerName:'',compilerOutShown:'0',execArgs:'',execStdin:'',fontScale:14,fontUsePx:'0',j:1,lang:c%2B%2B,libs:!(),options:'-std%3Dc%2B%2B20+-Wall+-Wextra+-pedantic',overrides:!(),runtimeTools:!(),source:1,stdinPanelShown:'1',tree:'1',wrap:'0'),l:'5',n:'0',o:'Executor+x86-64+gcc+(trunk)+(C%2B%2B,+Editor+%231)',t:'0')),header:(),k:50,l:'4',n:'0',o:'',s:0,t:'0')),l:'2',n:'0',o:'',t:'0')),version:4}
	\end{codesample}

	\begin{codesample}
		\sample{https://godbolt.org/#g:!((g:!((g:!((h:codeEditor,i:(filename:'1',fontScale:14,fontUsePx:'0',j:1,lang:c%2B%2B,selection:(endColumn:1,endLineNumber:1,positionColumn:1,positionLineNumber:1,selectionStartColumn:1,selectionStartLineNumber:1,startColumn:1,startLineNumber:1),source:'%23include+%3Ciostream%3E%0A%0Atemplate%3Ctypename+T%3E%0Arequires+(sizeof(T)+%3E+1+%26%26+sizeof(T)+%3C%3D+4)%0Avoid+foo(T)%0A%7B%7D%0A%0Aint+main()%0A%7B%0A++foo(5)%3B%0A%23if+0%0A++foo(!'a!')%3B%0A%23endif%0A%7D%0A'),l:'5',n:'0',o:'C%2B%2B+source+%231',t:'0')),k:50,l:'4',n:'0',o:'',s:0,t:'0'),(g:!((h:executor,i:(argsPanelShown:'1',compilationPanelShown:'0',compiler:gsnapshot,compilerName:'',compilerOutShown:'0',execArgs:'',execStdin:'',fontScale:14,fontUsePx:'0',j:1,lang:c%2B%2B,libs:!(),options:'-std%3Dc%2B%2B20+-Wall+-Wextra+-pedantic',overrides:!(),runtimeTools:!(),source:1,stdinPanelShown:'1',tree:'1',wrap:'0'),l:'5',n:'0',o:'Executor+x86-64+gcc+(trunk)+(C%2B%2B,+Editor+%231)',t:'0')),header:(),k:50,l:'4',n:'0',o:'',s:0,t:'0')),l:'2',n:'0',o:'',t:'0')),version:4}
	\end{codesample}

	\addproposal{P0734}{https://wg21.link/P0734R0}
	\addproposal{P0898}{https://wg21.link/P0898R3}
\end{frame}

\begin{frame}[fragile]
	\frametitle{Concepts -- Utilisation template}
	\begin{itemize}
		\item Support des \textit{parameters pack}
	\end{itemize}

	\begin{minted}{cpp}
		template<Constraint... T>
		void foo(T...);
	\end{minted}

	\begin{minted}{cpp}
		template<typename... T>
		requires (Constraint<T> && ... && true)
		void foo(T...);
	\end{minted}

	\begin{codesample}
		\sample{https://godbolt.org/#g:!((g:!((g:!((h:codeEditor,i:(filename:'1',fontScale:14,fontUsePx:'0',j:1,lang:c%2B%2B,selection:(endColumn:1,endLineNumber:1,positionColumn:1,positionLineNumber:1,selectionStartColumn:1,selectionStartLineNumber:1,startColumn:1,startLineNumber:1),source:'%23include+%3Ciostream%3E%0A%23include+%3Cvector%3E%0A%0Atemplate%3Ctypename...+T%3E%0Arequires+(std::incrementable%3CT%3E+%26%26...+%26%26+true)%0Avoid+foo(T...)%0A%7B%7D%0A%0Aint+main()%0A%7B%0A++foo(5)%3B%0A++foo(5,+6,+7)%3B%0A%23if+0%0A++foo(std::vector%3Cint%3E%7B1,+2%7D)%3B%0A%23endif%0A%7D%0A'),l:'5',n:'0',o:'C%2B%2B+source+%231',t:'0')),k:50,l:'4',n:'0',o:'',s:0,t:'0'),(g:!((h:executor,i:(argsPanelShown:'1',compilationPanelShown:'0',compiler:gsnapshot,compilerName:'',compilerOutShown:'0',execArgs:'',execStdin:'',fontScale:14,fontUsePx:'0',j:1,lang:c%2B%2B,libs:!(),options:'-std%3Dc%2B%2B20+-Wall+-Wextra+-pedantic',overrides:!(),runtimeTools:!(),source:1,stdinPanelShown:'1',tree:'1',wrap:'0'),l:'5',n:'0',o:'Executor+x86-64+gcc+(trunk)+(C%2B%2B,+Editor+%231)',t:'0')),header:(),k:50,l:'4',n:'0',o:'',s:0,t:'0')),l:'2',n:'0',o:'',t:'0')),version:4}
	\end{codesample}

	\begin{codesample}
		\sample{https://godbolt.org/#g:!((g:!((g:!((h:codeEditor,i:(filename:'1',fontScale:14,fontUsePx:'0',j:1,lang:c%2B%2B,selection:(endColumn:6,endLineNumber:12,positionColumn:6,positionLineNumber:12,selectionStartColumn:6,selectionStartLineNumber:12,startColumn:6,startLineNumber:12),source:'%23include+%3Ciostream%3E%0A%23include+%3Cvector%3E%0A%0Atemplate%3Cstd::incrementable...+T%3E%0Avoid+foo(T...)%0A%7B%7D%0A%0Aint+main()%0A%7B%0A++foo(5)%3B%0A++foo(5,+6,+7)%3B%0A%23if+0%0A++foo(std::vector%3Cint%3E%7B1,+2%7D)%3B%0A%23endif%0A%7D%0A'),l:'5',n:'0',o:'C%2B%2B+source+%231',t:'0')),k:50,l:'4',n:'0',o:'',s:0,t:'0'),(g:!((h:executor,i:(argsPanelShown:'1',compilationPanelShown:'0',compiler:gsnapshot,compilerName:'',compilerOutShown:'0',execArgs:'',execStdin:'',fontScale:14,fontUsePx:'0',j:1,lang:c%2B%2B,libs:!(),options:'-std%3Dc%2B%2B20+-Wall+-Wextra+-pedantic',overrides:!(),runtimeTools:!(),source:1,stdinPanelShown:'1',tree:'1',wrap:'0'),l:'5',n:'0',o:'Executor+x86-64+gcc+(trunk)+(C%2B%2B,+Editor+%231)',t:'0')),header:(),k:50,l:'4',n:'0',o:'',s:0,t:'0')),l:'2',n:'0',o:'',t:'0')),version:4}
	\end{codesample}

	\addproposal{P0734}{https://wg21.link/P0734R0}
	\addproposal{P0898}{https://wg21.link/P0898R3}
\end{frame}

\begin{frame}[fragile]
	\frametitle{Concepts -- Utilisation inférence de type}
	\begin{itemize}
		\item Contraintes sur les paramètres (lambdas et fonctions templates)
	\end{itemize}

	\begin{minted}{cpp}
		[](Constraint auto a) {}
		void foo(Constraint auto a);
	\end{minted}

	\begin{itemize}
		\item Contraintes sur les types de retour
	\end{itemize}

	\begin{minted}{cpp}
		Constraint auto foo();
		auto bar() -> Constraint decltype(auto);
	\end{minted}

	\begin{codesample}
		\sample{https://godbolt.org/#g:!((g:!((g:!((h:codeEditor,i:(filename:'1',fontScale:14,fontUsePx:'0',j:1,lang:c%2B%2B,selection:(endColumn:1,endLineNumber:1,positionColumn:1,positionLineNumber:1,selectionStartColumn:1,selectionStartLineNumber:1,startColumn:1,startLineNumber:1),source:'%23include+%3Ciostream%3E%0A%23include+%3Cvector%3E%0A%0Astd::incrementable+auto+foo()+%0A%7B%0A%23if+1%0A++return+5%3B%0A%23else%0A++return+std::vector%3Cint%3E%7B%7D%3B%0A%23endif%0A%7D%0A%0Aauto+bar()+-%3E+std::incrementable++decltype(auto)%0A%7B%0A%23if+1%0A++return+5%3B%0A%23else%0A++return+std::vector%3Cint%3E%7B%7D%3B%0A%23endif%0A%7D%0A%0Aint+main()%0A%7B%0A%7D%0A'),l:'5',n:'0',o:'C%2B%2B+source+%231',t:'0')),k:50,l:'4',n:'0',o:'',s:0,t:'0'),(g:!((h:executor,i:(argsPanelShown:'1',compilationPanelShown:'0',compiler:gsnapshot,compilerName:'',compilerOutShown:'0',execArgs:'',execStdin:'',fontScale:14,fontUsePx:'0',j:1,lang:c%2B%2B,libs:!(),options:'-std%3Dc%2B%2B20+-Wall+-Wextra+-pedantic',overrides:!(),runtimeTools:!(),source:1,stdinPanelShown:'1',tree:'1',wrap:'0'),l:'5',n:'0',o:'Executor+x86-64+gcc+(trunk)+(C%2B%2B,+Editor+%231)',t:'0')),header:(),k:50,l:'4',n:'0',o:'',s:0,t:'0')),l:'2',n:'0',o:'',t:'0')),version:4}
	\end{codesample}

	\begin{codesample}
		\sample{https://godbolt.org/#g:!((g:!((g:!((h:codeEditor,i:(filename:'1',fontScale:14,fontUsePx:'0',j:1,lang:c%2B%2B,selection:(endColumn:1,endLineNumber:1,positionColumn:1,positionLineNumber:1,selectionStartColumn:1,selectionStartLineNumber:1,startColumn:1,startLineNumber:1),source:'%23include+%3Ciostream%3E%0A%23include+%3Cvector%3E%0A%0Avoid+foo(std::incrementable+auto)%0A%7B%7D%0A%0Aint+main()%0A%7B%0A++foo(5)%3B%0A%23if+0%0A++foo(std::vector%3Cint%3E%7B%7D)%3B%0A%23endif%0A%0A++auto+bar+%3D+%5B%5D(std::incrementable+auto)+%7B%7D%3B%0A++bar(5)%3B%0A%23if+0%0A++bar(std::vector%3Cint%3E%7B%7D)%3B%0A%23endif%0A%7D%0A'),l:'5',n:'0',o:'C%2B%2B+source+%231',t:'0')),k:50,l:'4',n:'0',o:'',s:0,t:'0'),(g:!((h:executor,i:(argsPanelShown:'1',compilationPanelShown:'0',compiler:gsnapshot,compilerName:'',compilerOutShown:'0',execArgs:'',execStdin:'',fontScale:14,fontUsePx:'0',j:1,lang:c%2B%2B,libs:!(),options:'-std%3Dc%2B%2B20+-Wall+-Wextra+-pedantic',overrides:!(),runtimeTools:!(),source:1,stdinPanelShown:'1',tree:'1',wrap:'0'),l:'5',n:'0',o:'Executor+x86-64+gcc+(trunk)+(C%2B%2B,+Editor+%231)',t:'0')),header:(),k:50,l:'4',n:'0',o:'',s:0,t:'0')),l:'2',n:'0',o:'',t:'0')),version:4}
	\end{codesample}

	\addproposal{P0734}{https://wg21.link/P0734R0}
	\addproposal{P0898}{https://wg21.link/P0898R3}
\end{frame}

\begin{frame}[fragile]
	\frametitle{Concepts -- Utilisation inférence de type}
	\begin{itemize}
		\item Contraintes sur les variables
	\end{itemize}

	\begin{minted}{cpp}
		Constraint auto bar = ...;
	\end{minted}

	\begin{itemize}
		\item Contraintes sur les \textit{non-type template parameters}
	\end{itemize}

	\begin{minted}{cpp}
		template<Constraint auto S>
		void foo();
	\end{minted}

	\begin{itemize}
		\item Support des \textit{parameters pack}
	\end{itemize}

	\begin{minted}{cpp}
		void foo(Constraint auto... T);
	\end{minted}

	\begin{codesample}
		\sample{https://godbolt.org/#g:!((g:!((g:!((h:codeEditor,i:(filename:'1',fontScale:14,fontUsePx:'0',j:1,lang:c%2B%2B,selection:(endColumn:1,endLineNumber:1,positionColumn:1,positionLineNumber:1,selectionStartColumn:1,selectionStartLineNumber:1,startColumn:1,startLineNumber:1),source:'%23include+%3Ciostream%3E%0A%23include+%3Cvector%3E%0A%0Avoid+foo(std::incrementable+auto...)%0A%7B%7D%0A%0Aint+main()%0A%7B%0A++foo(5)%3B%0A++foo(1,+2,+3)%3B%0A%23if+0%0A++foo(std::vector%3Cint%3E%7B%7D)%3B%0A%23endif%0A%7D%0A'),l:'5',n:'0',o:'C%2B%2B+source+%231',t:'0')),k:50,l:'4',n:'0',o:'',s:0,t:'0'),(g:!((h:executor,i:(argsPanelShown:'1',compilationPanelShown:'0',compiler:gsnapshot,compilerName:'',compilerOutShown:'0',execArgs:'',execStdin:'',fontScale:14,fontUsePx:'0',j:1,lang:c%2B%2B,libs:!(),options:'-std%3Dc%2B%2B20+-Wall+-Wextra+-pedantic',overrides:!(),runtimeTools:!(),source:1,stdinPanelShown:'1',tree:'1',wrap:'0'),l:'5',n:'0',o:'Executor+x86-64+gcc+(trunk)+(C%2B%2B,+Editor+%231)',t:'0')),header:(),k:50,l:'4',n:'0',o:'',s:0,t:'0')),l:'2',n:'0',o:'',t:'0')),version:4}
	\end{codesample}

	\begin{codesample}
		\sample{https://godbolt.org/#g:!((g:!((g:!((h:codeEditor,i:(filename:'1',fontScale:14,fontUsePx:'0',j:1,lang:c%2B%2B,selection:(endColumn:1,endLineNumber:1,positionColumn:1,positionLineNumber:1,selectionStartColumn:1,selectionStartLineNumber:1,startColumn:1,startLineNumber:1),source:'%23include+%3Ciostream%3E%0A%23include+%3Cvector%3E%0A%0Atemplate%3Cstd::incrementable+auto+S%3E%0Avoid+foo()%0A%7B%7D%0A%0Aint+main()%0A%7B%0A++foo%3C5%3E()%3B%0A%23if+0%0A++foo%3Cstd::vector%3Cint%3E%7B%7D%3E()%3B%0A%23endif%0A%7D%0A'),l:'5',n:'0',o:'C%2B%2B+source+%231',t:'0')),k:50,l:'4',n:'0',o:'',s:0,t:'0'),(g:!((h:executor,i:(argsPanelShown:'1',compilationPanelShown:'0',compiler:gsnapshot,compilerName:'',compilerOutShown:'0',execArgs:'',execStdin:'',fontScale:14,fontUsePx:'0',j:1,lang:c%2B%2B,libs:!(),options:'-std%3Dc%2B%2B20+-Wall+-Wextra+-pedantic',overrides:!(),runtimeTools:!(),source:1,stdinPanelShown:'1',tree:'1',wrap:'0'),l:'5',n:'0',o:'Executor+x86-64+gcc+(trunk)+(C%2B%2B,+Editor+%231)',t:'0')),header:(),k:50,l:'4',n:'0',o:'',s:0,t:'0')),l:'2',n:'0',o:'',t:'0')),version:4}
	\end{codesample}

	\begin{codesample}
		\sample{https://godbolt.org/#g:!((g:!((g:!((h:codeEditor,i:(filename:'1',fontScale:14,fontUsePx:'0',j:1,lang:c%2B%2B,selection:(endColumn:1,endLineNumber:1,positionColumn:1,positionLineNumber:1,selectionStartColumn:1,selectionStartLineNumber:1,startColumn:1,startLineNumber:1),source:'%23include+%3Ciostream%3E%0A%23include+%3Cvector%3E%0A%0Aint+main()%0A%7B%0A++std::incrementable+auto+foo+%3D+5%3B%0A%23if+0%0A++std::incrementable+auto+bar+%3D+std::vector%3Cint%3E%7B%7D%3B%0A%23endif%0A%7D%0A'),l:'5',n:'0',o:'C%2B%2B+source+%231',t:'0')),k:50,l:'4',n:'0',o:'',s:0,t:'0'),(g:!((h:executor,i:(argsPanelShown:'1',compilationPanelShown:'0',compiler:gsnapshot,compilerName:'',compilerOutShown:'0',execArgs:'',execStdin:'',fontScale:14,fontUsePx:'0',j:1,lang:c%2B%2B,libs:!(),options:'-std%3Dc%2B%2B20+-Wall+-Wextra+-pedantic+-Wno-unused-variable',overrides:!(),runtimeTools:!(),source:1,stdinPanelShown:'1',tree:'1',wrap:'0'),l:'5',n:'0',o:'Executor+x86-64+gcc+(trunk)+(C%2B%2B,+Editor+%231)',t:'0')),header:(),k:50,l:'4',n:'0',o:'',s:0,t:'0')),l:'2',n:'0',o:'',t:'0')),version:4}
	\end{codesample}

	\addproposal{P0734}{https://wg21.link/P0734R0}
	\addproposal{P0898}{https://wg21.link/P0898R3}
\end{frame}

\begin{frame}[fragile]
	\frametitle{Concepts -- Standard}
	\begin{itemize}
		\item Nombreux concepts standards
		\begin{itemize}
			\item Relations entre types : \mintinline{cpp}|same_as|, \mintinline{cpp}|derived_from|, \mintinline{cpp}|convertible_to|, \mintinline{cpp}|common_with|, \ldots{}
			\item Types numériques : \mintinline{cpp}|integral|, \mintinline{cpp}|signed_integral|, \mintinline{cpp}|unsigned_integral|, \mintinline{cpp}|floating_point|, \ldots{}
			\item Opérations supportées : \mintinline{cpp}|swappable|, \mintinline{cpp}|destructible|, \mintinline{cpp}|default_constructible|, \mintinline{cpp}|move_constructible|, \mintinline{cpp}|copy_constructible|, \ldots{}
			\item Catégories de types : \mintinline{cpp}|movable|, \mintinline{cpp}|copyable|, \mintinline{cpp}|semiregular|, \mintinline{cpp}|regular|, \ldots{}

\note[item]{\mintinline{cpp}|semiregular| : \mintinline{cpp}|copyable| et \mintinline{cpp}|default_constructible|}
\note[item]{\mintinline{cpp}|regular| : \mintinline{cpp}|semiregular| et \mintinline{cpp}|equality_comparable|}

			\item Comparaisons : \mintinline{cpp}|boolean|, \mintinline{cpp}|equality_comparable|, \mintinline{cpp}|totally_ordered|, \ldots{}
			\item \textit{Callable concepts} : \mintinline{cpp}|invocable|, \mintinline{cpp}|predicate|, \mintinline{cpp}|strict_weak_order|, \ldots{}
			\item \ldots{}
		\end{itemize}
	\end{itemize}

	\addproposal{P0734}{https://wg21.link/P0734R0}
	\addproposal{P0898}{https://wg21.link/P0898R3}
\end{frame}

\begin{frame}[fragile]
	\frametitle{Concepts -- Définition}
	\begin{itemize}
		\item Peuvent être définis depuis des expressions
	\end{itemize}

	\begin{minted}{cpp}
		template<typename T>
		concept Addable = requires (T x) { x + x; };
	\end{minted}

	\begin{minted}{cpp}
		template <class T, class U = T>
		concept Swappable = requires(T&& t, U&& u) {
		  swap(forward<T>(t), forward<U>(u));
		  swap(forward<U>(u), forward<T>(t)); };
	\end{minted}

	\addproposal{P0734}{https://wg21.link/P0734R0}
	\addproposal{P0898}{https://wg21.link/P0898R3}
\end{frame}

\begin{frame}[fragile]
	\frametitle{Concepts -- Définition}
	\begin{itemize}
		\item Y compris en retirant des qualifieurs
	\end{itemize}

	\begin{minted}{cpp}
		template<class T>
		concept Addable = requires(
		  const remove_reference_t<T>& a,
		  const remove_reference_t<T>& b) { a + b; };
  	\end{minted}

	\begin{itemize}
		\item Ou en imposant les types de retour
	\end{itemize}

	\begin{minted}{cpp}
		template<class T>
		concept Comparable = requires(T a, T b) {
		  { a == b } -> boolean;
		  { a != b } -> boolean; };
	\end{minted}

	\addproposal{P0734}{https://wg21.link/P0734R0}
	\addproposal{P0898}{https://wg21.link/P0898R3}
\end{frame}

\begin{frame}[fragile]
	\frametitle{Concepts -- Définition}
	\begin{itemize}
		\item Depuis des \textit{traits}
	\end{itemize}

	\begin{minted}{cpp}
		template<class T>
		concept integral = is_integral_v<T>;
	\end{minted}

	\begin{minted}{cpp}
		template<class T, class... Args>
		concept constructible_from =
		  destructible<T> && is_constructible_v<T, Args...>;
  	\end{minted}

	\addproposal{P0734}{https://wg21.link/P0734R0}
	\addproposal{P0898}{https://wg21.link/P0898R3}
\end{frame}

\begin{frame}[fragile]
	\frametitle{Concepts -- Définition}
	\begin{itemize}
		\item Depuis d'autres concepts
	\end{itemize}

	\begin{minted}{cpp}
		template<class T> concept semiregular =
		  copyable<T> && default_constructible<T>;
  	\end{minted}

	\begin{itemize}
		\item En combinant différentes méthodes
	\end{itemize}

	\begin{minted}{cpp}
		template<class T> concept totally_ordered =
		  equality_comparable<T> &&
		  requires(const remove_reference_t<T>& a,
		           const remove_reference_t<T>& b) {
		    { a < b } -> boolean;
		    { a > b } -> boolean;
		    { a <= b } -> boolean;
		    { a >= b } -> boolean;
		  };
	\end{minted}

	\addproposal{P0734}{https://wg21.link/P0734R0}
	\addproposal{P0898}{https://wg21.link/P0898R3}
\end{frame}

\subsection*{Attributs}
\begin{frame}[fragile]
	\frametitle{Attributs}
	\begin{itemize}
		\item Ajout d'attributs
		\begin{itemize}
			\item \mintinline{cpp}|[[ likely ]]| et \mintinline{cpp}|[[ unlikely ]]| probabilité de branches conditionnelles

\note[item]{Fourni au compilateur des informations lui permettant des optimisations plus pertinentes}

			\item \mintinline{cpp}|[[ no_unique_address ]]| membre statique ne nécessitant pas une adresse unique

\note[item]{Et permettre l'EMO (\textit{Empty Member Optimisation})}
		\end{itemize}

		\item Extension de \mintinline{cpp}|[[ nodiscard ]]| aux constructeurs
		\begin{itemize}
			\item Marquage \mintinline{cpp}|[[ nodiscard ]]| des constructeurs autorisé
			\item Vérification également lors des conversions via les constructeurs
		\end{itemize}
		\item Possibilité d'associer un message à \mintinline{cpp}|[[ nodiscard ]]|
	\end{itemize}

	\begin{codesample}
		\sample{https://godbolt.org/#g:!((g:!((g:!((h:codeEditor,i:(filename:'1',fontScale:14,fontUsePx:'0',j:1,lang:c%2B%2B,selection:(endColumn:1,endLineNumber:1,positionColumn:1,positionLineNumber:1,selectionStartColumn:1,selectionStartLineNumber:1,startColumn:1,startLineNumber:1),source:'%23include+%3Ciostream%3E%0A%0A%5B%5B+nodiscard(%22Must+be+checked%22)+%5D%5D+static+int+foo()%0A%7B%0A++return+5%3B%0A%7D%0A%0Aint+main()%0A%7B%0A++foo()%3B%0A%7D%0A'),l:'5',n:'0',o:'C%2B%2B+source+%231',t:'0')),k:50,l:'4',n:'0',o:'',s:0,t:'0'),(g:!((h:executor,i:(argsPanelShown:'1',compilationPanelShown:'0',compiler:gsnapshot,compilerName:'',compilerOutShown:'0',execArgs:'',execStdin:'',fontScale:14,fontUsePx:'0',j:1,lang:c%2B%2B,libs:!(),options:'-std%3Dc%2B%2B20+-Wall+-Wextra+-pedantic',overrides:!(),runtimeTools:!(),source:1,stdinPanelShown:'1',tree:'1',wrap:'0'),l:'5',n:'0',o:'Executor+x86-64+gcc+(trunk)+(C%2B%2B,+Editor+%231)',t:'0')),header:(),k:50,l:'4',n:'0',o:'',s:0,t:'0')),l:'2',n:'0',o:'',t:'0')),version:4}
	\end{codesample}

	\begin{codesample}
		\sample{https://godbolt.org/#g:!((g:!((g:!((h:codeEditor,i:(filename:'1',fontScale:14,fontUsePx:'0',j:1,lang:c%2B%2B,selection:(endColumn:1,endLineNumber:1,positionColumn:1,positionLineNumber:1,selectionStartColumn:1,selectionStartLineNumber:1,startColumn:1,startLineNumber:1),source:'%23include+%3Ciostream%3E%0A%0Astruct+Bar+%7B%0A++%5B%5B+nodiscard+%5D%5DBar()+%7B%7D%0A%7D%3B%0A%0Aint+main()%0A%7B%0A%23if+0%0A++Bar+b%7B%7D%3B%0A%23else%0A++Bar%7B%7D%3B%0A%23endif%0A%7D%0A'),l:'5',n:'0',o:'C%2B%2B+source+%231',t:'0')),k:50,l:'4',n:'0',o:'',s:0,t:'0'),(g:!((h:executor,i:(argsPanelShown:'1',compilationPanelShown:'0',compiler:gsnapshot,compilerName:'',compilerOutShown:'0',execArgs:'',execStdin:'',fontScale:14,fontUsePx:'0',j:1,lang:c%2B%2B,libs:!(),options:'-std%3Dc%2B%2B20+-Wall+-Wextra+-pedantic',overrides:!(),runtimeTools:!(),source:1,stdinPanelShown:'1',tree:'1',wrap:'0'),l:'5',n:'0',o:'Executor+x86-64+gcc+(trunk)+(C%2B%2B,+Editor+%231)',t:'0')),header:(),k:50,l:'4',n:'0',o:'',s:0,t:'0')),l:'2',n:'0',o:'',t:'0')),version:4}
	\end{codesample}

	\addproposal{P0479}{https://wg21.link/P0479R5}
	\addproposal{P0840}{https://wg21.link/P0840R2}
	\addproposal{P1771}{https://wg21.link/P1771R1}
	\addproposal{P1301}{https://wg21.link/P1301R4}
\end{frame}

\subsection*{Programmation fonctionnelle}
\begin{frame}[fragile]
	\frametitle{Lambda}
	\begin{itemize}
		\item Utilisables dans des contextes non évalués

\note[item]{Contextes non évalués : \mintinline{cpp}|sizeof()|, \mintinline{cpp}|typeid()| ou \mintinline{cpp}|decltype()|}

		\item Utilisation de paramètres templates pour les lambdas génériques
		\begin{itemize}
			\item En complément de la syntaxe avec \mintinline{cpp}|auto|
			\item Permet de récupérer le type
		\end{itemize}
	\end{itemize}

	\begin{block}{Usage}
		\begin{itemize}
			\item Spécification de contraintes sur paramètres : types identiques, itérateur, \ldots{}
		\end{itemize}
	\end{block}

	\begin{minted}{cpp}
		auto foo = []<typename T>(T first, T second) {
		  return first + second; };
	\end{minted}


	\begin{minted}{cpp}
		auto foo = []<typename T>(vector<T> const& vec) {
		  cout<< size(vec) << '\n';
		  cout<< vec.capacity() << '\n';
		};
	\end{minted}

	\begin{codesample}
		\sample{https://godbolt.org/#g:!((g:!((g:!((h:codeEditor,i:(filename:'1',fontScale:14,fontUsePx:'0',j:1,lang:c%2B%2B,selection:(endColumn:1,endLineNumber:1,positionColumn:1,positionLineNumber:1,selectionStartColumn:1,selectionStartLineNumber:1,startColumn:1,startLineNumber:1),source:'%23include+%3Ciostream%3E%0A%23include+%3Cvector%3E%0A%0Aint+main()%0A%7B%0A++auto+foo+%3D+%5B%5D%3Ctypename+T%3E(std::vector%3CT%3E+vec)+%7B+return+std::size(vec)%3B+%7D%3B%0A%0A++std::cout+%3C%3C+foo(std::vector%3Cint%3E%7B1,+2,+3%7D)+%3C%3C+%22%5Cn%22%3B%0A++std::cout+%3C%3C+foo(std::vector%3Cdouble%3E%7B1.,+2.%7D)+%3C%3C+%22%5Cn%22%3B%0A%23if+0%0A++std::cout+%3C%3C+foo(5)+%3C%3C+%22%5Cn%22%3B%0A%23endif%0A%7D%0A'),l:'5',n:'0',o:'C%2B%2B+source+%231',t:'0')),k:50,l:'4',n:'0',o:'',s:0,t:'0'),(g:!((h:executor,i:(argsPanelShown:'1',compilationPanelShown:'0',compiler:gsnapshot,compilerName:'',compilerOutShown:'0',execArgs:'',execStdin:'',fontScale:14,fontUsePx:'0',j:1,lang:c%2B%2B,libs:!(),options:'-std%3Dc%2B%2B20+-Wall+-Wextra+-pedantic',overrides:!(),runtimeTools:!(),source:1,stdinPanelShown:'1',tree:'1',wrap:'0'),l:'5',n:'0',o:'Executor+x86-64+gcc+(trunk)+(C%2B%2B,+Editor+%231)',t:'0')),header:(),k:50,l:'4',n:'0',o:'',s:0,t:'0')),l:'2',n:'0',o:'',t:'0')),version:4}
	\end{codesample}

	\begin{codesample}
		\sample{https://godbolt.org/#g:!((g:!((g:!((h:codeEditor,i:(filename:'1',fontScale:14,fontUsePx:'0',j:1,lang:c%2B%2B,selection:(endColumn:1,endLineNumber:1,positionColumn:1,positionLineNumber:1,selectionStartColumn:1,selectionStartLineNumber:1,startColumn:1,startLineNumber:1),source:'%23include+%3Ciostream%3E%0A%23include+%3Cvector%3E%0A%0Aint+main()%0A%7B%0A++auto+foo+%3D+%5B%5D%3Ctypename+T%3E(T+first,+T+second)+%7B+return+first+%2B+second%3B+%7D%3B%0A%0A++std::cout+%3C%3C+foo(1,+5)+%3C%3C+%22%5Cn%22%3B%0A%23if+0%0A++std::cout+%3C%3C+foo(1.,+5)+%3C%3C+%22%5Cn%22%3B%0A%23endif%0A%7D%0A'),l:'5',n:'0',o:'C%2B%2B+source+%231',t:'0')),k:50,l:'4',n:'0',o:'',s:0,t:'0'),(g:!((h:executor,i:(argsPanelShown:'1',compilationPanelShown:'0',compiler:gsnapshot,compilerName:'',compilerOutShown:'0',execArgs:'',execStdin:'',fontScale:14,fontUsePx:'0',j:1,lang:c%2B%2B,libs:!(),options:'-std%3Dc%2B%2B20+-Wall+-Wextra+-pedantic',overrides:!(),runtimeTools:!(),source:1,stdinPanelShown:'1',tree:'1',wrap:'0'),l:'5',n:'0',o:'Executor+x86-64+gcc+(trunk)+(C%2B%2B,+Editor+%231)',t:'0')),header:(),k:50,l:'4',n:'0',o:'',s:0,t:'0')),l:'2',n:'0',o:'',t:'0')),version:4}
	\end{codesample}

	\addproposal{P0315}{https://wg21.link/P0315R4}
	\addproposal{P0428}{https://wg21.link/P0428R2}
\end{frame}

\begin{frame}[fragile]
	\frametitle{Lambda}
	\begin{itemize}
		\item Lambda \textit{stateless} assignables et constructibles par défaut

\note[item]{\textit{stateless}, c'est à dire qui ne capture rien}
	\end{itemize}

	\begin{minted}{cpp}
		auto greater = [](auto x,auto y) { return x > y; };

		map<string, int, decltype(greater)> foo;
	\end{minted}

	\begin{itemize}
		\item Dépréciation de la capture implicite de \mintinline{cpp}|this| par \mintinline{cpp}|[=]|
		\begin{itemize}
			\item Capture explicite par \mintinline{cpp}|[=, this]|

\note[item]{\mintinline{cpp}|this| est capturé par référence}

			\item Capture implicite par \mintinline{cpp}|[&]| toujours présente
		\end{itemize}
		\item Capture de \textit{structured binding}
	\end{itemize}

	\addproposal{P0624}{https://wg21.link/P0624R2}
	\addproposal{P0409}{https://wg21.link/P0409R2}
\end{frame}

\begin{frame}[fragile]
	\frametitle{Lambda}
	\begin{itemize}
		\item Expansion des \textit{parameter packs} lors de la capture
	\end{itemize}

	\begin{minted}{cpp}
		template<class F, class... Args>
		auto delay_invoke(F f, Args... args) {
		  return [f=move(f),...args=move(args)]()->decltype(auto) {
		    return invoke(f, args...);
		  }
		}
	\end{minted}

	\begin{itemize}
		\item Peuvent être \mintinline{cpp}|consteval|
	\end{itemize}

	\addproposal{P0780}{https://wg21.link/P0780R2}
\end{frame}

\begin{frame}[fragile]
	\frametitle{Binding}
	\begin{itemize}
		\item \mintinline{cpp}|std::bind_front()| assigne les arguments fournis aux premiers paramètres de l'appelable
	\end{itemize}

	\begin{minted}{cpp}
		int foo(int a, int b, int c, int d) { return a * b * c + d; }

		auto baz = bind_front(&foo, 2, 3, 4);
		baz(7);  // 31

		// Equivalent a

		auto bar = bind(&foo, 2, 3, 4, _1);
		bar(6);  // 30
	\end{minted}

	\begin{itemize}
		\item \mintinline{cpp}|std::reference_wrapper| accepte les types incomplets

\note[item]{Exemple de types incomplets : \textit{forward declaration} ou types abstraits}
	\end{itemize}

	\begin{codesample}
		\sample{https://godbolt.org/#g:!((g:!((g:!((h:codeEditor,i:(filename:'1',fontScale:14,fontUsePx:'0',j:1,lang:c%2B%2B,selection:(endColumn:1,endLineNumber:1,positionColumn:1,positionLineNumber:1,selectionStartColumn:1,selectionStartLineNumber:1,startColumn:1,startLineNumber:1),source:'%23include+%3Cfunctional%3E%0A%23include+%3Ciostream%3E%0A%0Aint+foo(int+a,+int+b,+int+c,+int+d)%0A%7B%0A++return+a+*+b+*+c+%2B+d%3B%0A%7D%0A%0Aint+main()%0A%7B%0A++std::cout+%3C%3C+foo(2,+3,+4,+5)+%3C%3C+%22%5Cn%22%3B%0A%0A++auto+bar+%3D+std::bind(%26foo,+2,+3,+4,+std::placeholders::_1)%3B%0A++std::cout+%3C%3C+bar(6)+%3C%3C+%22%5Cn%22%3B%0A%0A++auto+baz+%3D+std::bind_front(%26foo,+2,+3,+4)%3B%0A++std::cout+%3C%3C+baz(7)+%3C%3C+%22%5Cn%22%3B%0A%7D%0A'),l:'5',n:'0',o:'C%2B%2B+source+%231',t:'0')),k:50,l:'4',n:'0',o:'',s:0,t:'0'),(g:!((h:executor,i:(argsPanelShown:'1',compilationPanelShown:'0',compiler:gsnapshot,compilerName:'',compilerOutShown:'0',execArgs:'',execStdin:'',fontScale:14,fontUsePx:'0',j:1,lang:c%2B%2B,libs:!(),options:'-std%3Dc%2B%2B20+-Wall+-Wextra+-pedantic',overrides:!(),runtimeTools:!(),source:1,stdinPanelShown:'1',tree:'1',wrap:'0'),l:'5',n:'0',o:'Executor+x86-64+gcc+(trunk)+(C%2B%2B,+Editor+%231)',t:'0')),header:(),k:50,l:'4',n:'0',o:'',s:0,t:'0')),l:'2',n:'0',o:'',t:'0')),version:4}
	\end{codesample}

	\addproposal{P0356}{https://wg21.link/P0356R5}
	\addproposal{P0357}{https://wg21.link/P0357R3}
\end{frame}

\subsection*{Multi-threading}
\begin{frame}[fragile]
	\frametitle{\mintinline[style=white]{cpp}|std::atomic|}
	\begin{itemize}
		\item \mintinline{cpp}|std::atomic<std::shared_ptr<T>>|
		\item \mintinline{cpp}|std::atomic<>| sur les types flottant
		\item Initialisation par défaut de \mintinline{cpp}|std::atomic<>|
		\item \mintinline{cpp}|std::atomic_ref| applique des modifications atomiques sur des données non-atomiques qu'il référence
		\item \mintinline{cpp}|wait()|, \mintinline{cpp}|notify_one()| et \mintinline{cpp}|notify_all()| pour attendre le changement d'état d'un \mintinline{cpp}|std::atomic|
	\end{itemize}

	\addproposal{P0718}{https://wg21.link/P0718R2}
	\addproposal{P0020}{https://wg21.link/P0020R6}
	\addproposal{P0883}{https://wg21.link/P0883R2}
	\addproposal{P0019}{https://wg21.link/P0019R8}
\end{frame}

\begin{frame}[fragile]
	\frametitle{Thread}
	\begin{itemize}
		\item Nouvelle variante \mintinline{cpp}|std::jthread|
		\begin{itemize}
			\item Automatiquement arrêté et joint lors de la destruction
		\end{itemize}
	\end{itemize}

	\begin{minted}{cpp}
		int main() { thread t(foo); }  // Erreur (terminate)
	\end{minted}

\note[item]{Il faut rajouter un appel à \mintinline{cpp}|join()| pour corriger ce problème}

	\begin{minted}{cpp}
		int main() { jthread t(foo); } // OK
	\end{minted}

	\begin{itemize}
		\item [] \begin{itemize}
			\item Peut être arrêté par l'appel à \mintinline{cpp}|request_stop()|
		\end{itemize}
	\end{itemize}

	\begin{minted}{cpp}
		void foo(stop_token st) {
		  while(!st.stop_requested()) { ... }
		}

		jthread t(foo);
		...
		t.request_stop();
	\end{minted}

	\begin{codesample}
		\sample{https://godbolt.org/#g:!((g:!((g:!((h:codeEditor,i:(filename:'1',fontScale:14,fontUsePx:'0',j:1,lang:c%2B%2B,selection:(endColumn:1,endLineNumber:1,positionColumn:1,positionLineNumber:1,selectionStartColumn:1,selectionStartLineNumber:1,startColumn:1,startLineNumber:1),source:'%23include+%3Ciostream%3E%0A%23include+%3Cthread%3E%0A%23include+%3Cchrono%3E%0A%0Ausing+namespace+std::chrono_literals%3B%0A%0Avoid+foo(std::stop_token+st)%0A%7B%0A++while(!!st.stop_requested())%0A++%7B+%0A++++std::cout+%3C%3C+%22Foo%5Cn%22%3B%0A++++std::this_thread::sleep_for(500ms)%3B%0A++%7D%0A%7D%0A%0Aint+main()%0A%7B%0A++std::jthread+t(foo)%3B%0A++std::this_thread::sleep_for(2s)%3B%0A%0A++t.request_stop()%3B%0A%7D%0A'),l:'5',n:'0',o:'C%2B%2B+source+%231',t:'0')),k:50,l:'4',n:'0',o:'',s:0,t:'0'),(g:!((h:executor,i:(argsPanelShown:'1',compilationPanelShown:'0',compiler:gsnapshot,compilerName:'',compilerOutShown:'0',execArgs:'',execStdin:'',fontScale:14,fontUsePx:'0',j:1,lang:c%2B%2B,libs:!(),options:'-std%3Dc%2B%2B20+-Wall+-Wextra+-pedantic+-pthread',overrides:!(),runtimeTools:!(),source:1,stdinPanelShown:'1',tree:'1',wrap:'0'),l:'5',n:'0',o:'Executor+x86-64+gcc+(trunk)+(C%2B%2B,+Editor+%231)',t:'0')),header:(),k:50,l:'4',n:'0',o:'',s:0,t:'0')),l:'2',n:'0',o:'',t:'0')),version:4}
	\end{codesample}

	\begin{codesample}
		\sample{https://godbolt.org/#g:!((g:!((g:!((h:codeEditor,i:(filename:'1',fontScale:14,fontUsePx:'0',j:1,lang:c%2B%2B,selection:(endColumn:1,endLineNumber:1,positionColumn:1,positionLineNumber:1,selectionStartColumn:1,selectionStartLineNumber:1,startColumn:1,startLineNumber:1),source:'%23include+%3Ciostream%3E%0A%23include+%3Cthread%3E%0A%0Avoid+foo()%0A%7B%0A++std::cout+%3C%3C+%22Foo%5Cn%22%3B%0A%7D%0A%0Aint+main()%0A%7B%0A%23if+1%0A++std::jthread+t(foo)%3B%0A%23else%0A++std::thread+t(foo)%3B%0A%23endif%0A%7D%0A'),l:'5',n:'0',o:'C%2B%2B+source+%231',t:'0')),k:50,l:'4',n:'0',o:'',s:0,t:'0'),(g:!((h:executor,i:(argsPanelShown:'1',compilationPanelShown:'0',compiler:gsnapshot,compilerName:'',compilerOutShown:'0',execArgs:'',execStdin:'',fontScale:14,fontUsePx:'0',j:1,lang:c%2B%2B,libs:!(),options:'-std%3Dc%2B%2B20+-Wall+-Wextra+-pedantic+-pthread',overrides:!(),runtimeTools:!(),source:1,stdinPanelShown:'1',tree:'1',wrap:'0'),l:'5',n:'0',o:'Executor+x86-64+gcc+(trunk)+(C%2B%2B,+Editor+%231)',t:'0')),header:(),k:50,l:'4',n:'0',o:'',s:0,t:'0')),l:'2',n:'0',o:'',t:'0')),version:4}
	\end{codesample}

	\addproposal{P0660}{https://wg21.link/P0660R10}
\end{frame}

\begin{frame}[fragile]
	\frametitle{synchronisation -- sémaphores}
	\begin{itemize}
		\item \mintinline{cpp}|std::counting_semaphore|
		\begin{itemize}
			\item Création avec la valeur maximale de possesseurs
			\item \mintinline{cpp}|max()| retourne le nombre maximal de possesseurs
			\item \mintinline{cpp}|release()| relâche, une ou plusieurs fois, le sémaphore

\note[item]{Incrémente le compteur de la valeur en paramètre, 1 par défaut}

			\item \mintinline{cpp}|acquire()| prend le sémaphore en attendant si besoin

\note[item]{Décrémente le compteur de 1 en attendant tant qu'il est nul}

			\item \mintinline{cpp}|try_acquire()| tente de prendre le sémaphore et retourne le résultat de l'opération

\note[item]{Décrémente le compteur de 1 s'il n'est pas nul et retourne \mintinline{cpp}|true| dans ce cas, et \mintinline{cpp}|false| si le compteur est nul}

			\item \mintinline{cpp}|try_acquire_for()| tente de prendre le sémaphore en attendant la durée donnée si besoin
			\item \mintinline{cpp}|try_acquire_until()| tente de prendre le sémaphore en attendant jusqu'à un temps donné si besoin
		\end{itemize}
		\item \mintinline{cpp}|std::binary_semaphore|	instanciation de \mintinline{cpp}|std::counting_semaphore| pour un unique possesseur
	\end{itemize}

	\begin{codesample}
		\sample{https://godbolt.org/#g:!((g:!((g:!((h:codeEditor,i:(filename:'1',fontScale:14,fontUsePx:'0',j:1,lang:c%2B%2B,selection:(endColumn:1,endLineNumber:1,positionColumn:1,positionLineNumber:1,selectionStartColumn:1,selectionStartLineNumber:1,startColumn:1,startLineNumber:1),source:'%23include+%3Ciostream%3E%0A%23include+%3Cthread%3E%0A%23include+%3Cchrono%3E%0A%23include+%3Csemaphore%3E%0A+%0Astd::binary_semaphore+signalMainToThread%7B0%7D%3B%0Astd::binary_semaphore+signalThreadToMain%7B0%7D%3B%0A%0Ausing+namespace+std::literals%3B%0A%0Avoid+foo()%0A%7B%0A++signalMainToThread.acquire()%3B%0A++std::cout+%3C%3C+%22%5Bthread%5D+Reception+du+signal%5Cn%22%3B%0A+%0A++std::this_thread::sleep_for(2s)%3B%0A+%0A++std::cout+%3C%3C+%22%5Bthread%5D+Envoi+a+main%5Cn%22%3B%0A++signalThreadToMain.release()%3B%0A%7D%0A+%0Aint+main()%0A%7B%0A++std::thread+worker(foo)%3B%0A++std::this_thread::sleep_for(2s)%3B%0A%0A++std::cout+%3C%3C+%22%5Bmain%5D+Envoi+au+thread%5Cn%22%3B%0A++signalMainToThread.release()%3B%0A++signalThreadToMain.acquire()%3B%0A+%0A++std::cout+%3C%3C+%22%5Bmain%5D+Got+the+signal%5Cn%22%3B+//+response+message%0A++worker.join()%3B%0A%7D%0A'),l:'5',n:'0',o:'C%2B%2B+source+%231',t:'0')),k:50,l:'4',n:'0',o:'',s:0,t:'0'),(g:!((h:executor,i:(argsPanelShown:'1',compilationPanelShown:'0',compiler:gsnapshot,compilerName:'',compilerOutShown:'0',execArgs:'',execStdin:'',fontScale:14,fontUsePx:'0',j:1,lang:c%2B%2B,libs:!(),options:'-std%3Dc%2B%2B20+-Wall+-Wextra+-pedantic+-pthread',overrides:!(),runtimeTools:!(),source:1,stdinPanelShown:'1',tree:'1',wrap:'0'),l:'5',n:'0',o:'Executor+x86-64+gcc+(trunk)+(C%2B%2B,+Editor+%231)',t:'0')),header:(),k:50,l:'4',n:'0',o:'',s:0,t:'0')),l:'2',n:'0',o:'',t:'0')),version:4}
	\end{codesample}

	\addproposal{P1135}{https://wg21.link/P1135R6}
\end{frame}

\begin{frame}[fragile]
	\frametitle{synchronisation -- latch}
	\begin{itemize}
		\item \mintinline{cpp}|std::latch| compteur descendant permettant de bloquer des threads tant qu'il n'a pas atteint zéro
		\begin{itemize}
			\item Création avec la valeur initiale du compteur
			\item \mintinline{cpp}|count_down()| décrémente le compteur
			\item \mintinline{cpp}|try_wait()| indique si le compteur a atteint zéro
			\item \mintinline{cpp}|wait()| attend jusqu'à ce que le compteur atteigne zéro
			\item \mintinline{cpp}|arrive_and_wait()| décrémente le compteur et attend qu'il atteigne zéro
		\end{itemize}
	\end{itemize}

	\begin{alertblock}{Pas d'incrément}
		\begin{itemize}
			\item Impossible d'incrémenter un \mintinline{cpp}|std::latch| et de revenir à sa valeur initiale
		\end{itemize}
	\end{alertblock}

	\begin{codesample}
		\sample{https://godbolt.org/#g:!((g:!((g:!((h:codeEditor,i:(filename:'1',fontScale:14,fontUsePx:'0',j:1,lang:c%2B%2B,selection:(endColumn:1,endLineNumber:1,positionColumn:1,positionLineNumber:1,selectionStartColumn:1,selectionStartLineNumber:1,startColumn:1,startLineNumber:1),source:'%23include+%3Ciostream%3E%0A%23include+%3Clatch%3E%0A%23include+%3Cthread%3E%0A%0Aint+main()%0A%7B%0A++const+size_t+nbLatch+%3D+5%3B%0A++std::latch+l1(nbLatch)%3B%0A++std::latch+l2(1)%3B%0A%0A++auto+work+%3D+%5B%26%5D+(int+i)%0A++++++++++++++%7B%0A++++++++++++++++std::cout+%3C%3C+%22Entree+foo%22+%3C%3C+i+%3C%3C+%22%5Cn%22%3B%0A++++++++++++++++l1.count_down()%3B%0A++++++++++++++++l2.wait()%3B%0A++++++++++++++++std::cout+%3C%3C+%22Sortie+foo%22+%3C%3C+i+%3C%3C+%22%5Cn%22%3B%0A++++++++++++++%7D%3B%0A%0A++std::cout+%3C%3C+%22Demarrage%5Cn%22%3B%0A++std::jthread+t%5BnbLatch%5D%3B%0A++for(size_t+i+%3D+0%3B+i+%3C+nbLatch%3B+%2B%2Bi)%0A++%7B%0A++++t%5Bi%5D+%3D+std::jthread(work,+i)%3B%0A++%7D%0A%0A++l1.wait()%3B%0A++std::cout+%3C%3C+%22Thread+OK%5Cn%22%3B%0A++l2.count_down()%3B%0A%7D'),l:'5',n:'0',o:'C%2B%2B+source+%231',t:'0')),k:50,l:'4',n:'0',o:'',s:0,t:'0'),(g:!((h:executor,i:(argsPanelShown:'1',compilationPanelShown:'0',compiler:gsnapshot,compilerName:'',compilerOutShown:'0',execArgs:'',execStdin:'',fontScale:14,fontUsePx:'0',j:1,lang:c%2B%2B,libs:!(),options:'-std%3Dc%2B%2B20+-Wall+-Wextra+-pedantic+-pthread',overrides:!(),runtimeTools:!(),source:1,stdinPanelShown:'1',tree:'1',wrap:'0'),l:'5',n:'0',o:'Executor+x86-64+gcc+(trunk)+(C%2B%2B,+Editor+%231)',t:'0')),header:(),k:50,l:'4',n:'0',o:'',s:0,t:'0')),l:'2',n:'0',o:'',t:'0')),version:4}
	\end{codesample}

	\addproposal{P1135}{https://wg21.link/P1135R6}
\end{frame}

\begin{frame}[fragile]
	\frametitle{synchronisation -- barrière}
	\begin{itemize}
		\item \mintinline{cpp}|std::barrier| attend qu'un certain nombre de threads n'atteigne la barrière
		\begin{itemize}
			\item Création avec le nombre de threads attendus
			\item \mintinline{cpp}|arrive()| décrémente le compteur
			\item \mintinline{cpp}|wait()| attend que le compteur atteigne zéro
			\item \mintinline{cpp}|arrive_and_wait()| décrémente le compteur et attend qu'il atteigne zéro
			\item \mintinline{cpp}|arrive_and_drop()| décrémente le compteur ainsi que la valeur initiale
			\item Une fois zéro atteint, les threads en attente sont débloqués et le compteur reprend la valeur initiale décrémentée du nombre de threads \og \textit{droppés}\fg{}
		\end{itemize}
	\end{itemize}

	\begin{codesample}
		\sample{https://godbolt.org/#g:!((g:!((g:!((h:codeEditor,i:(filename:'1',fontScale:14,fontUsePx:'0',j:1,lang:c%2B%2B,selection:(endColumn:1,endLineNumber:1,positionColumn:1,positionLineNumber:1,selectionStartColumn:1,selectionStartLineNumber:1,startColumn:1,startLineNumber:1),source:'%23include+%3Ciostream%3E%0A%23include+%3Cbarrier%3E%0A%23include+%3Cthread%3E%0A%0Aint+main()%0A%7B%0A++const+size_t+nb+%3D+5%3B%0A++std::barrier+b(nb)%3B%0A%0A++auto+work+%3D+%5B%26%5D+(int+i)%0A++++++++++++++%7B%0A++++++++++++++++std::cout+%3C%3C+%22Entree+foo%22+%3C%3C+i+%3C%3C+%22%5Cn%22%3B%0A++++++++++++++++b.arrive_and_wait()%3B%0A++++++++++++++++std::cout+%3C%3C+%22Milieu+foo%22+%3C%3C+i+%3C%3C+%22%5Cn%22%3B%0A++++++++++++++++b.arrive_and_wait()%3B%0A++++++++++++++++std::cout+%3C%3C+%22Sortie+foo%22+%3C%3C+i+%3C%3C+%22%5Cn%22%3B%0A++++++++++++++%7D%3B%0A%0A++std::cout+%3C%3C+%22Demarrage%5Cn%22%3B%0A++std::jthread+t%5Bnb%5D%3B%0A++for(size_t+i+%3D+0%3B+i+%3C+nb%3B+%2B%2Bi)%0A++%7B%0A++++t%5Bi%5D+%3D+std::jthread(work,+i)%3B%0A++%7D%0A%7D%0A'),l:'5',n:'0',o:'C%2B%2B+source+%231',t:'0')),k:50,l:'4',n:'0',o:'',s:0,t:'0'),(g:!((h:executor,i:(argsPanelShown:'1',compilationPanelShown:'0',compiler:gsnapshot,compilerName:'',compilerOutShown:'0',execArgs:'',execStdin:'',fontScale:14,fontUsePx:'0',j:1,lang:c%2B%2B,libs:!(),options:'-std%3Dc%2B%2B20+-Wall+-Wextra+-pedantic+-pthread',overrides:!(),runtimeTools:!(),source:1,stdinPanelShown:'1',tree:'1',wrap:'0'),l:'5',n:'0',o:'Executor+x86-64+gcc+(trunk)+(C%2B%2B,+Editor+%231)',t:'0')),header:(),k:50,l:'4',n:'0',o:'',s:0,t:'0')),l:'2',n:'0',o:'',t:'0')),version:4}
	\end{codesample}

	\addproposal{P1135}{https://wg21.link/P1135R6}
\end{frame}

\begin{frame}[fragile]
	\frametitle{Politique d'exécution}
	\begin{itemize}
		\item Nouvelle politique d'exécution vectorisé \mintinline{cpp}|std::unsequenced_policy|
	\end{itemize}

	\addproposal{P1001}{https://wg21.link/P1001R2}
\end{frame}

\subsection*{Coroutines}
\begin{frame}[fragile]
	\frametitle{\mintinline[style=white]{cpp}|std::coroutine| -- Présentation}
	\begin{itemize}
		\item Fonction dont l'exécution peut être suspendue et reprise
		\item Simplification du développement de code asynchrone
		\item TS publié en juillet 2017
	\end{itemize}

	\addproposal{P0912}{https://wg21.link/P0912R5}
\end{frame}

\begin{frame}[fragile]
	\frametitle{\mintinline[style=white]{cpp}|std::coroutine| -- Définition}
	\begin{itemize}
		\item Fonctions contenant
		\begin{itemize}
			\item \mintinline{cpp}|co_await| suspend l'exécution
			\item \mintinline{cpp}|co_yield| suspend l'exécution en retournant une valeur
			\item \mintinline{cpp}|co_return| termine la fonction
		\end{itemize}

		\item Restrictions
		\begin{itemize}
			\item Pas de \mintinline{cpp}|return|
			\item Pas d'argument \textit{variadic}
			\item Pas de déduction de type sur le retour
			\item Pas sur les constructeurs, destructeurs, fonctions \mintinline{cpp}|constexpr|
		\end{itemize}
	\end{itemize}

	\addproposal{P0912}{https://wg21.link/P0912R5}
\end{frame}

\begin{frame}[fragile]
	\frametitle{\mintinline[style=white]{cpp}|std::coroutine| -- Mécanismes}
	\begin{itemize}
		\item \textit{Promise} utilisée pour renvoyer valeurs et exceptions
		\item \textit{Coroutine state} interne contenant promesse, paramètres, variables locales et état du point de suspension
		\item \textit{Coroutine handle} non possédant pour poursuivre ou détruire la coroutine
		\begin{itemize}
			\item \mintinline{cpp}|operator bool()| indique si le \textit{handle} gère effectivement une coroutine
			\item \mintinline{cpp}|done()| indique si la coroutine est suspendue dans son état final
			\item \mintinline{cpp}|operator()| et \mintinline{cpp}|resume()| poursuit la coroutine
			\item \mintinline{cpp}|destroy()| détruit la coroutine
		\end{itemize}
		\item Spécialisation de \textit{coroutine handle} sur une \textit{promise}
		\begin{itemize}
			\item \mintinline{cpp}|promise()| accès à la promesse
		\end{itemize}
	\end{itemize}

	\addproposal{P0912}{https://wg21.link/P0912R5}
\end{frame}

\subsection*{Système de fichier}
\begin{frame}[fragile]
	\frametitle{\mintinline[style=white]{cpp}|std::create_directory()|}
	\begin{itemize}
		\item Échec de \mintinline{cpp}|std::create_directory()| si l'élément terminal existe et n'est pas un répertoire
	\end{itemize}

	\begin{minted}{cpp}
		create_directory("a/b/c");
		// C++17
		// Erreur si a ou b existe mais ne sont pas des repertoires
		// Pas d'erreur si c existe mais n'est pas un repertoire

		// C++20
		// Erreur dans les deux cas
	\end{minted}

	\addproposal{P1164}{https://wg21.link/P1164R1}
\end{frame}

\subsection*{Variant}
\begin{frame}[fragile]
	\frametitle{Constructeur de \mintinline[style=white]{cpp}|std::variant|}
	\begin{itemize}
		\item Contraintes sur le constructeur et l'opérateur d'affectation de \mintinline{cpp}|std::variant|
		\begin{itemize}
			\item Pas de conversion en \mintinline{cpp}|bool|
			\item Pas de \textit{narrowing conversion}
		\end{itemize}
	\end{itemize}

	\addproposal{P0608}{https://wg21.link/P0608R3}
\end{frame}

\begin{frame}[fragile]
	\frametitle{\mintinline[style=white]{cpp}|std::visit()|}
	\begin{itemize}
		\item Possibilité d'expliciter le type de retour de \mintinline{cpp}|std::visit()|
		\begin{itemize}
			\item Via un paramètre template
			\item Sinon déduit de l'application du visiteur au premier paramètre
		\end{itemize}
	\end{itemize}

	\addproposal{p0655}{https://wg21.link/p0655r1}
\end{frame}
\end{document}