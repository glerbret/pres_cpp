\documentclass[C++.tex]{subfiles}
\begin{document}

\section{C++20}
\subsection*{Présentation}
\begin{frame}
	\frametitle{Présentation}
	\begin{itemize}
		\item Travaux lancés en juillet 2017
		\item Périmètre figé en juillet 2019
		\item Version finale validée en septembre 2020
		\item Publication finale en décembre 2020
		\item Dernier Working Draft : \href{https://github.com/cplusplus/draft/releases/download/n4861/n4861.pdf}{N4861}
		\item Très bon support par GCC, Clang et Visual C++
		
\note[item]{Support quasi-complet du langage par GCC, seuls les modules ne sont que partiellement supportés par GCC 11}
\note[item]{Support très faible par les autres compilateurs}
	\end{itemize}
\end{frame}

\subsection*{Organisation}
\begin{frame}
	\frametitle{Changements d'organisation du comité}
	\begin{itemize}
		\item Création d'un \textit{Direction Group}
		\item Création d'un Study Group pour l'éducation (SG20) : aide à l'apprentissage et à l'adoption des évolutions
	\end{itemize}

	\centering
	\adjincludegraphics[height=0.7\textheight]{input_src/wg21-structure.png}

\end{frame}

\subsection*{Dépréciations et suppressions}
\begin{frame}[fragile]
	\frametitle{Dépréciations et suppressions}
	\begin{itemize}
		\item Dépréciation du terme POD et de \lstinline|std::is_pod()|
		\item Dépréciation partielle de \lstinline|volatile|

\note[item]{Seules certaines constructions à base de \lstinline|volatile| sont dépréciées, les constructions jugées utiles sont conservées}

		\item Suppression des membres dépréciés de \lstinline|std::reference_wrapper| : \lstinline|result_type|, \lstinline|argument_type|, \lstinline|first_argument_type| et \lstinline|second_argument_type|
	\end{itemize}
\end{frame}

\subsection*{compilation conditionnelle et macro}
\begin{frame}[fragile]
	\frametitle{Fonctionnalités}
	\begin{itemize}
		\item \lstinline|__has_cpp_attribute| permet de tester le support d'un attribut
		\begin{itemize}
			\item Similaire à \lstinline|__has_include| pour la présence d'entête
			\item Extensible aux attributs propriétaires d'une implémentation
		\end{itemize}
		\item Macros testant le support de fonctionnalité du langage
		\begin{itemize}
			\item \lstinline|__cpp_decltype| : support de \lstinline|decltype|
			\item \lstinline|__cpp_range_based_for| : support du \textit{range-based for loop}
			\item \lstinline|__cpp_static_assert| : support de \lstinline|static_assert|
			\item \ldots
		\end{itemize}
		\item Macros testant le support de fonctionnalités par la bibliothèque standard
		\begin{itemize}
			\item \lstinline|__cpp_lib_any| : support de \lstinline|std::any|
			\item \lstinline|__cpp_lib_chrono| : support de \lstinline|std::chrono|
			\item \lstinline|__cpp_lib_gcd_lcm| : support des fonctions \lstinline|std::gcd()| et \lstinline|std::lcm|
			\item \ldots
		\end{itemize}
	\end{itemize}

	\begin{block}{Valorisation}
		Année et au mois de l'acceptation dans le standard ou de l'évolution

\note[item]{Ainsi en testant la valeur, on peut savoir l'état d'implémentation de la fonctionnalité si celle-ci a évolué}
	\end{block}
\end{frame}

\begin{frame}[fragile]
	\frametitle{Information à la compilation}
	\begin{itemize}
		\item Entête \lstinline|<version>| : informations de version
		\begin{itemize}
			\item Contenu \textit{implementation-dependent}
			\item Typiquement : version du standard, version de la bibliothèque, \textit{release date}, copyright, \ldots
		\end{itemize}
		
		\item \lstinline|source_location| : position dans le code source
		\begin{itemize}
			\item Fichier, ligne, colonne et fonction courante
			\item Contenu \textit{implementation-dependent}

\note[item]{En particulier, la numérotation des lignes et colonnes peut commencer à 0 ou à 1}
\note[item]{Il est toutefois encouragé de numéroter à partir de 1 et de retourner 0 lorsque le numéro n'est pas connu}
\note[item]{De même le format de la chaîne contenant la fonction n'est pas définie}

			\item Remplaçant de \lstinline|__LINE__|, \lstinline|__FILE__|, \lstinline|__func__| et autres macros propriétaires
		\end{itemize}
	\end{itemize}
\end{frame}

\begin{frame}[fragile]
	\frametitle{Compilation conditionnelle}
	\begin{itemize}
		\item Ajout d'un paramètre booléen, optionnel, à \lstinline|explicit|
		\begin{itemize}
			\item Pilotage de \lstinline|explicit| via un paramètre booléen \textit{compile-time}
			\item Possibilité de rendre des constructeurs templates explicites ou non en fonction de l'instanciation
			\item Alternative à des constructions à base de macros de compilation ou de SFINAE
		\end{itemize}
	\end{itemize}
\end{frame}

\subsection*{Types}
\begin{frame}[fragile]
	\frametitle{Types entiers}
	\begin{itemize}
		\item Types entiers signés obligatoirement en compléments à 2

		\begin{block}{Situation actuelle}
			\begin{itemize}
				\item Pas de contrainte en C++
				\item 3 choix en C : signe+mantisse, complément à 1 et complément à 2
			\end{itemize}
		\end{block}

\note[item]{A priori, le même changement va avoir lieu en C2x}
\note[item]{Complément à 1 : négation de tous les bits}
\note[item]{Complément à 2 : négation de tous les bits, puis ajout de 1}

		\begin{alertblock}{Rupture de compatibilité ?}
			En pratique, toutes les implémentations actuelles sont en complément à 2
		\end{alertblock}

		\item Précision de comportements sur des types entiers signés
		\begin{itemize}
			\item Conversion vers non signé est toujours bien défini

\note[item]{C'est l'unique valeur de la destination congruent à la source modulo $2^N$ où N est le nombre de bits de la destination}

			\item Décalage à gauche : même résultat que celui du type non signé correspondant
			\item Décalage à droite : décalage arithmétique avec extension du signe
		\end{itemize}
	\end{itemize}
\end{frame}


\begin{frame}[fragile]
	\frametitle{Caractères}
	\begin{itemize}
		\item Type \lstinline|char8_t| pour les caractères
		\begin{itemize}
			\item Pendant UTF-8 de \lstinline|char16_t| et \lstinline|char32_t|
			\item Similaire en terme de taille, d'alignement, de conversion à \lstinline|unsigned char|
			\item Pas un alias sur un autre type
			\item Prise en compte dans la bibliothèque standard
		\end{itemize}
		\item Type \lstinline|u8string| pour les chaînes UTF-8
	\end{itemize}

	\begin{block}{Motivation}
		\begin{itemize}
			\item Suppression de l'ambigüité caractère UTF-8 / littéral

\note[item]{Pour rappel, le type \lstinline|char| est signé ou non en fonction de l'implémentation}

			\item Suppression d'ambigüité sur les surcharges et spécialisation de template
		\end{itemize}
	\end{block}
\end{frame}

\begin{frame}[fragile]
	\frametitle{Définition d'agrégat}
	\begin{itemize}
		\item Modification de la définition d'agrégat :
		\begin{itemize}
			\item C++17 : pas de constructeur \textit{user-provided}
			\item C++20 : pas de constructeur \textit{user-declared}

\note[item]{Ni \lstinline|==default|, ni \lstinline|==delete|}
\note[item]{Y compris dans les classes de base}
		\end{itemize}
	
		\begin{lstlisting}[language=C++]
// Agregat en C++17 pas en C++20
class S {
  S() = default; };\end{lstlisting}
	\end{itemize}
\end{frame}

\begin{frame}[fragile]
	\frametitle{Initialisation des agrégats\titlehfill{}1/2}
	\begin{itemize}
		\item Initialisation nommée des membres d'un agrégat ou d'une union
	
\note[item]{Vient plus ou moins de C99}

		\begin{lstlisting}[language=C++]
struct S { int a; int b; int c; }; 
S s{.a = 1, .c = 2};

union U { int a; char* b};
U u{.b = "foo"}; \end{lstlisting}

		\begin{alertblock}{Restrictions}
			\begin{itemize}
				\item Uniquement sur les agrégats et les unions, pas sur toutes les classes
				\item Initialisation des champs dans leur ordre de déclaration
				\item Initialisation d'un unique membre d'une union

\note[item]{Par contre, il est permis de ne pas initialiser des champs d'un agrégat}
			\end{itemize}
		\end{alertblock}
	\end{itemize}
\end{frame}

\begin{frame}[fragile]
	\frametitle{Initialisation des agrégats\titlehfill{}2/2}
	\begin{itemize}
		\item Initialisation des agrégats via des données parenthésées

		\begin{block}{Différences entre \lstinline|\{\}| et \lstinline|()| }
			\begin{itemize}
				\item \lstinline|{}| permet l'utilisation d'\textit{initializer list}
				\item \lstinline|()| permet les conversions avec perte de précision
			\end{itemize}
		\end{block}

		\begin{block}{Motivations}
			Utilisation des fonctions transférant les arguments à un constructeur sur des agrégats
		\end{block}

\note[item]{Exemple de fonctions transférant les arguments à un constructeur :  (\lstinline|std::make_unique()| ou \lstinline|emplace_back()|)}

		\item Initialisation par défaut des champs de bits
	\end{itemize}
\end{frame}

\begin{frame}[fragile]
	\frametitle{Endianess}
	\begin{itemize}
		\item Définition d'une énumération \lstinline|std::endian|
		\begin{itemize}
			\item \lstinline|little| : \textit{little-endian}
			\item \lstinline|big| : \textit{big-endian}
			\item \lstinline|native| : \textit{endianess} du système
		\end{itemize}
	\end{itemize}
\end{frame}

\subsection*{Opérateurs}
\begin{frame}[fragile]
	\frametitle{\lstinline|operator<=>()|\titlehfill{}1/2}
	\begin{itemize}
		\item Effectue une \og \textit{Three-way comparison}\fg{}
		\begin{itemize}
			\item $(a <=> b) < 0$ si $a < b$
			\item $(a <=> b) > 0$ si $a > b$
			\item $(a <=> b) == 0$ si $a$ et $b$ sont équivalents

\note[item]{Mécanisme similaire au fonctionnement de \lstinline|strcmp|, \lstinline|memcmp|, \ldots}
\note[item]{Un retour non numérique, p.ex. \lstinline|Nan|, indique des valeurs non ordonnées}

		\end{itemize}
		\item Trois types de retour possibles :
		\begin{itemize}
			\item \lstinline|std::strong_ordering| : ordre et égalité
			\item \lstinline|std::weak_ordering| : ordre et équivalence
			\item \lstinline|std::partial_ordering| : ordre partiel
		\end{itemize}

\note[item]{Égalité implique substitualité (formellement $a == b <==> f(a) == f(b)$), ce n'est pas le cas pour l'équivalence}
\note[item]{On peut avoir équivalence et non égalité lorsqu'il y a plusieurs représentation d'une même valeur : comparaison sur certains champs seulement, codage des \lstinline|float| et \lstinline|double|}
\note[item]{Avant C++20, ça pouvait être aussi le cas des entiers en bit+mantisse ou complément à 1. En effet, il y a alors deux représentations de 0 qui peuvent donner des résultats différents après certaines opérations binaires}

		\item Peut être généré par le compilateur (\lstinline|=default|)
		\begin{itemize}
			\item \lstinline|operator<=>()| des bases et membres
			\item \lstinline|operator==()| et \lstinline|operator>()|
		\end{itemize}
	\end{itemize}
\end{frame}

\begin{frame}[fragile]
	\frametitle{\lstinline|operator<=>()|\titlehfill{}2/2}
	\begin{itemize}
		\item \lstinline|operator<=>()| déclenche la génération :
		\begin{itemize}
			\item Opérateurs d'ordre (\lstinline|<|, \lstinline|<=|, \lstinline|>| et \lstinline|>=|) via \lstinline|operator<=>()|
			\item \lstinline|operator==()| via \lstinline|operator==()| des bases et membres
			\item \lstinline|operator!=()| via \lstinline|operator==()|
		\end{itemize}

		\begin{block}{\lstinline|==, !=|}
			\lstinline|operator==()| et \lstinline|operator!=()| uniquement pour des retours de type \og égalité\fg{}
		\end{block}

		\begin{alertblock}{\lstinline|==, !=, <=>|}
			\lstinline|operator==()| et \lstinline|operator!=()| ne sont pas générés à partir de \lstinline|operator<=>()|
		\end{alertblock}

\note[item]{Initialement, \lstinline|operator==()| et \lstinline|operator!=()| aussi résolu via \lstinline|operator<=>()|. Modifié pour ne pas avoir des tests d'égalité non optimaux \og par erreur\fg{} via le \textit{fallback} sur \lstinline|operator<=>()| (p.ex. court-circuit sur la taille de conteneurs)}

		\item Possible de marquer ces autres opérateurs \lstinline|=default|

\note[item]{Mais non obligatoire pour les opérateurs d'ordre générés dès que \lstinline|operator<=>()| est présent. Mais \lstinline|=default| permet de prendre son adresse}
\note[item]{\lstinline|=default| utile pour générer \lstinline|operator!=()| ou \lstinline|operator==()| en l'absence de \lstinline|operator<=>()|}
\note[item]{Un opérateur \lstinline|=default| qui ne peut pas être généré reste bien \textit{deleted}}

		\item Utilisation de l'opérateur binaire déclaré s'il existe

\note[item]{La génération implicite n'a donc lieu que s'il n'y a pas de version explicite}

		\item Supporté par la bibliothèque standard
	\end{itemize}

	\begin{block}{Fun fact}
		Cet opérateur est surnommé \og \textit{spaceshift operator}\fg{}
	\end{block}
\end{frame}

\subsection*{Namespace}
\begin{frame}[fragile]
	\frametitle{Nested namespace}
	\begin{itemize}
		\item Extension des \textit{nested namespaces} aux \textit{inline namespaces}
	\end{itemize}

	\begin{lstlisting}[language=C++]
namespace A::inline B::C { 
  int i; }

// Equivalent a
namespace A {
  inline namespace B {
    namespace C {
      int i; } } }\end{lstlisting}

\note[item]{Rappel, un \textit{inline namespace} permet de traiter ses membres comme s'ils appartenaient au namespace supérieur}
\end{frame}

\subsection*{Modules}
\begin{frame}[fragile]
	\frametitle{Modules - Présentation}
	\begin{itemize}
\note[item]{Fusion du TS de mai 2018 et de la proposition concurrente de Clang (ATOM)}

		\item Alternative au mécanisme d'inclusion

		\begin{alertblock}{Et les \lstinline|namespace| ?}
			Ne replace pas les \lstinline|namespace|
		\end{alertblock}

		\begin{itemize}
			\item Réduction des temps de compilation

\note[item]{Pas plusieurs traitements complets d'un même fichier inclus plusieurs fois}

			\item Nouveau niveau d'encapsulation
			\item Plus grande robustesse (isolation des effets des macros)
			\item Meilleurs prises en charge des bibliothèques par l'analyse statique, les optimiseurs, \ldots
			\item Gestion des inclusions multiples sans garde
			\item Compatible avec le système actuel d'inclusion

\note[item]{Le mécanisme actuel d'inclusion est conservé et n'est pas, pour l'instant, déprécié}
		\end{itemize}

		\begin{alertblock}{Bibliothèque standard}
			En C++20, la bibliothèque standard n'utilise pas les modules

\note[item]{La bibliothèque standard est restée aux inclusions}
\note[item]{Mais peut être importée via \lstinline|import|}

		\end{alertblock}
	\end{itemize}
\end{frame}

\begin{frame}[fragile]
	\frametitle{Modules - Interface Unit}
	\begin{itemize}
		\item L'\textit{Interface Unit} commence par un préambule
		\begin{itemize}
			\item Nom du module à exporter
			\item Suivi de l'import d'autres modules
			\item \ldots Éventuellement ré-exportés par le module
		\end{itemize}

		\begin{lstlisting}[language=C++]
export module foo;
import a;
export import b;\end{lstlisting}

		\item Suivi du corps exportant des symboles via le mot-clé \lstinline|export|

		\begin{lstlisting}[language=C++]
export int i;
export void bar(int j);
export {
  void baz() {...}
  long l }\end{lstlisting}
	\end{itemize}
\end{frame}

\begin{frame}[fragile]
	\frametitle{Modules - Implementation Unit}
	\begin{itemize}
		\item L'\textit{Implementation Unit} commence par un préambule
		\begin{itemize}
			\item Nom du module implémenté
			\item Suivi de l'import d'autres modules
		\end{itemize}

		\item Suivi du corps contenant les détails d'implémentation

		\begin{lstlisting}[language=C++]
module foo;
void bar(int j) { return 3 * j; }\end{lstlisting}

		\begin{block}{Note}
			Une \textit{Implementation Unit} a accès aux déclarations non exportées du module

\note[item]{Les déclarations non exportées sont visibles de l'ensemble du module}
		\end{block}
	
\pause
	
		\begin{alertblock}{Mais \ldots}
			Mais pas les autres unités de compilation même si elles importent le module
		\end{alertblock}
	\end{itemize}
\end{frame}

\begin{frame}[fragile]
	\frametitle{Modules - Partitions\titlehfill{}1/2}
	\begin{itemize}
		\item Les modules peuvent être partitionnés sur plusieurs unités
		\item \ldots les partitions fournissent alors un nom de partition

		\begin{lstlisting}[language=C++]
// Interface Unit
export module foo:part;\end{lstlisting}

		\begin{lstlisting}[language=C++]
// Implementation Unit
module foo:part;\end{lstlisting}

		\begin{alertblock}{\textit{Primary Module Interface Unit}}
			Chaque module doit contenir un et un seul \textit{Interface Unit} sans nom de partition
		\end{alertblock}

		\item Un élément peut être déclaré dans une partition et défini dans une autre
	\end{itemize}
\end{frame}

\begin{frame}[fragile]
	\frametitle{Modules - Partitions\titlehfill{}2/2}
	\begin{itemize}
		\item Les partitions sont un détail d'implémentation non visibles hors du module

\note[item]{Concrètement, hors du module on importe le module dans son ensemble}

		\item Une partition peut être importée dans une \textit{Implementation Unit}
		\item \ldots En important uniquement le nom de la partition

		\begin{lstlisting}[language=C++]
module foo;
import :part;     // Importe foo:part
import foo:part;  // Erreur\end{lstlisting}

		\item Le \textit{Primary Module Interface Unit} peut exporter les partitions
		\begin{lstlisting}[language=C++]
export module foo;
export :part1;
export :part2;\end{lstlisting}
	\end{itemize}
\end{frame}

\begin{frame}[fragile]
	\frametitle{Modules - Export de namespace}
	\begin{itemize}
		\item Un nom de namespace est exporté s'il est déclaré \lstinline|export|
		\item \ldots Ou implicitement si un de ses éléments est exporté
	
		\begin{lstlisting}[language=C++]
export namespace A {  // A est exporte
  int n; }            // A::n est exporte
	
namespace B {
  export int n;       // B::n et B sont exportes
  int m; }            // B::m n'est pas exporte\end{lstlisting}
	
		\item Les éléments d'une partie exportée d'un namespace sont exportés
	
		\begin{lstlisting}[language=C++]
// C::m est exporte mais pas C::n
namespace C { int n; }
	
export namespace C { int m; }\end{lstlisting}
	\end{itemize}
\end{frame}

\begin{frame}[fragile]
	\frametitle{Modules - Implémentation inline}
	\begin{itemize}
		\item Interface et implémentation dans un unique fichier
		\item En séparant les deux parties

		\begin{lstlisting}[language=C++]
export module m;
struct s;
export using s_ptr = s*;

module : private;
struct s {};\end{lstlisting}

		\pause

		\begin{alertblock}{Restriction}
			Uniquement dans une \textit{Primary Module Interface Unit} qui devrait être la seule unité du module
		\end{alertblock}
	\end{itemize}
\end{frame}

\begin{frame}[fragile]
	\frametitle{Modules - Utilisation}
	\begin{itemize}
		\item Import des modules via la directive \lstinline|import|

		\begin{lstlisting}[language=C++]
import foo;

// Utilisation des symboles exportes de foo\end{lstlisting}

		\item Cohabitation possible avec des inclusions
		\begin{lstlisting}[language=C++]
#include <vector>
import foo;
#include "bar.h"\end{lstlisting}
	\end{itemize}
\end{frame}

\begin{frame}[fragile]
	\frametitle{Modules - Code non-modulaire\titlehfill{}1/2}
	\begin{itemize}
		\item Inclusion d'en-têtes avant le préambule du module

\note[item]{Seules des directives \lstinline|include| peuvent apparaître}

		\begin{lstlisting}[language=C++]
module;
#include "bar.h"
export module foo;\end{lstlisting}
	
		\item Ou import des en-têtes

\note[item]{C'est une des évolutions apportées par ATOM}

		\begin{lstlisting}[language=C++]
export module foo;
import "bar.h"
import <version>\end{lstlisting}
	\end{itemize}
\end{frame}

\begin{frame}[fragile]
	\frametitle{Modules - Code non-modulaire\titlehfill{}2/2}
	\begin{itemize}
		\item Export possible des symboles inclus

		\begin{lstlisting}[language=C++]
module;
#include "bar.h" // Definit X
export module foo;
export using X = ::X;\end{lstlisting}

		\item Ou de l'en-tête dans son ensemble

		\begin{lstlisting}[language=C++]
export module foo;
export import "bar.h"\end{lstlisting}
	\end{itemize}
\end{frame}

\subsection*{Conteneurs}
\begin{frame}[fragile]
	\frametitle{Chaînes de caractères\titlehfill{}1/2}
	\begin{itemize}
		\item \lstinline|std::basic_string::reserve()| ne peut plus réduire la capacité
		\begin{itemize}
			\item L'appel avec une capacité inférieure n'a pas d'effet

\note[item]{Auparavant, la norme permettait de réduire effectivement la capacité mais ne l'imposait pas (idem \lstinline|shrink_to_fit()|)}
\note[item]{Si la capacité demandée était inférieur à la taille effective, cet appel était équivalent à un appel à \lstinline|shrink_to_fit()|}

			\item Comportement similaire à \lstinline|std::vector::reserve()|
		\end{itemize}

		\begin{block}{Rappel}
			Après \lstinline|reserve()|, la capacité est \textbf{supérieure} ou égale à la capacité demandée
		\end{block}

		\item Dépréciation de \lstinline|reserve()| sans paramètre

		\begin{block}{Réduction à la capacité utile}

\note[item]{Oui car avant, \lstinline|std::basic_string::reserve()| sans paramètre permettait de réduire la capacité en fonction de la taille effective}

			Utilisez \lstinline|shrink_to_fit()| et non \lstinline|reserve()|

\note[item]{La capacité est alors supérieure ou égale à la taille}
		\end{block}
	\end{itemize}
\end{frame}

\begin{frame}[fragile]
	\frametitle{Chaînes de caractères\titlehfill{}2/2}
	\begin{itemize}
		\item Ajout à \lstinline|std::basic_string| et \lstinline|std::string_view|
		\begin{itemize}
			\item \lstinline|starts_with()| teste si la chaîne commence par une sous-chaîne
			\item \lstinline|ends_with()| teste si la chaîne termine par une sous-chaîne
		\end{itemize}

		\begin{lstlisting}[language=C++]
string foo = "Hello world";

foo.starts_with("Hello");   // true
foo.ends_with("monde");     // false\end{lstlisting}

		\item \lstinline|std::string_view| constructible depuis une paire d'itérateurs
	\end{itemize}
\end{frame}

\begin{frame}[fragile]
	\frametitle{Conteneurs associatifs\titlehfill{}1/2}
	\begin{itemize}
		\item \lstinline|contains()| teste la présence d'une clé

\note[item]{Pour tous les conteneurs associatif, aussi bien \lstinline|map| que \lstinline|set|, \textit{ordered} qu'\textit{unordered} et single que multi}

		\begin{lstlisting}[language=C++]
map<int, string> foo{{1, "foo"}, {42, "bar"}};

foo.contains(42);  // true
foo.contains(38);  // false\end{lstlisting}
	\end{itemize}
\end{frame}

\begin{frame}[fragile]
	\frametitle{Conteneurs associatifs\titlehfill{}2/2}
	\begin{itemize}
		\item Optimisation de la recherche hétérogène dans des conteneurs non-ordonnés
		\begin{itemize}
			\item Fourniture d'une classe exposant
			\begin{itemize}
				\item Les différents foncteurs de calcul du hash
				\item Le \textit{tag} \lstinline|transparent_key_equal|
			\end{itemize}
			\item Suppression des conversions de type inutiles
		\end{itemize}

		\begin{lstlisting}[language=C++]
struct string_hash {
	using transparent_key_equal = equal_to<>;
	size_t operator()(string_view txt) const   { 
	  return hash_type{}(txt); }
	size_t operator()(const string& txt) const {
	  return hash_type{}(txt); }
	size_t operator()(const char* txt) const   {
	  return hash_type{}(txt); } };

unordered_map<string, int, string_hash> map = ...;
map.find("abc");
map.find("def"sv);\end{lstlisting}
	\end{itemize}
\end{frame}

\begin{frame}[fragile]
	\frametitle{\lstinline|std::list| et \lstinline|forward_list|}
	\begin{itemize}
		\item \lstinline|remove()|, \lstinline|remove_if()| et \lstinline|unique()| retourne le nombre d'éléments supprimés
	\end{itemize}
\end{frame}

\begin{frame}[fragile]
	\frametitle{\lstinline|std::array|}
	\begin{itemize}
		\item \lstinline|std::to_array()| construit un \lstinline|std::array| depuis un tableau C

		\begin{lstlisting}[language=C++]
auto foo = to_array({1, 2, 5, 42});

long foo[] = {1, 2, 5, 42};
auto bar = to_array(foo);

auto foo = to_array<long>({1, 2, 5, 42});\end{lstlisting}

		\item Y compris une chaîne C

		\begin{lstlisting}[language=C++]
auto foo = to_array("foo");\end{lstlisting}

		\begin{alertblock}{\lstinline|0| terminal}
			Le \lstinline|0| terminal est un élément du tableau
		\end{alertblock}
	\end{itemize}
\end{frame}

\begin{frame}[fragile]
	\frametitle{Suppression d'éléments}
	\begin{itemize}
		\item \lstinline|erase()| supprime les éléments égaux à la valeur fournie
		\item \lstinline|erase_if()| supprime les éléments satisfaisant le prédicat fourni

		\begin{lstlisting}[language=C++]
vector<int> foo {5, 12, 2, 56, 18, 33};
erase_if(foo, [](int i) {return i > 20;});
// 5 12 2 18\end{lstlisting}

		\begin{lstlisting}[language=C++]
map<int, int> bar{{5, 1}, {12, 2}, {2, 3}, {42, 4}};
erase_if(bar, [](pair<int, int> i) {return i.first > 20;});
// 2-3 5-1 12-2 \end{lstlisting}

		\item Remplacent l'idiome \og \textit{Erase-remove}\fg{} et l'utilisation de la fonction membre \lstinline|erase()|
	\end{itemize}
\end{frame}

\begin{frame}[fragile]
	\frametitle{\lstinline|std::span|\titlehfill{}1/2}
	\begin{itemize}
		\item Fournit une vue sur un conteneur contigu

\note[item]{Donc entre autres sur des tableaux C, \lstinline|std::vector|, \lstinline|std::array|, et éventuellement \lstinline|std::string|}
\note[item]{Il s'agit d'une vue, les données restent stockées par le conteneur original}

		\item Similaire à \lstinline|std::string_view|
		\item Constructible depuis un conteneur, un couple début/taille, un couple début/fin, un \textit{range} ou un autre \lstinline|std::span|

	\begin{lstlisting}[language=C++]
array<int, 5> foo = {0, 1, 2, 3, 4};
span<int> s1{foo};
span<int> s2(foo.data(), 3);\end{lstlisting}
	\end{itemize}
\end{frame}

\begin{frame}[fragile]
	\frametitle{\lstinline|std::span|\titlehfill{}2/2}
	\begin{itemize}
		\item \lstinline|begin()|, \lstinline|end()|, \ldots : itérateurs sur le \lstinline|std::span|
		\item \lstinline|size()|, \lstinline|empty()| : taille et vacuité
		\item \lstinline|operator[]()|, \lstinline|front()|, \lstinline|back()| : accès à un élément

		\begin{lstlisting}[language=C++]
array<int, 5> foo = {0, 1, 2, 3, 4};
span<int> bar{ foo.data(), 4 };

bar.front();    // 0\end{lstlisting}

		\item \lstinline|first()|, \lstinline|last()| : construction de \textit{sous-span}

		\begin{lstlisting}[language=C++]
array<int, 5> foo = {0, 1, 2, 3, 4};
span<int> bar{ foo.data(), 4 };

span<int> baz = bar.first(2);   // 0, 1\end{lstlisting}

		\item \textit{structured binding} sur des \lstinline|std::span| de taille fixe
	\end{itemize}
\end{frame}

\subsection*{Algorithmes}
\begin{frame}[fragile]
	\frametitle{Décalages d'éléments}
	\begin{itemize}
		\item \lstinline|std::shift_left()| décale les éléments vers le début de l'ensemble
		\item \lstinline|std::shift_right()| décale les éléments vers la fin de l'ensemble
		\item \ldots retournent un itérateur vers la fin (resp. début) du nouvel ensemble

		\begin{block}{Taille et décalage}
			Si le décalage est plus grand que la taille de l'ensemble, l'opération est sans effet
		\end{block}

		\begin{lstlisting}[language=C++]
vector<int> foo{5, 10, 15, 20};
shift_left(foo.begin(), foo.end(), 2); // 15, 20

vector<int> bar{5, 10, 15, 20};
shift_right(bar.begin(), bar.end(), 1); // 5, 10, 15\end{lstlisting}
	\end{itemize}
\end{frame}


\begin{frame}[fragile]
	\frametitle{Manipulation de puissances de deux}
	\begin{itemize}
		\item \lstinline|std::ispow2()| teste si un entier est une puissance de deux
		\item \lstinline|std::ceil2()| plus petite puissance de deux non strictement inférieure
		\item \lstinline|std::floor2()| plus grande puissance de deux non strictement supérieure
		\item \lstinline|std::log2p1()| plus petit nombre de bits nécessaire pour représenter un entier

		\begin{lstlisting}[language=C++]
ispow2(4u);   // true
ispow2(7u);   // false
ceil2(7u);    // 8
ceil2(8u);    // 8
floor2(7u);   // 4\end{lstlisting}
	\end{itemize}

	\begin{alertblock}{Restriction}
		Uniquement sur des entiers non signés
	\end{alertblock}
\end{frame}

\begin{frame}[fragile]
	\frametitle{Manipulation binaire}
	\begin{itemize}
		\item \lstinline|std::rotl()| et \lstinline|std::rotr()| rotations binaires

\note[item ]{Respectivement vers la gauche et la droite}

		\item \lstinline|std::countl_zero| nombre consécutif de bits à zéro depuis le plus significatif
		\item \lstinline|std::countl_one| nombre consécutif de bits à un depuis le plus significatif
		\item \lstinline|std::countr_zero| nombre consécutif de bits à zéro depuis le moins significatif
		\item \lstinline|std::countr_one| nombre consécutif de bits à un depuis le moins significatif
		\item \lstinline|std::popcount| nombre de bit à un

		\begin{lstlisting}[language=C++]
rotl(6u, 2);   // 24
rotr(6u, 1);   // 3
popcount(6u);  // 2\end{lstlisting}

		\begin{alertblock}{Restriction}
			Uniquement sur des entiers non signés
		\end{alertblock}
	\end{itemize}
\end{frame}

\begin{frame}[fragile]
	\frametitle{Conversion binaire}
	\begin{itemize}
		\item \lstinline|std::bit_cast| ré-interprétation d'une représentation binaire en un autre type
		\begin{itemize}
			\item Conversions bit-à-bit
			\item Alternative plus sûre à \lstinline|reinterpret_cast| ou \lstinline|memcpy()|
			\item Conversion \lstinline|constexpr| si possible
		\end{itemize}
	\end{itemize}

	\begin{alertblock}{Restriction}
		Uniquement sur des types \textit{trivially copyable}
	\end{alertblock}
\end{frame}

\begin{frame}[fragile]
	\frametitle{Comparaison d'entiers}
	\begin{itemize}
		\item Ajout de fonctions de comparaison d'entier : \lstinline|std::cmp_equal|, \lstinline|std::cmp_not_equal|, \lstinline|std::cmp_less|, \lstinline|std::cmp_greated|, \lstinline|std::cmp_less_equal| et \lstinline|std::cmp_greater_equal|
		\item Permettent la comparaison signé / non signé sans promotion
	\end{itemize}
\end{frame}

\begin{frame}[fragile]
	\frametitle{Mathématiques\titlehfill{}1/2}
	\begin{itemize}
		\item Définition de constantes mathématiques : $e$, $\log_2 e$, $\log_{10} e$, $\pi$, $\dfrac{1}{\pi}$, $\dfrac{1}{\sqrt{\pi}}$, $\ln{2}$, $\ln{10}$, $\sqrt{2}$, $\sqrt{3}$, $\dfrac{1}{\sqrt{3}}$, $\gamma$, $\Phi$

\note[item]{$\gamma$ est la constante de Euler-Mascheroni et $\Phi$ le nombre d'or}

		\item \lstinline|std::midpoint()| : demi-somme de deux valeurs (entières ou flottantes)

		\begin{block}{Règle d'arrondi}
			La demi-somme d'entiers est entière et arrondie vers le premier paramètre
		\end{block}
		
		\begin{lstlisting}[language=C++]
midpoint(2, 4);     // 3
midpoint(2, 5);     // 3
midpoint(5, 2);     // 4\end{lstlisting}
	\end{itemize}
\end{frame}

\begin{frame}[fragile]
\frametitle{Mathématiques\titlehfill{}2/2}
	\begin{itemize}
		\item \lstinline|std::lerp()| interpolation linéaire entre deux valeurs flottantes

		\begin{lstlisting}[language=C++]
lerp(10, 20, 0);    // 10
lerp(10, 20, 0.1);  // 11
lerp(10, 20, 0.2);  // 12
lerp(10, 20, 0.3);  // 13
lerp(10, 20, 0.4);  // 14
lerp(10, 20, 0.5);  // 15
lerp(10, 20, 0.6);  // 16
lerp(10, 20, 0.7);  // 17
lerp(10, 20, 0.8);  // 18
lerp(10, 20, 0.9);  // 19
lerp(10, 20, 1);    // 20\end{lstlisting}
	\end{itemize}
\end{frame}

\begin{frame}[fragile]
	\frametitle{Évolutions de la bibliothèque standard}
	\begin{itemize}
		\item Utilisation de l'attribut \lstinline|[[nodiscard]]|
		\item Utilisation de \lstinline|noexcept|
		\item Optimisation d'algorithmes numériques via \lstinline|std::move()|

\note[item]{\lstinline|noexcept| est déjà présent dans la bibliothèque standard, cette présence est étendue en C++20}
	\end{itemize}
\end{frame}

\subsection*{Ranges}
\begin{frame}
	\frametitle{Ranges - Présentation}
	\begin{itemize}
		\item Abstraction de plus haut niveau que les itérateurs
		\item Manipulation d'ensemble d'éléments au travers d'algorithmes et de \textit{range adaptators}
		\item Vivent dans le \lstinline|namespace std::ranges|

\note[item]{TS publié en juillet 2017}

\note[item]{Pour être précis, les \textit{range adaptors} manipulent seulement les \textit{viewable ranges}}

		\begin{block}{Pour aller plus loin}
			\begin{itemize}
				\item \href{https://accu.org/content/conf2009/AndreiAlexandrescu_iterators-must-go.pdf}{\og Iterators Must Go\fg{}} d'Andrei Alexandrescu

\note[item]{\textit{Iterators Must Go} est l'article fondateur des \textit{ranges}}

				\item Le \href{http://ericniebler.com/}{blog d'Eric Niebler}
			\end{itemize}
		\end{block}
	\end{itemize}
\end{frame}

\begin{frame}[fragile]
	\frametitle{Ranges - Itérateurs}
	\begin{itemize}
		\item \lstinline|std::common_iterator| : adaptateur d'itérateur/sentinelle représentant un range itérateur/sentinelle de types différents en un range de types similaires

\note[item]{Grâce à l'opérateur de comparaison adéquate}

		\item \lstinline|std::counted_iterator| : adaptateur d'itérateur avec un fonctionnement similaire à l'itérateur sous-jacent mais conservant la distance à la fin du range
	\end{itemize}
\end{frame}

\begin{frame}[fragile]
	\frametitle{Ranges - Concepts\titlehfill{}1/2}
	\begin{itemize}
		\item \lstinline|Range|
		\begin{itemize}
			\item Un itérateur de début
			\item Une sentinelle de fin
			\begin{itemize}
				\item Une valeur particulière
				\item Un autre itérateur

\note[item]{Ce qui revient aux paires d'itérateurs classiques}

				\item Le type vide \lstinline|std::default_sentinel_t| marquant la fin d'un range et utilisable avec des itérateurs gérant la limite du range

\note[item]{\lstinline|std::default_sentinel_t| est donc utilisable avec \lstinline|std::counted_iterator|}
\note[item]{Ce qui est appelé \textit{counted range} et est un range représenté par un itérateur de début et un nombre d'éléments}
			\end{itemize}
		\end{itemize}

		\item \lstinline|SizedRange| : taille en temps constant
		\item \lstinline|View| : copie, déplacement et affectation en temps constant
		\item \lstinline|ViewableRange| : \textit{range} convertible en \lstinline|View|
		\item \lstinline|CommonRange| : itérateurs et sentinelle ont le même type
	\end{itemize}
\end{frame}

\begin{frame}[fragile]
	\frametitle{Ranges - Concepts\titlehfill{}2/2}
	\begin{itemize}
		\item \lstinline|InputRange| : fournit des \lstinline|input_iterator|
		\item \lstinline|OutputRange| : fournit des \lstinline|output_iterator|
		\item \lstinline|ForwardRange| : fournit \lstinline|forward_iterator|
		\item \lstinline|BidirectionalRange| : fournit \lstinline|bidirectional_iterator|
		\item \lstinline|RandomAccessRange| : fournit \lstinline|random_access_iterator|
		\item \lstinline|ContiguousRange| : fournit \lstinline|contiguous_iterator|
	\end{itemize}

	\begin{block}{En résumé}
		\begin{itemize}
			\item Conteneurs : possession, copie profonde
			\item Vues : référence, copie superficielle
		\end{itemize}
	\end{block}

\end{frame}

\begin{frame}
	\frametitle{Ranges - Opérations}
	\begin{itemize}
		\item \lstinline|begin()|, \lstinline|end()|, \lstinline|cbegin()|, \lstinline|cend()|, \ldots : récupération des itérateurs

\note[item]{En version simple, \textit{const}, \textit{reverse} et \textit{const reverse}}

		\item \lstinline|size()| : récupération de la taille
		\item \lstinline|empty()| : teste la vacuité
		\item \lstinline|data()| et \lstinline|cdata()| : récupération de l'adresse de début de la plage

		\begin{alertblock}{Restrictions}
			\lstinline|data()| et \lstinline|cdata()| sur des \textit{contiguous range} uniquement
		\end{alertblock}

		\item Surcharges de différents algorithmes pour prendre des \textit{ranges} en paramètre
	\end{itemize}
\end{frame}

\begin{frame}[fragile]
	\frametitle{Ranges - Factory}
	\begin{itemize}
		\item \lstinline|std::views::empty| crée une vue vide
		\item \lstinline|std::views::single| crée une vue d'un unique élément
		\item \lstinline|std::views::iota| crée une vue en incrémentant une valeur initiale
	
		\begin{lstlisting}[language=C++]
for(int i : views::iota{1, 10})
  cout << i << ' ';
  // 1 2 3 4 5 6 7 8 9\end{lstlisting}
	
		\item \lstinline|std::views::counted| crée un range depuis un itérateur et un nombre d'éléments
	
		\begin{lstlisting}[language=C++]
int a[] = {1, 2, 3, 4, 5, 6, 7, 8, 9, 10};
for(int i : views::counted(a, 3))
  cout << i << ' ';
  // 1 2 3\end{lstlisting}
	\end{itemize}
\end{frame}

\begin{frame}[fragile]
	\frametitle{Ranges - range adaptators\titlehfill{}1/6}
	\begin{itemize}
		\item Appliquent filtres et transformations aux \textit{ranges}
		\item Associés, pour certains, à un \textit{range adaptor closure object}
		\begin{itemize}
			\item Prends un unique paramètre \lstinline|viewable_range|
			\item Retourne une \lstinline|view|
		\end{itemize}
		\item Évaluation paresseuse des \lstinline|view|
	\end{itemize}
\end{frame}

\begin{frame}[fragile]
	\frametitle{Ranges - range adaptators\titlehfill{}2/6}
	\begin{itemize}
		\item Peuvent être chaînés avec une syntaxe \og appel de fonction\fg{} \ldots

		\begin{lstlisting}[language=C++]
D(C(R));\end{lstlisting}
		
		\item \ldots Ou une syntaxe \og pipeline\fg{}

		\begin{lstlisting}[language=C++]
R | C | D;\end{lstlisting}

		\item Peuvent prendre plusieurs arguments

\note[item]{Plusieurs syntaxes pour prendre les arguments}

		\begin{lstlisting}[language=C++]
adaptor(range, args...);
adaptor(args...)(range);
range | adaptor(args...);\end{lstlisting}

\note[item]{\lstinline|adaptor(args...)| est forcément une \textit{range adaptor closure object}}
	\end{itemize}
\end{frame}

\begin{frame}[fragile]
	\frametitle{Ranges - range adaptators\titlehfill{}3/6}
	\begin{itemize}
		\item Plusieurs adapteurs fournis par la bibliothèque standard :

\note[item]{Les adapteurs \lstinline|XXX_view| ont des \textit{closure object} de la forme \lstinline|views::XXX|}

		\begin{itemize}
			\item \lstinline|all_view| : tous les éléments du range
			\item \lstinline|ref_view| : références sur les éléments du range
			\item \lstinline|filter_view| : tous les éléments satisfaisants un prédicat

			\begin{lstlisting}[language=C++]
vector<int> ints{0, 1, 2, 3, 4, 5};
auto even = [](int i){ return (i % 2) == 0;},

auto rng = ints | view::filter(even); //0, 2, 4\end{lstlisting}
 
			\item \lstinline|transform_view| : les éléments transformés par l'application d'une fonction 
			\begin{lstlisting}[language=C++]
vector<int> ints{0, 1, 2, 3, 4, 5};
auto double = [](int i){ return 2 * i;},

//0, 2, 4, 6, 8, 10
auto rng = ints | view::transform(double);\end{lstlisting}
		\end{itemize}
	\end{itemize}
\end{frame}

\begin{frame}[fragile]
	\frametitle{Ranges - range adaptators\titlehfill{}4/6}
	\begin{itemize}
		\item Plusieurs adapteurs fournis par la bibliothèque standard :

		\begin{itemize}
			\item \lstinline|take_view| : les N premiers éléments
			\item \lstinline|take_while_view| : les éléments jusqu'au premier ne satisfaisant pas un prédicat
			\item \lstinline|drop_view| : tous les éléments sauf les N premiers
			\item \lstinline|drop_while_view| : tous les éléments depuis le premier ne satisfaisant pas un prédicat
			\item \lstinline|common_view| convertit une vue en \lstinline|common_range|
			\item \lstinline|reverse_view| : éléments en sens inverse
			\item \lstinline|istream_view| : vue par application successive de \lstinline|operator>>| sur un flux
		\end{itemize}
	\end{itemize}
\end{frame}

\begin{frame}[fragile]
	\frametitle{Ranges - range adaptators\titlehfill{}5/6}
	\begin{itemize}
		\item Plusieurs adapteurs fournis par la bibliothèque standard :
		\begin{itemize}
			\item \lstinline|join_view| \og aplati\fg{} les éléments d'un \textit{range}

\begin{lstlisting}[language=C++]
vector<string> ss{"hello", " ", "world", "!"};
join_view greeting{ss};
for(char ch : greeting)
  cout << ch;  // hello world!\end{lstlisting}

			\item \lstinline|split_view| sépare un \textit{range} en élément sur un délimiteur donné
			
			\begin{lstlisting}[language=C++]
string str{"the quick brown fox"};
split_view sentence{str, ' '};
for(auto word : sentence) {
  for(char ch : word)
    cout << ch;
  cout << " *"; }
  // the *quick *brown *fox *\end{lstlisting}

		\end{itemize}
	\end{itemize}
\end{frame}

\begin{frame}[fragile]
	\frametitle{Ranges - range adaptators\titlehfill{}6/6}
	\begin{itemize}
		\item Plusieurs adapteurs fournis par la bibliothèque standard :
		\begin{itemize}
			\item \lstinline|elements_view| : la vue des N\textsuperscript{e} éléments de chaque \textit{tuple} d'une vue de \textit{tuple-likes}

			\begin{lstlisting}[language=C++]
auto figures = map {
  {"Lovelace"s, 1815}, {"Turing"s,   1912},
  {"Babbage"s,  1791}, {"Hamilton"s, 1936} };

auto years = figures | views::elements<1>;
// 1791 1936 1815 1912\end{lstlisting}

			\item \lstinline|keys_view| : la vue des clés de chaque \lstinline|std::pair| d'une vue de \lstinline|std::pair|
			\item \lstinline|values_view| : la vue des valeurs de chaque \lstinline|std::pair| d'une vue de \lstinline|std::pair|
		\end{itemize}
	\item Possible d'utiliser les algorithmes opérants sur les \textit{ranges}
	\end{itemize}
\end{frame}

\begin{frame}[fragile]
	\frametitle{Ranges - Exemples}
	\begin{lstlisting}[language=C++]
vector<int> ints{0, 1, 2, 3, 4, 5};

auto even = [](int i){ return (i % 2) == 0; };
auto square = [](int i) { return i * i; };

for(int i : ints | 
            view::filter(even) | 
            view::transform(square))
  cout << i << ' '; // 0 4 16\end{lstlisting}

	\begin{lstlisting}[language=C++]
vector<int> foo{1, 1, 2, 2, 3, 3, 5, 5, 5, 6, 9}, bar;

auto odd = [](int i){ return (i % 2) == 1; };

copy(foo | view::filter(odd) | unique{}, 
     back_inserter(bar)); // 1 3 5 9\end{lstlisting}
\end{frame}

\begin{frame}[fragile]
	\frametitle{Ranges - Projections}
	\begin{itemize}
		\item Paramètres des algorithmes pré-traitant les éléments du \textit{range}

		\begin{lstlisting}[language=C++]
vector foo { -1, 2, -3, 4, -5, 6 };
sort(foo, {}, [](int i) { return abs(i); }); 
// -1 2 -3 4 -5 6\end{lstlisting}

\note[item]{La paire d'accolade vide indique d'utiliser la fonction de comparaison par défaut (\lstinline|ranges::less|)}

		\item Ou extrayant une données des éléments du \textit{range}

		\begin{lstlisting}[language=C++]
struct Task { string desc;
              unsigned int priority { 0 };};

vector<Task> tasks { { "AAA", 10 }, {"BBB", 5 }, 
                     { "CCC", 8 }, { "DDD", 12 }};
sort(tasks, greater{}, &Task::priority);
// DDD, AAA, CCC, BBB\end{lstlisting}
	\end{itemize}
\end{frame}

\subsection*{Flux}
\begin{frame}[fragile]
	\frametitle{Gestion des flux}
	\begin{itemize}
		\item Flux synchrones
		\begin{itemize}
			\item Classe tampon synchrone : \lstinline|std::basic_syncbuf|
			\item Classe flux bufferisé synchrone : \lstinline|std::basic_osyncstream|
			\item \lstinline|emit()| transfère le tampon vers le flux de sortie
		\end{itemize}

\note[item]{Exemple issue de la norme}

		\begin{lstlisting}[language=C++]
{ osyncstream s(cout);
  s << "Hello," << '\n'; // no flush
  s.emit(); // characters transferred, cout not flushed
  s << "World!" << endl; // flush noted, cout not flushed
  s.emit(); // characters transferred, cout flushed
  s << "Greetings." << '\n'; // no flush
} // characters transferred, cout not flushed\end{lstlisting}

		\item Limitation de la taille lue dans les flux avec \lstinline|std::setw()|

		\begin{lstlisting}[language=C++]
// Seuls 24 caracteres sont lus
cin >> setw(24) >> a;\end{lstlisting}
	\end{itemize}
\end{frame}

\begin{frame}[fragile]
	\frametitle{\lstinline|std::format| - Présentation}
	\begin{itemize}
		\item API de formatage inspiré de la bibliothèque \href{https://github.com/fmtlib/fmt}{\{fmt\}}
		
		\begin{block}{Motivations}
			\begin{itemize}
				\item Formatage \og à la C\fg{} non extensible et peu sûr

\note[item]{Non extensible : ne supporte pas le types utilisateurs}
\note[item]{Par formatage \og à la C\fg{} j'entends ce qui est utilisé par \lstinline|printf| et équivalent}

				\item Flux complexes, peu performants, peu propices à l'internationalisation et la localisation, formateurs globaux

\note[item]{Peu propices à la localisation car le format et l'ordre des éléments est dans le code lui-même et ne peut pas être sortie facilement}
\note[item]{Les formateurs sont globaux, ainsi l'injection de \lstinline|std::hex| dans un flux va passer tous les affichages des entiers en hexadécimal jusqu'à un changement explicite et pas uniquement celui concerné}
			\end{itemize}
		\end{block}

		\item Formatage \textit{locale-specific} ou \textit{locale-independent}

\note[item]{Par défaut, la locale n'est pas prise en compte, mais peut l'être si souhaité}

		\item Format sous forme de chaînes utilisant \lstinline|{}| comme \textit{placeholder}
		\item Compatible avec Unicode

		\begin{block}{En attendant C++20}
			Utilisez \{fmt\} ou Boost.Format
		\end{block}
	\end{itemize}
\end{frame}

\begin{frame}[fragile]
	\frametitle{\lstinline|std::format| - API\titlehfill{}1/2}
	\begin{itemize}
		\item \lstinline|format()| retourne une chaîne

		\begin{lstlisting}[language=C++]
format("{}", "a"); 	// "a"\end{lstlisting}

		\item \lstinline|format_to()| formate dans un \lstinline|output_iterator|

		\begin{lstlisting}[language=C++]
vector<char> foo;
format_to(back_inserter(foo), "{}", "a");\end{lstlisting}

\note[item]{L'\lstinline|output_iterator| est retourné par \lstinline|format_to()|}

		\item \lstinline|format_to_n()| formate dans un \lstinline|output_iterator| avec une taille limite

		\begin{lstlisting}[language=C++]
array<char, 4> foo;
format_to_n(foo.data(), foo.size(), "{}", "a");\end{lstlisting}

\note[item]{Troncature à la taille indiquée si nécessaire}

		\begin{block}{Retour}
			\lstinline|format_to_n()| retourne l'\lstinline|output_iterator| et la taille sans troncature
		\end{block}
	\end{itemize}
\end{frame}


\begin{frame}[fragile]
	\frametitle{\lstinline|std::format| - API\titlehfill{}2/2}
	\begin{itemize}
		\item \lstinline|formatted_size()| retourne la taille nécessaire au formatage
		\item \lstinline|vformat()| et \lstinline|vformat_to()| arguments regroupés dans un \og tuple\fg{}

		\begin{lstlisting}[language=C++]
vformat("{}", make_format_args("a"));\end{lstlisting}

		\item Variations \lstinline|wchar| et \lstinline|locale|
	\end{itemize}
\end{frame}

\begin{frame}[fragile]
	\frametitle{\lstinline|std::format| - Référencement des valeurs}
	\begin{itemize}
		\item Deux types de référencement :
		\begin{itemize}
			\item Positionnel
	
			\begin{lstlisting}[language=C++]
format("{} et {}", "a", "b"); 	// "a et b"\end{lstlisting}
	
			\item indexé
	
			\begin{lstlisting}[language=C++]
format("{1} et {0}", "a", "b"); // "b et a"
format("{0} et {0}", "a");      // "a et a"\end{lstlisting}
	
\note[item]{Le référencement indexé permet de répéter une valeur et de gérer la position dans la chaîne de format (utile pour la localisation)}
	
			\begin{alertblock}{Limite}
				Il n'est pas possible de mixer les deux types
			\end{alertblock}
		\end{itemize}
	\end{itemize}
\end{frame}

\begin{frame}[fragile]
	\frametitle{\lstinline|std::format| - Spécificateurs}
	\begin{itemize}
		\item Introduits par \og : \fg{}

			\begin{lstlisting}[language=C++]
format("{:6}", 42);    	// "    42"
format("{0:6}", 42);    // "    42"\end{lstlisting}

		\item Définissent l'alignement, le \textit{padding}, le format, \ldots
		\item Deux jeux standard
		\begin{itemize}
			\item Types arithmétiques, pointeurs et chaînes
			\item \lstinline|std::chrono| : \textit{clocks}, \textit{duration} et \textit{time points}
		\end{itemize}
	\end{itemize}
\end{frame}

\begin{frame}[fragile]
	\frametitle{\lstinline|std::format| - Alignement}
	\begin{itemize}
		\item Alignement par défaut dépendant du type

			\begin{lstlisting}[language=C++]
format("{:6}", 42);    	// "    42"
format("{:6}", 'x');   	// "x     "\end{lstlisting}

		\item Fourniture du caractère de \textit{padding}

			\begin{lstlisting}[language=C++]
format("{:*6}", 42);    // "****42"\end{lstlisting}

		\item Choix de l'alignement

			\begin{lstlisting}[language=C++]
format("{:*<6}", 'x'); 	// "x*****"
format("{:*>6}", 'x'); 	// "*****x"
format("{:*^6}", 'x'); 	// "**x***"\end{lstlisting}
	\end{itemize}
\end{frame}

\begin{frame}[fragile]
	\frametitle{\lstinline|std::format| - Signe}
	\begin{itemize}
		\item Uniquement sur les négatifs : '\lstinline|-|'

\note[item]{Uniquement sur les négatifs est le comportement par défaut}

		\item Sur toutes les valeurs : '\lstinline|+|'
		\item Uniquement sur les négatifs en réservant l'espace : '\lstinline| |'

		\begin{lstlisting}[language=C++]
format("{0:},{0:+},{0:-},{0: }", 1); 	// "1,+1,1, 1"
format("{0:},{0:+},{0:-},{0: }", -1);	// "-1,-1,-1,-1"\end{lstlisting}
	\end{itemize}
\end{frame}

\begin{frame}[fragile]
	\frametitle{\lstinline|std::format| - Zéros non significatifs}
	\begin{itemize}
		\item Gestion des zéros non significatifs

		\begin{lstlisting}[language=C++]
format("{:+06d}", 120);	// "+00120"\end{lstlisting}
	\end{itemize}
\end{frame}

\begin{frame}[fragile]
	\frametitle{\lstinline|std::format| - Format\titlehfill{}1/2}
	\begin{itemize}
		\item Entiers : décimal, octal, binaire ou hexadécimal

		\begin{lstlisting}[language=C++]
format("{:d}", 42);    // "42"
format("{:x}", 42);    // "2a"
format("{:X}", 42);    // "2A"
format("{:b}", 42);    // "101010"
format("{:o}", 42);    // "52"\end{lstlisting}

		\item Caractères : valeur numérique ou caractères

\begin{lstlisting}[language=C++]
	format("{:X}", 'A');   // "41"
	format("{:c}", 'A');   // "A"\end{lstlisting}

		\item Booléens : chaîne ou nombre

		\begin{lstlisting}[language=C++]
format("{:d}", true);    // "1"
format("{:s}", true);    // "true"\end{lstlisting}
	\end{itemize}
\end{frame}

\begin{frame}[fragile]
	\frametitle{\lstinline|std::format| - Format\titlehfill{}2/2}
	\begin{itemize}
		\item Flottants : fixe, scientifique, \og court\fg{} ou hexadécimal

		\begin{lstlisting}[language=C++]
format("{:.6f}", 392.65);   // "392.650000"
format("{:.6g}", 392.65);   // "392.65"
format("{:.6e}", 392.65);   // "3.9265e+02"
format("{:.6E}", 392.65);   // "3.9265E+02"
format("{:.6a}", 42.);    	// "1.5p+5"\end{lstlisting}

\note[item]{Format court : fixe sur les petites nombres, scientifique sur les grands}
\note[item]{Par défaut court : .6g}

		\item Choix de la précision pour les flottants

		\begin{lstlisting}[language=C++]
format("{:.6f}", 1.5);				// "1.500000"
format("{:.0f},{.22f}", 1.5, 1.3);	// "2 1.30...0444089"\end{lstlisting}

		\begin{lstlisting}[language=C++]
format("{0:05.2f}, {0:.2f}", 1.5); // "01.50, 1.50"\end{lstlisting}
	\end{itemize}
\end{frame}

\begin{frame}[fragile]
	\frametitle{\lstinline|std::format| - Forme alternative}
	\begin{itemize}
		\item Affichage de la base des entiers

		\begin{lstlisting}[language=C++]
format("{:#x}", 42);   // "0x2a"\end{lstlisting}

		\item Affichage du point décimal et de l'ensemble de la précision des flottants
		\begin{lstlisting}[language=C++]
format("{:.6g}", 392.65);   // "392.65"
format("{:#.6g}", 392.65);  // "392.650"\end{lstlisting}

	\end{itemize}
\end{frame}

\begin{frame}[fragile]
	\frametitle{\lstinline|std::format| - Dates et heures}
	\begin{itemize}
		\item Format basé sur \lstinline|strftime|
		\begin{itemize}
			\item \lstinline|%y| l'année sur deux digits
			\item \lstinline|%m| le mois
			\item \lstinline|%d| le jour dans le mois
			\item \lstinline|%u| ou \lstinline|%w| le jour dans la semaine

\note[item]{1 à 7 avec 1 lundi pour u, 0 à 6 avec 0 pour dimanche pour w}

			\item \lstinline|%H| ou \lstinline|%I| l'heure

\note[item]{Respectivement sur 24h et 12h}
			
			\item \lstinline|%M| les minutes
			\item \lstinline|%S| les secondes
			\item \ldots
		\end{itemize}

		\begin{lstlisting}[language=C++]
format("{:%F %T}", chrono::system_clock::now());
// AAAA-MM-JJ HH:mm:ss"\end{lstlisting}
	\end{itemize}
\end{frame}

\begin{frame}[fragile]
	\frametitle{\lstinline|std::format| - Gestion des erreurs}
	\begin{itemize}
		\item Émission d'une exception \lstinline|std::format_error|
		\begin{itemize}
			\item Chaîne de format invalide
			\item Spécificateurs non cohérents avec le type fournit
			\item Absence de valeur
			\item Exception levée par un formateur
		\end{itemize}

		\begin{block}{Valeur surnuméraire}
			Les valeurs surnuméraires ne sont pas des erreurs et sont ignorées
		\end{block}
	\end{itemize}
\end{frame}

\begin{frame}[fragile]
	\frametitle{\lstinline|std::format| - Types utilisateur\titlehfill{}1/2}
	\begin{itemize}
		\item Par spécialisation de \lstinline|std::formatter<>|
		
		\begin{lstlisting}[language=C++]
template<>
struct formatter<T> {
	template <class ParseContext>
	auto parse(ParseContext& parse_ctx);

	template <class FormatContext>
	auto format(const T& value, FormatContext& fmt_ctx);};\end{lstlisting}
	\end{itemize}
\end{frame}

\begin{frame}[fragile]
	\frametitle{\lstinline|std::format| - Types utilisateur\titlehfill{}2/2}
	\begin{lstlisting}[language=C++]
struct MyComplex { double real; double imag;};

template <>
struct formatter<MyComplex> {
	template <class ParseContext>
	auto parse(ParseContext& parse_ctx) { 
		return parse_ctx.begin(); }

	template <class FormatContext>
	auto format(const T& value, FormatContext& fmt_ctx) {
		return format_to(fmt_ctx.out(), "{}+{}i", 
		                 value.real, value.imag); } };

format("{}", MyComplex{1, 2});	// "(1+2i)"\end{lstlisting}
\end{frame}

\subsection*{Mémoire}
\begin{frame}[fragile]
	\frametitle{Gestion mémoire}
	\begin{itemize}
		\item Support des tableaux par \lstinline|std::make_shared()|

\begin{lstlisting}[language=C++]
shared_ptr<double[]> foo = make_shared<double[]>(1024);\end{lstlisting}

		\item Déduction de la taille des tableaux par \lstinline|new()|

		\begin{lstlisting}[language=C++]
double* a = new double[]{1, 2, 3};\end{lstlisting}
	\end{itemize}
\end{frame}

\subsection*{Date et heure}
\begin{frame}[fragile]
	\frametitle{Nouvelles horloges}
	\begin{itemize}
		\item Ajout de nouvelles horloges

\note[item]{Différentes horloges avec des \textit{epoch} différents ainsi qu'une gestion différentes des secondes intercalaires}

		\begin{itemize}
			\item \lstinline|std::chrono::utc_clock| : temps universel coordonné

\note[item]{UTC : \textit{epoch} 1 janvier 1970 00:00:00, support des secondes intercalaires}

			\item \lstinline|std::chrono::gps_clock|

\note[item]{GPS : \textit{epoch} 6 janvier 1980 00:00:00 UTC, pas de seconde intercalaire}

			\item \lstinline|std::chrono::tai_clock| : temps atomique universel

\note[item]{TAI \textit{epoch} 31 décembre 1957 23:59:50 UTC, pas de seconde intercalaire}

			\item \lstinline|std::chrono::file_clock| : alias vers le temps du système de fichier
		\end{itemize}
		\item Conversion des horloges vers et depuis UTC
		\item Conversion de \lstinline|std::chrono::utc_clock| vers et depuis le temps système
		\item Conversion des horloges entre-elles

\note[item]{Conversion via UTC ou temps système si nécessaire}

		\begin{alertblock}{Conversion de \lstinline|std::chrono::file_clock|}
			Le support des conversions entre \lstinline|std::chrono::file_clock| et \lstinline|std::chrono::utc_clock| ou \lstinline|std::chrono::system_clock| est optionnel
		\end{alertblock}

		\item Pseudo-horloge \lstinline|std::chrono::local_t| temps dans la \textit{timezone} locale
	\end{itemize}
\end{frame}

\begin{frame}[fragile]
	\frametitle{Évolution de \lstinline|std::chrono::duration|}
	\begin{itemize}
		\item Ajout de \textit{helper} pour le jour, la semaine, le mois ou l'année
		\item Ajout de \lstinline|to_stream()| pour afficher une \lstinline|std::chrono::duration|
		\item Ajout de \lstinline|from_stream()| pour lire une \lstinline|std::chrono::duration|
		\item Utilisation de chaîne de format utilisant des séquences préfixées par \lstinline|%|
		\begin{itemize}
			\item \lstinline|%H| et \lstinline|%I| : l'heure (au format 24h ou 12h)
			\item \lstinline|%M| : les minutes
			\item \lstinline|%S| : les secondes
			\item \lstinline|%Y| et \lstinline|%y| : l'année (4 ou 2 chiffres)
			\item \lstinline|%m| : le numéro du mois
			\item \lstinline|%b| et \lstinline|%B| :  le nom du mois dans la locale (abrégé ou complet)
			\item \lstinline|%d| : le numéro du jour dans le mois
			\item \lstinline|%U| : le numéro de la semaine
			\item \lstinline|%Z| : l'abréviation de la \textit{timezone}
			\item \ldots

\note[item]{Et bien d'autre, notamment en prenant en compte ou non la locale}
\note[item]{Similaire à \lstinline|strftime()|}
\note[item]{Et identique à ceux utilisés dans \lstinline|std::format|}

		\end{itemize}
	\end{itemize}
\end{frame}

\begin{frame}[fragile]
	\frametitle{Calendrier}
	\begin{itemize}
		\item Gestion du calendrier grégorien
		\begin{itemize}
			\item Différentes représentations
			\begin{itemize}
				\item Année, mois
				\item Jour dans l'année, dans le mois
				\item Dernier jour du mois
				\item Jour dans la semaine, n\textsuperscript{e} jour de la semaine dans le mois

				\begin{alertblock}{Convention anglo-saxonne}
					Le premier jour de la semaine est le dimanche
				\end{alertblock}

				\item Et différentes combinaisons permettant de construire une date complète
			\end{itemize}
			\item Constantes représentant les jours de la semaine et les mois
			\item Suffixes littéraux \lstinline|y| et \lstinline|d| marquant les années et les jours
			\item \lstinline|operator/()| pour construire une date depuis un format \og humain\fg{}
		\end{itemize}

		\begin{lstlisting}[language=C++]
auto date1 = 2016y/may/29d;
auto date2 = Sunday[3]/may/2016y;\end{lstlisting}
	\end{itemize}
\end{frame}

\begin{frame}[fragile]
	\frametitle{Timezone\titlehfill{}1/2}
	\begin{itemize}
		\item Gestion des \textit{timezones}
		\begin{itemize}
			\item Gestion de la base de \textit{timezones} de l'IANA

\note[item]{IANA : \textit{Internet Assigned Numbers Authority}}

			\item Récupération de la \textit{timezone} courante
			\item Recherche d'une \textit{timezone} depuis son nom
			\item Caractéristique d'une \textit{timezone}
			\item Informations sur les secondes intercalaires
			\item Récupération du nom d'une \textit{timezone}
			\item Conversion entre \textit{timezone}
			\item Gestion des ambigüité de conversion
		\end{itemize}
	\end{itemize}

	\begin{lstlisting}[language=C++]
// 2016-05-29 07:30:06.153 UTC
auto tp = sys_days{2016y/may/29d} + 7h + 30min + 6s + 153ms; 
// 2016-05-29 16:30:06.153 JST
zoned_time zt = {"Asia/Tokyo", tp};\end{lstlisting}
\end{frame}

\begin{frame}[fragile]
	\frametitle{Timezone\titlehfill{}2/2}
	\begin{block}{En attendant C++20}
		Utilisez Boost.DateTime
	\end{block}

	\begin{block}{Pour aller plus loin}
		\href{http://site.icu-project.org/home}{ICU} supporte de nombreux calendriers et mécanismes de localisation

\note[item]{ICU : \textit{International Components for Unicode}}
\note[item]{ICU peut être une alternative à Boost.DateTime}
	\end{block}
\end{frame}

\subsection*{Range-based for loop}
\begin{frame}[fragile]
	\frametitle{Évolutions des range-based for loop}
	\begin{itemize}
		\item Initialisation dans les range-based for loop

		\begin{lstlisting}[language=C++]
vector<int> foo{1, 8, 5, 56, 42};
for(size_t i = 0; const auto& bar : foo) {
  cout << bar << " " << i << "\n";
  ++i; }\end{lstlisting}

		\item Cohérence entre \lstinline|begin()| et \lstinline|end()| :
		\begin{itemize}
			\item \og Début\fg{} et \og début + taille\fg{}
			\item fonctions membres \lstinline|begin()| et \lstinline|end()|
			\item fonctions libres \lstinline|std::begin()| et \lstinline|std::end()|
		\end{itemize}

\note[item]{Auparavant, \textit{range-based for loop} était utilisable avec des éléments présentant une des deux fonctions membres seulement et l'autre en libre}
	\end{itemize}
\end{frame}

\subsection*{Constexpr}
\begin{frame}[fragile]
	\frametitle{\lstinline|consteval|}
	\begin{itemize}
		\item Spécificateur \lstinline|consteval| : impose une évaluation \textit{compile-time}

\note[item]{\lstinline|constrexpr| le permet mais ne l'impose pas}

		\begin{itemize}
			\item \lstinline|consteval| implique \lstinline|inline|
		\end{itemize}

		\begin{lstlisting}[language=C++]
consteval int sqr(int n) { return n * n; }
constexpr int r = sqr(100);  // OK
int x = 100;
int r2 = sqr(x);             // Erreur\end{lstlisting}

		\begin{alertblock}{Restriction}
			Pas de pointeur dans des contextes \lstinline|consteval|
		\end{alertblock}
	\end{itemize}
\end{frame}

\begin{frame}[fragile]
	\frametitle{\lstinline|constinit|}
	\begin{itemize}
		\item Spécificateur \lstinline|constinit| : impose une initialisation durant la phase \textit{static initialization}
		\begin{itemize}
			\item Uniquement sur des objets dont la \textit{storage duration} est \textit{static} ou \textit{thread}
			\item Mal-formé en cas d'initialisation dynamique
			\item Adresse le \textit{static initialization order fiasco}
		\end{itemize}
	\end{itemize}
\end{frame}

\begin{frame}[fragile]
	\frametitle{Évolutions de \lstinline|constexpr|\titlehfill{}1/2}
	\begin{itemize}
		\item Initialisation triviale dans des contextes \lstinline|constexpr|
		\item \lstinline|std::is_constant_evaluated()| pour savoir si l'évaluation est \textit{compile-time}
		\item Prise en compte étendue de \lstinline|constexpr| dans la bibliothèque standard (\lstinline|std::string|, \lstinline|std::vector|, algorithmes numériques, \ldots)
	\end{itemize}
\end{frame}

\begin{frame}[fragile]
	\frametitle{Évolutions de \lstinline|constexpr|\titlehfill{}2/2}
	\begin{itemize}
		\item Assouplissement des restrictions de \lstinline|constexpr|
		\begin{itemize}
			\item Utilisation d'\lstinline|union| dans du code \lstinline|constexpr|
			\item Utilisation de \lstinline|try {} catch()| dans du code \lstinline|constexpr|
			\begin{itemize}
				\item Comporte comme \textit{no-ops} en \textit{compile-time}
				\item Ne peut pas lancer d'exception \textit{compile-time}
			\end{itemize}
			\item Utilisation de \lstinline|dynamic_cast| et \lstinline|typeid| dans du code \lstinline|constexpr|
			\item Déclaration de fonctions virtuelles \lstinline|constexpr|
			\item Utilisation de \lstinline|asm|
		\end{itemize}
	\end{itemize}
\end{frame}

\subsection*{structured binding}
\begin{frame}[fragile]
	\frametitle{Évolutions des structured binding\titlehfill{}1/2}
	\begin{itemize}
		\item Extension à tous les membres visibles (et plus uniquement publics)

\note[item]{\textit{structured binding} sur les membres privés depuis une fonction membre ou amis}
	
		\item Plus proche de variables \og classiques\fg{}
		\begin{itemize}
			\item Capture par les lambdas (copie et référence)

\note[item]{En C++17, les lambdas ne peuvent pas capturer de \textit{structured binding}}

		\begin{lstlisting}[language=C++]
tuple foo{5, 42};

auto[a, b] = foo;
auto f1 = [a] { return a; };
auto f2 = [=] { return b; };\end{lstlisting}

			\item Possibilité de les déclarer \lstinline|inline|, \lstinline|extern|, \lstinline|static|, \lstinline|thread_local| ou \lstinline|constexpr|
			\item Possibilité de les marquer \lstinline|[[maybe_unused]]|
		\end{itemize}
	\end{itemize}
\end{frame}

\begin{frame}[fragile]
	\frametitle{Évolutions des structured binding\titlehfill{}2/2}
	\begin{itemize}
		\item Recherche de \lstinline|get()| : seules les fonctions membres templates dont le premier paramètre template n'est pas un type sont retenues

		\begin{block}{Motivation}
			Utiliser des classes possédant un \lstinline|get()| indépendant de l'interface \textit{tuple-like}
		\end{block}

		\begin{lstlisting}[language=C++]
struct X : shared_ptr<int> { string foo; };

template<int N> string& get(X& x) { 
  if constexpr(N==0) return x.foo;}
template<> class tuple_size<X> : 
  public integral_constant<int, 1> {};
template<> class tuple_element<0, X> {
  public: using type = string;};

X x;
auto& [y] = x;\end{lstlisting}

\note[item]{Exemple invalide en C++17 à cause du \lstinline|get()| de \lstinline|shared_ptr| qui n'était pas utilisable mais empêchait de rechercher dans le \lstinline|namespace| englobant. En C++20 cette fonction n'est pas retenue et on recherche dans le \lstinline|namespace|}
	\end{itemize}
\end{frame}

\subsection*{Template}

\begin{frame}[fragile]
	\frametitle{Non-Type Template Parameters}
	\begin{itemize}
		\item Utilisation possible de classes
		\begin{itemize}
			\item \textit{strong structural equality}
			\begin{itemize}
				\item Classes de base et membres non statiques avec une \textit{defaulted} \lstinline|operator==()|
				\item Pas de référence
				\item Pas de type flottant
			\end{itemize}
			\item Pas d'union
		\end{itemize}
	\end{itemize}

	\begin{lstlisting}[language=C++]
template<chrono::seconds seconds>
class fixed_timer { /* ... */ };\end{lstlisting}

\note[item]{En C++17, il fallait utiliser un type entier, p.ex. \lstinline|size_t|}

	\begin{lstlisting}[language=C++]
template<fixed_string Id>
class entity { /* ... */ };

entity<"hello"> e;\end{lstlisting}

\note[item]{En C++17, il faut utiliser un ensemble de \lstinline|char| : \lstinline|template<char... Id>|}
\end{frame}

\begin{frame}[fragile]
	\frametitle{Évolutions des templates}
	\begin{itemize}
		\item \lstinline|typename| optionnel lorsque seul un nom de type est possible
		
\note[item]{Il ne sert qu'à lever des ambigüités}

		\item Spécialisation possible sur des classes internes privées ou protégées
		\item \lstinline|std:type_identity<>| désactive la déduction de type

		\begin{lstlisting}[language=C++]
template<class T>
void f(T, T);

f(4.2, 0); // erreur, int ou double\end{lstlisting}

		\begin{lstlisting}[language=C++]
template<class T>
void g(T, type_identity_t<T>);

g(4.2, 0); // OK, g<double>\end{lstlisting}
	\end{itemize}
\end{frame}

\begin{frame}[fragile]
	\frametitle{Paramètres \lstinline|auto|}
	\begin{itemize}
		\item Création de fonctions templates via l'usage d'\lstinline|auto|

		\begin{lstlisting}[language=C++]
void foo(auto a, auto b) {...};\end{lstlisting}

		\item Similaire	à la création de lambdas polymorphiques
	\end{itemize}
\end{frame}

\subsection*{Concepts}
\begin{frame}[fragile]
	\frametitle{Concepts - Présentation}
	\begin{itemize}
		\item Histoire ancienne et mouvementée
		\begin{itemize}
			\item Prévu initialement pour C++0x
			\item \ldots Et cause des décalages successifs
			\item Retrait à grand bruit de C++11
			\item Finalement Concept lite TS publié en 2015
			\item Intégration du TS acceptée en juillet 2017

\note[item]{Mais pas en C++17, il fallait le support par les compilateurs avant pour valider le concept}
		\end{itemize}
		\item Définir des contraintes sur les paramètres templates et l'inférence de type
		\begin{itemize}
			\item Diagnostics plus clair
			\item Meilleur documentation du code
			\item Aide à la déduction de type
			\item Aide à la résolution de spécialisation

\note[item]{Basé du coup sur un \og nom\fg{} et non sur la structure (typage nominal/structurel)}

		\end{itemize}
		\item Propositions visiblement abandonnées
		\begin{itemize}
			\item \textit{Axiom} : spécification de propriétés sémantiques d'un concept

\note[item]{Exemple d'\textit{axiom} : associativité ou commutativité}
\note[item]{Intérêt des \textit{axiom} : meilleur documentation du code, informations au compilateur (meilleurs avertissement et diagnostics, meilleures optimisations)}

			\item \textit{Concept map} : transformation entre un concept et un type ne le satisfaisant pas

\note[item]{Exemple de \textit{concept map} : Utilisation de type définissant \lstinline|>| vers un concept attendant \lstinline|<|, implémentation du concept de pile (\textit{stack}) vers \lstinline|std::vector|}

		\end{itemize}
	\end{itemize}
\end{frame}

\begin{frame}[fragile]
	\frametitle{Concepts - Utilisation template\titlehfill{}1/4}
	\begin{itemize}
		\item Utilisable via une \textit{Requires clause}

		\begin{lstlisting}[language=C++]
template<typename T> requires Incrementable<T>
void foo(T);\end{lstlisting}

		\item \ldots via une \textit{Trailing requires clause}

		\begin{lstlisting}[language=C++]
template<typename T>
void foo(T) requires Incrementable<T>;\end{lstlisting}
		
		
		\item \ldots via des paramètres templates contraints

		\begin{lstlisting}[language=C++]
template<Decrementable T>
void foo(T);\end{lstlisting}

		\item \ldots ou via des combinaisons de ces syntaxes
	\end{itemize}
\end{frame}

\begin{frame}[fragile]
	\frametitle{Concepts - Utilisation template\titlehfill{}2/4}
	\begin{itemize}
		\item Utilisable depuis un concept nommé

		\begin{lstlisting}[language=C++]
// On suppose le concept Addable existant
template<typename T> requires Addable<T>
T add(T a, T b) { return a + b; }\end{lstlisting}

		\item \ldots Depuis des expressions
 
		\begin{lstlisting}[language=C++]
template<typename T>
requires requires (T x) { x + x; }
T add(T a, T b) { return a + b; }\end{lstlisting}

		\begin{lstlisting}[language=C++]
template<typename T>
requires (sizeof(T) > 1)
void foo(T);\end{lstlisting}

\note[item]{Oui, \lstinline|requires| est bien en double}
	\end{itemize}
\end{frame}

\begin{frame}[fragile]
	\frametitle{Concepts - Utilisation template\titlehfill{}3/4}
	\begin{itemize}
		\item Peuvent être composés

		\begin{lstlisting}[language=C++]
template<typename T>
requires (sizeof(T) > 1 && sizeof(T) <= 4)
void foo(T);\end{lstlisting}

		\begin{lstlisting}[language=C++]
template<typename T>
requires (sizeof(T) == 2 || sizeof(T) == 4)
void foo(T);\end{lstlisting}
	\end{itemize}
\end{frame}

\begin{frame}[fragile]
	\frametitle{Concepts - Utilisation template\titlehfill{}4/4}
	\begin{itemize}
		\item Support des \textit{parameters pack}

		\begin{lstlisting}[language=C++]
template<typename... T>
requires Concept<T> && ... && true
void foo(T...);\end{lstlisting}

		\begin{lstlisting}[language=C++]
template<Concept... T>
void foo(T...);\end{lstlisting}
	\end{itemize}
\end{frame}

\begin{frame}[fragile]
	\frametitle{Concepts - Utilisation inférence de type\titlehfill{}1/2}
	\begin{itemize}
		\item Contraintes sur les paramètres (lambdas et fonctions templates)

		\begin{lstlisting}[language=C++]
[](Constraint auto a) {...};
void foo(Constraint auto a) {...};\end{lstlisting}

		\item Contraintes sur les types de retour

		\begin{lstlisting}[language=C++]
Constraint auto foo();
auto bar() -> Constraint decltype(auto);\end{lstlisting}

	\end{itemize}
\end{frame}

\begin{frame}[fragile]
	\frametitle{Concepts - Utilisation inférence de type\titlehfill{}2/2}
	\begin{itemize}
		\item Contraintes sur les variables

		\begin{lstlisting}[language=C++]
Constraint auto bar = foo();
Constraint decltype(auto) baz = foo();\end{lstlisting}

		\item Contraintes sur les \textit{non-type template parameters}

		\begin{lstlisting}[language=C++]
template<Constraint auto S>
void foo();\end{lstlisting}

		\item Support des \textit{parameters pack}
	
		\begin{lstlisting}[language=C++]
void foo(Constraint auto... T);\end{lstlisting}
	\end{itemize}
\end{frame}

\begin{frame}[fragile]
	\frametitle{Concepts - Standard}
	\begin{itemize}
		\item De nombreux concepts définis dans la bibliothèque standard

\note[item]{Construit depuis les traits \lstinline|is_XXX_v| et la composition de ceux-ci généralement}

		\begin{itemize}
			\item Relations entre types : \lstinline|same_as|, \lstinline|derived_from|, \lstinline|convertible_to|, \lstinline|common_with|, \ldots
			\item Types numériques : \lstinline|integral|, \lstinline|signed_integral|, \lstinline|unsigned_integral|, \lstinline|floating_point|, \ldots
			\item Opérations supportées : \lstinline|swappable|, \lstinline|destructible|, \lstinline|default_constructible|, \lstinline|move_constructible|, \lstinline|copy_constructible|, \ldots
			\item Catégories de types : \lstinline|movable|, \lstinline|copyable|, \lstinline|semiregular|, \lstinline|regular|, \ldots

\note[item]{\lstinline|semiregular| : \lstinline|copyable| et \lstinline|default_constructible|}
\note[item]{\lstinline|regular| : \lstinline|semiregular| et \lstinline|equality_comparable|}

			\item Comparaisons : \lstinline|boolean|, \lstinline|equality_comparable|, \lstinline|totally_ordered|, \ldots
			\item \textit{Callable concepts} : \lstinline|invocable|, \lstinline|predicate|, \lstinline|strict_weak_order|, \ldots

\note[item]{\lstinline|predicate| : appelable et retourne un bool}

			\item \ldots
		\end{itemize}
	\end{itemize}
\end{frame}

\begin{frame}[fragile]
	\frametitle{Concepts - Définition\titlehfill{}1/4}
	\begin{itemize}
		\item Peuvent être définis depuis des expressions

		\begin{lstlisting}[language=C++]
template<typename T>
concept Addable = requires (T x) { x + x; };\end{lstlisting}

		\begin{lstlisting}[language=C++]
template <class T, class U = T>
concept Swappable = requires(T&& t, U&& u) {
  swap(forward<T>(t), forward<U>(u));
  swap(forward<U>(u), forward<T>(t)); };\end{lstlisting}

	\end{itemize}
\end{frame}

\begin{frame}[fragile]
	\frametitle{Concepts - Définition\titlehfill{}2/4}
	\begin{itemize}
		\item Y compris en retirant des qualifieurs

		\begin{lstlisting}[language=C++]
template<class T> concept Addable = requires(
  const remove_reference_t<T>& a,
  const remove_reference_t<T>& b) { a + b; };\end{lstlisting}

		\item Ou en imposant les types de retour

		\begin{lstlisting}[language=C++]
template<class T> concept Comparable = requires(
{ a == b } -> boolean;
{ a != b } -> boolean; };\end{lstlisting}
	\end{itemize}
\end{frame}

\begin{frame}[fragile]
	\frametitle{Concepts - Définition\titlehfill{}3/4}
	\begin{itemize}
		\item Depuis des \textit{traits}

		\begin{lstlisting}[language=C++]
template<class T>
concept integral = is_integral_v<T>;\end{lstlisting}

		\begin{lstlisting}[language=C++]
template<class T, class... Args>
concept constructible_from = 
  destructible<T> && is_constructible_v<T, Args...>;\end{lstlisting}
	\end{itemize}
\end{frame}

\begin{frame}[fragile]
	\frametitle{Concepts - Définition\titlehfill{}4/4}
	\begin{itemize}
		\item Depuis d'autres concepts

		\begin{lstlisting}[language=C++]
template<class T>
concept semiregular = copyable<T> && default_constructible<T>;\end{lstlisting}

		\item En combinant différentes méthodes

		\begin{lstlisting}[language=C++]
template<class T> concept totally_ordered =
  equality_comparable<T> &&
  requires(const remove_reference_t<T>& a,
           const remove_reference_t<T>& b) {
    { a < b } -> boolean;
    { a > b } -> boolean;
    { a <= b } -> boolean;
    { a >= b } -> boolean; };\end{lstlisting}
	\end{itemize}
\end{frame}

\subsection*{Attributs}
\begin{frame}[fragile]
	\frametitle{Évolutions des Attributs}
	\begin{itemize}
		\item Ajout de nouveaux attributs
		\begin{itemize}
			\item \lstinline|[[likely]]| et \lstinline|[[unlikely]]| : probabilité de branches conditionnelles

\note[item]{Fourni au compilateur des information lui permettant des optimisations plus pertinentes}

			\begin{alertblock}{Avec parcimonie}
				Les compilateurs savent déjà déterminer les branches les plus probables, généralement mieux que nous 
			\end{alertblock}

			\item \lstinline|[[no_unique_address]]| : l'adresse d'un membre peut être partagée
		
\note[item]{Et permettre l'EMO (\textit{Empty Member Optimisation})}
		\end{itemize}

		\item Extension de \lstinline|[[nodiscard]]| aux constructeurs
		\begin{itemize}
			\item marquage \lstinline|[[nodiscard]]| des constructeurs est autorisé

\note[item]{Auparavant, GCC levait un avertissement}

			\item Vérification également lors des conversions via les constructeurs

\note[item]{Constructeurs non \lstinline|explicit|}
		\end{itemize}
		\item Possibilité d'associer un message à \lstinline|[[nodiscard]]|

\note[item]{Pour indiquer la motivation du marquage \lstinline|[[nodiscard]]| ou l'erreur potentielle que cache la non-vérification du retour}
	\end{itemize}
\end{frame}

\subsection*{Programmation fonctionnelle}
\begin{frame}[fragile]
	\frametitle{Évolutions des lambdas\titlehfill{}1/2}
	\begin{itemize}
		\item Utilisables dans des environnements non évalués
		\item Utilisation de paramètres templates pour les lambdas génériques

		\begin{lstlisting}[language=C++]
auto foo = []<typename T>(vector<T> bar) { ... };\end{lstlisting}

		\begin{itemize}
			\item En complément de la syntaxe avec \lstinline|auto|

\note[item]{Les deux syntaxes peuvent cohabiter}

			\item Permet de récupérer le type

		\end{itemize}

		\begin{block}{Usage}
			Spécification de contraintes sur paramètres : types identiques, itérateur, \ldots
		\end{block}
	\end{itemize}

	\begin{lstlisting}[language=C++]
auto foo = []<typename T>(vector<T> const& vec) { 
  cout<< size(vec) << '\n';
  cout<< vec.capacity() << '\n'; };\end{lstlisting}

\note[item]{Cet exemple impose l'utilisation d'un vecteur}
\end{frame}

\begin{frame}[fragile]
	\frametitle{Évolutions des lambdas\titlehfill{}2/2}
	\begin{itemize}
		\item Lambda \textit{stateless} assignables et constructibles par défaut

\note[item]{\textit{stateless}, c'est à dire qui ne capture rien}
\note[item]{Le but est de pouvoir écrire des choses comme ci-dessous et de rendre les lambdas \textit{stateless} encore plus interchangeable avec les foncteurs}

		\begin{lstlisting}[language=C++]
auto greater = [](auto x,auto y) {return x > y; };
map<string, int, decltype(greater)> foo;\end{lstlisting}

		\item Dépréciation de la capture implicite de \lstinline|this| par \lstinline|[=]|
		\begin{itemize}
			\item Capture explicite par \lstinline|[=, this]|
			\item Capture implicite par \lstinline|[&]| toujours présente
		\end{itemize}
		\item Capture de \textit{structured binding}
		\item Expansion des \textit{parameter packs} lors de la capture

		\begin{lstlisting}[language=C++]
template<class F, class... Args>
auto delay_invoke(F f, Args... args) {
  return [f=move(f),...args=move(args)]()->decltype(auto) 
    {return invoke(f, args...);};}\end{lstlisting}

		\item Lambdas peuvent être \lstinline|consteval|
	\end{itemize}
\end{frame}

\begin{frame}[fragile]
	\frametitle{Binding}
	\begin{itemize}
		\item \lstinline|std::bind_front()| assigne les arguments fournis aux premiers paramètres de l'appelable

	\begin{lstlisting}[language=C++]
int foo(int a, int b, int c, int d) {
	return a * b * c + d; }

auto baz = bind_front(&foo, 2, 3, 4);
baz(7);  // 31

// Equivalent a

auto bar = bind(&foo, 2, 3, 4, _1);
bar(6);  // 30\end{lstlisting}

		\item \lstinline|std::reference_wrapper| accepte les types incomplets

\note[item]{Exemple de types incomplets : \textit{forward declaration} ou types abstraits}
	\end{itemize}
\end{frame}

\subsection*{Multi-threading}
\begin{frame}[fragile]
	\frametitle{\lstinline|std::atomic|}
	\begin{itemize}
		\item Ajout de \lstinline|std::atomic<std::shared_ptr<T>>|
		\item Ajout de \lstinline|std::atomic<>| sur les types flottant
		\item Ajout de l'initialisation par défaut de \lstinline|std::atomic<>|
		\item \lstinline|std::atomic_ref| applique des modifications atomiques sur des données non-atomiques qu'il référence
		\item \lstinline|wait()|, \lstinline|notify_one()| et \lstinline|notify_all()| pour attendre le changement d'état d'un \lstinline|std::atomic|
	\end{itemize}
\end{frame}

\begin{frame}[fragile]
	\frametitle{Thread}
	\begin{itemize}
		\item Nouvelle variante de thread : \lstinline|std::jthread|
		\begin{itemize}
			\item Automatiquement arrêté et joint lors de la destruction

\note[item]{Qui est finalement un motif classique de l'usage des threads}

\begin{lstlisting}[language=C++]
int main() { thread t(foo); }  // Erreur (terminate)\end{lstlisting}

\note[item]{Il faut rajouter un appel à \lstinline|join| pour corriger ce problème}

\begin{lstlisting}[language=C++]
int main() { jthread t(foo); } // OK\end{lstlisting}

			\item Peut être arrêté par l'appel à \lstinline|request_stop()|

\begin{lstlisting}[language=C++]
void foo(stop_token st) {
  while(!st.stop_requested()) { /* ... */ }}

jthread t(foo);
// ...
t.request_stop();\end{lstlisting}
		\end{itemize}
	\end{itemize}
\end{frame}

\begin{frame}[fragile]
	\frametitle{synchronisation - sémaphores}
	\begin{itemize}
		\item \lstinline|std::counting_semaphore|
		\begin{itemize}
			\item Création avec la valeur maximale de possesseurs
			\item \lstinline|max()| retourne le nombre maximal de possesseurs
			\item \lstinline|release()| relâche, une ou plusieurs fois, le sémaphore

\note[item]{Incrémente le compteur de la valeur en paramètre, 1 par défaut}

			\item \lstinline|acquire()| prend le sémaphore en attendant si besoin

\note[item]{Décrémente le compteur de 1 en attendant tant qu'il est nul}

			\item \lstinline|try_acquire()| tente de prendre le sémaphore et retourne le résultat de l'opération

\note[item]{Décrémente le compteur de 1 s'il n'est pas nul et retourne \lstinline|true| dans ce cas, et \lstinline|false| si le compteur est nul}

			\item \lstinline|try_acquire_for()| tente de prendre le sémaphore en attendant la durée donnée si besoin
			\item \lstinline|try_acquire_until()| tente de prendre le sémaphore en attendant jusqu'à un temps donné si besoin

		\end{itemize}
		\item \lstinline|std::binary_semaphore|	instanciation de \lstinline|std::counting_semaphore| pour un unique possesseur
	\end{itemize}
\end{frame}

\begin{frame}[fragile]
	\frametitle{synchronisation - latch}

\note[item]{Conservation du terme anglais car verrou est ambigu (\textit{latch} ou \textit{lock})}

	\begin{itemize}
		\item \lstinline|std::latch| compteur descendant permettant de bloquer des threads tant qu'il n'a pas atteint zéro 
		\begin{itemize}
			\item Création avec la valeur initiale du compteur
			\item \lstinline|count_down()| décrémente le compteur
			\item \lstinline|try_wait()| indique si le compteur a atteint zéro
			\item \lstinline|wait()| attend jusqu'à ce que le compteur atteigne zéro
			\item \lstinline|arrive_and_wait()| décrémente le compteur et attend qu'il atteigne zéro
		\end{itemize}

		\begin{alertblock}{Pas d'incrément}
			Il n'est pas possible d'incrémenter un \lstinline|std::latch| ni de revenir à sa valeur initiale
		\end{alertblock}
	\end{itemize}
\end{frame}

\begin{frame}[fragile]
	\frametitle{synchronisation - barrière}
	\begin{itemize}
		\item \lstinline|std::barrier| attend qu'un certain nombre de threads n'atteigne la barrière
		\begin{itemize}
			\item Création avec le nombre de threads attendus
			\item \lstinline|arrive()| décrémente le compteur
			\item \lstinline|wait()| attends que le compteur atteigne zéro
			\item \lstinline|arrive_and_wait()| décrémente le compteur et attends qu'il atteigne zéro
			\item \lstinline|arrive_and_drop()| décrémente le compteur ainsi que la valeur initiale
			\item Une fois zéro atteint, les threads en attente sont débloqués et le compteur reprends la valeur initiale décrémentée du nombre de threads \og \textit{droppés}\fg{}
		\end{itemize}
	\end{itemize}
\end{frame}

\begin{frame}[fragile]
	\frametitle{Politique d'exécution}
	\begin{itemize}
		\item Ajout d'une nouvelle politique d'exécution vectorisé \lstinline|std::unsequenced_policy|
	\end{itemize}
\end{frame}

\subsection*{Coroutines}
\begin{frame}[fragile]
	\frametitle{\lstinline|std::coroutine| - Présentation}
	\begin{itemize}
		\item Fonction dont l'exécution peut être suspendue et reprise
		\item Simplification du développement de code asynchrone
		\item TS publié en juillet 2017
	\end{itemize}
\end{frame}

\begin{frame}[fragile]
	\frametitle{\lstinline|std::coroutine| - Définition}
	\begin{itemize}
		\item Fonction contenant
		\begin{itemize}
			\item \lstinline|co_await| suspend l'exécution
			\item \lstinline|co_yield| suspend l'exécution en retournant une valeur
			\item \lstinline|co_return| termine la fonction
		\end{itemize}

		\item Des restrictions
		\begin{itemize}
			\item Pas de \lstinline|return|
			\item Pas d'argument \textit{variadic}
			
\note[item]{Pas d'arguments variables à la C via \lstinline|...|}
			
			\item Pas de déduction de type sur le retour
			
\note[item]{Ni \lstinline|auto| ni concepts}
			
			\item Pas sur les constructeurs, destructeurs, fonctions \lstinline|constexpr|
		\end{itemize}

	\end{itemize}
\end{frame}

\begin{frame}[fragile]
	\frametitle{\lstinline|std::coroutine| - Mécanismes}
	\begin{itemize}
		\item \textit{Promise} utilisée pour renvoyer valeurs et exceptions
		\item \textit{Coroutine state} interne contenant promesse, paramètres, variables locales et état du point de suspension
		\item \textit{Coroutine handle} non possédant pour poursuivre ou détruire la coroutine
		\begin{itemize}
			\item \lstinline|operator bool()| : le \textit{handle} gère effectivement une coroutine
			\item \lstinline|done()| : la coroutine est suspendue dans son état final

\note[item]{C'est à dire qu'elle est vraiment terminée}

			\item \lstinline|operator()()| et \lstinline|resume()| poursuit la coroutine
			\item \lstinline|destroy()| détruit la coroutine
		\end{itemize}
		\item Spécialisation de \textit{coroutine handle} sur une \textit{promise}
		\begin{itemize}
			\item \lstinline|promise()| accès à la promesse
		\end{itemize}
	\end{itemize}
\end{frame}

\begin{frame}[fragile]
	\frametitle{\lstinline|std::coroutine| - Exemple}
	\begin{lstlisting}[language=C++]
struct generator {
  ...
  bool next() { 
    return cor ? (cor.resume(), !cor.done()) : false; }
  int value() { 
    return cor.promise().current_value; }

  coroutine_handle<promise_type> cor; };

generator f() { co_yield 1; co_yield 2; }

auto g = f();
while(g.next()) cout << g.value() << endl;\end{lstlisting}
\end{frame}


\subsection*{Système de fichier}
\begin{frame}[fragile]
	\frametitle{\lstinline|std::create_directory()|}
	\begin{itemize}
		\item \lstinline|std::create_directory()| échoue si l'élément terminal existe et n'est pas un répertoire
	\end{itemize}

\begin{lstlisting}[language=C++]
create_directory("a/b/c");
// Erreur en C++17 si a ou b existe mais ne sont pas des repertoires
// Pas d'erreur en C++17 si c existe mais n'est pas un repertoire

// Erreur en C++20 dans les deux cas\end{lstlisting}
\end{frame}

\subsection*{Variant}
\begin{frame}[fragile]
	\frametitle{Constructeur de \lstinline|std::variant|}
	\begin{itemize}
		\item Contraintes sur le constructeur et l'opérateur d'affectation de \lstinline|std::variant|
		\begin{itemize}
			\item Pas de conversion en \lstinline|bool|

			\begin{lstlisting}[language=C++]
variant<string, bool> x = "abc";
// C++17 : bool, C++20 : string\end{lstlisting}

			\item Pas de \textit{narrowing conversion}

			\begin{lstlisting}[language=C++]
variant<float, long> v;
v = 0;
// C++17 : erreur, C++20 : long\end{lstlisting}
		\end{itemize}
	\end{itemize}
\end{frame}

\begin{frame}[fragile]
	\frametitle{\lstinline|std::visit()|}
	\begin{itemize}
		\item Possibilité d'expliciter le type de retour de \lstinline|std::visit()|
		\begin{itemize}
			\item Via un paramètre template
			\item Sinon déduit de l'application du visiteur au premier paramètre
		\end{itemize}
	\end{itemize}
\end{frame}
\end{document}