\documentclass[C++.tex]{subfiles}
\begin{document}

\section{C++14}
\subsection*{Présentation}
\begin{frame}
	\frametitle{Présentation}
	\begin{itemize}
		\item Approuvé le 16 août 2014
		\item Dernier \textit{Working Draft} : \href{https://timsong-cpp.github.io/cppwp/n4140/draft.pdf}{N4140\linklogo}
		\item Dans la continuité de C++11
		\item Changements moins importants
		\item Mais loin d'une simple version correctrice
	\end{itemize}
\end{frame}

\subsection*{constexpr}
\begin{frame}[fragile]
	\frametitle{\mintinline[style=white]{cpp}|constexpr|}
	\begin{itemize}
		\item Fonctions membres \mintinline{cpp}|constexpr| plus implicitement \mintinline{cpp}|const|
		\item Relâchement des contraintes sur les fonctions \mintinline{cpp}|constexpr|
		\begin{itemize}
			\item Variables locales (ni \mintinline{cpp}|static|, ni \mintinline{cpp}|thread_local|, obligatoirement initialisées)
			\item Objets mutables créés lors l'évaluation de l'expression constante
			\item \mintinline{cpp}|if|, \mintinline{cpp}|switch|, \mintinline{cpp}|while|, \mintinline{cpp}|for|, \mintinline{cpp}|do while|
		\end{itemize}
		\item Application de \mintinline{cpp}|constexpr| à plusieurs éléments de la bibliothèque standard
	\end{itemize}

	\addproposal{N3652}{https://wg21.link/N3652}
\end{frame}

\subsection*{Déduction de type}
\begin{frame}[fragile]
	\frametitle{Généralisation de la déduction du type retour}
	\begin{itemize}
		\item Utilisable sur les lambdas complexes
	\end{itemize}

	\begin{minted}{cpp}
		[](int x) { 
		  if(x >= 0) return 2 * x; 
		  else return -2 * x;
		};
	\end{minted}

	\begin{itemize}
		\item Mais aussi sur les fonctions
	\end{itemize}

	\begin{minted}{cpp}
		auto bar(int x) {
		  if(x >= 0) return 2 * x; 
		  else return -2 * x;
		}
	\end{minted}

	\addproposal{N3638}{https://wg21.link/N3638}
\end{frame}

\begin{frame}[fragile]
	\frametitle{Généralisation de la déduction du type retour}
	\begin{itemize}
		\item Y compris récursive
	\end{itemize}

	\begin{minted}{cpp}
		auto fact(unsigned int x) {
		  if(x == 0) return 1U;
		  else return x * fact(x - 1);
		}
	\end{minted}

	\begin{alertblock}{Contraintes}
		\begin{itemize}
			\item Un \mintinline{cpp}|return| doit précéder l'appel récursive
			\item Tous les chemins doivent avoir le même type de retour
		\end{itemize}
	\end{alertblock}

	\begin{tikzpicture}[remember picture,overlay]
	\node[xshift=-5mm,yshift=6mm] at (current page.south east){%
		\href{https://godbolt.org/#g:!((g:!((g:!((h:codeEditor,i:(filename:'1',fontScale:14,fontUsePx:'0',j:1,lang:c%2B%2B,selection:(endColumn:1,endLineNumber:26,positionColumn:1,positionLineNumber:26,selectionStartColumn:1,selectionStartLineNumber:26,startColumn:1,startLineNumber:26),source:'%23include+%3Ciostream%3E%0A%0Astatic+auto+bar(int+x)%0A%7B%0A++if(x+%3E%3D+0)+return+2+*+x%3B%0A++else+return+-2+*+x%3B%0A%7D%0A%0Astatic+auto+fact(unsigned+int+x)%0A%7B%0A++if(x+%3D%3D+0)+return+1U%3B%0A++else+return+x+*+fact(x-1)%3B%0A%7D%0A%0Aint+main()%0A%7B%0A++%7B%0A++++std::cout+%3C%3C+bar(5)+%3C%3C+!'%5Cn!'%3B%0A++++std::cout+%3C%3C+bar(-2)+%3C%3C+!'%5Cn!'%3B%0A++%7D%0A%0A++%7B%0A++++std::cout+%3C%3C+fact(4)+%3C%3C+!'%5Cn!'%3B%0A++%7D%0A%7D%0A'),l:'5',n:'0',o:'C%2B%2B+source+%231',t:'0')),k:50,l:'4',n:'0',o:'',s:0,t:'0'),(g:!((h:executor,i:(argsPanelShown:'1',compilationPanelShown:'0',compiler:gsnapshot,compilerName:'',compilerOutShown:'0',execArgs:'',execStdin:'',fontScale:14,fontUsePx:'0',j:1,lang:c%2B%2B,libs:!(),options:'-std%3Dc%2B%2B14+-Wall+-Wextra',overrides:!(),runtimeTools:!(),source:1,stdinPanelShown:'1',tree:'1',wrap:'0'),l:'5',n:'0',o:'Executor+x86-64+gcc+(trunk)+(C%2B%2B,+Editor+%231)',t:'0')),header:(),k:50,l:'4',n:'0',o:'',s:0,t:'0')),l:'2',n:'0',o:'',t:'0')),version:4}{\codelogo}};
	\end{tikzpicture}
\end{frame}

\begin{frame}[fragile]
	\frametitle{\mintinline[style=white]{cpp}|decltype(auto)|}
	\begin{itemize}
		\item Déduction du type retour en conservant la référence
	\end{itemize}

	\begin{minted}{cpp}
		string bar("bar");

		string  foo1() { return string("foo"); }
		string& bar1() { return bar; }

		decltype(auto) foo2() { return foo1(); } // string
		decltype(auto) bar2() { return bar1(); } // string&
		auto foo3() { return foo1(); }           // string
		auto bar3() { return bar1(); }           // string
	\end{minted}

	\addproposal{N3638}{https://wg21.link/N3638}
\end{frame}

\subsection*{Initialisation}
\begin{frame}[fragile]
	\frametitle{Aggregate Initialisation}
	\begin{itemize}
		\item Compatible avec l'initialisation par défaut des membres
		\item Initialisation par défaut des membres non explicitement initialisés
	\end{itemize}

	\begin{minted}{cpp}
		struct Foo {int i, int j = 5};

		Foo foo{42};   // i = 42, j = 5
	\end{minted}

	\begin{tikzpicture}[remember picture,overlay]
	\node[xshift=-5mm,yshift=6mm] at (current page.south east){%
		\href{https://godbolt.org/#g:!((g:!((g:!((h:codeEditor,i:(filename:'1',fontScale:14,fontUsePx:'0',j:1,lang:c%2B%2B,selection:(endColumn:1,endLineNumber:23,positionColumn:1,positionLineNumber:23,selectionStartColumn:1,selectionStartLineNumber:23,startColumn:1,startLineNumber:23),source:'%23include+%3Ciostream%3E%0A%0Astruct+Foo%0A%7B%0A++int+a%3B%0A++int+b+%3D+42%3B%0A%7D%3B%0A%0Aint+main()%0A%7B%0A++%7B%0A++++Foo+foo%7B6%7D%3B%0A%0A++++std::cout+%3C%3C+foo.a+%3C%3C+!'+!'+%3C%3C+foo.b+%3C%3C+!'%5Cn!'%3B%0A++%7D%0A%0A++%7B%0A++++Foo+foo%7B6,+5%7D%3B%0A%0A++++std::cout+%3C%3C+foo.a+%3C%3C+!'+!'+%3C%3C+foo.b+%3C%3C+!'%5Cn!'%3B%0A++%7D%0A%7D%0A'),l:'5',n:'0',o:'C%2B%2B+source+%231',t:'0')),k:50,l:'4',n:'0',o:'',s:0,t:'0'),(g:!((h:executor,i:(argsPanelShown:'1',compilationPanelShown:'0',compiler:gsnapshot,compilerName:'',compilerOutShown:'0',execArgs:'',execStdin:'',fontScale:14,fontUsePx:'0',j:1,lang:c%2B%2B,libs:!(),options:'-std%3Dc%2B%2B14+-Wall+-Wextra',overrides:!(),runtimeTools:!(),source:1,stdinPanelShown:'1',tree:'1',wrap:'0'),l:'5',n:'0',o:'Executor+x86-64+gcc+(trunk)+(C%2B%2B,+Editor+%231)',t:'0')),header:(),k:50,l:'4',n:'0',o:'',s:0,t:'0')),l:'2',n:'0',o:'',t:'0')),version:4}{\codelogo}};
	\end{tikzpicture}

	\addproposal{N3653}{https://wg21.link/N3653}
\end{frame}

\subsection*{Itérateurs}
\begin{frame}[fragile]
	\frametitle{Itérateurs}
	\begin{itemize}
		\item Fonctions libres \mintinline{cpp}|std::cbegin()| et \mintinline{cpp}|std::cend()|
		\item Fonctions libres \mintinline{cpp}|std::rbegin()| et \mintinline{cpp}|std::rend()|
		\item Fonctions libres \mintinline{cpp}|std::crbegin()| et \mintinline{cpp}|std::crend()|
		\item \textit{Null forward iterator} ne référençant aucun conteneur valide
	\end{itemize}

	\begin{minted}{cpp}
		auto ni = vector<int>::iterator();
		auto nd = vector<double>::iterator();

		ni == ni;    // true
		nd != nd;    // false
		ni == nd;    // Erreur de compilation
	\end{minted}

	\begin{alertblock}{Attention}
		\begin{itemize}
			\item \textit{Null forward iterator} non comparables avec des itérateurs classiques
		\end{itemize}
	\end{alertblock}

	\addproposal{N3644}{https://wg21.link/N3644}
\end{frame}

\subsection*{Conteneur}
\begin{frame}[fragile]
	\frametitle{Recherche hétérogène}
	\begin{itemize}
		\item Optimisation de la recherche hétérogène dans les conteneurs associatifs ordonnés
		\item Fourniture d'une classe exposant
		\begin{itemize}
			\item Fonction de comparaison
			\item \textit{Tag} \mintinline{cpp}|is_transparent|
		\end{itemize}
		\item Suppression de conversions inutiles
	\end{itemize}

	\addproposal{N3657}{https://wg21.link/N3657}
\end{frame}

\subsection*{Algorithmes}
\begin{frame}[fragile]
	\frametitle{Algorithmes}
	\begin{itemize}
		\item Surcharge de \mintinline{cpp}|std::equal()|, \mintinline{cpp}|std::mismatch()| et de \mintinline{cpp}|std::is_permutation()| prenant deux paires complètes d'itérateurs

\note[item]{Il n'est donc plus nécessaire de tester la taille auparavant}
	\end{itemize}

	\begin{minted}{cpp}
		vector<int> foo{1, 2, 3};
		vector<int> bar{10, 11};

		equal(begin(foo), end(foo), begin(bar), end(bar));
	\end{minted}

	\begin{itemize}
		\item \mintinline{cpp}|std::exchange()| change la valeur d'un objet et retourne l'ancienne
	\end{itemize}

	\begin{minted}{cpp}
		vector<int> foo{1, 2, 3};

		vector<int> bar = exchange(foo, {10, 11});
		// foo : 10 11, bar : 1, 2, 3
	\end{minted}

	\begin{block}{Dépréciation}
		\begin{itemize}
			\item Dépréciation de \mintinline{cpp}|std::random_shuffle()|
		\end{itemize}

\note[item]{Remplacé par \mintinline{cpp}|std::shuffle()| qui permet un meilleur aléa}
	\end{block}

	\addproposal{N3668}{https://wg21.link/N3668}
	\addproposal{N3671}{https://wg21.link/N3671}
\end{frame}

\subsection*{string}
\begin{frame}[fragile]
	\frametitle{Quoted string}
	\begin{itemize}
		\item Insertion et extraction de chaînes avec guillemets
	\end{itemize}

	\begin{minted}{cpp}
		stringstream ss;
		string in = "String with spaces and \"quotes\"";
		string out;

		ss << quoted(in);
		cout << "in:  '" << in << "'\n"
		     << "stored as '" << ss.str() << "'\n";
		// in : 'String with spaces and "quotes"'
		// stored as '"String with spaces and \"quotes\""'

		ss >> quoted(out);
		cout << "out: '" << out << "'\n";
		// out: 'String with spaces, and "quotes"'
	\end{minted}

\note[item]{Exemple issue de cppreference.com}

	\begin{tikzpicture}[remember picture,overlay]
	\node[xshift=-5mm,yshift=6mm] at (current page.south east){%
		\href{https://godbolt.org/#g:!((g:!((g:!((h:codeEditor,i:(filename:'1',fontScale:14,fontUsePx:'0',j:1,lang:c%2B%2B,selection:(endColumn:1,endLineNumber:18,positionColumn:1,positionLineNumber:18,selectionStartColumn:1,selectionStartLineNumber:18,startColumn:1,startLineNumber:18),source:'%23include+%3Ciostream%3E%0A%23include+%3Cstring%3E%0A%23include+%3Csstream%3E%0A%23include+%3Ciomanip%3E%0A%0Aint+main()%0A%7B%0A++std::stringstream+ss%3B%0A++std::string+in+%3D+%22String+with+spaces+and+%5C%22quotes%5C%22%22%3B%0A++std::string+out%3B%0A%0A++ss+%3C%3C+std::quoted(in)%3B%0A++std::cout+%3C%3C+%22in:++!'%22+%3C%3C+in+%3C%3C+%22!'%5Cn%22+%3C%3C+%22stored+as+!'%22+%3C%3C+ss.str()+%3C%3C+%22!'%5Cn%22%3B%0A%0A++ss+%3E%3E+std::quoted(out)%3B%0A++std::cout+%3C%3C+%22out:+!'%22+%3C%3C+out+%3C%3C+%22!'%5Cn%22%3B%0A%7D%0A'),l:'5',n:'0',o:'C%2B%2B+source+%231',t:'0')),k:50,l:'4',n:'0',o:'',s:0,t:'0'),(g:!((h:executor,i:(argsPanelShown:'1',compilationPanelShown:'0',compiler:gsnapshot,compilerName:'',compilerOutShown:'0',execArgs:'',execStdin:'',fontScale:14,fontUsePx:'0',j:1,lang:c%2B%2B,libs:!(),options:'-std%3Dc%2B%2B14+-Wall+-Wextra',overrides:!(),runtimeTools:!(),source:1,stdinPanelShown:'1',tree:'1',wrap:'0'),l:'5',n:'0',o:'Executor+x86-64+gcc+(trunk)+(C%2B%2B,+Editor+%231)',t:'0')),header:(),k:50,l:'4',n:'0',o:'',s:0,t:'0')),l:'2',n:'0',o:'',t:'0')),version:4}{\codelogo}};
	\end{tikzpicture}

	\addproposal{N3654}{https://wg21.link/N3654}
\end{frame}

\subsection*{Littéraux}
\begin{frame}[fragile]
	\frametitle{Littéraux binaires}
	\begin{itemize}
		\item Support des littéraux binaires préfixés par \mintinline{cpp}|0b|
	\end{itemize}

	\begin{minted}{cpp}
		int foo = 0b101010; // 42
	\end{minted}

	\begin{tikzpicture}[remember picture,overlay]
	\node[xshift=-5mm,yshift=6mm] at (current page.south east){%
		\href{https://godbolt.org/#g:!((g:!((g:!((h:codeEditor,i:(filename:'1',fontScale:14,fontUsePx:'0',j:1,lang:c%2B%2B,selection:(endColumn:1,endLineNumber:9,positionColumn:1,positionLineNumber:9,selectionStartColumn:1,selectionStartLineNumber:9,startColumn:1,startLineNumber:9),source:'%23include+%3Ciostream%3E%0A%0Aint+main()%0A%7B%0A++int+foo+%3D+0b101010%3B%0A%0A++std::cout+%3C%3C+foo+%3C%3C+!'%5Cn!'%3B%0A%7D%0A'),l:'5',n:'0',o:'C%2B%2B+source+%231',t:'0')),k:50,l:'4',n:'0',o:'',s:0,t:'0'),(g:!((h:executor,i:(argsPanelShown:'1',compilationPanelShown:'0',compiler:gsnapshot,compilerName:'',compilerOutShown:'0',execArgs:'',execStdin:'',fontScale:14,fontUsePx:'0',j:1,lang:c%2B%2B,libs:!(),options:'-std%3Dc%2B%2B14+-Wall+-Wextra',overrides:!(),runtimeTools:!(),source:1,stdinPanelShown:'1',tree:'1',wrap:'0'),l:'5',n:'0',o:'Executor+x86-64+gcc+(trunk)+(C%2B%2B,+Editor+%231)',t:'0')),header:(),k:50,l:'4',n:'0',o:'',s:0,t:'0')),l:'2',n:'0',o:'',t:'0')),version:4}{\codelogo}};
	\end{tikzpicture}

	\addproposal{N3472}{https://wg21.link/N3472}
\end{frame}

\begin{frame}[fragile]
	\frametitle{Séparateurs}
	\begin{itemize}
		\item Utilisation possible de ' dans les nombres littéraux
	\end{itemize}

	\begin{minted}{cpp}
		int foo = 0b0010'1010;  // 42
		int bar = 1'000;        // 1000
		int baz = 010'00;       // 512
	\end{minted}

	\begin{block}{Note}
		\begin{itemize}
			\item Purement esthétique, aucune sémantique ni place réservée
		\end{itemize}
	\end{block}

	\begin{tikzpicture}[remember picture,overlay]
	\node[xshift=-5mm,yshift=6mm] at (current page.south east){%
		\href{https://godbolt.org/#g:!((g:!((g:!((h:codeEditor,i:(filename:'1',fontScale:14,fontUsePx:'0',j:1,lang:c%2B%2B,selection:(endColumn:1,endLineNumber:9,positionColumn:1,positionLineNumber:9,selectionStartColumn:1,selectionStartLineNumber:9,startColumn:1,startLineNumber:9),source:'%23include+%3Ciostream%3E%0A%0Aint+main()%0A%7B%0A++std::cout+%3C%3C+0b0010!'1010+%3C%3C+!'%5Cn!'%3B%0A++std::cout+%3C%3C+1!'000+%3C%3C+!'%5Cn!'%3B%0A++std::cout+%3C%3C+010!'00+%3C%3C+!'%5Cn!'%3B%0A%7D%0A'),l:'5',n:'0',o:'C%2B%2B+source+%231',t:'0')),k:50,l:'4',n:'0',o:'',s:0,t:'0'),(g:!((h:executor,i:(argsPanelShown:'1',compilationPanelShown:'0',compiler:gsnapshot,compilerName:'',compilerOutShown:'0',execArgs:'',execStdin:'',fontScale:14,fontUsePx:'0',j:1,lang:c%2B%2B,libs:!(),options:'-std%3Dc%2B%2B14+-Wall+-Wextra',overrides:!(),runtimeTools:!(),source:1,stdinPanelShown:'1',tree:'1',wrap:'0'),l:'5',n:'0',o:'Executor+x86-64+gcc+(trunk)+(C%2B%2B,+Editor+%231)',t:'0')),header:(),k:50,l:'4',n:'0',o:'',s:0,t:'0')),l:'2',n:'0',o:'',t:'0')),version:4}{\codelogo}};
	\end{tikzpicture}

	\addproposal{N3781}{https://wg21.link/N3781}
\end{frame}

\begin{frame}[fragile]
	\frametitle{User-defined literals standards}
	\begin{itemize}
		\item Suffixe \mintinline{cpp}|s| sur les chaînes : \mintinline{cpp}|std::string|
	\end{itemize}

	\begin{minted}{cpp}
		auto foo = "abcd"s;   // string
	\end{minted}

	\begin{block}{Note}
		\begin{itemize}
			\item Remplace \mintinline{cpp}|std::string{"abcd"}|
		\end{itemize}
	\end{block}

	\begin{alertblock}{Attention}
		\begin{itemize}
			\item Nécessite l'utilisation de \mintinline{cpp}|using namespace std::literals|
		\end{itemize}

\note[item]{Ou \mintinline{cpp}|using namespace std::string_literals| ou \mintinline{cpp}|using namespace std::literals::string_literals|}
	\end{alertblock}

	\begin{tikzpicture}[remember picture,overlay]
	\node[xshift=-5mm,yshift=6mm] at (current page.south east){%
		\href{https://godbolt.org/#g:!((g:!((g:!((h:codeEditor,i:(filename:'1',fontScale:14,fontUsePx:'0',j:1,lang:c%2B%2B,selection:(endColumn:1,endLineNumber:17,positionColumn:1,positionLineNumber:17,selectionStartColumn:1,selectionStartLineNumber:17,startColumn:1,startLineNumber:17),source:'%23include+%3Ciostream%3E%0A%23include+%3Cstring%3E%0A%23include+%3Ccassert%3E%0A%0Ausing+namespace+std::literals%3B%0A%0Aint+main()%0A%7B%0A++auto+s1+%3D+%22Abcd%22%3B%0A++auto+s2+%3D+%22Abcd%22s%3B%0A%0A%23if+1%0A++assert(typeid(s1)+%3D%3D+typeid(std::string))%3B%0A%23endif%0A++assert(typeid(s2)+%3D%3D+typeid(std::string))%3B%0A%7D%0A'),l:'5',n:'0',o:'C%2B%2B+source+%231',t:'0')),k:50,l:'4',n:'0',o:'',s:0,t:'0'),(g:!((h:executor,i:(argsPanelShown:'1',compilationPanelShown:'0',compiler:gsnapshot,compilerName:'',compilerOutShown:'0',execArgs:'',execStdin:'',fontScale:14,fontUsePx:'0',j:1,lang:c%2B%2B,libs:!(),options:'-std%3Dc%2B%2B14+-Wall+-Wextra',overrides:!(),runtimeTools:!(),source:1,stdinPanelShown:'1',tree:'1',wrap:'0'),l:'5',n:'0',o:'Executor+x86-64+gcc+(trunk)+(C%2B%2B,+Editor+%231)',t:'0')),header:(),k:50,l:'4',n:'0',o:'',s:0,t:'0')),l:'2',n:'0',o:'',t:'0')),version:4}{\codelogo}};
	\end{tikzpicture}

	\addproposal{N3642}{https://wg21.link/N3642}
\end{frame}

\begin{frame}[fragile]
	\frametitle{User-defined literals standards}
	\begin{itemize}
		\item Suffixe \mintinline{cpp}|h|, \mintinline{cpp}|min|, \mintinline{cpp}|s|, \mintinline{cpp}|ms|, \mintinline{cpp}|us| et \mintinline{cpp}|ns| : \mintinline{cpp}|std::chrono::duration|
	\end{itemize}

	\begin{minted}{cpp}
		auto foo = 60s;   // chrono::seconds
		auto bar = 5min;  // chrono::minutes
	\end{minted}

	\addproposal{N3642}{https://wg21.link/N3642}
\end{frame}

\begin{frame}[fragile]
	\frametitle{User-defined literals standards}
	\begin{itemize}
		\item Suffixe \mintinline[escapeinside=||]{cpp}{|if|} : nombre imaginaire de type \mintinline{cpp}|std::complex<float>|
		\item Suffixe \mintinline{cpp}|i| : nombre imaginaire de type \mintinline{cpp}|std::complex<double>|
		\item Suffixe \mintinline{cpp}|il| : nombre imaginaire de type \mintinline{cpp}|std::complex<long double>|
	\end{itemize}

	\begin{minted}{cpp}
		auto foo = 5i;  // complex<double>
	\end{minted}

	\addproposal{N3642}{https://wg21.link/N3642}
\end{frame}

\subsection*{tuple}
\begin{frame}[fragile]
	\frametitle{Adressage des \mintinline[style=white]{cpp}|std::tuple| par le type}
	\begin{itemize}
		\item Utilisation du type plutôt que de l'indice
	\end{itemize}

	\begin{minted}{cpp}
		tuple<int, long, long> foo{42, 58L, 9L};

		get<int>(foo);  // 42
	\end{minted}

	\begin{alertblock}{Attention}
		\begin{itemize}
			\item Uniquement s'il n'y a qu'une occurrence du type dans le \mintinline{cpp}|std::tuple|
		\end{itemize}

		\begin{minted}{cpp}
			get<long>(foo);  // Erreur
		\end{minted}
	\end{alertblock}

	\begin{tikzpicture}[remember picture,overlay]
	\node[xshift=-5mm,yshift=6mm] at (current page.south east){%
		\href{https://godbolt.org/#g:!((g:!((g:!((h:codeEditor,i:(filename:'1',fontScale:14,fontUsePx:'0',j:1,lang:c%2B%2B,selection:(endColumn:1,endLineNumber:13,positionColumn:1,positionLineNumber:13,selectionStartColumn:1,selectionStartLineNumber:13,startColumn:1,startLineNumber:13),source:'%23include+%3Ciostream%3E%0A%23include+%3Ctuple%3E%0A%0Aint+main()%0A%7B%0A++std::tuple%3Cint,+long,+long%3E+foo%7B42,+58L,+9L%7D%3B%0A%0A++std::cout+%3C%3C+std::get%3Cint%3E(foo)+%3C%3C+!'%5Cn!'%3B%0A%23if+0%0A++std::cout+%3C%3C+std::get%3Clong%3E(foo)+%3C%3C+!'%5Cn!'%3B%0A%23endif%0A%7D%0A'),l:'5',n:'0',o:'C%2B%2B+source+%231',t:'0')),k:50,l:'4',n:'0',o:'',s:0,t:'0'),(g:!((h:executor,i:(argsPanelShown:'1',compilationPanelShown:'0',compiler:gsnapshot,compilerName:'',compilerOutShown:'0',execArgs:'',execStdin:'',fontScale:14,fontUsePx:'0',j:1,lang:c%2B%2B,libs:!(),options:'-std%3Dc%2B%2B14+-Wall+-Wextra',overrides:!(),runtimeTools:!(),source:1,stdinPanelShown:'1',tree:'1',wrap:'0'),l:'5',n:'0',o:'Executor+x86-64+gcc+(trunk)+(C%2B%2B,+Editor+%231)',t:'0')),header:(),k:50,l:'4',n:'0',o:'',s:0,t:'0')),l:'2',n:'0',o:'',t:'0')),version:4}{\codelogo}};
	\end{tikzpicture}

	\addproposal{N3670}{https://wg21.link/N3670}
\end{frame}

\subsection*{Template}
\begin{frame}[fragile]
	\frametitle{Variable template}
	\begin{itemize}
		\item Généralisation des templates aux variables
		\item Y compris les spécialisations
	\end{itemize}

	\begin{minted}{cpp}
		template<typename T>
		constexpr T PI = T(3.1415926535897932385);

		template<>
		constexpr const char* PI<const char*> = "pi";

		PI<int>;          // 3
		PI<double>;       // 3.14159
		PI<const char*>;  // pi
	\end{minted}

	\begin{tikzpicture}[remember picture,overlay]
	\node[xshift=-5mm,yshift=6mm] at (current page.south east){%
		\href{https://godbolt.org/#g:!((g:!((g:!((h:codeEditor,i:(filename:'1',fontScale:14,fontUsePx:'0',j:1,lang:c%2B%2B,selection:(endColumn:1,endLineNumber:15,positionColumn:1,positionLineNumber:15,selectionStartColumn:1,selectionStartLineNumber:1,startColumn:1,startLineNumber:1),source:'%23include+%3Ciostream%3E%0A%0Atemplate%3Ctypename+T%3E%0Aconstexpr+T+PI+%3D+T(3.1415926535897932385)%3B%0A%0Atemplate%3C%3E%0Aconstexpr+const+char*+PI%3Cconst+char*%3E+%3D+%22pi%22%3B%0A%0Aint+main()%0A%7B%0A++std::cout+%3C%3C+PI%3Cint%3E+%3C%3C+!'%5Cn!'%3B%0A++std::cout+%3C%3C+PI%3Cdouble%3E+%3C%3C+!'%5Cn!'%3B%0A++std::cout+%3C%3C+PI%3Cconst+char*%3E+%3C%3C+!'%5Cn!'%3B%0A%7D%0A'),l:'5',n:'0',o:'C%2B%2B+source+%231',t:'0')),k:50,l:'4',n:'0',o:'',s:0,t:'0'),(g:!((h:executor,i:(argsPanelShown:'1',compilationPanelShown:'0',compiler:gsnapshot,compilerName:'',compilerOutShown:'0',execArgs:'',execStdin:'',fontScale:14,fontUsePx:'0',j:1,lang:c%2B%2B,libs:!(),options:'-std%3Dc%2B%2B14+-Wall+-Wextra',overrides:!(),runtimeTools:!(),source:1,stdinPanelShown:'1',tree:'1',wrap:'0'),l:'5',n:'0',o:'Executor+x86-64+gcc+(trunk)+(C%2B%2B,+Editor+%231)',t:'0')),header:(),k:50,l:'4',n:'0',o:'',s:0,t:'0')),l:'2',n:'0',o:'',t:'0')),version:4}{\codelogo}};
	\end{tikzpicture}

	\addproposal{N3651}{https://wg21.link/N3651}
\end{frame}

\subsection*{Programmation fonctionnelle}
\begin{frame}[fragile]
	\frametitle{Generic lambdas}
	\begin{itemize}
		\item Lambdas utilisables sur différents types de paramètres
		\item Déduction du type des paramètres déclarés \mintinline{cpp}|auto|
	\end{itemize}

	\begin{minted}{cpp}
		auto foo = [] (auto in) { cout << in << '\n'; };

		foo(2);
		foo("azerty"s);
	\end{minted}

	\begin{tikzpicture}[remember picture,overlay]
	\node[xshift=-5mm,yshift=6mm] at (current page.south east){%
		\href{https://godbolt.org/#g:!((g:!((g:!((h:codeEditor,i:(filename:'1',fontScale:14,fontUsePx:'0',j:1,lang:c%2B%2B,selection:(endColumn:1,endLineNumber:13,positionColumn:1,positionLineNumber:13,selectionStartColumn:1,selectionStartLineNumber:13,startColumn:1,startLineNumber:13),source:'%23include+%3Ciostream%3E%0A%23include+%3Cstring%3E%0A%0Ausing+namespace+std::literals%3B%0A%0Aint+main()%0A%7B%0A++auto+foo+%3D+%5B%5D+(auto+in)+%7B+std::cout+%3C%3C+in+%3C%3C+!'%5Cn!'%3B+%7D%3B%0A%0A++foo(2)%3B%0A++foo(%22azerty%22s)%3B%0A%7D%0A'),l:'5',n:'0',o:'C%2B%2B+source+%231',t:'0')),k:50,l:'4',n:'0',o:'',s:0,t:'0'),(g:!((h:executor,i:(argsPanelShown:'1',compilationPanelShown:'0',compiler:gsnapshot,compilerName:'',compilerOutShown:'0',execArgs:'',execStdin:'',fontScale:14,fontUsePx:'0',j:1,lang:c%2B%2B,libs:!(),options:'-std%3Dc%2B%2B14+-Wall+-Wextra',overrides:!(),runtimeTools:!(),source:1,stdinPanelShown:'1',tree:'1',wrap:'0'),l:'5',n:'0',o:'Executor+x86-64+gcc+(trunk)+(C%2B%2B,+Editor+%231)',t:'0')),header:(),k:50,l:'4',n:'0',o:'',s:0,t:'0')),l:'2',n:'0',o:'',t:'0')),version:4}{\codelogo}};
	\end{tikzpicture}

	\addproposal{N3649}{https://wg21.link/N3649}
\end{frame}

\begin{frame}[fragile]
	\frametitle{Variadic lambdas}
	\begin{itemize}
		\item Lambda à nombre de paramètres variable
		\item Suffixe \mintinline{cpp}|...| à \mintinline{cpp}|auto|
	\end{itemize}

	\begin{minted}{cpp}
		auto foo = [] (auto... args) { 
		  std::cout << sizeof...(args) << '\n';
		};

		foo(2);           // 1
		foo(2, 3, 4);     // 3
		foo("azerty"s);   // 1
	\end{minted}

	\begin{tikzpicture}[remember picture,overlay]
	\node[xshift=-5mm,yshift=6mm] at (current page.south east){%
		\href{https://godbolt.org/#g:!((g:!((g:!((h:codeEditor,i:(filename:'1',fontScale:14,fontUsePx:'0',j:1,lang:c%2B%2B,selection:(endColumn:1,endLineNumber:14,positionColumn:1,positionLineNumber:14,selectionStartColumn:1,selectionStartLineNumber:14,startColumn:1,startLineNumber:14),source:'%23include+%3Ciostream%3E%0A%23include+%3Cstring%3E%0A%0Ausing+namespace+std::literals%3B%0A%0Aint+main()%0A%7B%0A++auto+foo+%3D+%5B%5D+(auto...+args)+%7B+std::cout+%3C%3C+sizeof...(args)+%3C%3C+!'%5Cn!'%3B+%7D%3B%0A%0A++foo(2)%3B%0A++foo(2,+3,+4)%3B%0A++foo(%22azerty%22s)%3B%0A%7D%0A'),l:'5',n:'0',o:'C%2B%2B+source+%231',t:'0')),k:50,l:'4',n:'0',o:'',s:0,t:'0'),(g:!((h:executor,i:(argsPanelShown:'1',compilationPanelShown:'0',compiler:gsnapshot,compilerName:'',compilerOutShown:'0',execArgs:'',execStdin:'',fontScale:14,fontUsePx:'0',j:1,lang:c%2B%2B,libs:!(),options:'-std%3Dc%2B%2B14+-Wall+-Wextra',overrides:!(),runtimeTools:!(),source:1,stdinPanelShown:'1',tree:'1',wrap:'0'),l:'5',n:'0',o:'Executor+x86-64+gcc+(trunk)+(C%2B%2B,+Editor+%231)',t:'0')),header:(),k:50,l:'4',n:'0',o:'',s:0,t:'0')),l:'2',n:'0',o:'',t:'0')),version:4}{\codelogo}};
	\end{tikzpicture}
\end{frame}

\begin{frame}[fragile]
	\frametitle{Capture généralisée}
	\begin{itemize}
		\item Création de variables capturées depuis des variables locales ou des constantes
	\end{itemize}

	\begin{minted}{cpp}
		int foo = 42;

		auto bar = [ &x = foo ]() { --x; };
		bar();  // foo : 41

		auto baz = [ y = 10 ]() { cout << y << '\n'; };
		baz();  // 10

		auto qux = [ z = 2 * foo ]() { cout << z << '\n'; };
		qux();  // 82
	\end{minted}

	\addproposal{N3648}{https://wg21.link/N3648}
\end{frame}

\begin{frame}[fragile]
	\frametitle{Capture généralisée}
	\begin{itemize}
		\item Capture par déplacement
	\end{itemize}

	\begin{minted}{cpp}
		auto foo = make_unique<int>(42);
		auto bar = [ foo = move(foo) ](int i) {
		  cout << *foo * i << '\n';
		};

		bar(5);  // Affiche 210
	\end{minted}

	\begin{itemize}
		\item Capture des variables membres
	\end{itemize}

	\begin{minted}{cpp}
		struct Bar {
		  auto foo() { return [s=s] { cout << s << '\n'; }; }

		  string s;
		};
	\end{minted}

	\begin{tikzpicture}[remember picture,overlay]
	\node[xshift=-5mm,yshift=6mm] at (current page.south east){%
		\href{https://godbolt.org/#g:!((g:!((g:!((h:codeEditor,i:(filename:'1',fontScale:14,fontUsePx:'0',j:1,lang:c%2B%2B,selection:(endColumn:1,endLineNumber:17,positionColumn:1,positionLineNumber:17,selectionStartColumn:1,selectionStartLineNumber:17,startColumn:1,startLineNumber:17),source:'%23include+%3Ciostream%3E%0A%23include+%3Cstring%3E%0A%0Ausing+namespace+std::literals%3B%0A%0Aint+main()%0A%7B%0A++int+foo+%3D+42%3B%0A%0A++auto+bar+%3D+%5B+%26x+%3D+foo+%5D()+%7B+--x%3B+%7D%3B%0A++bar()%3B%0A++std::cout+%3C%3C+foo+%3C%3C+!'%5Cn!'%3B%0A%0A++auto+baz+%3D+%5B+y+%3D+2*foo+%5D()+%7B+std::cout+%3C%3C+y+%3C%3C+!'%5Cn!'%3B+%7D%3B%0A++baz()%3B%0A%7D%0A'),l:'5',n:'0',o:'C%2B%2B+source+%231',t:'0')),k:50,l:'4',n:'0',o:'',s:0,t:'0'),(g:!((h:executor,i:(argsPanelShown:'1',compilationPanelShown:'0',compiler:gsnapshot,compilerName:'',compilerOutShown:'0',execArgs:'',execStdin:'',fontScale:14,fontUsePx:'0',j:1,lang:c%2B%2B,libs:!(),options:'-std%3Dc%2B%2B14+-Wall+-Wextra',overrides:!(),runtimeTools:!(),source:1,stdinPanelShown:'1',tree:'1',wrap:'0'),l:'5',n:'0',o:'Executor+x86-64+gcc+(trunk)+(C%2B%2B,+Editor+%231)',t:'0')),header:(),k:50,l:'4',n:'0',o:'',s:0,t:'0')),l:'2',n:'0',o:'',t:'0')),version:4}{\codelogo}};
	\end{tikzpicture}
\end{frame}

\begin{frame}[fragile]
	\frametitle{Améliorations des lambdas}
	\begin{itemize}
		\item Type de retour complètement facultatif

\note[item]{Il n'y a plus les restrictions de C++11 (une seule instruction, de type \mintinline{cpp}|return|)}

		\item Conversion possible de lambda sans capture en pointeur de fonction

\note[item]{Donc passable à des fonctions C attendant un pointeur de fonction en paramètre}
	\end{itemize}

	\begin{minted}{cpp}
		void foo(void(* bar)(int))

		foo([](int x) { cout << x << endl; });
	\end{minted}

	\begin{itemize}
		\item Peuvent être \mintinline{cpp}|noexcept|
		\item Ajout des paramètres par défaut aux lambdas
	\end{itemize}

	\begin{minted}{cpp}
		auto foo = [] (int bar = 12) { cout << bar << '\n'; };
	\end{minted}
\end{frame}

\subsection*{Type traits}
\begin{frame}[fragile]
	\frametitle{\mintinline[style=white]{cpp}|std::is_final|}
	\begin{itemize}
		\item Indique si la classe est finale ou non
	\end{itemize}

	\begin{minted}{cpp}
		class Foo {};
		class Bar final {};

		is_final<Foo>::value;   // false
		is_final<Bar>::value;   // true
	\end{minted}
\end{frame}

\begin{frame}[fragile]
	\frametitle{Alias transformation}
	\begin{itemize}
		\item Simplification de l'usage des transformations de types
		\item Ajout du suffixe \mintinline{cpp}|_t| aux transformations
		\item Suppression de \mintinline{cpp}|typename| et \mintinline{cpp}|::type|
	\end{itemize}

	\begin{minted}{cpp}
		typedef add_const<int>::type A;
		typedef add_const<const int>::type B;
		typedef add_const<const int*>::type C;

		// Deviennent

		add_const_t<int> A;
		add_const_t<const int> B;
		add_const_t<const int*> C;
	\end{minted}
\end{frame}

\subsection*{Pointeurs intelligents}
\begin{frame}[fragile]
	\frametitle{\mintinline[style=white]{cpp}|std::make_unique|}
	\begin{itemize}
		\item Allocation et construction de l'objet dans le \mintinline{cpp}|std::unique_ptr|
	\end{itemize}

	\begin{minted}{cpp}
		unique_ptr<int> foo = make_unique<int>(42);
	\end{minted}

	\begin{alertblock}{Don't}
		\begin{itemize}
			\item Plus de \mintinline{cpp}|new| dans le code applicatif
		\end{itemize}
	\end{alertblock}

	\begin{block}{Note}
		\begin{itemize}
			\item Utilisable pour construire dans un conteneur
		\end{itemize}
	\end{block}

	\addproposal{N3656}{https://wg21.link/N3656}
\end{frame}

\subsection*{Attributs}
\begin{frame}[fragile]
	\frametitle{Attribut \mintinline[style=white]{cpp}|[[ deprecated ]]|}
	\begin{itemize}
		\item Indique qu'une entité (variable, fonction, classe, \ldots{}) est dépréciée
		\item Émission possible d'avertissement sur l'utilisation d'une entité \mintinline{cpp}|deprecated|

\note[item]{Possible car il n'y a pas d'obligation dans la norme. En pratique c'est le cas}
	\end{itemize}

	\begin{minted}{cpp}
		[[ deprecated ]]
		void bar() {}

		class [[ deprecated ]] Baz {};

		[[ deprecated ]]
		int foo{42};
	\end{minted}

	\addproposal{N3760}{https://wg21.link/N3760}
\end{frame}

\begin{frame}[fragile]
	\frametitle{Attribut \mintinline[style=white]{cpp}|[[ deprecated ]]|}
	\begin{itemize}
		\item Possibilité de fournir un message explicatif
	\end{itemize}

	\begin{minted}{cpp}
		[[ deprecated("utilisez foo") ]]
		void bar() {}
	\end{minted}

	\begin{minted}{shell}
		warning: 'void bar()' is deprecated: utilisez foo
	\end{minted}

	\begin{tikzpicture}[remember picture,overlay]
	\node[xshift=-5mm,yshift=6mm] at (current page.south east){%
		\href{https://godbolt.org/#g:!((g:!((g:!((h:codeEditor,i:(filename:'1',fontScale:14,fontUsePx:'0',j:1,lang:c%2B%2B,selection:(endColumn:8,endLineNumber:14,positionColumn:8,positionLineNumber:14,selectionStartColumn:8,selectionStartLineNumber:14,startColumn:8,startLineNumber:14),source:'%23include+%3Ciostream%3E%0A%0A%5B%5B+deprecated(%22Utilisez+Foo%22)+%5D%5D%0Astatic+void+bar()%0A%7B%0A%7D%0A%0Aclass+%5B%5B+deprecated+%5D%5D+Baz%0A%7B%0A%7D%3B%0A%0Aint+main()%0A%7B%0A++bar()%3B%0A%0A++Baz+baz%3B%0A%0A++%5B%5B+deprecated+%5D%5D%0A++int+foo%7B42%7D%3B%0A++std::cout+%3C%3C+foo+%3C%3C+!'%5Cn!'%3B%0A%7D%0A'),l:'5',n:'0',o:'C%2B%2B+source+%231',t:'0')),k:50,l:'4',n:'0',o:'',s:0,t:'0'),(g:!((h:executor,i:(argsPanelShown:'1',compilationPanelShown:'0',compiler:gsnapshot,compilerName:'',compilerOutShown:'0',execArgs:'',execStdin:'',fontScale:14,fontUsePx:'0',j:1,lang:c%2B%2B,libs:!(),options:'-std%3Dc%2B%2B14+-Wall+-Wextra',overrides:!(),runtimeTools:!(),source:1,stdinPanelShown:'1',tree:'1',wrap:'0'),l:'5',n:'0',o:'Executor+x86-64+gcc+(trunk)+(C%2B%2B,+Editor+%231)',t:'0')),header:(),k:50,l:'4',n:'0',o:'',s:0,t:'0')),l:'2',n:'0',o:'',t:'0')),version:4}{\codelogo}};
	\end{tikzpicture}
\end{frame}

\subsection*{Multi-threading}
\begin{frame}[fragile]
	\frametitle{\mintinline[style=white]{cpp}|std::shared_timed_mutex|}
	\begin{itemize}
		\item Similaire à \mintinline{cpp}|std::timed_mutex| avec deux niveaux d'accès
		\begin{itemize}
			\item Exclusif : possible si le verrou n'est pas pris
			\item Partagé : possible si le verrou n'est pas pris en exclusif
		\end{itemize}
		\item Même API que \mintinline{cpp}|std::timed_mutex| pour l'accès exclusif
		\item API similaire pour l'accès partagé
		\begin{itemize}
			\item \mintinline{cpp}|lock_shared|
			\item \mintinline{cpp}|try_lock_shared|
			\item \mintinline{cpp}|try_lock_shared_for|
			\item \mintinline{cpp}|try_lock_shared_until|
			\item \mintinline{cpp}|unlock_shared|
		\end{itemize}
	\end{itemize}

	\begin{alertblock}{Attention}
		\begin{itemize}
			\item Un thread ne doit pas prendre un mutex qu'il possède déjà
			\item Même en accès partagé
		\end{itemize}
	\end{alertblock}

	\addproposal{N3659}{https://wg21.link/N3659}
\end{frame}

\begin{frame}[fragile]
	\frametitle{\mintinline[style=white]{cpp}|std::shared_lock|}
	\begin{itemize}
		\item Capsule RAII sur les mutex partagés
		\item Support des mutex verrouillés ou non
		\item Relâche le mutex à la destruction
		\item Similaire à \mintinline{cpp}|std::unique_lock| mais en accès partagée
	\end{itemize}

	\begin{minted}{cpp}
		shared_timed_mutex foo;
		{
		  shared_lock<shared_timed_mutex> bar(foo, defer_lock);
		  ...
		  bar.lock();  // Prise du mutex
		  ...
		}  // Liberation du mutex
	\end{minted}
\end{frame}
\end{document}