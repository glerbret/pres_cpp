\documentclass[C++.tex]{subfiles}
\begin{document}

\section{C++14}
\subsection*{Présentation}
\begin{frame}
	\frametitle{Présentation}
	\begin{itemize}
		\item Approuvé le 16 août 2014
		\item Dernier \textit{Working Draft} : \href{https://timsong-cpp.github.io/cppwp/n4140/draft.pdf}{N4140}
		\item Dans la continuité de C++11
		\item Changement moins important
		\item Mais loin d'une simple version correctrice
		\item Support complet par GCC, Clang et Visual C++

\note[item]{Et de tous les compilateurs récents}
	\end{itemize}
\end{frame}

\subsection*{constexpr}
\begin{frame}[fragile]
	\frametitle{\lstinline|constexpr|}
	\begin{itemize}
		\item Fonctions membres \lstinline|constexpr| ne sont plus implicitement \lstinline|const|
		\item Relâchement des contraintes sur les fonctions \lstinline|constexpr|
		\begin{itemize}
			\item Variables locales (ni \lstinline|static|, ni \lstinline|thread_local| et obligatoirement initialisées)
			\item Objets mutables créés lors l'évaluation de l'expression constante
			\item \lstinline|if|, \lstinline|switch|, \lstinline|while|, \lstinline|for|, \lstinline|do while|
		\end{itemize}
		\item Application de \lstinline|constexpr| à plusieurs éléments de la bibliothèque standard
	\end{itemize}
\end{frame}

\subsection*{Déduction de type}
\begin{frame}[fragile]
	\frametitle{Généralisation de la déduction du type retour\titlehfill{}1/2}
	\begin{itemize}
		\item Utilisable sur les lambdas complexes
	\end{itemize}

	\begin{lstlisting}[language=C++]
[](int x) { 
  if(x >= 0) return 2 * x; 
  else return -2 * x;};\end{lstlisting}

	\pause

	\begin{itemize}
		\item Mais aussi sur les fonctions
	\end{itemize}

	\begin{lstlisting}[language=C++]
auto bar(int x) {
  if(x >= 0) return 2 * x; 
  else return -2 * x;}\end{lstlisting}
\end{frame}

\begin{frame}[fragile]
	\frametitle{Généralisation de la déduction du type retour\titlehfill{}2/2}
	\begin{itemize}
		\item Y compris récursive
	\end{itemize}

	\begin{lstlisting}[language=C++]
auto fact(unsigned int x) {
  if(x == 0) return 1U;
  else return x * fact(x-1);}\end{lstlisting}

	\begin{alertblock}{Contraintes}
		\begin{itemize}
			\item Un \lstinline|return| doit précéder l'appel récursive
			\item Tous les chemins doivent avoir le même type de retour
		\end{itemize}
	\end{alertblock}
\end{frame}

\begin{frame}[fragile]
	\frametitle{\lstinline|decltype(auto)|}
	\begin{itemize}
		\item Détermine le type retour en conservant la référence
	\end{itemize}

	\begin{lstlisting}[language=C++]
string bar("bar");

string  foo1() { return string("foo"); }
string& bar1() { return bar; }

decltype(auto) foo2() { return foo1(); } // string
decltype(auto) bar2() { return bar1(); } // string&
auto foo3() { return foo1(); }           // string
auto bar3() { return bar1(); }           // string\end{lstlisting}
\end{frame}

\subsection*{Initialisation}
\begin{frame}[fragile]
	\frametitle{\textit{Aggregate Initialisation}}
	\begin{itemize}
		\item Devient compatible avec l'initialisation par défaut des membres
		\item \ldots{} les membres non explicitement initialisés le sont par défaut
	\end{itemize}

	\begin{lstlisting}[language=C++]
	struct Foo {int i, int j = 5};

	Foo foo{42};   // i = 42, j = 5\end{lstlisting}
\end{frame}

\subsection*{Itérateurs}
\begin{frame}[fragile]
	\frametitle{Itérateurs}
	\begin{itemize}
		\item Fonctions libres \lstinline|std::cbegin()| et \lstinline|std::cend()|
		\item Fonctions libres \lstinline|std::rbegin()| et \lstinline|std::rend()|
		\item Fonctions libres \lstinline|std::crbegin()| et \lstinline|std::crend()|
		\item \textit{Null forward iterator} ne référençant aucun conteneur valide
	\end{itemize}

	\begin{lstlisting}[language=C++]
auto ni = vector<int>::iterator();
auto nd = vector<double>::iterator();

ni == ni;    // true
nd != nd;    // false
ni == nd;    // Erreur de compilation\end{lstlisting}

	\begin{alertblock}{Attention}
		\textit{Null forward iterator} non comparables avec des itérateurs \og classiques\fg{}

\note[item]{But ? Pouvoir créer un itérateur sans l'initialiser avec un conteneur existant. Assez limité de mon point de vue}
	\end{alertblock}
\end{frame}

\subsection*{Algorithmes}
\begin{frame}[fragile]
	\frametitle{Algorithmes}
	\begin{itemize}
		\item Surcharge de \lstinline|std::equal()|, \lstinline|std::mismatch()| et de \lstinline|std::is_permutation()| prenant deux paires complètes d'itérateurs
	\end{itemize}

	\begin{block}{Note}
		Il n'est donc plus nécessaire de tester la taille auparavant
	\end{block}

	\begin{itemize}
		\item \lstinline|std::exchange()| change la valeur d'un objet et retourne l'ancienne

\note[item]{Pas un algorithme à proprement parler, comme \lstinline|std::swap|}
	\end{itemize}

	\begin{lstlisting}[language=C++]
vector<int> foo{1, 2, 3};
vector<int> bar = exchange(foo, {10, 11});
// foo : 10 11, bar : 1, 2, 3\end{lstlisting}

	\begin{block}{Dépréciation}
		\lstinline|std::random_shuffle()| est déprécié

\note[item]{Remplacé par \lstinline|std::shuffle()| qui permet un meilleur aléa}
	\end{block}
\end{frame}

\subsection*{string}
\begin{frame}[fragile]
	\frametitle{\textit{Quoted string}}
	\begin{itemize}
		\item Insertion et extraction de chaînes avec guillemets
	\end{itemize}

	\begin{lstlisting}[language=C++]
stringstream ss;
string in = "String with spaces and \"quotes\"";
string out;

ss << quoted(in);
cout << "in:  '" << in << "'\n"
     << "stored as '" << ss.str() << "'\n";
// in : 'String with spaces and "quotes"'
// stored as '"String with spaces and \"quotes\""'

ss >> quoted(out);
cout << "out: '" << out << "'\n";
// out: 'String with spaces, and "quotes"'\end{lstlisting}

\note[item]{Exemple issue de cppreference.com}
\end{frame}

\subsection*{Littéraux}
\begin{frame}[fragile]
	\frametitle{Littéraux binaires}
	\begin{itemize}
		\item Support des littéraux binaires grâce au préfixe \og 0b\fg{}
	\end{itemize}

	\begin{lstlisting}[language=C++]
int foo = 0b101010; // 42\end{lstlisting}
\end{frame}

\begin{frame}[fragile]
	\frametitle{Séparateurs}
	\begin{itemize}
		\item Utilisation possible de ' dans les nombres littéraux
	\end{itemize}

	\begin{lstlisting}[language=C++]
int foo = 0b0010'1010;  // 42
int bar = 1'000;        // 1000
int baz = 010'00;       // 512\end{lstlisting}

	\begin{block}{Note}
		Purement esthétique, aucune sémantique ni place réservée
	\end{block}
\end{frame}

\begin{frame}[fragile]
	\frametitle{\textit{User-defined literals} standard\titlehfill{}1/3}

\note[item]{Dans la bibliothéque et non dans le langage}
\note[item]{Notons l'absence de \_}

	\begin{itemize}
		\item Suffixe \og s\fg{} sur les chaînes : \lstinline|std::string|
	\end{itemize}

	\begin{lstlisting}[language=C++]
auto foo = "abcd"s;   // string\end{lstlisting}

	\begin{block}{Note}
		Remplace avantageusement \lstinline|std::string("abcd")| dans de nombreux contextes
	\end{block}
\end{frame}

\begin{frame}[fragile]
	\frametitle{\textit{User-defined literals} standard\titlehfill{}2/3}
	\begin{itemize}
		\item Suffixe \og h\fg{}, \og min\fg{}, \og s\fg{}, \og ms\fg{}, \og us\fg{} et \og ns\fg{} : \lstinline|std::chrono::duration| dans l'unité correspondante
	\end{itemize}

	\begin{lstlisting}[language=C++]
auto foo = 60s;       // chrono::seconds
auto bar = 5min;      // chrono::minutes\end{lstlisting}

	\begin{block}{Note}
		Suffixe \og s\fg{} utilisé pour \lstinline|std::string| et pour les secondes mais sans ambiguïté car dépendant du type de littéral auquel il s'applique
	\end{block}
\end{frame}

\begin{frame}[fragile]
	\frametitle{\textit{User-defined literals} standard\titlehfill{}3/3}
	\begin{itemize}
		\item Suffixe \og if\fg{} : nombre imaginaire de type \lstinline|std::complex<float>|
		\item Suffixe \og i\fg{} : nombre imaginaire de type \lstinline|std::complex<double>|
		\item Suffixe \og il\fg{} : nombre imaginaire de type \lstinline|std::complex<long double>|
	\end{itemize}

	\begin{lstlisting}[language=C++]
auto foo = 5i;        // complex<double>\end{lstlisting}
\end{frame}

\subsection*{tuple}
\begin{frame}[fragile]
	\frametitle{Adressage des tuples par le type}
	\begin{itemize}
		\item Utilisation du type plutôt que de l'indice
	\end{itemize}

	\begin{lstlisting}[language=C++]
tuple<int, long, long> foo{42, 58L, 9L};

get<int>(foo);	// 42\end{lstlisting}

	\pause

	\begin{alertblock}{Attention}
		Uniquement s'il n'y a qu'une occurrence du type dans le tuple
	\begin{lstlisting}[language=C++]
get<long>(foo);     // Erreur\end{lstlisting}
	\end{alertblock}
\end{frame}

\subsection*{Template}
\begin{frame}[fragile]
	\frametitle{Variable template}
	\begin{itemize}
		\item Généralisation des templates aux variables
		\item Y compris les spécialisations
	\end{itemize}

	\begin{lstlisting}[language=C++]
template<typename T>
constexpr T PI = T(3.1415926535897932385);

template<>
constexpr const char* PI<const char*> = "pi";

PI<int>;          // 3
PI<double>;       // 3.14159
PI<const char*>;  // pi\end{lstlisting}

	\begin{block}{Sous silence \ldots}
		\lstinline|std::integer_sequence|
	\end{block}
\end{frame}

\subsection*{Programmation fonctionnelle}
\begin{frame}[fragile]
	\frametitle{\textit{Generic lambdas}}
	\begin{itemize}
		\item Lambdas utilisables sur différents types de paramètres
		\item Déduction du type des paramètres déclarés \lstinline|auto|
	\end{itemize}

	\begin{lstlisting}[language=C++]
auto foo = [] (auto in) { cout << in << '\n'; };

foo(2);
foo("azerty"s);\end{lstlisting}
\end{frame}

\begin{frame}[fragile]
	\frametitle{\textit{Variadic lambdas}}
	\begin{itemize}
		\item Lambda à nombre de paramètres variable
		\item Suffixe \lstinline|...| à \lstinline|auto|
	\end{itemize}

	\begin{lstlisting}[language=C++]
auto foo = [] (auto... args) { 
  std::cout << sizeof...(args) << '\n'; };

foo(2);           // 1
foo(2, 3, 4);     // 3
foo("azerty"s);   // 1\end{lstlisting}
\end{frame}

\begin{frame}[fragile]
	\frametitle{Capture généralisée\titlehfill{}1/2}

\note[item]{On parle parfois de renommage des captures}

	\begin{itemize}
		\item Création de variables capturées depuis des variables locales ou des constantes
	\end{itemize}

	\begin{lstlisting}[language=C++]
int foo = 42;

auto bar = [ &x = foo ]() { --x; };
bar();  // foo : 41

auto baz = [ y = 10 ]() { cout << y << '\n'; };
baz();  // 10

auto qux = [ z = 2*foo ]() { cout << z << '\n'; };
qux();  // 82\end{lstlisting}
\end{frame}

\begin{frame}[fragile]
	\frametitle{Capture généralisée\titlehfill{}2/2}
	\begin{itemize}
		\item Capture par déplacement
	\end{itemize}
	
	\begin{lstlisting}[language=C++]
auto foo = make_unique<int>(42);
auto bar = [ foo = move(foo) ](int i) {
  cout << *foo * i << '\n'; };

bar(5);  // Affiche 210\end{lstlisting}

	\begin{itemize}
		\item Capture des variables membres
	\end{itemize}

	\begin{lstlisting}[language=C++]
struct Bar {
	auto foo() {
		return [s=s] { cout << s << '\n'; }; }

	std::string s;};\end{lstlisting}
\end{frame}

\begin{frame}[fragile]
	\frametitle{Améliorations des lambdas}
	\begin{itemize}
		\item Type de retour complètement facultatif

\note[item]{Il n'y a plus les restrictions de C++11 (une seule instruction, de type \lstinline|return|)}

		\item Une lambda qui ne capture rien peut être convertie en pointeur de fonction

\note[item]{Donc passable à des fonctions C attendant un pointeur de fonction en paramètre}

\begin{lstlisting}[language=C++]
void foo(void(* bar)(int))
		
foo([](int x) { std::cout << x << std::endl; });\end{lstlisting}

		\item Peuvent être \lstinline|noexcept|
		\item Ajout des paramètres par défaut aux lambdas
	
		\begin{lstlisting}[language=C++]
auto foo = [] (int bar = 12) { cout << bar << '\n'; };\end{lstlisting}
	\end{itemize}
\end{frame}

\subsection*{Type traits}
\begin{frame}[fragile]
	\frametitle{\lstinline|std::is_final|}
	\begin{itemize}
		\item Indique si la classe est finale ou non
	\end{itemize}

	\begin{lstlisting}[language=C++]
class Foo {};
class Bar final {};

is_final<Foo>::value;   // false
is_final<Bar>::value;   // true\end{lstlisting}
\end{frame}

\begin{frame}[fragile]
	\frametitle{Alias transformation}
	\begin{itemize}
		\item Simplification de l'usage des transformations de types
		\item Ajout du suffixe \lstinline|_t| aux transformations
		\item Suppression de \lstinline|typename| et \lstinline|::type|
	\end{itemize}

	\begin{lstlisting}[language=C++]
typedef add_const<int>::type A;
typedef add_const<const int>::type B;
typedef add_const<const int*>::type C;

// Deviennent

add_const_t<int> A;
add_const_t<const int> B;
add_const_t<const int*> C;\end{lstlisting}
\end{frame}

\subsection*{Pointeur intelligent}
\begin{frame}[fragile]
	\frametitle{\lstinline|std::make_unique|}
	\begin{itemize}
		\item Alloue et construit l'objet dans le \lstinline|std::unique_ptr|
	\end{itemize}

	\begin{lstlisting}[language=C++]
unique_ptr<int> foo = make_unique<int>(42);\end{lstlisting}

	\begin{alertblock}{Don't}
		Plus de \lstinline|new| dans le code applicatif
	\end{alertblock}

	\begin{block}{Note}
		Utilisable pour construire dans un conteneur
	\end{block}
\end{frame}

\subsection*{Attributs}
\begin{frame}[fragile]
	\frametitle{Attribut \lstinline|[[ deprecated ]]|\titlehfill{}1/2}
	\begin{itemize}
		\item Indique qu'une entité (variable, fonction, classe, \ldots) est dépréciée
		\item Émission possible d'un warning sur l'utilisation d'une entité \lstinline|deprecated|

\note[item]{Peut car il n'y a pas d'obligation dans la norme. En pratique c'est le cas}
	\end{itemize}

	\begin{lstlisting}[language=C++]
[[ deprecated ]]
void bar() {}

class [[ deprecated ]] Baz { };

[[ deprecated ]]
int foo{42};\end{lstlisting}
\end{frame}

\begin{frame}[fragile]
	\frametitle{Attribut \lstinline|[[ deprecated ]]|\titlehfill{}2/2}
	\begin{itemize}
		\item Possibilité de fournir un message explicatif
	\end{itemize}

	\begin{lstlisting}[language=C++]
[[ deprecated("utilisez foo") ]]
void bar() {}\end{lstlisting}

	\begin{lstlisting}
warning: 'void bar()' is deprecated: utilisez foo\end{lstlisting}
\end{frame}

\subsection*{Multi-threading}
\begin{frame}[fragile]
	\frametitle{\lstinline|std::shared_timed_mutex|}
	\begin{itemize}
		\item Similaire à \lstinline|std::timed_mutex|
		\item \ldots{}avec deux niveaux d'accès
		\begin{itemize}
			\item Exclusif : possible si le verrou n'est pas pris
			\item Partagé : possible si le verrou n'est pas pris en exclusif
		\end{itemize}
		\item Même API que \lstinline|std::timed_mutex|  pour l'accès exclusif
		\item API similaire pour l'accès partagé
		\begin{itemize}
			\item \lstinline|lock_shared|
			\item \lstinline|try_lock_shared|
			\item \lstinline|try_lock_shared_for|
			\item \lstinline|try_lock_shared_until|
			\item \lstinline|unlock_shared|
		\end{itemize}
	\end{itemize}

	\begin{alertblock}{Attention}
		Un thread ne doit pas prendre un mutex qu'il possède déjà, même en accès partagé
	\end{alertblock}
\end{frame}

\begin{frame}[fragile]
	\frametitle{\lstinline|std::shared_lock|}
	\begin{itemize}
		\item Capsule RAII sur les mutex partagés
		\item Supporte les mutex verrouillés ou non
		\item Relâche le mutex à la destruction
		\item Similaire à \lstinline|std::unique_lock| mais en accès partagée
	\end{itemize}

	\begin{lstlisting}[language=C++]
shared_timed_mutex foo;
{
  shared_lock<shared_timed_mutex> bar(foo, defer_lock);
  ...
  bar.lock();  // Prise du mutex
  ...
}  // Liberation du mutex\end{lstlisting}
\end{frame}
\end{document}